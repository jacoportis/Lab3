Allora, vi dicevo oggi tratteremo l'ultimo argomento sui rivelatori, abbiamo abbiamo a trattato i rivelatori che abbiamo utilizzato il laboratorio per le prime esperienze, quindi quindi rivelatori a gas, i rivelatori a scintillazione. Adesso l'ultima tipologia di rivelatore che rimane da ad introdure sono i rivelatori a semiconduttore, è detto anche rivelatore stato solido. Quindi per introdure questi rivelatori dobbiamo richiamare qualche concetto di base della fisica dei semiconduttori che io immagino abbiate affrontato già in altre materie, probabilmente in strutture della materia o in altri corsi, datevi un feedback su questo. Ma che laboratorio giude? Ha il laboratorio 2 quindi abbiamo fatto qualche simile su semiconduttori, semiconduttori drogati, immagino in vio di semiconduttori, quindi qualche concetto lo abbiamo. Noi andremo invece a analizzare i semiconduttori per il loro utilizzo come rivelatori. Allora, questi rivelatori e semiconduttori come diceva la stessa parola si basano sull'utilizzo di materiali semiconduttori che sapete sono dei materiali cristallini come ad esempio quello che vedete qui in questa figura, quindi una struttura ordinata di atomi in un reticolo. Generalmente i rivelatori e semiconduttori sono rivelatori tipicamente o al silicio o al germanio, in in adesso esistono anche altre tipologie di rivelatori però quelle più comunemente utilizzate, comunque i trindi che sono stati inventati si basano sull'utilizzo di silicio e germanio, ovviamente non insieme, silicio o germanio. Sono anche detti rivelatori a stato solido perché hanno una densità che è circa mille volte maggiore rispetto a quella dei gas, quindi questa è una notevole differenza rispetto a quello che abbiamo studiato al primo semestre, rivelatori a gas si basano sull'utilizzo di un gas che sapete a densità molto basse, qui abbiamo invece un materiale solido e questo comporta delle differenze delle caratteristiche che andremo ad approfondire. Sono rivelatori che ormai hanno una loro storia perché i primi prototipi sono stati sviluppati negli anni 60, da allora chiaramente ci sono stati enormi sviluppi e oggi abbiamo rivelatori al silicio particolarmente performanti pensati per diversi tipi di applicazione, quindi quello che noi faremo in questa presentazione è introdurre il concetto di base di rivelazione basata sul semiconduttore, quindi quindi sostanzialmente è stato inventato e rivelato dal semiconduttore, però poi alla fine di questa presentazione vi farò vedere alcune applicazioni, alcuni esempi di come i rivelatori al semiconduttore sono stati sviluppati e migliorati nel corso degli anni. Quelli sono i principali vantaggi di questi rivelatori. Allora in anzitutto hanno un'ottima risoluzione energetica. Se vi ricordate durante il primo ciclo di esperienze avevate la possibilità di misurare la risoluzione in energia con un rivelatore a scintillazione. Se vi ricordate venivano fuori quegli spettri acquisiti utilizzando l'esorgentiga, ma è andando a guardare la larghezza del picco cocaelettrico possiamo valutare la risoluzione in energia. A qualcuno l'ha fatto lì sul momento perché magari ha avuto il tempo e gli strumenti per poterlo fare, altri lo rifaranno quando magari dovranno analizzare i dati in sede di stessura della tessina. Comunque se qualcuno l'ha fatto si è risolvonto che un sistema basato sull'utilizzo di uno scintillatore e di un fotomoltiplicatore porta ad avere risoluzioni in energia che non sono particolarmente spinte, stiamo parlando di risoluzioni dell'ordine del 5-10\%. Vi ricordo che la risoluzione in energia esprime la capacità di uno rivelatore di distinguere due valori in energia molto vicini tra di loro. Quindi se la risoluzione non è buona quello che succede è che si rischia di non riuscire a distinguere due valori in energia molto vicini tra di loro. I rivelatori semi-conduttore invece presentano delle buone risoluzioni energetiche.

Altra caratteristica che noi ritengiamo un vantaggio è l'elevato stopping power, che infatti il fatto di avere a disposizione un materiale solido fa sì che la radiazione che penetra all'interno del rivelatore venga più facilmente arrestata, quindi perda più facilmente energia. Pensiamo ad esempio a un elettrone che deve entrare in un rivelatore a gas per delle energie oppure entrare in un rivelatore a silicio per delle energie. Nel rivelatore a silicio percorre veramente pochi millimetri, un rivelatore a gas potrebbe percorre diversi centimetri. Quindi questo fa sì che affinché ad esempio il rivelatore venga utilizzato per misurare tutta l'energia di una particella sono sufficienti spessori e dimensioni compatte. Ad esempio lo vedrete in laboratorio, questi rivelatori noi utilizzeremo per misurare particelle alfa da 5 mezzo, grosso modo, e le particelle alfa dei 5 mezzo vengono arrestati in poche decine, di micron, di silicio. Quindi è sufficiente il rivelatore ad esempio spesso 50 micron come quello che utilizzate il laboratorio per essere sicuri che le particelle alfa vengono arrestate all'interno del rivelatore, quindi andiamo a misurare tutta l'energia della particella alfa. Altro vantaggio è che in generale questi rivelatori che vedremo sono sostanzialmente delle raggiunzioni PN polarizzate inversamente richiedono delle basse tensioni di alimentazione. E questo è un vantaggio da un punto di vista pratico. Se Se confrontiamo ad esempio con i fotomoltiplicatori che richiedono centinaia di volt, se vi ricordate abbiamo modificato la tensione di lavoro del fotomoltiplicatore usato per i $\gamma$, e li arrivate anche 600 volt, ma ci sono fotomoltiplicatori che richiedono tensioni anche più elevate. Oppure il contatore Geiger magari non vi siatere si conto dell'attenzione di alimentazione perché non l'abbiamo modificata, però anche lì siamo dell'ordine di 300-400 volt. Invece per il rivelatore al silicio lo vedrete sono sufficienti decine di volt per la polarizzazione. Un altro vantaggio riguarda la risposta del rivelatore con una risposta abbastanza veloce. Questo è utile soprattutto per le misure di timing. Cosa che invece non avevamo ad esempio per i rivelatori a gas. Se vi ricordate vi ho sempre detto sono dei rivelatori piuttosto solenti proprio per i meccanismi e i fenomeni che avvengono all'interno del rivelatore a gas. Chiaramente ci sono anche degli svantaggi. Il Il svantaggio è il fatto che questi rivelatori sono parecchio sensibili alla temperatura, quindi le condizioni di lavoro possono cambiare a seconda della temperatura, ambiente della temperatura in cui si sta operando. A volte alcuni di questi richiedono addirittura proprio un sistema di raffreddamento perché altrimenti il rumore sarebbe eccessivamente elevato e questo è il caso del germanio. In generale i rivelatori al germanio necessitano di un raffreddamento. Altro svantaggio che però a noi non interessa per le cose che facciamo in laboratorio riguarda il danneggiamento della radiazione. Infatti essendo il semiconduttore un reticolo, una struttura ordinata in cristallo quando viene sottoposto a un'elevata dose di radiazione questa dose può produrre dei danni al reticolo quindi modificare in qualche modo la struttura ordinata del reticolo e causare delle conseguenze sul funzionamento del rivelatore. Ora vi dicevo questo non è un problema per le esperienze che svolgiamo in laboratorio, dal momento che noi utilizziamo una sorgente alpha che è comunque un'attività abbastanza bassa. Questi problemi si presentano laddove questi rivelatori devono essere adoperati in presenza di alte radiazioni, frussi di dose alta radiazione come può essere ad esempio il caso di esperimenti sotto fascio, quindi quindi un acceleratore oppure esperimenti nello spazio dove la radiazione cosmica diventa importante perché non si è più schermati dal filtro dovuto all'atmosfera terrestre. Infine un ultimo svantaggio che qui non è riportato, in realtà era stato riportato come vantaggio cioè le dimensioni compatte perché da un lato è bella avere il rivelatore piccolino facile da trasportare. Dall'altro immaginate però di dover rivestire una superficie molte stesse con un rivelatore semiconduttore. Questo diventa estremamente costoso e anche impegnativo dal punto di vista, lo vedremo di elettronica, di consumo, quindi in generale quando si deve realizzare un rivelatore al silicio o un rivelatore con un cassone conduttore di dimensioni stesse, nascono altre problematiche e può essere non facile affrontarle. Quindi richiamiamo molto velocemente le proprietà dei semiconduttori. Voi sapete che i materiali solidi si distinguono in tre categorie diverse a seconda dello struttura a ban de livelli di energia. Quindi in generale noi sappiamo che se queste ban di valenze di conduzione sono separati da un gap abbastanza esteso abbiamo un materiale isolante, se invece il gap è inesistente il materiale è un materiale conduttore, una situazione intermedia invece l'abbiamo per i materiali solidi conduttori dove questa gap è presente, è una gap dove non possono essere presenti dei livelli energetici, una gap di energia proibita, però non ha una dimensione molto grande.

Parliamo di alcuni electron volt quindi esiste ma è abbastanza piccola. Questo cosa comporta? Comporta il fatto che se noi proviamo ad applicare un campo elettrico a questi tre eticologie di materiale, chiaramente un campo elettrico applicato in isolante non comporta il passaggio di corrente perché tutti gli elettroni si trovano nella banda di valenza e quindi non abbiano conduzione. In un conduttore invece si dovrebbe osservare il passaggio di una certa corrente e il semiconduttore si osserva una piccolissima corrente legata al fatto che alcuni elettroni che noi definiamo elettroni termici che hanno acquisito un'energia proprio perfetti termici, sufficiente a superare il getto energetico, questi elettroni possono condurre, però chiaramente capite che è un effetto proprio curamente statistico dovuta alla agitazione termica e quindi si genera una corrente e una corrente abbastanza debole. Ne semiconduttori abbiamo un'altra caratteristica fondamentale, cioè il fatto che i portatori di carica, cioè le cariche che generano una corrente possono essere di due tipologie, elettroni e lacune. Voi sapete che le lacune sapete che sono delle assenze, delle vacanze di elettroni e anche se concorrono alla valutazione della corrente in un semiconduttore, quindi ogni volta che noi andremo a parlare di questi effetti di corrente, faremo riferimento sia elettroni che lacune. In un semiconduttore, chiaramente poi ci siamo fermando ancora i semiconduttori puri, ok? In un semiconduttore normalmente questi sono due fenomeni che sono in competizione l'uno con l'altro. Il primo fenomeno è quello della creazione di coppie elettrone lacuna che abbiamo già accennato, cioè per effetti termici effettivamente un elettrone potrebbe passare dalla banda di valenza alla banda di conduzione e quindi si viene a creare una coppia elettrone lacuna. Viceversa potremmo avere un fenomeno posto, cioè quello della ricombinazione, cioè un elettrone potrebbe ricombinarsi con una lacuna, perché ovviamente valori di energia e di impulso lo consentono. Capite che in generale in un semiconduttore puro, ho detto anche in trinceco, il numero di elettroni che si trova nella banda di conduzione è esattamente uguale al numero delle lacune in banda di valenza. Questo è abbastanza chiaro perché ogni volta che un elettrone viene promosso, salta dalla banda di valenza, la banda di conduzione, crea la sua lacuna corrispondente. E vi dicevo che sia gli elettroni che le lacune contribuiscono alla conducibiliate elettrica del semiconduttore. E allora, quanto vale questa concentrazione di elettroni e di lacune? La indichiamo genericamente con N con I, dal momento che in questa fase c'è un semiconduttore auguro il numero di elettroni corrisponde al numero di lacuna. Quindi la chiamiamo N con I, che rappresenta la concentrazione di elettroni e di di E chiaramente questo N con I dipende innanzitutto dal numero di possibili stati che abbiamo nella banda di conduzione e nella banda di valenza. Quindi ritroviamo qui questa radice di N con C per N con I. E poi abbiamo questo fattore esponenziale, un po' la Botsman, dove compare la temperatura e compare E con G, cioè la gap energetica. L'energia della gap proibita, quanto è ampia. Ed è chiaro che queste dipendenze sono abbastanza intuitive perché se la gap è piccola sarà più facile per un elettrone lasciare la banda di valenza e andare verso la banda di conduzione. Ed effettivamente avere qui un esponenziale di meno e con G esprime proprio questo aspetto.

Viceversa la temperatura più elevata, la temperatura più sarà probabile che un elettrone perfetto termico possa nuovamente passare alla banda di conduzione. Ed ecco perché ci ritroviamo qui al denominatore in questo esponente. Ora, utilizzando la statistica di Fermidira, è possibile andare a valutare questo numero di stati nella banda di conduzione e dalla banda di valenza e si trova che corrisponde a un fattore A, una costante per T inalzata a tre mezzi. Quindi abbiamo semplicemente espresso il fatto che la concentrazione di elettroni di lacune che mi ritrovo libere per effetto termico semplicemente dipende dalle caratteristiche del semiconduttore, quindi in particolare dai con G e dalla temperatura. Ora quanti sono questi elettrone e queste lacune che ci ritroviamo magari a temperatura ambiente? Supponiamo di avere una temperatura di 300 gradi Kelvin e andiamo a sostituire questo valore qui all'interno di questa formula considerando due semiconduttori di tipo diverso il germanio e il silicio che si differenziano per l'energia e con G, la dimensione del laghetto. E allora se fate questo conto troverete che nel caso del germanio questa N con I corrisponde all'incirca 2,5 per 10 alla 13 su centimetro cubo che sembra un numero enorme così presa a sé. Ma in realtà se considerate che in media abbiamo 10 alla 22 atomi ogni centimetro cubo questo numero che abbiamo estratto prima corrisponde in realtà a dire che soltanto un elettrone su un miliardo di atomi si trova nella banda di conduzione quindi effettivamente non sono tantissimi e questo corrisponde a quello che avevamo detto all'inizio se vi ricordate se applichiamo un campo elettrico a un materiale sempliconduttore si genera una corrente ma una corrente molto piccola ed effettivamente così ti roviamo poche coppie elettrone, le lacune lacune ed effetti termici. Il silicio che è una gap leggermente più ampia ha in media 1,5 per 10 alla 10 e le coppie elettrone lacune per centimetro cubo che corrisponde in questo caso a uno su 10 alla 12 atomi quindi anche qui è un numero abbastanza basso. Se noi appliciamo un campo elettrico sotto razione di un campo elettrico e elettrone lacune cominciano a muoversi, cominciano a derivare e possiamo valutare la velocità con cui abbia questa deriva. La velocità di elettroni e la velocità delle lacune sono proporzionali al valore del campo elettrico come giusto che sia e alla mobilità che sono diverse per elettroni e lacune. Queste mobilità dipendono non sono costanti, dipendono in realtà dal valore del campo elettrico e della temperatura e quindi in generale ad esempio per il silicio a temperature normali troviamo che le mobilità risultano essere costanti per valori di campo elettrico inferiori a mille volte per centimetro quindi finché vi mantengo con campi elettrici abbastanza piccoli miucone e miuconacca le posso ritenere costanti. Poi incominciano ad avere delle dipendenze dal valore del campo elettrico quindi in particolare una dipendenza come uno sul radice di E nella zona intermedia tra 10 alla 3 e 10 alla 4 volte su centimetro è una dipendenza come uno su E per valori di campo elettrico maggiori di 10 alla 4 volte su centimetro. Questo comporta il fatto sostanzialmente che è arrivato a un certo punto si giunge a una saturazione della velocità a un valore massimo di 10 alla 7 centimetri al secondo legato al fatto che l'energia che viene acquisita da questi elettroni e queste lacune viene poi persa per gli urti con il reticolo quindi non si va oltre comunque una certa velocità di saturazione. La conuttività quindi si può esprimere andando a considerare entrambi i contributi. Quella dovuta gli elettroni e quella dovuta le lacune quindi è semplicemente dato dal prodotto della carita per la concentrazione di elettroni o la punna per la mobilità. Vedete appunto è la somma di entrambi i contributi. E questa parte che abbiamo fatto fino adesso riguarda la produzione di coppie che era il primo dei fenomeni che avviene in un semi-conductor in trincepo quindi creazione di coppie elettrone e lacuna per effetti termici. Ora andiamo a vedere l'altro processo, quello posso di riconvinazione. Può venire una riconvinazione spontanea cioè legato al fatto che elettrone e lacune che hanno opportuni valori di energia ed impulso possono riconvinarsi e da luogo all'emissione di un fotone. Ora questo che sembra un meccanismo molto facile in realtà è un meccanismo raro perché capite che non è sufficiente che elettroni e lacune si trovino vicine per ricominarsi ma davvero aveva anche dei valori opportuni di energia e impulso. Ed è un processo raro tanto che la vita media di una coppia che è stata creata per effetti termici è dell'ordine di un secondo quindi se si viene a creare una coppia a causa di effetti termici questa coppia il media sopravvive un secondo e poi per qualche motivo c'è una ricombinazione e quindi questa coppia cessa di esistere. Ora un secondo sembra un tempo piccolo ma in realtà è confrontato con i tempi della rivelazione, tempi che che tuttavia quello che si osserva sperimentalmente è che la vita media di una coppia è molto più bassa di un secondo quindi evidentemente ci sono altri meccanismi oltre alla ricombinazione spontanea che portano alla scomparsa della coppia. Questi meccanismi sono legati alla presenza dei cosiddetti centri di ricombinazione. Infatti a causa di difetti della struttura cristallina si possono presentare dei livelli nella zona proibita, sono dei livelli non definiamo profondi nel senso che sono abbastanza distanziati sia dalla banda di conduzione che dalla banda di valenza, li vedete ad esempio qui tratteggiati.

In questi livelli si può verificare una ricombinazione perché sono livelli che possono attrarre elettroni dalla banda di conduzione e la pune della banda di valenza permettendo una ricombinazione, ecco perché vengono definiti come centri di ricombinazione, quindi oltre alla ricombinazione spontanea possono essere presenti dei centri di ricombinazione che fanno sì che alla fine la vita media di una coppia sia effettivamente più bassa. Da un punto di vista della rivelazione cosa ci interessa? Ci interessa che se il rivelatore semi-conduttore come effetto del passaggio di una particella produce coppie elettrone e lacuna, un po' in analogia a quella quella abbiamo visto nelle rivelatori a gas. Se vi ricordate il passaggio della particella produceva coppie elettrone che noi poi andavamo a raccogliere agli elettrodi. Qui vedremo che in un rivelatore semi-conduttore il segnale sarà basato sulla creazione di coppie elettrone lacuna, quindi io sono interessata a raccogliere queste cariche e non voglio che si ricombinino prima che io riesco a raccoglierle quindi non voglio che la vita media sia effettivamente eccessivamente bassa perché se queste coppie si ricombinano troppo rapidamente io il segnale lo perdo. Quindi questo che cosa mi porta a dire? Mi porta a dire che i semi-conduttori che devo utilizzare nei rivelatori devono essere dei semi-conduttori esseramente puri, devono presentare pochi difetti perché più difetti ci sono più aumentano i centri di ricombinazione. Alcuni centri di ricombinazione sono anche definiti dei centri trappola, sono in realtà dei livelli dove magari viene catturato soltanto una tipologia di carica quindi ad esempio viene catturato un elettrone dalla banda di conduzione e l'elettrone rimane di per parecchio tempo, ecco perché venivano definiti dei livelli trappola. I difetti ci possono essere in un reticolo cristallino come quello di un semi-conduttore. Vedete in questa figura sono rappresentate le principali difetti di un reticolo allora potremmo avere ad esempio una vacanza come quella che vedete qui cioè l'assenza di un atomo nel reticolo qui era previsto un atomo non è presente, questa è una vacanza. Ogni difetto del reticolo comporta una modifica dello schermo dei livelli energetici a questa conseguenza poi potremmo avere un altro caso come quello che vedete qui il numero 2 dove il difetto è autointerstiziale cioè è proprio l'opposto della vacanza vedete un atomo dello stesso tipo ad esempio se questo è silicio un atomo di silicio si trova in più rispetto a quanto previsto dello schermo del reticolo. Poi potremmo avere un altro difetto come quello che vedete qui in alto interstiziale che si differenze dal precedente per il fatto che abbiamo sempre un atomo in più rispetto a quanto previsto però è un atomo di natura diversa. E poi gli ultimi due il 4 e il 5 vedete sono dei difetti sostituzionali cioè un atomo di silicio ad esempio viene sostituito con un atomo di tipologia diversa però affinché questa venga le dimensioni devono essere abbastanza simili, i raggi dei due automi devono essere similari. Scegliendo in maniera opportuna degli elementi da aggiungere al reticolo cristallino possiamo andare a generare degli ulteriori livelli energetici che questa volta sono superficiali quindi da un lato abbiamo visto che i difetti comportano la formazione di centri di riconvinazione o centritratola che sono uno svantaggio per noi quindi vorremmo effettivamente dei cristalli puri. Dall'altro però ci si resi conto che con opportuni drogaggi del semiconduttore si è in grado di modificare la struttura dei livelli energetici al nostro favore in particolare allo scopo di generare dei livelli che sono vicini o alla banda di conduzione o alla banda di valenza quindi sono dei livelli superficiali. E quindi in realtà si utilizzano semiconduttori drogati. Che cosa cambia un semiconduttore drogato? Sostanzialmente si vanno a introdurre, a sostituire gli atomi del semiconduttore quindi può essere siddicio germano con atomi di elementi diversi ad esempio normalmente nel caso del silicio si dice il germano sono entrambi i tetravalenti quindi hanno quattro elettroni di valenza e è necessario introdurre per drogare o atomi trivalenti o atomi pentavalenti. I semiconduttori drogati sono anche detti semiconduttori estrinseci. Capite che quello che succede sarà a cambiare la proporzione tra elettroni e lacune mentre in un semiconduttore puro il numero di elettroni coincide con il numero delle lacune generati per effetti termici qui drogando il materiale chiaramente andiamo a creare unis ecolibrio. E allora ad esempio un semiconduttore drogato di tipo N si va a sostituire un atomo del reticolo con un atomo pentavalente come può essere l'ascenico, il fosforo, l'antimonio quindi vedete che questo atomo chiaramente avrà 5 elettroni di cui quattro vanno a legarsi con i quattro atomi di sitiscio che si tromano attorno, il quinto elettrone rimane come un elettrone in eccesso che rappresenta un portatore di carica, può essere appunto un elettrone di conduzione. Questo equivale sostanzialmente a generare nella zona dell'elettro prohibito, nell'energia di cap proibita, è un livello che è un livello superficiale, un livello molto vicino alla banda di conduzione. Quindi questo elettrone si troverà in questo livello discreto, molto vicino alla banda di conduzione quando dicono vicino intendiamo ad esempio a una distanza di 0,01 elettronvolte nel caso del Germania o 0,05 nel caso del Siricio e quindi un elettronico che con una piccolissima energia è in grado di passare alla banda di conduzione, quindi un elettrone quasi libero sostanzialmente. Con valori di drogaggio tipici si possono raggiungere ad esempio un numero di elettroni di conduzione di 10 alla 17 su 100 metro cubo, mentre il numero di lacune si riduce a 10 alla 3 su 100 metro cubo. Ecco perché i simiconduttori drogati di tipo N, i portatori di carica maggioritari sono gli elettroni, 

mentre le lacune rappresentano i portatori di carica minoritari. Situazioni opposte in simiconduttore drogato di tipo P, in questo caso si va a sostituire un atomo del reticolo con un atomo trivalente come il gallio, il borro, l'indio e in questo caso quello che rimane in eccesso è una lacuna. Questa lacuna si troverà in un livello discreto, vedete qui, che si trova nella banda di energia probita, a una distanza molto piccola dalla banda di valenza. In questo caso le lacune rappresentano i portatori di carica maggioritari e gli elettroni portatori di carica minoritari, quindi abbiamo la situazione esattamente speculare. Si può dimostrare che, indipendentemente del tipo di drogaggio, il prodotto della concentrazione di elettroni per la concentrazione di lacune è sempre pari a N con I al quadrato dove N con I è la concentrazione del simiconduttore intrinceco, quella che avevamo visto precedentemente. Quindi che cosa succede? Che se aumentano gli elettroni per effetto di un drogaggio, la quantità delle lacune che si formano nel simiconduttore non rimane costante ma diminuisce e viceversa nel caso opposto. Questa legge prende il nome di legge dell'azione di massa. Allora, partendo dalla legge precedente, ad esempio, un insiamiconduttore di tipo N, mi aspetto che il numero di atomi accettori sia zero, gli atomi accettori sono come gli atomi trivalenti, mentre il numero di atomi donori N con I comporta la formazione di un certo numero N di elettroni. Allora, andando a sostituire alla legge di prima apposso di N e N con I, troviamo questa relazione e possiamo andare a esprimere la conducibilità del materiale in questo modo. Alla logamente, per un simiconduttore drogato di tipo P. Quindi vedete la differenza di prima dove la conducibilità veniva espressa come sommatoria di due termini, uno dovuta agli elettroni o uno dovuta alla elettrocone. Qui alla fine la conduzione elettrica è affidata ai portatori maggiori di carica, quindi gli elettroni nel caso dei simiconduttori drogati di tipo N e le lacune nei casi semiconduttori drogati di tipo P. Quindi se volete avere una corrente più elevata, quello che dovete fare è drogare maggiormente il materiale. Quindi se abbiamo un drogaggio di tipo N dovete aumentare la concentrazione degli atomi donori. Mi c'è versato il simiconduttore di tipo P, dovete aumentare la concentrazione di atomi acettori. Troverete a volte delle sigle un po diverse, P più N e più EI, che cosa stanno a indicare queste sigle? Può essere utile andare a drogare i simiconduttori con elevate concentrazioni. Si parla anche di 10 a 20 atomi su centimetro cubo, anziché i classici 10 a 13 a 1 su centimetro cubo. Questi materiali risultano essere altamente conduttivi per quello che abbiamo detto prima. Questi materiali sono indicati proprio per indicare che la concentrazione è elevata e vengono tipicamente adoperati per l'enerizzazione dei contatti elettrici. Voi sapete che in qualsiasi rivelatore per poter estrarre un segnale elettrico abbiamo bisogno di contatti elettrici. Quindi nel caso, ad esempio, del del a gas, se vi ricordate, avevate a Anodo e Cato Do, da cui potevate prelevare un segnale. Qui il materiale semiconduttore non ha dei contatti, bisogna crearli e non è facile crearli, lo vedremo un po più là, perché nel momento in cui io provo a utilizzare un materiale metallico a contatto con un semiconduttore purtroppo introduco degli effetti collaterali. Allora, per evitare questo, quello che si fa è andare a realizzare delle zone ad alto drogaggio proprio là dove voglio andare a creare il contatto omico con un metallo. Comunque questo lo approfondiremo dopo. Per adesso mi interessa semplicemente darvi la definizione di P più N più. Addirittura troverete anche N più più P più più per dire che sono concentrazione ancora più elevate. Invece la I, perché cosa sta? Sembrerebbe stare per intrinceco, ma in realtà non è esattamente così. Nel senso un materiale di tipo I si comporta come un materiale intrinceco, un materiale puro, ma in realtà prende il nome di materiale compensato, perché comunque sia un semiconduttore che è stato drogato, sia di tipo N che di tipo P con la stessa concentrazione. Quindi l'effetto risultante è semplicemente riportare il semiconduttore alla condizione di intrinceco, di puro, ma dovete sempre ricordarvi che è il testato drogato. Chiaro? E sono materiali che hanno una resistività elevata, ovviamente. Allora facciamo un passo avanti e parliamo delle giunzioni, perché questo è il principio di base di un rivelatore, quindi una giunzione PN o Np. Se andate a prendere due materiali, uno drogato di tipo N e uno drogato di tipo P e li mettete accostati uno vicino all'altro, in realtà non è così semplice come lo sto dicendo, ma ma che sia così, ma vengono dei fenomeni. State a metterne a contatto dei materiali che hanno concentrazione di carite diverse l'una dall'altro, quindi quindi mettendo vicino un materiale di tipo N dove c'è un'elevata concentrazione di elettroni liberi. È un materiale di tipo P dove abbiamo un'elevata concentrazione di la puna. Cosa succede? Chiaramente comincia a avere una diffusione di elettroni verso il materiale di tipo P e di la puna verso il materiale di tipo N, a causa proprio di questa differente concentrazione. Ora gli elettroni che si diffondono nella zona di tipo P incontrono le lacune e si ricombinano con le lacune. Viceversa lo stessa viene nel materiale di tipo N, con le lacune che si sono diffuse nel materiale di tipo N. Ora siccome le regioni inizialmente erano neutre, ecco mi sono dimenticata di specificare una cosa, torno un attimo indietro, qui anche se noi andiamo a introdurre un'impurezza, un attimo di tipo diverso, il materiale rimane pur sempre neutro, questo deve essere chiaro perché ad esempio qui dove è introdotto un'impurità di tipo N, un donore, vedete c'è l'elettrono in eccesso, ma non vuol dire che il 

materiale carico negativamente perché abbiamo la carica del nucleo che compensa ovviamente l'elettrone libero. Quindi complessivamente abbiamo un materiale neutro, stessa cosa per il materiale di tipo P. Quindi detto questo inizia di un'impurità un'impurità tipo tipo tipo che si viene invece a creare una carica nella regione dell'aggiunzione, ora lo vediamo come un disegno, ecco. Dovete immaginare che questo è il il N, questo è il il P, quindi inizialmente in N erano presenti gli elettroni liberi portato di maggiorità di carica, qui qui avevamo le lacune, gli elettroni sono passati da N a P e si sono ricombinati con le lacune. Cosa è successo? Che qui nel reticolo sono rimasti degli ioni negativi, fissi nel reticolo. Biceversa nella zona di tipo N, le lacune si sono ricombinati con gli elettroni e quindi sono rimasti fissi nel reticolo gli ioni positivi. Cosa comporta questo? Comporta la formazione di un potenziale, si crea un campo elettrico, un potenziale di contatto. Vedete qui come si sono deformate i livelli energetici. In questo disegno vedete sull'asse verticale sempre i livelli in energia. L'asse orizzontale lo dovete immaginare come un'asse spaziale che corrisponde alla figura che vedete in alto. Qui ci ritroiamo nella zona dell'aggiunzione, qui nella zona B e qui nella zona N. Se creato una differenza di potenziale, dovuta la presenza di queste cariche fisse, nel neritico dei materiali di tipo P di tipo N, una differenza di potenziale che infidisce ulteriore passaggio e diffusione di cariche libera. Se creato una regione che dal punto di vista della rivelazione è ottimale, perché è una regione in cui non sta circolando carica libera. E se per caso dovesse passare una particella che comporta la formazione di nuove cariche, queste cariche possono essere spazzate via da questo potenziale e raccolte. Sulla destra vedete grafici analoghi sempre relativi a questa situazione della aggiunzione, dove vedete la densità di carica e il relativo campo elettrico. Questo potenziale di contatto è abbastanza piccolino e circa un volt. La zona che si è venuta a creare prende il nome di zona di suotamento, perché è una zona dove ho eliminato tutte le cariche libere. Sono presenti delle cariche fisse, che determinano un campo elettrico, ma non sono presenti i cariche liberi, che potrebbero rappresentare l'umore per me, perché sono cariche che non sono dettate dalla passaggio di una radiazione, ma sono cariche che erano presenti nel materiale semiconductor. Invece non ci sono in questa regione, quindi una regione utile per la rivelazione. Ma quanto si estende, quanto è grande questa regione? Lo possiamo valutare andando a guardare la concentrazione degli atomi acettori e degli atomi donori. Se ad esempio la concentrazione di carica, la densità di carica, ha un andamento che vedete qui nella figura in alta a destra, quindi è un andamento sostanzialmente costante fino a una certa profondità. X con m, in questo caso, è x con p nel caso del materiale di tipo p. Questa densità di carica la posso esprimere, ad esempio, con questa espressione, con questa funzione, per entrambe le zone, zona di tipo n, quelle in alto, e zona di tipo p, quelle in basso. Utilizzando le equazioni di Poisson, posso andare a determinare la forma del potenziale e la profondità, x con n e x con p, in cui si va a estendere la regione di svuotamento. Vedete che x con n e x con p dipendono dal potenziale di con zero, il potenziale di contatto e dalle concentrazioni. All'inizio della formula, che può essere complicata, ci interessa vedere una cosa che, se, torniamo un attimo indietro, ho un materiale n estremamente drogato con un alto drogaggio, allora la giunzione si estenderà maggiormente nella zona p. O viceversa, se il materiale p è più drogato rispetto a quello n, la giunzione si estenderà di più nella regione n. E tutto dipende, sostanzialmente, da come è stato drogato il materiale. Tuttavia, in ogni caso, sono regioni di svuotamento molto piccole, parliamo tipicamente di dimensioni dell'organicento micron. In queste condizioni, un rivelatore basato su una giunzione pn avrà prestazioni abbastanza limitate in termini di rumore, risoluzione e stop-in-power. Quindi, come possiamo migliorarlo? Semplicemente ampliando questa regione di svuotamento e questo si fa polarizzando la giunzione, cioè applicando una differenza di potenziale esterna per aumentare la regione in cui si verifica lo svuotamento. Quindi è una polarizzazione inversa, cioè vuol dire che, vedete, il potenziale positivo viene applicato al materiale di tipo n, quello negativo al materiale di tipoppi. In questo modo si allarga la zona di svuotamento, che rappresenterà proprio il volume sensibile per la rivelazione delle particelle. Cosa succede se la giunzione viene polarizzata direttamente? L'abbiamo studiato questo? Si contrae e si conduce. Sì, sostanzialmente si produce luce. Per esempio, un LED si basa su questo principio. Qui abbiamo esattamente l'opposto. Noi siamo interessati ad aumentare la regione di svuotamento e a utilizzare questa zona per la rivelazione. In questo modo possiamo aumentare la regione di svuotamento anche a valori dell'ordine del millimetro. Non si può andare oltre un certo valore perché poi abbiamo limite dettato dalla resistività del materiale. Quindi, vi assumendo, rivelato alla semiconductor come funziona, è un'aggiunzione PN polarizzata inversamente. La regione di svuotamento è la regione attiva, quella che noi utilizziamo sostanzialmente per la rivelazione. Vedete ad esempio qui schematizzato un rivelatore a semiconductor, dove abbiamo la zona N, che vedete qui, che rappresenta la parte principale di questa aggiunzione. Poi abbiamo dei contatti, qui realizzati con un P più, e con un N più dall'altro lato. Quando passa una particella all'interno della regione di svuotamento, la particella deposita energia. Queste energie viano utilizzate per produrre coppie, 

elettrone e lacuna, e queste coppie cominciano a migliorare verso gli elettrodi per poi raccolte, per innurre il segnale che poi noi andiamo a misurare. Quindi vedete, è qualcosa di molto simile a quello che abbiamo visto in un rivelatore a gas, dove quello che cambia è il mezzo in cui avviene il processo, e anche il tipo di processo cui produciamo coppie elettrone e lacuna, nel caso di un rivelatore a gironizzazione si produce i gironizzazioni. Questo è un po' uno schema riassuntivo di quello che abbiamo visto finora nel campo della rivelazione. Vedete che infatti i rivelatori a gas, i rivelatori a semiconductor, presentano sostanzialmente uno schema molto similare. I rivelatori a scintillazione ovviamente hanno un comportamento diverso, si basano su principi fisici abbastanza diversi. Ma come si realizzano un rivelatore a semiconductor? Quindi come si realizzano queste junzioni? Ci sono diverse tecniche, diciamo diversi processi, non utilizzati per creare la barriera. E noi andremo a vedere molto velocemente, molto rapidamente, alcuni di questi, quelli più utilizzati. Ad esempio, i rivelatori a diffusione, questi vengono realizzati facendo diffondere delle impurità di tipo N, come ad esempio il fosforo, in un'estremità di un semiconduttore di tipo P. E per fare questo sono necessarie delle elevate temperature, anche mille gradi. Si gioca un po' sui tempi di diffusione, sulle concentrazioni, in maniera tale da avere una giunzione adeguata. Tuttavia, il principale problema di questo tipo di rivelatori è che la giunzione non si forma in superficie, veloce si forma, vedete qui in questa figura, a una profondità di alcune decine di micron. Questo vuol dire che ad esempio una particella che incide sul rivelatore dovrà attraversare prima questa zona, che è una zona per noi morta, perché non è una zona utile per la rivelazione, e poi entrare nella regione di suotamento. Quindi perdiamo comunque sia parte dell'informazione trasportata dalla particella. Quindi il principale svantaggio è questo, che limita certamente le misure di energia. E un altro svantaggio sono le alte temperature che si adobrano, perché aumentano il rumore e tendono a diminuire la vita media dei portatori di carica, che abbiamo detto non è una cosa che vanno a trovare un svantaggio, perché noi vorremmo andare a misurare queste cariche prodotte. I vantaggi sono certamente la robustezza e le basse contaminazioni superficiali. Superano questo problema della barriera, a una certa profondità quindi della presenza di una zona morta, questi rivelatori, i rivelatori è barriera superficiale. Questi sono basati su dei dio di shopki, cioè dei dio di che si formano non con due semiconduttore, pensi con un semiconduttore e un metallo. Infatti quando andate ad accostare, vedete, un metallo con un semiconduttore, quello che succede è la formazione anche in questo caso di un aggiunzione. Quindi si può ad esempio adoperare dell'oro su un materiale di Tqn o dell'alluminio su un materiale di Tqnp. La produzione di questi rivelatori avviene innanzitutto trattando la superficie chimicamente, ossidandola e poi depositando lo strato metallico per evaporazione. Il tutto infine viene montato su un anello isolante con delle superficie metallizzate per assicurare il contatto 

elettrico. Questo ad esempio è il rivelatore che noi adobberemo per l'esperienza della misura della radiazione alfa e si presenta così con questo aspetto. Quindi vedete qui l'esterno e l'anello su cui viene montato il tutto. All'interno è posizionato il rivelatore, che è questo. Il connettore ad habitare permette appunto il passaggio del segnale. I vantaggi è che in questo caso abbiamo dei rivelatori totalmente svuotati, non abbiamo nessuna zona morta come avveniva invece nel caso precedente. Possono essere profondi anche diversi millimetri. Nel nostro caso non ci interessa, però potrebbe essere utile per la rivelazione di altre particelle. Il processo di lavorazione avviene a temperatura ambiente, al solito a differenza di quanto abbiamo visto per il rivelatore di diffusione. Lo spontaggio quale è che lo strato depositato è talmente sottile, un strato metallico depositato così sottile che purtroppo non isola dalla luce e quindi sono rivelatori che possono essere sensibili alla luce. Infatti la radiazione luminosa, i fotoni del visibile hanno un'energia sufficiente a poter creare delle coppie elettorane la puna. Inoltre sono anche sensibili a possibili contaminazioni superficiali. Altre tipologie di rivelatori si basano invece sull'utilizzo dell'impiantazione ionica, quindi quello che avviene, il drogaggio, avviene attraverso l'utilizzo di acceleratori che accelerano dei fasci di ioni che sono le nostre impurità per impiantarle all'interno di un materiale semiconduttore. Capite di un processo proprio violento, un bombardamento sostanzialmente del materiale semiconduttore e questo fa sì che alla fine sia necessario un annealing a 500 gradi per poter ripristinarli da anni eventualmente causati da questo processo di impiantazione. I vantaggi che sono dei rivelatori molto starabili con finestre di ingresso molto sottili, quindi la zona morte eventualmente di poche decine di nanometri, ma l'osvantaggio che sono parecchio costosse, soprattutto per i processi di produzione che richiedono l'utilizzo di un acceleratore. L'ultimo rivelatore di cui volevo parlare è il rivelatore che prende il nome di Silly, rivelatore a deriva di Lithio, Silly, appunto che viene denominato Silly. Questi rivelatori cercano di risolvere il problema delle piccole dimensioni della regione di suotamento. Vi ho detto che comunque c'è sempre un limite a questa regione di suotamento, perché poi il diolo ha rottura. Allora per risolvere questo problema si utilizza una cosiddetta aggiunzione PIN, dove al materiale di tipo P e al materiale di tipo N si frappone un materiale di tipo compensato, il tipo di cui avevo parlato prima. Allora in questo modo è possibile sostanzialmente ampliare la zona in cui può avvenire la rivelazione. Si possono arrivare anche a spessori di 10-15 mm, che è molto utile laddove, abbiamo bisogno di range elevati per poter fermare una parte cella e misurare nell'energia. Questo è il processo con cui avviene la realizzazione di questi rivelatori, quindi nello specifico. Qui abbiamo del materiale dopato di tipo P, il litio viene posizionato in superficie, viene fatto diffondere attraverso l'applicazione di un campo elettrico, e alla fine quello che si ottiene è un'aggiunzione di tipo PIN, perché qui rimane la zona P, perché il litio non è arrivato fino a questa estremità. La regione centrale diventa di tipo compensato, perché è la drogata di tipo P e gli aggiungiamo il litio di tipo N, e poi invece rimane la zona di tipo N dovuta essenzialmente a un'alta concentrazione di tipo P. I vantaggi sono ovviamente questi spessori elevati, quindi vennero utilizzati per la spettroscopia beta o per i raggi X, bassa energia. I svantaggi è il fatto di si debba doverare a temperature basse, a causa di un rumore termico, e inoltre anche la conservazione dovrebbe avvenire a temperatura basse per mantenere inalterata la zona intrinzeca. L'ultimo aspetto che volevo discutere sul rivolto da Siriccio riguarda proprio la realizzazione dei contatti. Ve l'avevo accennato poco fa. Non è possibile banalmente creare un contatto omico utilizzando un materiale metallico sul semiconduttore, perché l'abbiamo visto, quello che si viene a creare è sostanzialmente un dio dosciocchi, quindi un ulteriore aggiunzione, un ulteriore regione di svuotamento. Questo non è desiderato. Allora quello che si fa, sostanzialmente prima di applicare un contatto metallico è realizzare una regione altamente dopata. Quindi, ad esempio, qui vediamo una regione di tipo N più prima del contatto metallico. Adesso, una volta concretata questa generalità sul funzionamento del rivolto da Siriccio, vi chiede innanzitutto se ci sono domande o qui o anche da fuori. Sì, dimmi. Madonna, penso sia un stupide, ma come mai il rivelatore si dice se ci serve per qualcun motivo di rivestire una superficie ampia non diciamo fare banalmente più grandi? In essenzione, Nici. Allora, si possono fare più grandi. Considera, ti puoi fare un'idea anche dai processi di produzione. Un po' hai visto i meccanismi con cui vengono prodotti. Fa sì che realizzare tutto in una superficie piccola d'esempio, pensa l'impiantazione, l'ionica.

Tu hai un acceleratore, un discorso è realizzare un rivelatore di superficie piccolina, un altro di superficie stesa. Ma il vero motivo è legato al fatto che questi rivolatori di Siriccio normalmente venono utilizzati come rivolatori segmentati per dare una informazione aggiuntiva alla posizione quindi dove ha inciso la radiazione. E quindi, ora lo vedremo nel corso di questa presentazione, si utilizzano delle tecnologie diverse di rivolatori al Siriccio che possono essere strip, drift o pixel. Ad esempio, pensa i pixel. I pixel possono avere anche dimensioni molto piccole e rivestire superfici dell'ordine del metroquadro con pixel di dimensioni di 20 micron per 20 micron. Non capisci che praticamente diventa una matrice fittissima. E hai a che fare con una equivalente elettronica, perché poi ogni canale ogni pixel ha la sua elettronica associata, che consuma anche energia e potenze e quindi può portare anche a un riscaldamento. In più considera la complicazione di andare a leggere un numero di canali enormi dell'ordine di miliardi di canali, miliardi di segnali. Quindi la difficoltà, oltre a una difficoltà realizzativa che può essere legata al processo di produzione, è anche poi una difficoltà pratica, perché spesso questi rivelateli si disciò, vengono adoperati più che altro come rivelatori di posizione per andare a rivestire delle superficie stesa. Quindi questa è un po' la problematica. Certo, anche questo si può creare una zona morta tra un pixel e il successivo, quindi esiste un free factor anche qui che normalmente viene mantenuto il più piccolo possibile. Ci sono altre domande? Ok, mi sembra di no. Allora andiamo a guardare un po' le caratteristiche invece di questi rivelatori. Abbiamo accennato all'inizio che sono dei rivelatori che da un, per diversi aspetti, ci aiutano, hanno delle prestazioni migliori rispetto ad altri rivelatori che abbiamo studiato. Allora, partiamo da una cosa fondamentale, la linearità. In ogni rivelatore che misura l'energia, ci aspettiamo un ottimo grado di linearità. Quindi se arriva una particella con un'energia E, mi aspetto che il segnale prodotto sia proporzionale a E, in maniera tale da avere una risposta lineare. In quel caso di un rivelatore semiconductor, chiaramente il punto di partenza è che la particena, innanzitutto, si debba fermare all'interno del rivelatore, quindi abbiano spessore sufficiente ad arrestare la particella e a misurare tutta l'energia. Allora, a quel punto mi aspetto che il segnale intenzione che viene indotto sia proporzionale alla carica prodotta, quindi alle coppie elettrone e lacuna che sono state create a seguito della perdida di energia della particella all'interno del rivelatore, diviso la capacità. Ovviamente, perché alla fine il dio non è altro che una capacità sostanzialmente. Questo è uguale a cosa posso indarli esprimere in un seguito modo, q, che è il numero di coppie elettrone e lacuna, lo posso indarli esprimere come l'energia E, che è stata persa all'interno del rivelatore, diviso q2, che invece è l'energia media per creare una coppia. In realtà qui viene introdotto un ulteriore fattore Epsilon, perché è vero che magari si produce un certo numero di coppie. La generale è l'energia depositata, ma magari non tutte le coppie vengono effettivamente raccolte, quindi abbiamo un certo grado di efficienza che è espressa da questo fattore Epsilon, potrebbe essere ad esempio il 90\%, quindi 90 su 100 coppie vengono raccolte, le altre purtroppo le perdiamo, però l'importante è conoscere questo fattore e l'importante è che si mantenga comunque sia la linearità tra il segnale prodotto e l'energia depositata nel rivelatore. Ora, la risposta dei semi-conduttori in generale è abbastanza indipendente dal tipo di particella, quindi se entro nell'elettrone o se entro nella particella alfa, normalmente se si deposita la stessa energia viene prodotto lo stesso segnale. Tutto questo è vero, almeno che non si va a considerare l'arrivo di Ioni, quindi nel caso di Ioni purtroppo non si generano degli effetti di plasma, quindi delle nuvole di elettroni particolarmente dense che vanno a modificare e distorcere il campo elettrico all'interno del rivelatore, quindi la linearità non è del tutto assicurata. Capite che se il rivelatore non è sufficientemente spesso, viene invece misurata semplicemente una perdida di energia, quindi a volte i rivelatori a semi-conduttore venivano utilizzati come rivelatori in trasparenza, nel senso con lo scopo di essere attraversati per perdere alle particelle una quantità di energia molto piccola. In questo caso capite che la risposta non è lineare con l'energia della particella. La risoluzione in energia, l'abbiamo detto all'inizio, la risoluzione di questi rivelatori è tipicamente abbastanza buona. Infatti, se vi ricordate, servono pochi elettron volt per generare una coppia elettrone lacuna, che dobbiamo confrontare con l'energia media per creare ad esempio una coppia ionelettrone in un rivelatore a gas. Se vi ricordate, in quel caso, erano necessarie energie dell'ordine dei venti, 40 elettron volt per creare una coppia, mentre qui siamo un fattore 10 più basso, quindi abbiamo bisogno di meno energia per creare una coppia. Qual è la conseguenza? Che a pariutà di energia depositata, nel caso di un rivelatore al silicio, si produce un numero di coppia che è 10 volte superiore rispetto a quanto si produrrebbe in un rivelatore a gas. E il fatto di avere un numero di coppia elevato fa sì che la risoluzione sia migliore. Ora faremo dei passaggi matematici molto semplici proprio per vedere questo effetto. Comunque alla base c'è proprio il fatto che si generano un numero di coppia più elevato rispetto ad altri rivelatori.

Ora, quanta energia è necessaria? Vi ho detto qualche elettro molto, ovviamente dipende dalla gap di energia prohibita. Vedete qui, sono riportate quelle per il silicio e per il germaneo a due temperature diverse. Vedete che per il silicio dell'ordine di 3-4 letto, molto per il germaneo dell'ordine di 3 letto molto più basso. Queste sono in realtà le energie per produrre una coppia. Le energie del gap sono in realtà più piccole. Se vogliamo vedere il letteratura quanto vale il gel del silice del germaneo, sono un gel molto leggermente più piccoli. Vi ritrovate invece un'energia per creare una coppia un po' più grande, legata a fatto che una parte dell'energia viene spesa, viene persa per eccitare il reticolo per creare delle vibrazioni reticolari. Quindi ci ritroviamo ad avere bisogno di almeno 3-4 letto molto per creare una coppia. Allora possiamo fare qualche passaggio, sono di funzione al tablet. La miglior condivisione. Allora quando vogliamo misurare l'energia, in realtà l'energia che noi misuriamo la coppia è stata, è viene depositata nel rivelatore diviso W, energia media per creare una coppia. Abbiamo detto questo vale sia per i rivelatori a gas che per i rivelatori al silice, con l'unica differenza che nei rivelatori a gas, W è dell'ordine all'incirca di 30 eV, mentre nel rivelatore al silice o comunque al semiconduttore W è dell'ordine di circa 3 eV, quindi 1.10. Chiaramente quando io dico che il numero di coppia è N, ovviamente parlo di un valore medio, perché sì è vero che si producono queste coppia ma poi le coppie che danno l'uomo al segnale sono una parte perché appunto ci sono dentro tanti fenomeni di fluttuazioni statistiche e quindi il numero N medio che io misuro è che indice della mia energia, il numero N che io misuro in realtà può variare da evento a evento, quindi ad esempio se io invio il rivelatore al silice delle particelle alfa come quello che vi troverete a fare a breve elaboratorio di 5 MeV, questi 5 MeV vengono persi nel rivelatore e deranno luogo a un certo numero di coppie che produrranno il segnale, ma questo numero di coppie fluttua ovviamente a seguito di fenomeni in natura sadistica e quindi quando io dico che sono andando a misurare lo spettro in energia e le particelle alfa e trovo un picco molto stretto come quello in figura, questo rappresenta il numero di particelle alfa che hanno una certa energia. Ora quello che io riporto sull'asse orizzontale, l'energia, in realtà equivale a N, equivale a dir a quante cariche, quante coppie ho generato e quindi il fatto di non avere una delta di dirac come io mi dovrei aspettare perché sto mandando 5 MeV, mi dovrei, dovrei trovare sempre 5 MeV, in realtà trovo a volte dei valori un po' più grandi a volte un po' più piccoli per quale motivo perché ci sono delle flutuzioni dovute proprio al numero di coppie che creano il segnale, questo numero fluttua, coccilla leggermente quindi è come dire che quello che gli osservo solo spettro in energia è una conseguenza del numero di coppie N che ogni volta si producono e che vengono raccolte che non sono sempre le stesse. Quindi posso valutar un N medio, quindi dalla formula che abbiamo visto prima ma poi rispetto a questo N meglio ogni volta ho delle fluttuazioni. La risoluzione di energia come viene valutata, se vado a ricordare, andando a guardare il picco e studiandone la larghezza, in particolare mi concentro sulla larghezza a metà altezza, quindi se questo ad esempio è il massimo del picco vado a considerare la metà altezza, quindi il massimo diviso 2 e vado a vedere la larghezza del picco a metà altezza, quella che è indicata normalmente con la sigla FWHM Full Width At Alph Maximum. La risoluzione in energia la possiamo indicare con R e data dalla larghezza a metà altezza diviso l'energia, possiamo moltibricarla per 100, per averla in percentuale però questa è la formula. Se il picco ha una forma gaussiana si può dimostrare che questa larghezza a metà altezza è all'incirca 2,35 per la deviazione standard della gaussiana diviso l'energia. Quindi se abbiamo una gaussiana siete agevolati perché ad esempio possiamo realizzare un best fit della vostra distribuzione con una funzione gaussiana e strare il parametro sigma che rappresenta la deviazione standard e utilizzarlo per valutarla la larghezza a metà altezza semplicemente moltirricando per 2,35. Altrimenti l'alternative è guardare la distribuzione dei valori che abbiamo misurato, guardare la larghezza a metà altezza materialmente come abbiamo fatto qui adesso. Allora come mettiamo in relazione la risoluzione al numero di coppia? Abbiamo detto abbiamo questa relazione che abbiamo scritto qui in alto n medio uguale ae su w quindi quando vado a riscrivere la risoluzione al denominatore posso andare a sostituire ae il numero n per w. Al numeratore mi riscrivo 2,35 per come sigma? Sigma è la deviazione standard di questa distribuzione che posso immaginare appunto essere anch'essa legata alla sigma della distribuzione del numero di coppia quindi posso andare a scrivere che sigma con n non è altro che sigma con e diviso w. Andando quindi a sostituire questo diventa w per sigma con n, w e w posso semplificare e quindi la mia risoluzione banalmente la posso andare a scrivere la scrivo qua sotto r come 2,35 sigma con n diviso n. Quindi ho riscritto banalmente la risoluzione in termine di n numero di coppia quindi la risoluzione sarà dato dalla larghezza di questa distribuzione del numero di coppia su numero medio di coppia generato. Se la distribuzione, dato che stiamo parlando di numero di coppia, la distribuzione che regola la produzione di queste coppia, la distribuzione di tipo possaniana, posso approssimare sigma alla radice di n e quindi capite che questo diventa 2,35 radice di n su n. Quindi se canciglie un attimo, andiamo a fare il calcolo per i due casi. Quindi r l'abbiamo scritto come 2,35, 1 su radice di n. E allora ad esempio se nel selicio si producono, abbiamo detto in media, 10 volte in più di coppia, capite che la risoluzione non rivoluta alla selicio sarà dato da 2,35 diviso 1 sulla radice di 10n. E quindi sostanzialmente vedete c'è un fattore all'incirca un terzo, quindi la risoluzione di un rivelatore alla selicio è all'incirca un terzo rispetto alla risoluzione di un rivelatore a grassa. Suffizialmente per il fatto che si producono un fattore di 10 in più di coppia. Quindi dire che è un terzo vuol dire che è più piccolo e quindi vuol dire che ovviamente è una risoluzione migliore. Nei rivolatori al selicio si possono raggiungere anche le risoluzioni dell'1\% e 

anche più bassa volta, dipende dal modo di operare. Ci sono domande su questi passaggi. Sensibilità ed efficienza in trince. In trince che ha di un rivelatore a semiconduttore è dell'ordina del 100\%. Vi ricordate cos'è l'efficienza in trince che rappresenta il rapporto tra il numero di particelle che hanno dato l'uogo a un segnale diviso il numero di particelle incidenti. Quindi arriva una particella, incide sul rivelatore, perde energia. Questo da luogo a un segnale rivolabile non è detto, in questo caso sì, praticamente prossima al 100\%. In altri rivolatori l'efficienza in trince dovrebbe essere più bassa anche a secondo del tipo di particella. Quali sono i limiti? I limiti potrebbero essere la presenza di una soglia minima necessaria sul segnale da rivelare. Legato al fatto che comunque sì e ci sono delle correnti di dispersione e l'elettronica che introducono un minimo di rumore. Quindi possono creare dei segnali che in realtà sono segnali di rumore. Sono segnali ovviamente di bassa ampiezza. Questo fa sì che se non vogliono essere confusi con segnali fisici devono essere in qualche modo discriminati, cioè eliminati dell'acquisizione. Questo può essere fatto imponendo una soglia minima al di sotto della quale non rivelare nulla, non acquisire nulla. Quindi questo però da un lato ovviamente vi aiuta a eliminare il rumore. Dall'altro però vi potrebbe far perdere qualche segnale fisico di interesse, quindi abbassare leggermente l'efficienza in trincega. La presenza di una zona morta potrebbe anche qui imporre una certa perdita nell'efficienza perché magari i particelli di bassa energia riescono ad attraversare la zona morte, quindi non vengono misurate. Quindi hanno inciso sul rivelatore ma non sono state misurate. La densità, vi dicevo questo è un vantaggio perché permette di avere un elevato stopping power. Ad esempio guardate qui in questo grafico, come non è facilmente leggibile a causa del della qualità. Viene riportata l'energia della particella da 1 a 100 membo e sulla sé verticale viene riportato il range in micron e i grafici che vedete qui le linee corrispondono a diverse tipologie di particelle, protone ed autoni, trizzi, ediotree e alfa. Concentriamoci ad esempio sulla situazione di terci e di troveremmo il laboratorio, quindi i particelle e l'alfa da 5 Mav che incidono sul sidicio. Questo grafico ci permette di capire che spessore deve avere il rivelatore per assicurarci che tutta l'energia venga persa all'interno del rivelatore. Ad esempio a alfa da 5 Mav ci ritroviamo qui grosso modo, vedete che siamo a range dell'ordine della ventina di micron e quindi è sufficiente un rivelatore di questo spessore per poter far sì che le particelle alfa perdano tutta la loro energia. Questo vi dicevo è un vantaggio perché sono necessari spessori veramente ridotti, quindi rivelatori compatti per poter andare a misurare energia di particelle anche molto energetiche. Infine, un'ultima qualità di questi rivelatori che abbiamo una diffusione estremamente più bassa, più piccola rispetto a quelle rivelatori a gas, si fa sì che questi rivelatori possono essere utilizzati più facilmente come rivelatori di posizione, quindi per andare a misurare la posizione della particelle incidente.

Il tempo di risposta, vi ho detto, sono rivelatori molto veloci in generale possono essere utilizzati come rivelatori di timing certamente molto più performanti rispetto ai classici rivelatori a gas. Tipicamente i tempi di salidità di un segnale sono dell'ordine dei nano secondi quindi estremamente veloci. Altre rivelatori veloci che avevamo visto erano ad esempio gli scintillatori, se vi ricordate abbiamo fatto qualche misura lo sceloscopio. Un problema però di questi rivelatori è il cosiddetto danneggiamento a rada raviazione, cioè cosa succede se questi rivelatori vengono esposti a notevoli dosi di raviazione. Purtroppo questo avviene soprattutto in alcuni ambienti quando possono essere appunto gli esperimenti sotto fascio, sotto accederatore o esperimenti nello spazio e questo modifica le qualità e le proprietà del rivelatore. In generale i danni che si possono creare sono di due diversi tipi, possono essere o danni del bike, cioè proprio del corpo, della massa che costituisce leaponduttore oppure dei danni sulla superficie, quindi sull'ossido che normalmente rivesse questi rivelatori. Gli effetti sul bike appunto sono effetti che si manifestano come danneggiamento al arreticolo e in generale sono dovuti alle cosiddette radiazioni nile cioè radiazioni non ionizzanti. Cosa comporta poi ne satti concretamente che magari può cambiare l'attenzione di svuotamento. Quindi se ad esempio voi stavate lavorando con il vostro rivelatorio a 10 volt e attenevate determinate prestazioni, vi rendete conto che a seguito di questo deneggamento non sono più sufficienti 10 volt, ma dovete doverare 12 volt, 15 volt. Quindi cambia la vostra attenzione di lavoro a cui ottenevate un buon svuotamento della regione. Potrebbe aumentare la corrente. Quindi una corrente che comporta un rumore e quindi questo potrebbe richiedermi l'utilizzo di un raffreddamento, oppure diminuire l'efficienza nella raccolta della carica. Qui ad esempio possiamo vedere come cambia l'attenzione di lavoro man mano che aumenta la dose. Questi sono dei dati sperimentali realizzati su un determinato tipo di rivelatore. Vedete come l'attenzione di lavoro man mano cambia. L'importante è ovviamente studiare per bene i possibili deneggiamenti di radiazione. Questo è una parte importante di quello che costituisce hierarchica la progettazione di un esperimento. Ad esempio tutti gli esperimenti acceleratori come può essere l'HCO o altri acceleratori richiedono anni e anni di progettazione anche per studiare questi effetti. Perché sono arrivelatori che ovviamente richiedono lo sforzo di anni di lavoro e devono operare per almeno 5, 10 anni. quindi bisogna assicurarsi che il rivalatore non subnisca un deterioramento eccessivo e che quindi possa lavorare per l'intervallo di tempo in cui dovrebbe lavorare. Quindi spesso la tecnologia non esiste nel momento in cui si immagina di dover costruire il rivalatore, quindi si lavora per costruire e arrivare a questi obiettivi. Gli effetti sulla superficie invece sono tipicamente dovuti a radiazione ionizzante e questo comporta sostanzialmente la presenza di cariche positive che si accumulano sugli ossidi e poi può anche influire sull'umore e sulla tensione di rottura. Gli effetti che vengono prodotti dal deneggiamento e da radiazione possono essere o effetti cumulativi che si vanno sommando nel tempo e sono dei danni che non si possono riparare, sono dute proprio all'esposizione prolungata alla radiazione che è un po' quello che vi ho descritto prima, ha immaginato il rivalatore deve essere adoperato in un esperimento sotto fascio, chiaramente è un accumulo costante nel tempo che comporta ovviamente dei danni irreparabili oppure possono verificarsi e questi si studiano pure degli eventi singoli che possono essere proprio o eventi transitori, immaginate ad esempio banalmente il cambio di un bit da 0 a 1 questo potrebbe essere un effetto transitorio o eventi catastrofici permanenti ma questi sono singoli eventi quindi che potrebbero accadere oggi come tra 10 anni quindi non sono legati proprio a un acuno come abbiamo visto nel caso precedente ragazzi mi fermerei qui oggi e riprendiamo questa parte la prossima volta quindi sono ormai soltanto applicazioni del rivalatore a ciricio e parleremo poi del introdurro una delle esperienze che farete quella appunto un rivalatore a ciricio e particelle alfa e parleremo delle tecniche di vuoto perché appunto quando adopererete questo rivelatore dovrete lavorare con una cameretta da vuoto quindi introdurremo qualche concetto di questo tipo allora noi ci vediamo giovedì pomeriggio mi sembra in aula e mi sbaglio ci hanno spostati allora ragazzi qua interrompo la registrazione e ci vediamo giovedì

\textbf{lez 15}

Allora ragazzi, riprendiamo quello che avevamo iniziato la volta scorsa. Se vi ricordate avevamo in prodotto i rivelatori a semiconduttore, era l'ultima tipologia di rivelatori che affrontiamo in questo corso. Vi ricordo molto molto velocemente la struttura di base di un rivelatore a semiconduttore e vi ricordo che si basa su un'aggiunzione PN, quindi quindi realizzata con un materiale semiconduttore drogato di tipo N e accanto un materiale drogato di tipo P. In questo questo è possibile realizzare nella zona dell'aggiunzione una regione di svuotamento che è una regione priva di cariche libere. Invece sono presenti delle cariche fisse che sono i milioni presenti nel nerreticolo che generano una potenziale, detto potenziale di contatto. Quindi vedete qui ad esempio la forma del campo elettrico proprio in corrispondenza della giunzione e qui come vengono deformate le bande energetiche dei materiali. Questa è una regione che si è adatta bene allo scopo della rivelazione perché nel momento in cui dovesse passare una particella e depositare dell'energia questa energia verrà utilizzata per produrre nuove coppie elettrone e lacuna, chiaramente il numero di coppie proporzionale all'energia che viene depositata nel materiale e queste coppie possono essere raccolte proprio grazie a questa differenza di potenziale. Chiaramente poi è necessario andare a mettere degli elettrodi per poter raccogliere la tarica, tuttavia essendo un potenziale molto piccolo la regione di svuotamento comunque molto ridotta e quindi per aumentarla quello che si fa è polarizzare inversamente l'aggiunzione e quindi andare ad applicare un potenziale positivo dalla parte n è un potenziale negativo del lado p. In questo modo la regione di svuotamento aumenta e diventa ovviamente una regione ancora più adatta alla rivelazione. Tante volte questi rivelatori semi-conduttore vengono utilizzati per misurare radiazione alfa che viene arrestata in poche decine di micron e quindi in realtà queste giunzioni non devono avere delle regioni di svuotamento particolarmente stesse, però per la rivelazione di altre radiazioni come ad esempio gli x, le regioni anche dell'ordine del millimetro e quindi abbiamo visto anche delle tecniche costruttive di realizzazione di questi rivelatori per arrivare a estensioni dell'ordine dei millimetri. Avevamo concluso un po' tutta questa parte, eravamo arrivati anche a discutere le caratteristiche di questi rivelatori al silicio e il danneggiamento della radiazione e dovevamo concludere questo argomento parlando delle applicazioni nella spettroscopia di particelle cariche. Questi rivelatori hanno delle proprietà molto positive, abbiamo detto hanno una risposta temporale abbastanza pronta, abbastanza veloce quindi sono adatti ad esempio per applicazioni di timing, dove volete andare a misurare delle differenze di tempo tra segnali elettrici, quindi forniti ovviamente da rivelatori. La risoluzione è ottimale, stiamo parlando di risoluzioni dell'ordine dell'1\%, quindi molto più bassa rispetto a quelle che abbiamo visto con altre tipologie di rivelatori e quindi proprio per queste proprietà questi rivelatori venivano utilizzati nel campo della spettroscopia di particelle cariche. Inoltre sono disponibili come abbiamo detto in un'ampia varietà sia di spessori che anche di area sensibile, hanno un'efficienza prossima al 100\% nel caso di protoni, particelle alfa ma anche ioni pesanti e di ricevo che lo spessore da doperare chiaramente si deve valutare in base all'applicazione, quindi in base al tipo di particella che si vuole andare a parlare in base anche alla sua energia. E proprio per questo motivo nel campo della fisica moderna, più che fisica moderna della fisica attuale, soprattutto fisica delle alte energie, questi rivelatori vengono utilizzati ampiamente, hanno appunto le prestazioni proprio adatte per andare ad affrontare anche degli ambienti spavorevoli, come ambienti in cui si ha una elevata dose di radiazione, una elevata densità di particelle e quindi negli ultimi decenni ci sono stati enormi sviluppi soprattutto in questo campo con ovviamente ritadute poi in altre tipologie di applicazioni come ad esempio il campo della medicina, quindi tante volte alcune tecnologie si sviluppano nel campo della ricerca e poi vengono trasferite in automatico nel campo, in altri tipi di campo, campo arespaziale, campo della medicina. Un altro vantaggio di questi rivelatori è che spesso è utile avere dei rivelatori molto subtili, vi ho detto che questi rivelatori possono essere adoperati anche come rivelatori in trasparenza, vengono attraversati della particella, viene effettivamente depositata dell'energia, ma questa energia è molto piccola, quindi è come se fosse una piccola perturbazione effettivamente del percorso e della particella nelle sue proprietà. E questo è un vantaggio perché da un lato abbiamo comunque sia un segnale che mi indica il fatto che è passato una particella e quindi eventualmente per problematiche di tracciamento di tracking, quindi sapere attraverso quali punti la particella è passata è utile avere questi rivelatori perché sono in grado proprio di misurare la posizione della particella senza perturbarne eccessivamente le caratteristiche. Quindi ormai tutti, quasi tutti gli esperimenti di alta energia e nello spazio utilizzano ampiamente questi rivelatori al silicio. Questo è un grafico che ormai è un po' datato perché insomma risale ormai quasi a vent'anni fa, però era giusto per darvi un'idea vent'anni fa di quali erano gli esperimenti che tipicamente utilizzavano rivelatori in questo caso al silicio di tecnologia strip, ora vi dirò un po' qualcosa qualcosa questa tecnologia. Comunque al di là di questo dettaglio vedete colorati le diverse applicazioni, questi sono acceleratori, quindi quindi verde e rosso si riferiscono a esperimenti prestoacceleratori e l'HC è l'acceleratore al momento più potente che riesce ad accelerare alle massime energie fino a la realizzata mentre in blu vengono riportati gli esperimenti nello spazio. Quindi vedete che appunto... ma che è questo rumore ragazzi? Sto riuscendo a favori. Ansì rispetto a qualche giorno fa, a qualche qualche di giorni fa. A inizio febbraio abbiamo fatto un esame del dottorato, un esame fino a il dottorato qui in quest'aula, è stato un inferno veramente, non si capiva niente. Comunque, vi dicevo, vedete appunto la varietà di applicazioni qui sotto sono riportati proprio nel dettaglio le sigle degli esperimenti che fanno uso di questa tecnologia al siri. Cioè, ad esempio, quello che mostra la maggiora utilizza, ad esempio l'esperimento CMS, non so se l'abbiamo mai seguito nominare, un esperimento presso l'acceleratore e l'HC. Diciamo uno dei esperimenti che ha contribuita alla scoperta del bosone X. E vedete qui il grafico riporta l'area in metri quadri. Quindi Quindi questo caso, vedete, CMS utilizza 214 metri quadri di rivelatori, in questo caso al siriscio in strippo. Ma altri rivelatori non sono da meno, comunque una tecnologia particolarmente utilizzata. Guardiamo ora un po' più più dettaglio alcune forme, alcuni rivelatori più moderni che utilizzano sempre la tecnologia basata sul siriscio. In generale, si possono avere due possibili scelte quando si vuole utilizzare il siriscio per ricostruire la posizione della particella. Quindi la particella incide sul rivelatore, voglio sapere esattamente in che punto è passata la particella. Allora, in questo caso si possono adoperare due scelte distinte. O si utilizza un readout continuo o un readout discreto. Capiremo ora la differenza. E in particolare faremo riferimento a queste tipologie di rivelatori al siriscio. I rivelatori a strip, che sono quelli di cui vi ho parlato nella slide precedente, quelli a pixel o pad e quelli a drift. Quindi vedete già un'immediata differenza, soltanto nella rappresentazione. Strip sta a indicare in inglese una striscia. E infatti vedete che qui questo rivelatore, ora andremo a vedere un po' più nel dettaglio, presenta queste strisce parallele, l'un all'altro, che prendono nel nome appunto di strip. Pixel o pad, ovviamente conoscete i pixel perché siete abituati ovviamente con le fotocamere, ecco qualcosa di similare anche in questo caso. Vedete una matrice di elementi che prendono nel nome appunto di pixel. Infine quelli drift vedete qui nuovamente una striscia a strip, però evidentemente c'è qualche differenza a rispetto al caso delle strip. E a seconda del tipo di rivelatore avremo un readout continuo o un readout discreto. 

Partiamo ad esempio a rivelatori a strip o micro strip quando queste strip hanno dimensioni di pochi micron. Allora vediamo un po' la struttura di questo rivelatore. Si basa tutto sempre su giunzioni piene quindi non è qualcosa di totalmente diverso da quello che abbiamo fatto fino ad esso. Però in più si sfrutta il fatto di poter realizzare gli elettrodi con determinate geometrie. Quindi qui la caratteristica è che gli elettrodi di lettura sono costituiti da queste strip che vengono posizionati parallelamente, l'un all'altro ha una distanza opportuna tipicamente della decina di micron. Il segnale che viene prodotto all'interno del rivelatore induce il segnale sull'elettrodi lettura e chiaramente andrà in dur, il segnale sull'elettrodi più vicino. Quindi in base a dove passa la particella, ad esempio guardate questa linea che rappresenta una particella che è attraversata il rivelatore, qui questa particella avrà prodotto un certo numero di coppie elettrone e lacuna che migliano verso gli elettrodi inducendo un segnale ed è chiaro che la strip che viene interessata ci fornirà la coordinata spaziale, cioè il punto attraverso cui è passata la particella. Chiaramente è un'informazione monodimensionale perché l'unica informazione che abbiamo è lungo questa direzione. Quindi se viene colpita ad esempio il rivelatore al centro immagino che viene interessata la strip centrale se invece la particella colpisce il rivelatore più a sinistra sarà la strip di sinistra dal segnale e così via. Chiaramente qui vedete soltanto una porzione del rivelatore ma normalmente sono presenti centiraia di canali in base all'essenzione del rivelatore. Quindi questo rivelatore è diciamo di posizione in questo caso monodimensionale perché ci dà l'innicazione su una sola coordinata spaziale sul piano ovviamente. Poi il meccanismo, il principio di funzionamento è esattamente sempre lo stesso quello che abbiamo visto prima quindi sempre giunzioni pn polizzate. Se volessimo avere un'informazione bidimensionale sul piano cosa potremmo fare? Una tecnica che viene aggoperata è semplicemente andare a sovrapporre due strati di rivelatore a strip come vedete qui con una certa inclinazione ovviamente. Questo vi fornirà l'informazione bidimensionale grazie al fatto di conoscere le strip interessate su ciascun piano però può portare eventualmente a dei casi di ambiguità perché se passa una sola particella attraverso entrambi gli strati allora non c'è alcuna ambiguità perché verrà interessata una strip per piano ma se per caso simultaneamente come può avvenire ad esempio in un esperimento sotto fascio sono più particelle ad attraversare i due strati allora inevitabilmente si crea un'ambiguità. Guardate qui l'esempio riportato in basso a destra. I punti verdi sono i punti reali attraverso cui è passata la particella, sono passate due particelle simultaneamente allora in basso a questa posizione ora magari il grafico è piccolino però possiamo immaginare qual è la strip di ciascun strato interessato anzi ciascun strato avrà due strip interessate una per particella quindi nel momento in cui voglio risalire alla posizione devo andare a considerare tutte le possibili combinazioni di queste coppie di strip e quindi cosa succederà che ne ricostruirò due correttamente ma ce le saranno altre due che sono quelle segnate in rosso che sono diciamo legittime nel senso derivano proprio dalla combinazione delle strip colpite ma in realtà nascono insomma sono in realtà spuri non sono collegata al reale passaggio di particelle quindi è vero che è utile sono a porre questi strati di strip però bisogna stare attenti a questi casi di ambiguità quanto vale la risoluzione chiaramente la risoluzione spaziale quindi la capacità di ricostruire la precisione con cui si ricostruisceotomia la posizione dipende dalla distanza tra le strip più queste sono vicino tra di loro migliore sarà la risoluzione spaziale in particolare se vi ricordate già questo discorso l'avevamo fatto nel caso delle camera fili, qui è un discorso analogo, data la distanza tra due elettrodi, la risoluzione data da questa distanza diviso la radice di 12. Quindi se ad esempio la distanza è 20 micron, la risoluzione reale sarà 20 diviso la radice di 12. Questa è una conseguenza del teorema del limite centrale, dava un discorso durante il primo semestre. Quindi vedete una risoluzione molto spinta, si si arrivare anche a 5 micron, tempi di raccolta delle cariche molto veloci, anche decine di nanosecondi. Un unico difetto è questa eventuale ammiguità nella ricostruzione della posizione. E questo è un esempio di rivelatori al redout discreto, perché qui la posizione viene data grazie alla discretizzazione dell'elettrodo, cioè il fatto di avere tanti elettrodi separati. In un rivelatore ad drift invece abbiamo un redout continuo. Che cosa cambia rispetto a prima? In anzitutto il rivelatore ad drift fornisce entrambe le coordinate su un piano, quindi un'informazione bidimensionale. Come fa? La struttura sembra molto simile a quella strip. Vedete qui abbiamo sempre degli elettrodi che sono realizzati come strip, paralleli l'un all'altro. Ma queste in realtà hanno una funzione diversa rispetto a quella che abbiamo visto prima. Ciò che raccoglie il segnale sono in realtà questi anodi che vedete qui sotto forma di pad, cioè di piazzola. Andiamo a vedere cosa succede quando passa una particella in grado di produrre coppie elettroni di lacuna. Queste sono gli elettroni e le lacune che vengono prodotte, gli elettroni incominciano a migrare verso gli anodi grazie a una differenza di potenziale che viene realizzata attraverso queste strip. Quindi questo strip se vedete qui c'è un partitore di tensione. Questi strip vengono messe a potenziale evidentemente via via crescente in maniera tale che gli elettroni che si producono a seguito del rilascio di energia nel rivelatore, questi elettroni vengono guidati verso l'anodo. Quindi come facciamo ad avere la coordinata bidimensionale? Allora una coordinata che è quella lungo questo ass, penso che segue la manina del cursore, viene fornita da quale anodo è stato interessato, quindi quale anodo ha raccolto la carica. Evidentemente l'anodo più vicino raccoglierà la carica. L'altra coordinata questa, quella verso cui, lungo cui si muovono gli elettroni, viene fornita dal tempo di deriva, cioè da quanto tempo impiedono gli elettroni ad arrivare all'anodo. Capite che se la produzione di questi elettroni avviene qui a questa estremità, impiegaranno un certo tempo, se venisse invece in questa estremità gli elettroni arriverebbero subito, quindi il tempo di deriva sarebbe più piccolo. Quindi il tempo con cui questi elettroni derivano verso l'anodo ci fornisce la coordinata lungo questa direzione. Ecco che quindi abbiamo una informazione di natura bidimensionale. Sto vedendo che la risoluzione è completamente diversa. Io sono molto più allungata, voi vede tutto più compatto. Tuttavia qual è l'osvantaggio? Questi sono rivelatori più lenti perché bisogna aspettare una deriva che può essere anche abbastanza lunga degli elettroni e inoltre necessitano di temperature stabili. Questo è un esempio di rivelatore al redauto continuo perché a questo punto è vero che abbiamo una discretizzazione lungo questa direzione dovuta agli anodi, ma lungo è l'altra direzione. La lettura è continua, cioè questo tempo di deriva assume un valore che non è discreto, ma è continuo. È chiaro? Se ci sono domande interrompete. Vede qualcosa che è molto simile a un dispositivo che voi adoperate spesso. Il CCD, il Charged Couple Device, voi lo utilizzate spesso perché è l'elemento di base delle fotocamere. Anche questi CCD sono rivelatori al redauto continuo, si dicono anche rivelatori a memoria perché gli elettroni che vengono prodotti a seguito della ionizzazione non sono rimossi immediatamente, quindi non vengono raccolti immediatamente ad un elettrodo, ma vengono fatti passare, innanzitutto vengono confinati all'interno di un pixel e poi a poco a poco vengono fatti derivare fino alla raccolta finale, un po' come avviene qui, che gli elettroni vengono prodotti e poi man mano vengono fatti derivare. Ora vedremo che qualcosa di simile vale anche per i CCD. Vediamo proprio un'animazione che fa sia la cosa più semplice se fa per messerlo a questa. Niente, è peccato perché la vengono se clicco. Ah, perché funziona. Allora, vedete il CCD è costituito da la matrice di pixel, che sono questi quadratini che vedete. Questi pixel costituiscono righe e colonne. Quando viene attraversato, questo sensore viene attraversato in questo caso da luce, viene generata della carica e questa carica non viene raccolta immediatamente da un elettrodo, ma in realtà a seconda del punto in cui si è creata viene guidato lungo la colonna corrispondente. Quindi, su per esempio di trovarvi qui, la carica è stata prodotta qui grazie a un campo elettrico opportuno, quindi adere differenze di potenziale, è possibile guidare la carica prodotta da pixel a pixel fino ad arrivare a quest'ultima riga dove avviene la lettura. Quindi proprio grazie a questo sistema è possibile andare a risalire a quale pixel è interessato. Qui sotto vedete nel dettaglio proprio la fase di deriva. Questi sono i pixel di una colonna, vedete che inizialmente la carica è confinata qui dove si trova un potenziale positivo, dopo di che il potenziale positivo viene creato nel secondo pixel e azzerato nel primo, quindi in questo modo riuscite a far derivare le cariche fino all'ultima riga. I cct sono quindi rivelatori bidimensionale, vi viene fornita effettivamente la posizione in cui è venuta la rivelazione, hanno acuratezze spaziali anche molto elevate dell'ordine di 10 nigron, però hanno come difetto principale in un tempo di lettura molto lungo, quindi almeno hanno dell'ordine dei 10 mili secondi. E questo fa sì che questi rivelatori normalmente non possono essere adoperato nel campo della fisica delle alte energie perché sono troppo lenti. Succederebbe che ad esempio se pensiamo in un esperimento in un acceleratore ovviamente avvengono un certo numero di collisioni al secondo, se il rivelatore ancora è impegnato a misurare ciò che è avvenuto nell'evento precedente, succede il cosiddetto pile up, una sovrapposizione di eventi che chiaramente deve evitare. Quindi Quindi rivelatori non vengono adoperati in questo campo, ma vengono adoperati parecchio nel campo dell'astronomia. Hanno una bassa risoluzione energetica e spaziale, quindi vengono utilizzati nel caso dell'astronomia per spettroscopie di ragilis. Questo è un altro esempio giusto per farvi capire come vengono identificati i pixel interessati. Vedete, questa è una matrice piccolina di 4 x 3 pixel, anzi 3 x 3 per questo poi è la riga di rettura. Vedete incidono quattro fotoni che sono sostanzialmente quelli che colpiscono i pixel sottostanti, se vedete qui vedete i pixel interessati. Si produce in ciascuna di queste regioni la carica e questa carica ovviamente lungo ogni colonna viene fatta migrare verso la riga di rettura e alla fine quello che succede è che si ricostruiscono i pixel che sono stati effettivamente interessati nella radiazione per andare a ricostruire l'immagine. Chiaramente questo è un esempio esempio di 3 x 3 ma dovete immaginare matrici con milioni di pixel. Funziano in maniera analoga i cosiddetti rivelatori a pixel, quindi quelli sono i ccd. Ora vediamo invece nel caso della fisica nucleare cosa possiamo usare di analogo che sia però molto più veloce rispetto a un ccd. In questo caso noi adoperiamo il rivelatorio a pixel anche si costituiti da matrici di pixel che possono essere anche dell'ordine delle centinaia di milioni compressivamente a seconda della superficie da rivestire e quindi sono ovviamente dei canali indipendenti. Qui ritorniamo a un caso di redout discreto quindi ognuno è come se fosse un rivelatore indipendente dall'altro. Le dimensioni di queste pixel ovviamente capite che più sono piccole migliore sarà la risoluzione spaziale del mio rivelatore. Ormai si è arrivato anche a pixel dell'ordine di 15-20 micron di lato, quindi veramente piccoli. Sono rivelatori bidimensionali perché viene fornita sia la coordinata x che la coordinata y. Vi dicevo ormai abbiamo dimensioni particolarmente spinte dell'ordine delle poche decine di micron. Sono rivelatori che hanno un livello di rumore molto basso, il il dipende dalla capacità di questi rivelatori e quindi un buon rapporto segnale sul rumore. Cioè vuol dire che quando vengono attraversati da una radiazione si produce una carica quindi un segnale che si sopraeleva rispetto al fondo, quindi il rapporto segnale sul rumore è ottimale. Sono rivelatori veloci e sono particolarmente adatti nel caso di alte densità di particelle. Quindi immaginate di avere una situazione in cui abbiamo tante particelle l'una vicino all'altra, ma le volete distinguere e rivelare indipendentemente. Quindi la probabilità che due particelle cadano in un 

pixel di dimensioni così piccole è veramente remota, veramente bassa. Ecco perché questi rivelatori normalmente vengono posizionati nella regione a più alta densità. Quindi se immaginate un esperimento, un collider dove avviene un'interazione tra fasci, una contro l'altra, capite che è la zona più vicina alla collisione, al punto di collisione sarà quella con la più più densità di particelle. Più vi allontanate, più avere rivelatori così discreti non è necessario. Quindi tipicamente rivelatori al silizio non sono posizionati proprio all'inizio in prossimità della collisione. quali sono gli svantaggi ovviamente. Capite che abbiamo un numero di canali enormi da leggere? Ora vedremo un esempio concreto, giusto per rendersci conto di che numero li stiamo parlando. In lì è di principio ogni rivelatore ha bisogno di un connettore e di un cavo. Se parliamo di un rivelatore macroscopico, immaginate ad esempio il rivelatore che abbiamo utilizzato, lo scintillatore, col fotomultiplicatore. Dal fotomultiplicatore usciva un cavetto che trasportava il segnale. Ora immaginate di doverlo fare per miliardi di pixel di queste dimensioni. L'ansi è tutto già realizzare un filo di quelle dimensioni capite che diventano un problema, ma ora lo discuteremo, chiaramente una sfida tecnologica. Qui se vedete un ingrandimento di quello che è un rivelatore a pixel, questo è una matrice e qua sotto non so se riuscire a riferli. Sono dei piccoli fili microscopici che sono dei bonding, c'è proprio delle connessioni che vengono realizzate con delle opportune macchine, perché non si possono realizzare a mano, ma si realizzano con delle macchine, si chiamano bondatrici e vanno a realizzare come se fosse un filo però di natura microscopica, quindi dal rivelatore verso l'elettronica. Sono ovviamente molto fragili e richiedono tecnologia all'avanguardia. Perché? Perché innanzitutto il primo problema è vi dicevo quello di trasportare questo segnale fuori da ciascun pixel e allora come si fa? Capite ogni pixel è un rivelatore a 60, quindi ad esempio se il segnale che fu riesce da questo rivelatore ha bisogno di un'opportuna catena elettronica, questa catena elettronica dovrebbe essere replicata per tutti i pixel. E quello che si fa normalmente, che si faceva tempo addietro, ora questa è una tecnologia che si sta superando, era quella di realizzare l'elettronica, ovviamente sempre su strade di silicio, miniaturizzata e delle stesse dimensioni dei pixel. Vete quindi si aveva sostanzialmente una matrice di pixel che erano i rivelatori e sotto una matrice con la corrispondente elettronica. Come si fa la connessione si realizzava attraverso la tecnica che prende il nome di bump bonding, cioè un bonding non realizzato con un filo, ben sì con una pallina, come la vedete qui. Qui sotto abbiamo il sensore, qui sopra abbiamo l'elettronica, questa pallina di materiale conduttore costituisceurasse il contatto. Chiaramente la corrispondenza deve essere esatta di sensore e di pixel di elettronica. Il tutto viene a temperatura controllata perché comunque questa è una sorta di non dico stagno ma comunque un materiale che si deve sciogliere deve realizzare un contatto stabile e tuttavia questa tecnica aveva grossi difetti. Qui vedete degli ingrandimenti con il microscopio, per farvi rendere al conto della forma di questi contatti. Tuttavia aveva dei difetti enormi, un'efficienza estremamente bassa quindi spesse volentieri si buttavano tanti buffer di silicio perché il contatto, il bonding non era venuto bene. Inoltre era un processo molto costoso e capite che il limitato era la dimensione del pixel quindi anche se la tecnologia ci permetteva di ridurre le dimensioni del pixel, l'elettronica non si riuscì a compatterli più di tanto e anche il bonding non poteva essere effettuato su pixel di dimensioni troppo piccole. Inoltre sempre nell'ottica di rendere il materiale il rivelatore il più possibile è trasparente inserire una pallina di materiale, di un qualsiasi materiale, introduce ovviamente una perdida di energia se la particella deve attraversare anche questa piccola pallina è una perdida di energia. Sembra una cosa ridicola perché hanno dimensioni veramente di decine, venti micron, tuttavia per la rivolazione delle particelle può rappresentare un problema anche di multiposcheting. Come siamo andati oltre? Adesso la tecnologia ci permette di realizzare i così detti maps, cioè rivelatori monolitici. Cosa vuol dire che non lo stesso substrato di silicio si realizza sia il sensore che l'elettronica. Capite che non abbiamo più bisogno del bonding quindi guadagniamo in termini di efficienza e diventato ormai un processo abbastanza commerciale. La carica, vedete questo è il sensore e qui abbiamo tutta la parte di elettronica sostanzialmente. La carica che si produce viene fatta migrare verso l'elettrodo e poi c'è la relativa elettronica. Non sono necessarie connessioni elettriche ed è possibile diminuire le dimensioni del pixel. Tutta via qual è stato il problema iniziale di questa tecnologia era il fatto che questa tecnologia era poco resistente alla radiazione quindi perdeva in prestazioni man mano che si accumulava a dose. Ormai questi limiti sono stati superati quindi questa tecnologia viene utilizzata in tantissimi esperimenti. Come esempio vi riporta appunto il nuovo tracciatore dell'esperimento Alicia che è stato installato proprio lo scorso anno. Settestrati interamente di pixel, una superficie di 10 metri quadri per un totale di oltre 10 miliardi di pixel. Quindi come se restavò a disposizione una macchina fotografica di un smartphone un miliardi di pixel. La vostra macchina fotografica è un esempio del cellulare che il numero di pixel ha in grosso modo. 12 mega pixel ormai forse anche di più. Giusto per avere un ordine di grandezza. Ok, ci sono domande? Allora, visto che abbiamo parlato dei rivolatori al silicio, presentiamo una delle esperienze che dovete fare al secondo semestre. Per spiegare queste esperienze ora non parleremo delle esperienze, del rivelatore, delle modalità con cui viene seguita. Alla fine di questa presentazione introdurremo le tecniche da vuoto perché questa è un'esperienza che deve essere svolta utilizzando una cameretta da vuoto. Quindi magari alcuni concetti sul vuoto prendete li così come sono. Ora nella presentazione successiva ne parleremo. Allora, in che cosa consiste questa misura? La misura è la rivelazione di particelle alfa, in particolare spettrometria. Per spettrometria si intende l'analisi, lo studio di uno spettro in energia. In questo caso di particelle alfa, il studio della perdita di energia di queste particelle in opportuni materiali. Il rivelatore che doveremo è un rivelatore al silicio. Qual è lo scopo della misura? Allora, innanzitutto utilizzare un rivelatore al silicio che ancora non abbiamo adoperato per la misura di particelle cariche. Valutare la risoluzione del rivelatore per vedere se effettivamente la risoluzione in energia è molto più piccola rispetto a quella che abbiamo misurato nel caso dello scintillatore più fotomultiplicatore. Utilizzare un apparato per la produzione e la misura del vuoto, questa è un'altra cosa che per vuoto risulterà nuova. Misurare la perdita di energia di particelle cariche in un materiale per poi confrontarlo con quanto atteso dalla relazione di Bethe Block. Questa era una delle esercitazioni che vi avevano 

suggerito nel corso dell'anno proprio anche a questo scopo, perché attraverso la forma di Bethe Block se siete in grado di integrare questa relazione possiamo benissimo confrontare quello che misurate sperimentalmente con quanto previsto della teoria. Infine, non solo osservare ma anche quantificare gli effetti di straggling nella perdita di energia. Quando introdurre un materiale tra l'assorgente e il rivelatore, questo materiale, oltre a far perdere parte dell'energia alla particella, produrrà un effetto di straggling energetico, quindi con un relativo allargamento del picco misurato. Ora vedremo alcuni esempi. In cosa consiste l'appato sperimentale? Tutto molto contatto, perché in realtà è tutto contenuto all'interno di questo modulo. Questo è un modulo che non vedete la figura completa e inserito, vedete su queste guide, all'interno di quello che noi chiamiamo crate, che poi vedete... ah, l'abbiamo visto in crate laboratorio? Ad esempio, quando abbiamo fatto la misura con lo scintillatore, c'era il modulo, quello per la tensione di alimentazione che era posizionato proprio su quella struttura. In realtà quello è semplicemente un contenitore che permette di inserire questi moduli e prendere dal retro delle alimentazioni che servono per far funzionare il modulo. Quindi magari il crate è collegato di per sé alla 220, insomma, la tella corrente che noi utilizziamo. E da questa avevo utilizzando dei trasformatori, vi produce sul retro, su quello che vedete qui, su questo connettore, delle tensioni standard che sono una più 5 volt, una 12 volt, continue, che sono quelle che vengono adoperate tipicamente da questi moduli di elettronica. Quindi una volta che voi inserite il modulo nel crate, magari questo poi quando torneremo il laboratorio lo guarderemo dal vivo così me ne rendete conto più facilmente, quindi quando si inserisce ine sceglie questo modulo avviene il contatto sul retro. Il modulo può prendere le tensioni che li servono. Quindi questo è tutto un modulo di questa azienda Ortec, pensato per studiare le alfa particelle attraverso il rivelato dal sidiccio. Questo sportellino si può aprire e all'interno è presente la cameretta da vuoto. In questa cameretta si può montare un rivelatore sul soffitto della cameretta, che è il nostro rivelatore al silicio, e posizionare sotto una sorgente alfa. Quindi l'apparato sperimentale prevede diverse sorgenti alfa, poi una serie di assorbitori che noi possiamo andare a interporre tra la sorgente e il rivelatore per vedere quanta energia viene persa in questi spessori. Capite che sono spessori piccolissimi, perché se le alfa si fermano in venti micron di silicio, io non posso mettere uno spessore di alluminio di un millimetro, perché è chiaro che non passano. Quindi sono spessori di decine di micron al massimo. Poi abbiamo un rivelatore al silicio che, come vi dicevo, viene posizionato sul soffitto della cameretta. Abbiamo la cameretta da vuoto per realizzare un vuoto che non è un vuoto spinto. Ora ne parleremo di questo, per capire se il vuoto che produciamo è sufficiente. Per produrre il vuoto è necessario un sistema di pompaggio, quindi l'utilizzo di una pompa per strarre l'aria a residua nella cameretta. E poi, sempre contenuta all'interno del modulo, abbiamo un minimo di elettronica per la gestione del segnale prodotto del rivelatore al silicio e un sistema di acquisizione datica, è esattamente identico a quello che abbiamo adoperato, ad esempio, nel caso degli spettri in $\gamma$. Quindi vi ritroverete esattamente lo stesso software. Qui, appunto, oltre alla fotografia, abbiamo riportato anche un elenco delle caratteristiche di questo modulo. In questo modulo vedete, c'è anche questa valvola che può essere posizionata in diverse modalità. Serve sostanzialmente per aprire o chiudere la connessione con il sistema di aspirazione. Quindi immaginate di avere una pompa che attraverso un condotto aspirala l'aria, tra il condotto e l'ingresso della cameretta c'è questa valvola. Quindi se voi la chiudete e come se stessi solando la camera rispetto al sistema da vuoto, altrimenti se l'aprite permette alla pompa di aspirare l'aria della cameretta. Poi l'ultima posizione che è possibile questo vento serve per far rientrare l'aria dentro la cameretta, perché altrimenti capite che si si può aprire con un po' di forza, però la differenza di pressione non aiuta sinceramente in questa operazione. E allora cosa vogliamo misurare? Abbiamo detto particelle alfa, quindi ritiamiamo alcune caratteristiche delle particelle alfa per capire cosa ci aspettiamo di osservare. Il decadimento alfa prevede l'emissione di particelle che sono dei nuclei di helio con un'energia specifica proprio perché è un decadimento che avviene semplicemente mettendo questa particella, quindi nello stato finale abbiamo il nucleo residuo più la particelle alfa. La particella alfa si prende tutta l'energia a disposizione mentre il nucleo residuo chiaramente rimane fermo. Questa energia quanto vale? Tipicamente abbiamo energia dell'ordine di 5 MeV, 4 o 5 MeV per i principali i soto, Pi, Alfa, in natura e posso in realtà avvenire anche diversi decadimenti verso i livelli eccitati dell'isotopofilio. 

Ad esempio qui vedete lo schema di decadimento dell'americio 241 che è uno di questi soto, Pi, utilizzeremo il laboratorio. Nella maggior parte dei casi vedete nel 86\% dei casi questo decadimento avviene verso il primo livello eccitato del nettugno 237. Cosa vorrà dire? Che questo nettugno poi decadrà e mettendo $\gamma$ in questo caso, però noi i $\gamma$ non li misuriamo quindi vedremo sostanzialmente delle alfa derivanti da questo decadimento con un'opportuna energia. Tuttavia questo non è l'unica modalità di decadimento, vedete ad esempio nel 12,5\% l'americio decade verso il secondo livello energetico del nucleo figlio. Questo vuol dire che le alfa emesse avranno un'energia un po' più bassa. Ok? Poi c'è un'altra probabilità di decadere a questo livello eccitato e una probabilità anche questa piccola dello 0,3\% di decadere allo stato fondamentale. Insomma da questo schema di decadimento cosa traiamo? Che informazioni traiamo? Ci aspettiamo che l'americio presenti 4 picchi sostanzialmente. Uno ovviamente che sarà più popolato rispetto agli altri, che corrisponde a questo decadimento verso il primo livello e gli altri in proporzione, ovviamente al branching resso, avranno una popolazione differente. Ora riusciamo effettivamente a distinguere questi picchi l'uno dall'altro. Se vediamo l'energia in gioco sono energie molto vicine all'uno all'altro che differiscono magari di poche decine di keV. Come faccio a capire se riesco a vederli oppure no, devo conoscere la risoluzione in energia. Quindi immaginiamo ad esempio che questo rivelatore si dice abbia una risoluzione in energia dell'1\%, questo mi dice che io posso distinguere energie che differiscono dell'1\% di 5 Mb. Ok? Quindi questo già mi dice se io posso andare a distinguere questi picchi o se sono sovrapposti l'uno all'altro. Quello che noi osserviamo sperimentalmente sono quindi dei picchi che sono abbastanza stretti perché la risoluzione buona ma chiaramente non sono piccolissimi e questo deriva dalla risoluzione del nostro rivelatore. Giusto per farvi capire appunto che il rivelatore che stiamo adoperando è sufficiente fermare le particelle in questo grafico vedete sull'asse verticale l'energia della particella che noi stiamo mandando verso il materiale e questo caso verso il silicio e sull'asse orizzontale il range quindi quanto spazio viene percorso da quella particella fino ad arrestarsi. E allora se io vado a vedere in questa scala dove si trova l'energia corrispondente grosso modo delle particelle alfa siamo su questa linea rossa, sono circa 5 Mb. A me interessa andare a vedere le alfa che sostanzialmente corrispondono a questa linea continua e quindi è la linea più in alto questo mi dice vedete che il range delle alfa in silicio, alfa da 5 Mb in silicio grosso modo se io scendo sotto, corrisponde a 10-20 Mb. Quindi mettere un rivelatore di 50 Mb certamente mi assicura di fermare tutte le particelle alfa che vengono emesse da questi sottopi. Come mai dobbiamo realizzare questa misura con un sistema da vuoto? Andiamo a vedere quanta energia perdono le alfa in aria. Questo è un grafico simile a quello di prima però gli assi sono scambiati quindi sull'asse orizzontale ci ritroviamo l'energia sull'asse verticale il range questa volta espressi in centimetri. Le curve si riferiscono probabilmente a due stime diverse, a due conti diversi però sono molto simili un all'altro e sono riferite a particelle alfa quindi se io vado a concentrarmi nella legione dei 5 Mb e vado a vedere quanti centimetri si percorrono prima di 

arrestare le particelle vedete che mi ritrovo intorno a 3,5 cm quindi la cameretta non è immensa, non è grande, è piccolina, però la distanza tra l'assorgente rivelatore e di alcuni centimetri certamente questo fa sì che magari le particelle alfa riescano ad arrivare al rivelatore ma nel frattempo in quei 2 cm di aria hanno perso praticamente quasi tutta la loro energia quindi la misura che vogliamo andate ad effettuare effettivamente la misura non realistica non è andata a misurare la reale energia delle particelle alfa quindi è necessario praticare un cerchio a livello di uoto. Addirittura gli elettroni hanno un range ancora più elevato se guardate appunto 3 Mb di elettroni vedete in aria che distanze percorrono anche 1200 cm quindi ad esempio nel caso di di elettroni chiaramente la perdita di energia in questo caso è molto più piccola in aria e non si vorrebbe il problema del stesso problema delle alfa. Allora la domanda è devo praticare il vuoto che vuoto devo creare? Un vuoto spinto ora vedremo esistono diversi gradi di vuoto a seconda della pressione che si raggiunge. Praticare il vuoto vuol dire sostanzialmente diminuire la pressione rispetto al valore della pressione atmosferica. Come faccio a stimare che pressione devo raggiungere anche perché vedremo che praticare il vuoto non è un'operazione facile e richiede delle opportune apparecchiature che cambiano a seconda del livello di vuoto che voglio raggiungere. E allora banalmente si può fare questa considerazione abbiamo detto che in generale pressionato atmosferica le particelle alfa percorrono pochi centimetri in aria. Se io diminuissi la pressione di un fattore mille cosa vuol dire? Che il range delle particelle alfa aumenterebbe di un fattore mille quindi se prima percorrevano due mille e due centimetri di minuello di un fattore mille la pressione arriverebbero a percorrere due mille centimetri. Quindi capite che la partia di energia diventa questo punto del tutto trascura. Quindi già soltanto diminuire di un fattore mille mi va più che bene. Chiaramente capite l'effetto perché avviene questa proporzionalità inversa tra pressione e range. Semplicemente perché variare la pressione e qui va a cambiare la densità e quindi il numero di urti che avvengono all'interno del gas. Quindi certamente diminuendo la pressione da mille mille bar che la pressionata atmosferica sta andando grossomodo a un mille bar mi permette di assicurarmi che queste alfa percorrerebbero anche decine di metri prima di fermarsi. Quindi in un paio di centimetri la perdita di energia ed è tutto irrisoria. Noi raggiungeremo pressioni dell'ordine di 10 alla men 1, 10 alla men 2 mille bar. Quindi va più che bene. Quindi un fattore 10 alla 4, 10 alla 5 è più basso rispetto alla pressionata atmosferica. E allora quali sono le misure che andremo a fare? In anzitutto andremo a misurare lo spettro in energia per realizzare anche una calibrazione che è un po' la stessa operazione che abbiamo fatto nel caso dei $\gamma$, anche lì mi suravate dei $\gamma$ di energia nota, però dovete andare a vedere in termini di canali, quindi la scala orizzontale che vi compariva sul grafico, ogni canale a che energia corrispondereva. Allo stesso modo lo dobbiamo fare per le alfa e si partirà con l'utilizzo di una sorgente mixed, cioè una sorgente dove sono presenti più i sottopinoti, in particolare gli sottopi che sono presenti in questa sorgente sono questi elentati qui, vedete, Nettunio, Americe e Cuyo. La Americe e Nettunio hanno anche, e anche Cuyo, hanno anche dei picchi satellite, cioè a causa dello schema di detadimento che può avvenire verso i livelli eccitati del nucleo o figlio, abbiamo l'emissione di alfa di diverse energia, chiaramente con intensità diverse, quindi è molto più probabile andare a misurare, ad esempio nel caso dell'Americe, questo decalimento piuttosto che quest'altro, però potrebbero essere presenti picchi satellite che possono essere anche identificate se andate a effettuare una misura ad alta statistica. Ora quando andrete a fare questa misura, cosa dovete considerare? Innanzitutto il tempo che abbiamo a disposizione, questa è una delle misure che dovete fare, quindi dovete decidere quanto tempo dedicare a questa prima misura, che è una misura comunque sia importante, perché da questa deve venire fuori la calibrazione e in più possiamo studiare la risoluzione del vostro sistema, del vostro e giusto come esempio vi riporto qui un tipico aspetto che viene fuori, addirittura qua già è stato realizzato un best fit con una funzione gaussiana, una per picco, ecetto che nel caso del netto unico che vedete qui dove fortunatamente la statistica ci permette di mettere in evidenza la presenza di un picco satellite, quindi capite che di tutto questo schema noi riusciamo a vedere solamente uno di questi picchi satellite, probabilmente le date al fatto, ma questo dovete valutare voi durante l'esperienza, due tal fatto sia che alcuni picchi hanno intensità molto basse, sia che alcuni picchi sono troppo vicini di uno all'altro e venono sostanzialmente confusi col picco principale. È chiaro che aumentare la statistica migliorerebbe certamente questo aspetto però abbiamo a disposizione un tempo limitato e quindi dovete decidere dove fermare la misura. Dalla posizione di questi picchi che vedete qui riportate in canali possiamo andare a fare una corrispondenza, canale, energia e chiaro che i picchi che vedete qui sono i picchi principali quindi ad esempio nel caso del netto unico e questo picco qui a 4.788 MeV, nel caso dell'americio e quello a 5.486 e infine nel caso del curio è 5.805. Tenendo questi valori che sono riportati sulla severità e facendoli corrispondere al centroide di questi picchi è possibile realizzare la calibrazione in energia, quindi una retta che vi vedete 

\textbf{parte 2}