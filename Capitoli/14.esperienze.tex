\textbf{lez 13}

Allora mi sentite da fuori? Sì, però se adesso la sentiamo. Scusate, è ritardo e ci sono state una serie di domande quindi abbiamo attivato il collegamento con un po' di ritardo. Allora iniziamo con David al monitor. Allora ragazzi possiamo iniziare. Come vi dicevo oggi descriverò le esperienze che vi ritroveremo da fare a partire dalla prossima settimana. Quindi Joe Vidi vi dico che è scalta la lezione perché nuovamente sono fuori per lavoro però inizieremo con i turni di laboratorio. Ok a breve vi farò avere questo schema dei turni. Ho completato i gruppi perché molti di voi già avevate inserito il nome e altri li ho inseriti io dato che non erano presenti e quindi li ho inseriti io casualmente nei gruppi che erano rimasti scoperti. Quindi se hai andato a consultare quel Fone XL che aveva condiviso trovate i gruppi come sono stati definiti se ci sono dei problemi che sono mai fatte nello presente. Siccome già è stato ammirato se per caso abbiamo difficoltà per alcune date di laboratorio perché vi ha detto non tutti con molti in tutti i giorni previste per il laboratorio eventualmente se riuscite a trovare una soluzione tra di voi cioè far uno scambio con un collegamento e magari magari altro giorno deve fare la stessa esperienza possiamo farlo basta di te ne lo comunicate quando venite il laboratorio altrimenti ne lo fate presente proviamo a trovare un'altra soluzione. Comunque si aveva previsto anche dei turni di recupero eventualmente per chi proprio perde troppe lezioni eventualmente alla fine del mesi di gennaio ho previsto qualche turno di recupero. Vi ricordo nuovamente il discorso del certificato di sicurezza insomma il corso di sicurezza che abbiamo eseguito io ho ricevuto dalla secretaria una lista di studenti che aveva sostenuto questo corso infatti mi sono appuntata a alcuni nominativi a quelli di voi mi hanno inviato via email o un screenshot della scirmata insomma del cosso che abbiamo sostenuto gli altri mi raccomando cercate di averlo pronto prima di entrare il laboratorio e me lo consegnate. Chi l'ha fatto il primo anno dovrebbe valere perché se non mi sbagli non aveva di di di di di 5 anni io vi posso far vedere che lista mi è arrivata perché a me è arrivata una lista però non era per niente completa cioè nel senso ne mancavano parecchi. Eccolo chi ha l'X era in questa lista quindi non riuscite a leggerla? Vendranno il discontomo. Quindi verificate questa è la lista che mi è stata data circa una settimana fa e in ordine alfabetico. Vedete? Vedete? Molti mancano ora non so se perché non l'abbiamo fatto pura perché... Ok vabbè ma comunque risolviamo se poi semplicemente mi fate uno screenshot e mi mandate la scirmata. Non c'è bisogno di portarmi nulla il laboratorio, io so che voi abbiamo fatto il corso quindi chi ha l'X certamente non mi deve dare nulla. Chi non ha l'X magari se si fa uno screenshot semplicemente che compare scritto ora qualcuno di voi me l'ha mandato il corso e il punteggio non vi ricordo come arriva. È un problema, ci sono difficoltà su questo... come non so come funziona, non non qualcuno sociato così no? Ecco magari se mi scambiate tradito di questa informazione perché io non lo so sinceramente. Oppure se mi manda gli elenchi passati io non lo mando, io mi mangio di sì infatti è sempre lo stesso quindi basta di mi trasmette un elenco delle persone. Comunque i primi apporzi il problema saranno quel è il gruppo 1 perché entreranno in laboratorio il giorno 29, prossima settimana. Allora ragazzi vi dicevo in questi turni di laboratorio faremo parte delle esperienze di la fine della tesina. Per questioni logissiche ho selezionato 3 di queste esperienze rispetto a quelle che sono elencate nel programma più vi farò fare anche un'esercitazione che però non è un oggetto di tesina quindi un'esercitazione che farete con l'attività laboratoriale solo utilizzando i contatori guide per la distribuzione di quassonna quindi per lo studio della distribuzione di quassonna. Quindi contemporaneamente ci saranno 4 gruppi che lavoreranno in laboratorio, suddivisi in due ambienti non so se abbiamo avuto modo di vedere dove sono gli ambienti del laboratorio 3 sono 2 e uno è all'inizio di questo corridoio e l'altro praticamente alla fine del corridoio che porta verso i laboratori 1 e 2. Quindi vado ad arrivare poi vi dirò in quale ambiente lavorerete perché sono due esperienze in un locale e due esperienze in un altro locale. Quindi una esperienza che farete riguarderà certamente la misura del coefficiente di assorbimento degli elettroni in diversi tipi di materiale utilizzando come rivelatore il contatore guide. Quindi a questo punto date le lezioni che abbiamo fatta abbiamo tutte le conoscenze necessarie per poter svolgere questa esperienza. Cosa state facendo? State semplicemente andando a studiare come gli elettroni vengono assorbiti da spessori di materiale di natura diversa. Quindi la perdida di energia degli elettroni in un dato materiale che è una cosa che abbiamo visto, abbiamo detto avviene attraverso diversi meccanismi può essere anche parametrizata con un esempio formato di vede blocco soprattutto se stiamo lavorando a base energia come sarà ovviamente il laboratorio e si tratta di verificare sperimentalmente una legge semimilpirica che abbiamo descritto a lezione. I rivelatore che doverate è un contatore guide. L'abbiamo visto anche a lezione la volta scorsa, l'abbiamo studiato dal punto di vista del principio di funzionamento. In realtà il rivelatore guide che doverete è un po' diverso, non è proprio quello che vi ho dato l'altra volta lezione ma è sempre un contatore guide. Quindi la solita geometria cilindrica che conoscete e in particolare verrà gestito con una apparatosperimenta, un dispositivo che permette sia di applicare la differenza di potenziale tra gli elettrodi sia di conteggiare le particelle che arrivano quindi a un display che vi permette anche di sapere quante particelle sono passate in un determinato intervallo di tempo. Quindi l'apparatosperimentale si presenta grosso modo così schematizzato. abbiamo sostanzialmente una sorgente beta in particolare una sorgente di stronzi 90 e 390 che sono due isotopi che decadono beta. Ora vedremo anche lo schema di decadimento per capire l'endpoint e il tempo di dimezamento. Dopodiché abbiamo una serie di assorbitori di diverso tipo con vari spessori. In particolare c'è un supporto dove possiamo andare a posizionare diversi spessori, direttura possiamo andare a sommare insieme quindi non so ad esempio abbiamo diversi spessori di alluminio, uno da mezzo millimetro, uno da un millimetro, possiamo metterli insieme per fare uno spessore da un millimetro e mezzo. Ok quindi possiamo combinarli in maniera tale da sabbiliere diverse valori di spessore. In particolare in base al tipo di materiale che decidete di utilizzare ha senso arrivare fino a uno spessore massimo perché quello che avviene che quando superate uno spessore pari a range chiaramente andate ad assorbire tutti gli elettroni e messi dell'assorgente quindi anche se aggiungete altri spessori chiaramente non cambierà nulla nella vostra misura. Aveva disposizione in particolare alluminio, ottone che è una leda e quindi l'assorbimento dipende anche un po' dalla percentuale di questi elementi presenti nella leda e plexiglass. La rivelazione vi diceva che avviene attraverso un contatore Geiger che ha in particolare una finestra di ingresso che in questo momento non la vedete comunque dovete immaginare l'altra base del cilindro con la esposta verso l'assorgente quindi una finestra sottile proprio per ridurre al minimo la perdita di energia degli elettroni che entrano nel contatore Geiger. Sono altra visuale diciamo dello stesso apparato sperimentale, grosso modo e in particolare il contatore Geiger attraverso un carbo e collegato vi dicevo per chiatura elettronica. Questa vi permetterà di stabilire la differenza di potenziale tra gli elettroni del contatore Geiger perché se vi ricordate di principi di funzionamento del Geiger è basato su una camera cilindrica dove l'involgore esterno rappresenta il cattodo e un filo che percorre l'asse del cilindro 

rappresenta l'anodo. La differenza di potenziale applicata tra cattodo e anodo permette la raccolta del segnale quindi quando una particella carita attra come l'elettrone attraversa il contatore Geiger produce coppie elettrone e ione in particolare poi attraverso questo campo elettrico queste coppie migliano ovviamente gli elettroni verso l'anodo e gli ioni positivi verso il cattodo e gli elettroni riescono a produrre delle ionizzazioni successive quindi con un meccanismo diciamo avvalanga. Questo porta formazione di un segnale e questo segnale sempre attraverso questo cavo viaggia verso questo dispositivo e viene contato quindi ogni qualvolta si presenta un segnale e nel display il numero che è presente nel display aumenta la immunità. Quindi diciamo dal punto di vista operativo non c'è nulla da stabilire ad esempio la differenza di potenziale si può essere variata però lavorerete con la differenza di potenziale fissa perché se vi ricordate per lavorare con il contatore il Geiger bisogna lavorare nel piano rotto lo Geiger ha mostrato una curva e quindi bisogna scegliere un'opportuna attenzione di lavoro ma voi la troverete impostata quindi non dovete fare nessuna operazione si tratterà soltanto di far partire il contatore e a zerarlo. Chiaramente dovete effettuare tante misure con diversi spessori per poter costruire una curva di assorbimento. Il timer non è disponibile perché possiamo utilizzare benissimo quello del cellulare quindi vi basate su questo tanto non è necessaria una precisione, appartato del timer del cellulare una buona precisione ma comunque non è certamente l'errore sul tempo a comandare in questo tipo di misura. Ripassiamo un po cosa vi dovete aspettare. La legge di assorbimento degli elettroni grosso nodo segue un andamento di tipo esponenziale di crescente questa è una una semi-empirica se vi ricordate in particolare quindi vi aspettate che l'intensità si riduca attraverso una funzione esponenziale dove la pennenza dell'esponenziale è determinato da questo parametro mu che è il coefficiente di assorbimento proprio ciò che volete determinare da questa misura. Il coefficiente mu solle sprimete in unità lineari e espresso in centimetri alla meno uno o metri alla meno uno insomma una lunghezza alla meno uno. Tuttavia se andate a dividerlo per la densità ottenete il coefficiente di assorbimento massivo che diventa praticamente quasi indipendente dal materiale cioè otterrete un valore che è molto simile anche per materiali condensità diversa quindi ad esempio troverete che per l'alluminio il mu lineare è molto diverso da l'uniu lineare dell'ottone ma se andate a considerare il coefficiente di attenuazione massivo in quel caso i due mu di due coefficienti di assorbimento per ottone e dalluminio saranno molto simile tra di loro. In questa legge che vedete qui chiaramente voi non andate a misurare un'intensità ma andate a misurare un rate di conteggio una frequenza di conteggio quindi quante particelle arrivano sul conto Turing diver al secondo. Per questo motivo è importante che durante tutte le misure non venga modificata la distanza trasorgente in rivelatore perché altrimenti chiaramente non sono più misure confrontabili quindi questa i sarà espressa in Hertz. I con zero è una misura che si effettua senza alcuno spessore di materiale quindi tutto ciò che proviene della sorgente che intercetta il rivelatore quindi rispetto a questo I con zero che il valore massimo che possiamo misurare è man mano che andate a interporre degli spessori vi aspettate che questa I diminuisca. Quello che si otterrà quindi dovrebbe essere un andamento di questo tipo vedete questa è una misura realizzata con diversi spessori di alluminio ogni punto rappresentano una misura realizzata per un determinato spessore quindi vedete abbiamo lo spessore nullo che vuol dire non interporre nulla e poi abbiamo via via spessori che variano di mezzo millimetro quindi zero 5 1 5 2 e così via la scala che trovate riportato è i millimetri sull'assorizzontale perché lo spessore è sulla sé verticale trovate proprio il rate quindi quante particelle si misura al secondo in scala logaritmica in questo modo relazione precedente capite si trasforma viene linearizzata quindi si si trasforma sostanzialmente in una relazione lineare quindi in questo modo è più facile verificare che effettivamente i punti seguano un andamento esponenziale decrescente vedete i valori in grosso modo che sono più o meno quelli che otterrete il laboratorio è chiaro che la cosa migliore da fare e cercare di posizionare il rivelatore quanto più vicino alla sorgente però lasciando uno spazio utile per l'inserimento di tutti gli spessori che volete inserire perché è importante fare questa questa cortezza perché in questo questo diminuite la distanza e capite aumentate l'angolo solido sul teso del rivelatore quella che noi avevamo definito come accettanza geometrica questo comporta il fatto di permettere una un rate di opposizione più elevato quindi da un punto di vista statistico riuscire a raccogliere più dati in meno tempo sostanzialmente quindi cercate di mantenere non solo la distanza fissa ma quella più piccola possibile tra sorgente e rivelatore e quindi con questa cortezza grosso modo quello che vi aspettate sono valori di questo tipo quindi vedete quando non interponete nessuno spessore siamo a valori intorno a 20 hertz quindi comunque sono valori elevati capite che già misure di pochi minuti portano all'accumulo di centinaio di conteggi e lo dovete pensare da un punto di vista statistico perché ogni volta che poi effettuato una misura dovete pensare all'incertezza che abbiamo su quella misura quindi se io arriva ad acquisire 100 conteggi l'incertezza sarà la randesce di 100 quindi 10 e 10 su 100 quindi avere una incertezza di 10 su 100 misurate vuol dire un'incertezza del 10 per cento che comunque è grande si può migliorare certamente quindi più statistiche accumulate migliori sarà la vostra incertezza relativa e chiaramente il problema non si pone quando non abbiamo alcun alcun spessore o spessori molto sottili perché vedete qui le frequenze sono molto elevate il problema si pone quando gli spessori diventano notevoli tanto da praticamente arrestare tutte le particelle che propengono dalla sorgente e infatti qua desentù possiamo avere anche valori di 0,1 quindi abbastanza bassi e capite che se volete avere la stessa incertezza statistica su tutti i punti questo vorrà dire che le prime misure dureranno di meno le misure con spessori più elevati dovranno durare di più quindi quello che mi viene chiesto il laboratorio innanzitutto è anche fare una stima dei tempi delle diverse misure e organizzarvi il lavoro abbiamo a disposizione due ore abbiamo a disposizione diversi materiali questo non vuol dire che dovete fare tutti i materiali magari ne fate almeno due per confrontarlo ovviamente e ma per ogni materiale dovete esplorare diversi spessori quindi per vi dire vi richiediamo una maturità maggiore rispetto ai laboratori precedenti perché in questo caso vi ritrovate proprio a progettare una misura a stabilire voi come suddividere il tempo a disposizione ok quindi questo cosa vuol dire vuol dire che magari andate a valutare quanto è la frequenza nelle condizioni migliori quanto è la frequenza di conteggio nelle condizioni peggiori e a quel punto stabilite quanti spessori analizzare quanto tempo dedicarci a spuna misura in maniera da suddividere bene il tempo ok infazio successiva di analisi se poi non verrà estratta questa esperienza chiaramente il tutto consisterà poi nel estrarve da questi dati quindi dalla pennenza di questa curva il coefficiente di assorbimento e confrontarlo con una valore tabulata in letteratura guardiamo un po più nel dettaglio la sorgente la sorgente che è stata doverando io ho detto una sorgente mista non è misto scusate una sorgente di stronzi o 90 e 3 o 90 sapete che

le sorgenti beta non sono monocromatiche proprio perché si tratta di un decadimento a tre corbi quindi abbiamo nello stato finale il nupio residuo più l'elettrone l'anti neutrino e questo passì che l'elettrone può assumere valori di energia diverse abbiamo un doppio decadimento lo stronzi o 90 decade in e poi il 90 decada a sua volta non so se lo sclama di determinato no comunque nella nella scheda dell dell esperienza lo troverete ma la cosa che ci interessa quindi è un decadimento possiamo dire a cadena da lo stronzi 90 abbiamo come è prodotto l'itrio 90 poi solta decade nuovamente in beta sono neque di imbrio secolare perché appunto i tempi di dimazzamento sono molto diversi tra di loro e quindi comanda sostanzialmente quello che ha il tempo di dimazzamento più grande quindi vuol dire che ogni qual volta c'è un decadimento dello stronzi 90 subito dopo c'è il decadimento del littrio 90 e come spettri chiaramente lo spettro o conflessivo sarà la sovrapposizione dei due spettri quindi quello singolarmente dello stronzi 90 quello singolarmente del littrio 90 quindi quando andate a effettuare una misura dello spettro in energia o lo andate a simulare con la teoria di fermi e la forma dello spettro ha questo andamento che in realtà deriva dalla sovrapposizione di due spettri diversi uno relativo al decadimento dello stronzi o è uno relativo al decadimento dell'itrio e quindi vedete una prima parte a bassa energia un primo spettro che ha come un po in grosso modo e mezzo mezzo e poi un secondo spettro che si estende energia e pellevate dell'ordine di 2.2 mezzo come un po in ovviamente non possiamo distinguere quali sono gli elettroni provenienti del litrio quali provenienti dello stronzi non ci interessano eppure ci interessa sapere soltanto che è molto probabile che l'elettrone che state misurando ha un energia bassa perché vedete questa distribuzione questo spettro presenta dei valori molto elevati proprio basse energie diventa più improvabile avere elettroni più energetici cosa comporta questo comporta il fatto che man mano che inserite degli sfessori è chiaro che i primi elettroni essere sorbiti sono quelli relativi alla parte di bassa energia soltanto quelli più energetici riescono ad attraversare ovviamente i spessori maggiori questo è giusto per capire appunto cosa ci si aspetta attraverso delle simulazioni e riguardo appunto questo assorbimento degli elettroni ad esempio sapessimo elettroni monocromatici l'andamento sarebbe completamente diverso se vi ricordate ci aspettiamo l'andamento quasi al gradino però un gradino molto smussato non vi fa di niente in in da questo grafico semplicemente qui abbiamo una scala logarittimica verticale se rappresentassimo una scala lineare avremo un gradino molto smussato quando invece vai a considerare degli spetti di continui come quello dello stronzio novanta dell'itria novanta l'andamento è appunto un andamento in un modo disponenziale decrescente questa è la componente dello stronzio novanta che è quella che comanda bassa energia quindi questo primo spettro e questo dell'itria novanta come si vedono le scala orizzontali ma si si può vedere che vediamo uscendo scosate dalla presentazione se si veda le scale sono molto diverse vedete l'astronzione novanta che ha con end point o in questo questo mezzo MeV fa sì che gli elettroni arrivino al massima per correre 0,25 mm quindi un quarto di millimetro e lo stronzio e l'itria novanta invece che mette elettroni con energia più elevata anche di qualche MeV in questo caso vediamo che gli elettroni possono percorrere anche alcuni millimetri di questo materiale mi sbaglio all'uminio. Ovviamente noi non andiamo a distinguere le mie componenti vi ho detto vedremo il risultato complessivo però giusto per aver l'idea di come incidono i due isotopi. Se si vanno a confrontare i diversi materiali cosa ci si aspetta? Chiaramente l'annamento di queste curve è differente quindi il valore di miù mi aspetto che sia più elevato nel caso di materiali più densi come ad esempio l'ottone e quindi vedete abbiamo una discesa molto veloce addirittura con un millimetro di ottone già fate fuori tutti gli elettroni sostanzialmente mentre l'alluminio e la situazione intermedia ad esempio rispetto a utilizzare del cartone che è stato riportato un altro un altro materiale però non abbiamo a disposizione comunque il plexi ovviamente si comporta in maniera più simile al cartone piuttosto che all'alluminio con un coefficiente più basso. Vi dicevo quella forma di queste curve si può approssimare un'espolenziale detrescente quindi vedrete ovviamente delle leggeri deviazioni rispetto a quanto ho premisto da una pure espolenziale detrescente. Se si vuole essere approfondire questi aspetti di perdere energia bisognerà ricorrere assimolazioni molto dettagliate con dei software professionali come il software giant che vi ho ci dato più volte. Giant non è complesso da adoperare ci sono diverse versioni il giant giant e il giant giant vi dico che qualche vostro collega ad esempio in salle di esame presentando la tessina sia anche cimentato a fare dei conti cini con questi software un po più evoluti proprio per capire cosa mi aspetto in determinate condizioni si può simulare tutto con giant a far fare dalla sorgente, il materiale che abbiamo interposto, il rivelatore si può veramente simulare tutto e ad esempio questo è un un risultato che si ottiene con giant a 3 dove è stato riprodotto l'apparato sperimentale e sono stati riportati i dati quindi che ci si attende con questo apparato sperimentale poi si può fare un confronto con i dati sperimentali. Chiaramente non è richiesto non è un obbligatore però io ve ne parlo perché appunto sono anche espunti eventualmente per la tessina insieme al materiale che vi fornirò per questa esperienza vi fornisco anche un articolo che abbiamo scritto il professor Riggi riguatto proprio a questa questa quindi questo è l'ultimo che è un'autilizzare per lo sviluppo della tessina. Cosa andrete a fare quindi materialmente abbiamo scritto l'apparato sperimentale andrete innanzitutto studiare il fondo perché è la prima cosa di cui vi accorgerete che anche in assenza disorgente il conto della idea è di armisura qualcosa e questo è legato alla radiazione naturale quindi chiaramente tutto ciò che viene messo dai muri la radiazione cosonica interviene quindi fa parte costituzione una sorta di fondo che è presente sempre nelle vostre misure quindi anche quando sarà presente la sorgente parte delle particelle che abbiamo misurato in realtà non è un problema della sorgente ma sono legate al fondo di radiatività naturale quindi bisogna stimarlo per poterlo andare a sottrarre dalle misure. A quel punto si vanno a effettuare le misure per 

valutare il coefficiente di assorbimento con un dato materiale ad esempio possiamo partire dall'alluminio che quello più semplice anche perché ha spessori diciamo macroscopici anche più facile da misurare rispetto all'ottone e poi ripetere tutto per almeno un altro materiale quindi ho l'ottone o il plexiglass. I aspetti su cui fare attenzione ve lo detto riguardano i tempi certamente di misura perché questo stabilisceombra l'errore statistico da attribuire a ciascun punto sperimentale e l'aspetto della sottrazione del fondo e che è importante soprattutto per le misure con spesso rippie elevati perché quando andate a interporre 3 millimetri di alluminio che fermano praticamente tutte le particelle messa dalla sorgente è chiaro che quello che è messurato è praticamente al fondo quindi dovete stimarlo bene il fondo per poterlo sottrarre e quindi quindi quindi quindi quindi quindi quindi la la su questa esperienza quindi io vi fornirò. Oltre alla presentazione di oggi ovviamente una scheda ve la faccio vedere che descrive l'esperienza quindi in maniera più discorsiva rispetto alla presentazione di viene descritta l'esperienza quindi richiamando la parte teorica ciò che vi aspettate l'apparato sperimentale e poi le diverse parti dell'esperienza. Vi fornirò una scheda di attività che è qualcosa che dovete man mano fare durante l'esperienza anche per rendermi conto che tutto sta andando come effettivamente dovrebbe andare e quindi ad esempio quando andate a fare la misura del fondo vi chiedo a valutare il tasso di conteggi del fondo e l'incertezza statistica corrispondente quindi non so abbiamo fatto una misura l'unica abbiamo acquisito mille conteggi, andate a valutare anche l'incertezza statistica qui ad esempio vi viene fornita una tabella da riempire in base alla misura che abbiamo settualato la possibilità di andare a già costruire un primo grafico ma non lo mente perché ogni volta che si fa una misura è utile comunque non solo guardare il dato che abbiamo ottenuto ma anche provare a rappresentarlo banalmente in grafico, giusto per capire se le cose torno non pure no, sapete quanta volta mi arrivano tessine dove c'è l'andamento di tutti i punti che segue un certo trend e poi c'è un punto sparato da un'altra parte ma se le sono accorti dopo a posteriori la misura andava bene se invece si fa un controllo puntuale insomma bisogna saper valutare quello che si sta misurando quindi questa scheda di attività vi può aiutare anche in questo ovviamente il laboratorio sono presente io, il tecnico di laboratorio saranno presenti anche il ragazzo che farà il tutorial, ci siamo qui si abbiamo il dubbio ma manovitiamo durante le due ore di laboratorio e l'articolo che vi avevo detto prima quindi questo sarà il materiale accorrendo di questa esperienza. Ci sono domande su questa esperienza? Sì. C'è l'ostronzo dei caldi con elettroni un poche energetici, tanto che ci sia un enno spessore di aria tra la sagente e il rivelatore, quanto può influire come errore? Alla rada. Che Che si sono particelle altre, quindi non si cargano subito.

Il collega ha chiesto per chi ha casa ha chiesto nel soprattutto per quanto riguarda gli elettroni meno energetici, se ci sono dei problemi per il fatto che questi elettroni dovranno attraversare una piccola porzione di aria prima di essere livellati in realtà hanno praticamente la perdita di energia in aria di elettroni veramente trascurabile rispetto a quella che avviene dei materiali solidi quindi si può riteneri praticamente trascurabile. Altra domanda? Ok, mi pari di no. Allora andiamo avanti con le altre esperienze. Altra esperienza che vi ritroverete a fare è la spettrometria $\gamma$. Quindi in questo caso la sorgente riadoperata è una sorgente $\gamma$ e per misurare in $\gamma$ con una buona efficienza se vi ricordate abbiamo detto è utile ad operare scintillatori. In particolare quello che adopererete in laboratorio è uno uno aiuduro di sodio drogato al tallio che è uno scintillatore in organico, un cristallo. Uno di quei cristalli che vi diceva la difficoltà di essere igroscomico e infatti non lo vedrete nel suo stato naturale, nel senso lo vedrete encapsulato dentro un rivestimento di alluminio che non solo permette la riflessione della luce, la raccolta della luce e lo scintillatore ma in più lo protegge dall'umidità. Quindi avrete a disposizione diverse sorgenti $\gamma$, in particolare certamente il Cessio 137 e il Cobalto 60 che sono delle sorgenti di cui vi ho fatto vedere gli spettri più volte. Perché vi cito questi? Perché innanzitutto sono quelli che hanno un'attività maggiore nel nostro caso e quindi permettono delle misure anche con buona statistica in tempi brevi ma in più utilizzando solamente queste due sorgenti possiamo già effettuare una calibrazione nell'energia. Cioè abbiamo a disposizione già tre valori energetici perché il Cessio 137 è decada in $\gamma$ e mette $\gamma$ a 662 keV. Il Cobalto 60 addirittura ha due dettaglimenti $\gamma$ e quindi abbiamo altri due valori di energia e questo vedremo e ci ci utile per effettuare una calibrazione in energia. Quindi basterebbero soltanto queste tre sorgenti ma in realtà ne avrete a disposizione anche altri. Poi abbiamo un rivelatore scintillatore che abbiamo discusto la volta scorsa, vi ho detto di un duro di sodio drogato al taglio e ovviamente un foto sensore, in questo caso è un foto multiplicatore. In questo caso troverete un minimo di elettronica. Ancora l'elettronica non l'abbiamo affrontata come argomento quindi non abbiamo parlato di preamplificatore però diciamo una catena elettronica molto semplice, quello che faremo è vedere innanzitutto il segnale lo sceloscopio di questo foto multiplicatore così cominciate a approcciarvi per la prima volta con i segnali elettrici quindi anche l'elettronica che c'è di conseguenza. Però non vi preoccupate per questo discorso dell'elettronica perché questo è stesso appareto sperimentale e poi verrà utilizzato per la seconda esperienza il secondo turno di laboratorio. Quindi in quel caso poi avremo già presentato in diversi modi elettroniche quindi potremmo guardarli con altri occhi. Questo è lo schema dei livelli delle due sorventi di cui vi parlavo prima, quindi il CSI 137 vedete che il decadimento $\gamma$ avrebbe sempre assaigliato di un altro decadimento e quindi in particolare qui vedete è messo un $\gamma$ 662 keV. Mentre per quanto riguarda il cobalto vengono emessi due $\gamma$ 1 a 1 e 17 keV e l'altro a 1 e 33. Capite ad esempio che nel caso del successio questo $\gamma$ che arriva sullo scintillatore interagisce solamente o per effetto fotoelettrico o per effetto quantum. Ad esempio la produzione di coppie qui non può avvenire perché non siamo all'obbi di sopra della famosa energia di soglia per cui può avvenire produzione di coppie. Cosa ci aspettiamo di osservare? Questo è un aspetto che non abbiamo discosso. Noi abbiamo parlato da un lato dei $\gamma$. Abbiamo detto sono monocromatici e da un lato abbiamo parlato di scintillatore. Abbiamo detto in via di principio con uno scintillatore siamo in grado di misurare l'energia. Quindi cosa mi aspetterei da uno spettro $\gamma$? Mi aspetterei uno spettro mono energetico. Quindi sempre lo stesso valore di energia all'ina di principio, dato che in $\gamma$ sono mon energetici. Pensiamo ad esempio in $\gamma$ emessi dalla sorgente di Cesio. Sono tutti emessi con energia di 662 keV. Quindi se io avessi un rivelatore come lo scintillatore in grado di misurare l'energia dovrei dire che l'energia di misura è sempre 662 keV. Non dovrei ottenere altri valori. In realtà la situazione è più complessa. Se ad esempio il rivelatore avesse dimensioni infinite. In quel caso sono sicure che effettivamente tutta l'energia del $\gamma$ viene misurata. Perché cosa può fare il $\gamma$ quando incide sullo scintillatore? Può dare l'uovo a quei tre effetti che abbiamo detto prima. Effetto fotoelettrico, effetto quantum, produzione di coppie. Se si produce effetto fotoelettrico siamo fortunati perché come prodotto dell'effetto fotoelettrico abbiamo l'emissione di un elettrone. Questo elettrone si prende praticamente tutta l'energia del fotone incidente. Perché ad esempio se il fotone ha 662 keV, questi 662 keV servono in parte per liberare l'elettrone. Quindi una piccolissima parte viene utilizzata per strapare l'elettrone all'atomo. Tutta la restante parte se la prende l'elettrone come energia cinetica. Ora quanto vale quanto è l'energia necessaria per strapare un elettrone lo abbiamo detto pochi elettronvolta, una decina di elettronvolta, venti elettronvolta non di più. Quindi vuol dire che su 662 keV venti elettronvolta venono spensi per strapare l'elettrone. E capite che è una quantità ridicola in risoria e possiamo arrivare ad affermare che l'elettrone viene messo con l'energia per far far quella quella diamo incidente. Ok, questo elettrone nel materiale e e ovviamente interagisce laundry laundry i meccanismi che abbiamo visto quando abbiamo parlato della perdida di energia degli elettroni percorre qualche millimetro non l'abbiamo visto in caso dell'alluminio però ho detto detto fa percorre tra millimetri in grosso modo nel caso del Plexiglas quindi nel caso di unuscitillatore percorrere a un po di più 5-6 mm ma in pochissimo spazio perde tutta la sua energia e questa energia viene convertita in luce e quindi chiaramente a maggior ragione se il rivelatore è in dimensioni infinite sono sicure che tutta l'energia viene dissipata all'interno dello scintillatore e quindi in qualche modo il segnale più dotto sarà una misura della energia del $\gamma$ incidente e questo vi dicevo è il caso più semplice quando avviene effetto fotolettrico se avviene effetto Comton oltre a un elettrone diffuso abbiamo il $\gamma$ diffuso come prodotti dell'effetto Comton ora l'elettrone segue la stessa cosa lo stesso percorso che abbiamo detto prima quindi perde la sua energia e viene fermato in pochissimo spazio l'altro fotofotone diffuso può interagire con lo scintillatore attraverso sempre quei meccanismi che abbiamo detto effetto fotolettrico effetto Comton in produzione di coppia dipende ovviamente dalla sua energia e della probabilità di ciascun processo tuttavia se le dimensioni del rivelatore sono infinite anche se il $\gamma$ non interagisce Cut, Cut, prima o poi dobra interagire e quindi attraverso questi meccanismi prima o poi tutta l'energia verrà nuovamente depositata nello scintillatore e quindi in questo caso anche se avvenuto effetto Comton tutta l'energia viene ripostruita per il semplice fatto che ho un 

rivelatore di dimensioni infinite pensate invece di avere un rivelatore dimensioni finite come le le d'avere di laboratorio che ne so 5 cm per 5 cm per 5 cm per 5 cm cm cm il $\gamma$ prodotto dall effetto Comton potrebbe anche non interagire potrebbe fuori uscire dal materiale abbiamo detto il $\gamma$ ha una probabilità di interazione molto più basse rispetto a quella delle particelle cariche quindi potrebbe fuori uscire e noi lo perdiamo quindi quello che noi andiamo a misurare lo scintillatore soltanto l'energia che si è trasportato l'elettrone una parte dell'energia del $\gamma$ incidente quindi viene un po' a mancare quello che avevamo detto c'è che con un scintillatore io sono in grado di misurare l'energia della particelle incidente non è detto dipende cosa viene se avviere effetto Comton c'è una forte probabilità che parte dell'energia mena persa perché il fotone diffuso fuoriesce senza interagire dopo a mente per la produzione di coppia invece possiamo dire anche qui dato che si produce una coppia elettrone positrone che anche in questo caso si ricostuisce l'energia complessiva del $\gamma$ incidente quindi il grosso problema sta nell'effetto Comton se avessimo un rivelatore di dimensioni infinite non sarebbe un problema prima o poi raccogliamo tutta l'energia ma se il rivelatore ha dimensioni finita come è ovviamente nella realtà potrebbe accadere che parte dei prodotti delle interazioni come ad esempio i fotoni 

diffusi perfetto Comton fuoriescano dal rivelatore e non vendono misurati quindi l'energia vero riuscirà solamente una parte dell'energia iniziale questo porta ha uno spettro abbastanza complesso quindi quello che misurerete il laboratorio non sarà una riga in corrispondenza di un determinato valore di energia ma uno spettro che alla fine uno spettro continuo dove si intravedono determinate strutture quindi in particolare avremo un picco che è il il picco fotoelettrico questi eventi che poi misurate a questo valore di energia corrispondono agli eventi in cui si raccoglie tutta l'energia del $\gamma$ incidente e si chiama picco fuor elettrico perché questo si verifica soprattutto quando c'è effetto fotoelettrico abbiamo detto se avviene effetto fotoelettrico l'elettrone emesso viene sostanzialmente assorbito dal materiale cioè viene targa e tutta la sua energia materiale quindi ricostruiamo tutta l'energia e quindi è ragionevole che questo picco si trovi mi dica che l'energia è pari a 662 KG nel caso del senso dell'assorbiente di 637 quindi l'energia del $\gamma$ incidente ma oltre a questi casi possono capitare una serie di eventi in cui l'energia ricostruite solo una parte e quindi i valori di energia vedete più bassi e qui è tutto un continuo dove eventualmente si distinguono solamente due picchi picchi, cioè le strutture che prendere nel nome di picco di backscattering e di spalla cotton si possono valutare quindi il valore di energia a cui attendersi questi due picchetti si vogliamo chiamare picchi si può calcolare ricorrendo alla formula che conoscete benissimo valida per l'effetto cotton dove è possibile calcolare l'energia del fotone diffuso conoscendo l'energia del fotone incidente e l'angolo di scattering questa è una forma che conoscete e in particolare ad esempio per il 637 dove l'energia HN del fotone incidente 662 keV si può trovare che il picco di backscattering corrisponde a l'energia che ha il fotone quando l'angolo di scattering è di 180 gradi quindi quando teate 180 gradi si può fare il conto calcolare questo HN primo attraverso questa formula e si trova che il picco di backscattering è atteso a 184 keV e infatti se torniamo indietro ci ritroviamo qui questo per il 637 ovviamente per un altro risotto può cambiare la posizione di questo picco mentre la spalla cotton si valuta a considerare l'energia che ha l'elettrone in corrispondenza di questa condizione quando teate 180 in fatto banalmente si calcola come differenza tra l'energia del fotone iniziale e l'energia del fotone scatterato e quindi in questo caso 478 keV ed è proprio che abbiamo trovato qui quindi interessante quando andate a fare una misura verificare che il vostro spettro si presentino effettivamente questi picchi il picco foteletrio è quello più importante ovviamente è che ha una forma vedete grosso modo blausiano comunque non è strettissimo ma questo deriva dalla risoluzione in energia del vostro appartamento sperimentale e poi abbiamo questa struttura continua dove possiamo mettere in evidenza queste queste ulteriori picchetti Cosa dovrete fare il laboratorio? All'alena, anzi, tutta, studierete un po' il fotosensore. In questo caso, ti ho detto, un foto moltiplicatore. Il foto moltiplicatore lavora con un'altra tensione, in particolare quello che adopererete voi sui 600 keV. E quindi possiamo cercare di capire come va l'area, l'ampiezza del segnale prodotto quando cambiate la tensione, quindi apparità di sorgente. Stati inviando sempre gli 60 $\gamma$, ottenete ad esempio un picco fotoelettrico in corrispondenza di un determinato canale dell'ATC. Se cambiate la tensione, sostanzialmente cambiate il fattore di guadagno del foto moltiplicatore, quindi l'amplificazione cambia ed è il segnale elettrico. Quindi possiamo avere un segnale elettrico con ampiezza maggiore se aumentate la tensione e ampiezza minore se la diminuita. E quindi l'effetto sarà sostanzialmente che questo picco si sposta a destra o a sinistra, al secondo del valore di tensione che state considerando. Justo per capirci, vedete ad esempio qui abbiamo uno spettro espresso ovviamente in canali di ADC e questi sono dei picchi che sono stati ottenuti con diversi valori di tensione partendo da 550 volte fino ad arrivare al 600 volte. Vedete che man mano il picco si sposta verso destra come ci si aspetta perché il segnale all'aumentare della tensione del foto moltiplicatore aumenta la sua ampiezza e quindi l'ADC che è quel modulo che serve per andare a misurare l'ampiezza del segnale fornirà un valore in uscita che sarà via via più grande all'aumentare della tensione. Questi sono proprio i picchi fode elettrici come visto prima, non stiamo guardando la spalla conto, tutta la parte conto è tutta a sinistra e tagliate il grafico. Quindi la prima parte dell'esperienza è consistere a modificare la tensione e andare a studiare di quanto si sposta questo canale, quindi la posizione del picco ad esempio il passare da 550 550 e 5 valuto di quanti canali distano i due picchi, quindi questo delta C lo divido per la differenza di tensione, quindi quel caso 5 volt, lo divido per il valore di canale meglio tra i due e per i due picchi, ottengo quindi la variazione percentuale del canale per un volt di tensione, il coefficiente di guadagno, quindi di quanto varie il percentuale il mio segnale quando varie la tensione di un volt. Capite che se l'appalato è molto sensibile anche piccole variazioni di tensione comportano un grosso spostamento del picco, quindi un coefficiente di guadagno elevato, invece se lo strumento è meno sensibile anche se la tensione scilla fluttua di poco non mi accorvo il quasi nulla, il picco rimane sempre nella sua posizione. La seconda parte dell'esperienza consiste nel sabilire una calibrazione in energia, perché vi ho detto alla fine, voi abbiamo a disposizione una DC che vi dice l'ampiezza del segnale espressa in un numero che va da 0 a in questo caso 2048 perché abbiamo una DC a 11 bit. Come facciamo a capire ogni canale a che valore di energia corrisponde? Semplicemente utilizzando delle energie note e vedendo il picco fotoelettrico dove cade e quindi basta ad operare delle sorgenti per andare a realizzare una retta di calibrazione dove da un lato riportate le energie note quindi ad esempio abbiamo misurato lo spettro del cesio che emette a 662 keb, quindi siamo qui questo punto, abbiamo misurato un picco fotoelettrico che cade al canale 510, ecco questo rappresenta un punto di questo grafico. Chiaramente se abbiamo a disposizione diverse sorgenti, quindi con diverse energie, possiamo costruirvi questo grafico con diversi punti e valutare la retta di calibrazione, la relazione dovrebbe essere lineare è quello che ci si appende. L'altra parte dell'esperienza consiste a poi nell'analizzare lo spettro, quindi c'era dove si trova il picco fotoelettrico, quanto è il lavbo, perché questo mi da un'indicazione sulla risoluzione in energia, dove si trova la spalla conto, il picco in backscattering. In tutto ciò bisogna anche realizzare una misura del fondo, perché abbiamo sempre a che fare con un rivelatore, quindi questo rivelatore come nel caso il contatore Giver misurerà anche una radiazione di fondo e quindi dobbiamo fare una misura senza sorgente, che poi andrà sottratta dagli spettri misurati con le sorgenti. Considerate che nel caso del fotomultiplicatore, oltre alla radiazione di fondo, in realtà ci sarà una serie di misure, di eventi che vogliamo misurare di piccola ampiezza e questi sono 

il rivelca segnali dovuti al rumore del fotomultiplicatore. Vi ricordo che il fotomultiplicatore è uno strumento molto sensibile, ma anche molto rumoroso, tanto che, diciamo, bisogna avere la cortezza di non esporlar la luce, di non aumentare troppo la temperatura, perché ad esempio per effetto di emissione termica potrebbe essere prodotto un segnale spurso, un segnale che non deriva dalla misura di luce, ma semplicemente a causa di questa emissione di elettroni per effetto termico. Quindi in realtà ci saranno sempre una serie di segnali e di passi a un piazzale rappresenta un rumore elettronico, che può però cuocere sottratto realizzando una misura, in assenza di sorgente, che poi va sottratta allo spettro misurato. Vedete qui un esempio di spettro dove vedete il picco foteletrico che è stato fittato con una gaussiana e gli altri due bicchetti di backscattering e di contour, e si può andare a vedere appunto la corrispondenza con l'energia a cui ci si aspetta, si presentino queste strutture. In realtà troverete anche un bicchetto dovuto all'emissione di X, questo deriva dal decadimento sempre della sorgente, quindi eventualmente sapete una buona statistica, si può anche intravedere questo piccolo piccetto all'inizio. Si possima la l'attività della sorgente, poi abbiamo a disposizione una sorgente, diverse sorgenti, che hanno la loro attività. Chiaramente l'attività è quella che era stata fornita dal produttore tempo addietro, adesso bisogna valutare in base al tempo di dimensione a quanto si è ridotta l'attività, quindi voi se fate una misura la fate dell'attività odierna, quindi eventualmente dovreste provare a capire cosa bisogna attendersi a seguito del fatto che sono trascorsi anni da quando la sorgente è stata acquistata. Comunque detto questo l'attività è chiaramente non consiste semplicemente nel valutare quanti eventi andate a misurare al secondo, perché se vi ricordate c'è il problema che il vostro rivalatore non va a misurare tutto l'angolo solido, quindi tutte le particelle messe alla sorgente, ne misura solo una certa porzione legata alla sua accettanza geometrica e quindi con questo apparato si può provare a valutare l'attività conoscendo ovviamente l'efficienza geometrica. L'efficienza geometrica la possiamo valutare con delle formula un po' approssimate in base alla geometria che abbiamo all'isposizione oppure con delle simulazioni montetarlo che poi faremo materialmente durante le esercitazioni di rulto. Quindi avrete gli strumenti per poter valutare l'efficienza geometrica. Capite che conoscendo l'efficienza geometrica, l'efficienza intrinsica la possiamo considerare uno. Conoscete il rate di misura, la frequenza di conteggio, il branching resso e il noto in base al tipo di sorgente che conoscete possiamo valutare l'attività e confontarla quanto effettivamente vi aspettate in base agli altri resso e il risorgente. Ok, con questa esperienza oltre ovviamente la presentazione mi fornirò una nota descriviva dell'esperienza più una scheda di attività. Andiamo a vedere la terza esperienza di esame che in questo caso è lo spettrofotometro. Lo spettrofotometro è uno strumento per andarmi a suonare spettri luminosi, quindi spettri disorgente luminose. Avrete a disposizione di diversi tipi di disorgente che dovrete effettuare una rispura. In particolare vi farò vedere i vantaggi di operare un spettrofotometro rispetto alla strumentazione che abbiamo adoperato fino ad essere elaborato in un web specifico. Che cosa avrete a disposizione? Lo strumento per misurare lo spettro, quindi lo spettrofotometro digitale, diverse sorgenti luminose sia di spettri continui che di spettri all'iglia. Ovviamente ci limitiamo alla zona visibile, quindi ciò che avvede l'occhio umano anche se in realtà lo spettrofotometro permette di misurare anche un po' dell'UV e un po' dell'infrarosso. Esempi di spettri che andranno a misurare. In particolare come spettri continui possiamo adoperare la lampada neon, il ciasso fitto, una lampada incandescenza, faretti all'olgeni, quindi abbiamo a disposizione diversi sorgenti. Lo spettro che misurate vi permette di stabilire le caratteristiche di quella sorgente, quindi tipicamente ad esempio una sorgente calda, non so se vedete mai questa lampadina, penso che si è raccapitata, tante volte le lampadine si caratterizzano dalla temperatura, del valore di temperatura, perché in base alla temperatura il colore della luce è più caldo o più freddo. Quindi vi renderete conto anche analizzando lo spettro di una sorgente che viene definita calda, quindi con una componente molto forte del rosso e del giallo rispetto a una sorgente di luce fredda, come può essere ad esempio un LED bianco del cellulare, la luce del cellulare con una sorgente di luce molto fredda. Ma abbiamo anche a disposizione delle sorgenti che mettono spettri a righe, in particolare ovviamente una sorgente migliore in questo caso sono le lampade spettrali, alcuni di queste probabilmente le abbiamo adoperate, quindi le gassarare fatti in cui viene prodotto una scarica e quindi a seguito della diseccitazione del litato mi viene messa della luce nemo cromatica, caratteristica di quella sostanza. Anche la luce solare, in principio un esempio di spettro che potreste andare a misurare, il problema è che non entra luce il laboratorio, già ho visto di cosa sarebbe bella perché si vanno a vedere dei rivi di fra un'over però una

misura è purtroppo il laboratorio non si può fare. Questo ad esempio è un tipico spettro di una lampada in condescenza, cosa stiamo vedendo vedete la lunghezza donna è l'intensità, in questo caso viene data un'intensità relativa rispetto a un valore massimo che corrisponde al 100\%, quindi non è espresse l'unità assolute però è importante per capire le diverse componenti cromatiche come giocano all'interno dello spettro. Vedete qui una fascia nera da un certo valore di lunghezza donna in poi, corrisponde alla parte dell'infrarosso quindi più legata al calore però il sensore che viene adoperato permette anche di misurare questa porzione di spettro. Chiaramente dal punto di vista vissivo ci interessa più la parte dei colori e quindi ad esempio questo è un tipico esempio di luce calda perché vedete la componente fredda è parecchio attenuata rispetto alla parte calda. Chiaramente un spettro continuo non essendo presenti ai veri propri picchi si possono fare delle analisi del tipo su di vedere lo spettro in due regioni definirla magari come regione fredda, regione calda, cercare di capire la percentuale di spettro in una data regione per classificare il tipo insurgente. Più interessante mi va a vedere spettri a riga quindi spettri che esprimono le caratteristiche della sorgente quindi questo ad esempio una lampa da popoli di mercurio dove si trovano delle rivi caratteristiche si possono andare a confrontare anche con ciò che riportate il letteratura. Oltre a ciò che si osserva in emissione, quindi andate a posizionare il vostro spettro fotometro e andate a ricevere della luce emessa da una sorgente, si può andare anche a affrontare un'altra problematica cioè l'assorbimento quindi che cosa viene a mancare quando io interpongo qualcosa nel mezzo tra la mia fibra ottica e l'assorgente che un po' quello che si studia ad esempio per l'astronomia per andare a capire la composizione delle stelle. E tutt'è reddè qualche misura in assorbimento, un tempo ne facevamo anche diverse con delle soluzioni di sostanze dimiche poi non capito che era una cosa un po' complicata da gestire il laboratorio quindi ci limiteremo l'utilizzo di filtri colorati quindi l'effetto del filtro è ovviamente quello di attenuare una determinata componente cromatica e farne passare dell'altra, quindi ad esempio un filtro rosso fa passare la luce rossa e dovrebbe attenuare tutte le altre componenti cromatiche. Cosa abbiamo adoperato poi finora il laboratorio per misurare degli spettri? Tipicamente degli spettroscopi o al reticolo di diffrazione o a prisma, penso più a reticolo, a tutte e due le abbiamo adoperati, ok perfetto. In quel caso capite l'apparato sperimentale è puramente meccanico, non c'è nulla di elettronico, la precisione di questa tecnica sta avvenito sostanzialmente dal goniometro, quindi alla precisione con cui andrà da misurare l'angolo, il corso che è il quale andrà da misurare una ben precisa riga e mi dico che questa strumentazione ovviamente è molto precisa, cioè mi permette di stabilire dall'ungezza donna di una riga con una ottima precisione. Qual è lo subantaggio? Che chiaramente è una misura manuale, quindi voi dovete percorrere diversi gradi fino a quando non incontrate la riga e appuntate manualmente il valore e inoltre non abbiamo alcuna informazione sull'intensità diminosa di quella riga. Pentre lo spettro fotometro vi permette con una sola misura di andare a misurare tutto lo spettro della l'ungezza donna, anche la dove non è presente lucidia di principio, e darvi un'idea delle intensità relative dello spettro in corrispondenza delle diverse lumezza seconda, quindi un'informazione più completa a scapito però della precisione, quindi ad esempio nelle righe spettramoli. Abbiamo una risoluzione che chiaramente non può competere con quella di uno spettroscopio al reticolo o a prisma, quindi lo spettro fotometro permette di misurare l'intensità a diverse lumezza donna, ma ha una risoluzione più scarsa rispetto a quello dello spettroscopio. Come fatto all'interno lo spettro fotometro, che vedrete con una scatoletta chiusa. Innanzitutto abbiamo una fibra ottica, che già vi dico da adesso, maneggiate con cura perché la fibra ottica si può curvare in nata per questa, ma c'è anche il produttore, un angolo massimo di curvatura, oltre a cui si rompe, quindi evitate curvature eccessive. Questa fibra ottica vi servirà proprio per guidare la luce messa dalla sola luce all'interno del dispositivo, quindi sostanzialmente l'estremità libera verrà puntata verso la corrente luminosa. La luce potrà entrare all'interno dello spettro fotometro e c'è una serie di specchi collimatori, vedete qui la luce entra in tutte le sue componenti cromatiche, viene riflessa da questo specchio che ha la funzione di collimare la luce verso questo reticolo di diffrazione, quindi alla fine il principio è 

sempre lo stesso, c'è sempre un reticolo all'interno. La luce a questo punto viene riflessa e diffratta in tutte le sue componenti cromatiche verso uno specchio anche qui focalizzatore e uno specchio fa incidere la radiazione sul sensore. Il sensore che c'è all'interno è un CCD, un charge de caplet de l'ice che è lo stesso sensore che vi è ritrovato sostanzialmente nelle fotocamera e dei cellulari, un sensore di luce. Ne parleremo un po' quando parleremo del rivelatore e semiconduttore. Comunque dovete immaginare che questo rivelatore è un rivelatore a pixel, quindi ha tanti quadratini, ovvio uno verrà investito da una radiazione di colore diverso, perché proprio geometricamente vedete la luce rosse incide in questa estremità del sensore, la luce gru in corrispondenza dell'altra estremità. Quindi ogni pixel sarà sensibile in base alla sua posizione e alla sua larghezza, a una porzione di lunghezza e donna. Capite che la risoluzione a questo punto del sensore, tutto l'apparato, la risoluzione interna di lunghezza e donna è stabilita da quanti pixel abbiamo e delle loro dimensioni. Quindi più piccoli sono i pixel, più pixel abbiamo, ma già la risoluzione del vostro apparato è sperimentale. Nello specifico quello che vi aperete in laboratorio ha una risoluzione di circa mezzonanometro di che vuol dire che insomma questo mezzonanometro sarà stabilito dai pixel di questo sensore, il suo CCD. Se avessimo avuto un sensore più performante con una maglia di pixel, di dimensioni più piccole, questa risoluzione potrebbe essere anche più piccola. Abbiamo questo utilizziamo questo. Dal CCD ovviamente esce fuori esce un segnale, questo segnale viene elaborato anche in questo caso da un ADC, ovviamente voi non vedete nulla e tutto dentro la scatoletta, e alla fine abbiamo un software di acquisizione che vi permette proprio di produrre degli spettri di questo tipo. Quindi vi viene già la volata informazione, vi viene formita la lunghezza d'onda e l'intensità. Guardiamo un po' molto velocemente le sorgenti che abbiamo a disposizione. Per spettere continui vi dicevo l'ampada in caddescienza dei LED, anche se i LED sembra monocromatici in realtà sono delle sorgenti di luce, possiamo dire continua, ma c'è comunque interessa in una porzione di lunghezza d'onda non trascurabile. Faretti all'oggi l'ampa dei flore scenti che sono quelle del soffitto. Anche le fibre Web and Shifter che abbiamo visto l'altra volta allezione, le possiamo illuminare con una luce qualsiasi e la luce che viene emessa dalla fibra può essere misurata. Per andare a vedere se l'ospettro che è andato a misurare normalmente sul verde corrisponde con lo spettro che viene fornito dal costruttore. Spettri di emissione righe in particolari lampade spettrali, lampade miscelate al vapore di mercurio, un piccolo montatore laser, quindi vedrete un po' delle sorgenti monocromatiche. Questo spettro di assorbimenti di soluzione non lo facciamo, facciamo soltanto, vi dicevo, i filtri. Qui vediamo una carrella di immagini, questo ad esempio una lampada miscelata al vapore di mercurio. Quindi vedete è una sorta di spettro intermedio di spettro arricche e spettro continuo perché abbiamo una base continuo dove si vedono dei picchi caratteristici degli elementi presenti in questa lampada. La lampada interdessenza l'abbiamo vista anche prima, che è un classico esempio di spettro continuo. Un'infaretta l'oggiere è un altro spettro continuo che si può analizzare anche diverse tensioni di lavoro. Chiaramente la parte più interessante vi dicevo riguarda le lampade spettrali, ne abbiamo a disposizione diverse e quindi possiamo ad esempio fare una misura dello spettro di una lampade idrogeno e andare a confrontare quello che ottenete sperimentalmente con la serie di Balmer che abbiamo studiato, in struttura sì, sì, ok, comunque quando farete la tesina certamente l'avrete studiata. Chiaramente lampade più complessi, gli elementi più complessi sono un po' più difficili da verificare con dei conti o con delle formule, non esistono delle formule, ma trovate sicuramente dei valori tabulati e letteratura. Questo ad esempio è una, questo è una lampada spianéon, lampade spettrali non ne ho qui, comunque ci sono lampade spettrali che hanno spettri anche molto complessi con diverse righe. Vedete gli spettri LED, non sono per niente monocromatici, guardate la riviezza ad esempio di un LED rosso e di un LED giallo, quindi quello che ha, che sembra un colore definito in realtà non è per nulla, monocromatico. Questo è il tipico spettro di emissione della fibra Webless shifter, quella che abbia fatto vedere la volta scorsa che mette tipicamente nel verde, verde e blu, quindi questo è quello che si ottiene. Questa parte vi diceva che non si fa più l'assorbimento attraverso delle soluzioni chimiche, ma farete certamente la parte di assorbimento attraverso un filtro, quindi prendete una sorgente di riferimento che può essere ad esempio il faretta l'ogge, non fate una misura di questa luce e poi fate una misura interponendo il filtro. Quello che si osserva è ad esempio che una parte della componente cromatica viene totalmente soppressa come questo caso e rimane la componente del relativo al colore del filtro, quindi in questo caso è un filtro rosso e quindi quello che viene fatto passare soprattutto alla lunghezza d'ondo del rosso. Chiaramente il filtro è tanto più professionale quanto più filtra. Questi ovviamente sono le filtri didattici, dei filtri professionali, hanno delle bande passanti, lo stiamo a dire molto più selezionate. Guardate ad esempio questo il filtro verde, ma il riferimento è un po' di verde 

questa, ma oltre al verde passa anche un faretto rosso e un faretto rosso, quindi certamente non è un filtro professionale questo. L'ultima esperienza non ha una presentazione e quella diciamo dei Geiger che non è esperienza ad esame, avrete a disposizione questa scheda di attività, la scorriamo molto velocemente. La prima parte è una parte del tutto generica introduttiva sul riferatore Geiger che già abbiamo discusso insieme, comunque la possiamo leggere per ricordarvi alcune cose. A questo punto viene presentato il contatore come l'abbiamo visto la volta scorsa, ne avrete a disposizione per ogni gruppo tre contatori, quindi ovviamente siete in quattro per lo più, ma magari c'è una persona che prende appunti, prende le misure degli altri che misurano. Vedete che questa scheda di attività vi presenta anche la distribuzione di quassone e vi chiede di fare delle attività intermedie, ad esempio sua distribuzione di quassone vi chiede di valutare il valore predetto della distribuzione per un valore o meno di due e cinque, oppure di riportarlo graficamente, oppure realizzare delle misure brevi con il contatore Geiger, riportarle qui, valutare in certezza soluta, in certezza relativa, poi effettuare delle misure più lunghe e ogni qualmolta appunto vi viene chiesto anche di riflettere sui risultati che abbiamo ottenuto. Comunque al momento non si prende a discutere durante il turno. Quindi una scheda che prevede diverse attività e che alla fine ha come scopo quella di verificare che le misure che abbiamo effettuato il laboratorio seguono effettivamente la distribuzione di quassone. Quindi da un lato dovete calcolare il ciò che è predetto dalla distribuzione di quassone, dall'altro dovete fare delle misure e quindi avrete a disposizione questa scheda. L'ultima cosa che vi volevo mostrare oggi si ricollega agli spettri $\gamma$. Vi ho detto che lo spettro $\gamma$ ha una forma abbastanza caratteristica. Io voglio far vedere come lo spettro $\gamma$ che è misurato il laboratorio è caratteristico del vostro apparato sperimentale. Addirittura abbiamo detto che se avessimo rivelatore di dimensioni infinite lo spettro $\gamma$ sarebbe semplicissimo, avremmo solamente il picco fotoelettrico. Quindi capite che la geometria del rivelatore incide tantissimo su la percentuale di eventi fotoelettrici e di eventi con ton. Vediamo in che modo. Chiaramente questo aspetto si può studiare sperimentalmente proprio utilizzando rivelatore di forme diverse oppure si può simulare. In particolare questi risultati che vi faccio vedere sono un affutto di una simulazione, ma una simulazione sempre proprio per questo. Dal momento che non ho rivelatore di qualsiasi dimensione posso provare a simulare e vedere l'effetto, cosa attendermi. Immaginate di simulare quindi un rivelatore come quello che ha operato il laboratorio, io dico di sodio d'orga taltaglio, due pollici per due pollici, un cilindro di due pollici per due pollici, questo è la sorgente di cesio 137 e inizialmente adopero questa configurazione per cui la distanza tra la sorgente di cesio e il centro del mio rivelatore è di 3 cm. Immagino di avere una certa risoluzione, di ripostruire l'energia, questo è un altro dato che bisogna passare agente, in qualche modo sporcare il risultato per tener conto del reale funzionamento di un rivelatore e vedete ad esempio gli effetti delle dimensioni del rivelatore, questo è quello che ci si aspetta quando il rivelatore ha una lunghezza di 2,54 cm. Vedete questa è la parte relativa al Comton, al continuo Comton questo è l'effetto fotoelettrico e vedete possiamo dire l'effetto fotoelettrico insomma si verifica la maggior parte dei casi quindi nella maggior parte dei casi riusciamo a ripostruire quasi per intero l'energia del fotone incidente, cosa cambia se modifico la lunghezza del rivelatore ad esempio la dimesso, anziché 2,54 la faccio di 1,27 che torna indietro per vedere l'effetto e chiaramente succede quello che ci aspettiamo ma un rivelatore diventa sempre più piccolo e è probabile che parte dell'energia men la persa perché magari qualche prodotto secondario di fotone diffuso può fuori uscire e infatti la percentuale di eventi Comton incomincia a crescere rispetto agli eventi del fotoelettrico questo si accendo ancora di più se si riduce nuovamente la lunghezza del rivelatore di quel cilindro e vedete come ad esempio per 0,254 quindi un decimo della dimensione di partenza a questo punto interviene una forte componente conto non rispetto al fotoelettrico veniamo qui a confronto appunto i tre casi questo era quello di iniziare 2 pollici per 2 pollici e man mano un dimezamento della lunghezza del rivelatore è un passore 10 la risoluzione immaginiamo l'un per un certo di risoluzione vedete come viene ricostruito molto bene il picco fotoelettrico molto stretto si ottiene sempre lo stesso valoro quasi se peggioriamo la risoluzione effetto chiaramente quello di osservare un picco via via più largo 10 per 100 20 per 100 e capito che questo diventa un problema quando ad esempio come nel caso della sorgente del combattuto dove vengono emessi due $\gamma$ con valori di energia abbastanza vicini tra di loro allora in quel caso i due picchi possono arrivare anche a confondersi si sovrappongono certamente di troverete 

sovrapposti perché noi abbiamo un 10\% di risoluzione in grosso modo ma ancora di più se la risoluzione fosse peggiora potrebbe ad avvertura non essere più distinti l'uno dall'altro la direzione dei $\gamma$ i $\gamma$ vengono emessi in tutte le direzioni della sorgente cosa comporta qualsiasi interme di rivelazione a un discorso che i $\gamma$ incidono ortogonalmente la superficie del rivelatore un altro discorso che magari incidono in maniera inclinata quindi promieni da direzioni inclinate quello che succede che se lo spessore di rivelatore è attraversato è piccolo cosa che avviene quando la direzione è molto inclinata magari toccata un angolo del rivelatore non vi detto che il fotone incida interagisca ok perché il fotone ha una certa probabilità di interazione chiaramente più è lungo il percorso che effettuano interno del rivelatore maggiore sarà la probabilità che questo interagisca se invece rivelatore viene toccato solamente nel bordo è chiaro che potrebbe capitare che il fotone non riesce a interangire il $\gamma$ interagisca quindi non venga misurato che cosa cambia questo tambien termini efficienza l'efficienza se vi ricordate rappresenta il numero di particelle rivelate su un numero di particelle incidente dipende cosa si va a vedere se efficienza è in trinsega o efficienza complessiva comunque l'efficienza in trinsega in particolare rappresenta il numero di particelle rivelate su quelle che incidono ok quindi la particella è inciso la toccata il rivelatore ma qual è la probabilità che produga un segnale e quindi media una rivelazione completa allora dipende molto anche dalla direzione dei $\gamma$ se per assurdo vessimo $\gamma$ provenienti tutti quanti della stessa direzione in gilenti perpendicularmente rivelatore efficienza complessiva sarebbe del 75\% invece con una distribuzione di solo troppa quindi tutte le direzioni dello spazio vedete come l'efficienza si ammassa naturalmente e questo ovviamente uno svantaggio per l'esperimentatore perché per accumulare statistica sono richiesse dei tempi ovviamente più lunghi bene questo era un corollario possiamo dire la parte degli spettri che avevamo discorso in precedenza quindi le tre esperienze che vi ho detto sono esperienze di same spettrometria $\gamma$ assorbimento dei beta e spettrofotometra digitale e poi c'è l'esercitazione sui geiger allora per regolarvi comunque ora ve lo mando questo specchietto e lunedì inizierà il gruppo 1 gruppo 1 gruppo 2 gruppo 3 gruppo 4 esatto esatto delle 11 quindi i primi 4 gruppi sono questi forse qualcuno già l'ha fatto presente che ha la difficoltà per lunedì prossimo del caso appunto vi ho detto provate a trovare qualche scambio in grandisco allora adesso io procederò con la consegna dei kit perché a me rimandare in circa la ventina di persona che deveva dire a kit invece che aveva il kit multimetro può portarlo qui e lasciarlo qui ovviamente se ci sono dei problemi me lo segnalate la prossima consegna aspetto ve lo dico quando verrà allora per i kit ulti vedrò la prossima consegna la facciamo giorno 29 quindi la prossima settimana però il laboratorio ok quindi dovete venire alle 11, le lezione prima, le lezione 11 e la prossima con arduino chi deve andare può andare ovviamente si si allora c'è stata una domanda scusate ragazzi nel caso dello spettrofotometro in oltre all'immagine che viene per me c'è un modo per stare arredato e si vi farò vedere che lo spettrofotometro produce tre file due sono delle immagini una come quella che vi possiamo vedere è un'altra è un'immagine proprio del sessibili dominato e il terzo file è un file testo che vi riporta proprio i dati quindi per ogni lunghezza donna si intensita la mesurata quindi poi apposteriò ovviamente insieme ad analisi si può lavorare su questi dati dovete lavorare su questi dati allora quindi mi dicevo un edifico di prossimo consegno i prossimi kit ultimetro volete già sapere chi deve venire forse meglio perché sono venite tutti poi non lo vado tutti ovviamente allora quindi cominciò il nodi alfabetico francesca lì può venire non c'è arena santo si può venire lo ha disegno ok alisia attuglio non ti sento scusa aiarduino ok va bene allora rimandiamo la rimandiamo la passiamo l'auto e barbita già l'ho avuto luca buon anno non c'è fabbrizio buon coraglio non c'è gabriele bruno ti posso segnare poi luca garbona posso segnarti ok cerruto simona aiarduino ok andiamo avanti compagnini domenico andorno di bella posso segnare ok toscati piaci non vedo giordana di fede andrea di fini andrea d'unzo ok ti posso segnare per la prossima volta per il 29 sei tu si ok francesco fiorino si professoressa si dimmi si posso segnarti ok va bene salvatore garofi ok devi prender tu dito di letto e gincoli arduino emmanuele della ferla posso segnarti martina rossa ok lucia lattuga posso segnarti giù e alaudani di mandatuino alessandro li pani li pani scusate non mi ricordo mai e la cazzo guarda la faccia la no non mi ha mi ok niente e alfio lo castro ademri il cardino secondo me non ti consulta in 30 secondi non potrei ed e guarda lo dici francesca lo che però ha preso ha preso il ruino si giusto e antonio macchione ah no l'ha avuto già e marco micci cani si professoresso posso segnarti ok giorgio migliora deve si prender al ruino e andrea mia bella grazie grazie a non moravico posso segnarti ok alessia mosumeci aggacelai scusa era gian e giallucci giallucci bacino posso segnare mattheo marisi e ricapisci vello gioelle proietto gian arugusa che fossera samisante io si posso venire a giorno 29 e donata ragnolo deve prender duino carmen al spagnola non c'è simone rocca ti posso segnare ok salvo sant'angelo posso venire ok simona scalora disegno e francesco scarcevino ok saras crudo per estressa ancora me manca arduino ok si ok però quindi devi venire a ritirare anche arduino si quindi a me quando puoi poter rarduino e considera che ora questo questo profumo terraria arduino per almeno una decina di giorni in questo modo quindi verrà riconsegnato non questo giovetti ma l'altro e quindi la riconsegna viene tra due settimane ok ok giorgio sessa prenderà a due i no giuglia spina ha c'è la risposta la ragione e dario ballone prenderà a due i no so ce la chiara zisa ok va bene il massimo abbiamo intorno che sono risciotto però la chi è stato segnato che è la prossima lunettiglia non è necessario che prenda la corsa ah ok si forse la provo segnato è una mia collega che doveva portarle oggi testa ma non si segnate bene si deve consegnare chi deve consegnare anche chi arduino quindi chi i testa negli auguri ne devo consegnerli oggi e chi ce l'ho si chi non ce l'ha me lo consegna la che si segnata la corsa e il giorno domani per il binario metteteli tutti qua per il diodo si si ma lo sai che si prega per la corsa lo posso mancare controllo di ccia e di chi ha risciletto si, la la si grazie mille, qui te lasciate qui allora si 40 otto le listane si, non penso di aver veduto ah va bene si sta rantendo si preoccupa, 40-50 la non fa di ferma si, riguarda più si, possiamo lasciarla qua si, non si non Alicia a curio si ciao si, non si non si fatto un'imprensione con lei però non lo so si, non so che mi sia capito non so, non so va bene, allora, lo fai va bene, ok se lo trovo, posso venire a dare la mano si, si, si, si, si, si, un'ora, è un'ora e per un'ora, è un'ora è un'ora è va bene, va bene, va bene, ah, va bene ok ha dato benvenno si, si, si ah, va bene ok, Simone, è ciò di mudo? si, si va bene chi ne ha avratto sale? facciamo conto con tre, ragazzi chi ha bisogno? io, Danna Danna, mettiamo quattro ah, ma non c'è benvenuto ha dato saltato va bene, va bene va bene, va bene per il cinque, come è scusato? Garuffo ah, garuffo garuffo per il cinque, poi gente che ha avuto un cinque di diretto ok la vi consegno, se abbiamo tutte quelli ragazzi, che non avrò andato

\textbf{credo sia lez 14 poi vedo dove metterla}