La distribuzione di Poisson è alla base di molti esperimenti di rivelazione ed è utilizzata spesso in fisica spesso. Vediamo di farne un breve ripasso.

\section{Distribuzioni di probabilità di interesse in fisica}

In fisica le distribuzioni più importanti sono:

\begin{itemize}
   \item Distribuzione binomiale;
   \item Distribuzione di Poisson;
   \item Distribuzione di Gauss.
\end{itemize}

Esse sono delle distribuzioni limite, cioè sono delle distribuzioni verso cui si tende quando si aumenta verso $\infty$ il numero di misure. A seconda dell'evento che si studia può valere una distribuzione o un'altra.

\subsection{Distribuzione binomiale}

La distribuzione binomiale si applica quando effettuiamo un certo numero $n$ di prove indipendenti che possono dar luogo o a un successo oppure ad un insuccesso. Il successo viene valutato con una probabilità $p$, l'insuccesso con una probabilità $q=1-p$. 

Secondo tale distribuzione, la probabilità di ottenere $\nu$ successi su $n$ tentativi è data da
\begin{equation*}
   P_n(\nu)
   =\frac{n!}{\nu!(n - \nu)!}p^\nu q^{n - \nu}=
   \begin{pmatrix}
      n\\
      \nu
   \end{pmatrix}
   p^\nu q^{n - \nu}
\end{equation*}

Facciamo subito un esempio numerico:

\begin{esempio}[La probabilità di fare tre volte 1 a dadi con tre tiri]
   Supponiamo di lanciare tre volte un dado, dunque il numero di tentativi sarà $n=3$ e supponiamo che il successo consista nell'ottenere la faccia con il numero 1.\\
   \E chiaro che, poiché il dado ha sei facce, la probabilità di successo sarà $p=\frac{1}{6}$, mentre quella di insuccesso sarà $q=1-\frac{1}{6}=\frac{5}{6}$. Vediamo allora come varia la probabilità di ottenere un certo numero di successi $\nu$, il cui valore è riportato nella seguente tabella:
   \begin{center}
      \begin{tabular}{|l|l|l|}
         \hline
         &&\\[-0.4cm]
         $\nu$ & $P(\nu)$ & $P(\nu)$ in \%\\[0.1cm]
         \hline
         &&\\[-0.4cm]
         0 & $\frac{5}{6} \cdot \frac{5}{6} \cdot \frac{5}{6}=0.579$ & $57.9\%$\\[0.2cm]
         1 & $3 \cdot \frac{1}{6} \cdot \qty(\frac{5}{6})^2=0.347$ & $34.7\%$\\[0.2cm]
         2 & $3 \cdot \frac{5}{6} \cdot \qty(\frac{1}{6})^2=0.069$ & $6.9\%$\\[0.2cm]
         3 & $\frac{1}{6} \cdot \frac{1}{6} \cdot \frac{1}{6}=0.005$ & $0.5\%$\\[0.2cm]
         \hline
      \end{tabular}
   \end{center}
   Notiamo che, nei casi in cui otteniamo una o due volte la faccia col numero 1, ciò che cambia rispetto agli altri casi è che si possono avere diverse configurazioni in base a qual è il lancio in cui si ha successo.
\end{esempio}

La distribuzione binomiale è una distribuzione discreta, quindi si può calcolare soltanto per variabili discrete, cioè $\nu$ può assumere solo valori discreti. Inoltre è chiaro che $\nu$ non può superare il numero di prove $n$. 

questo numero di posturazioni sono tanto valori, ma in questo caso è numero 0 poi nel caso specifico la distribuzione di non-iala viene limitata per il numero di prove, quindi non può superare il numero al numero di altri possiamo applicare queste istituzioni di non-iala non solo al gioco delle caratteri, la volga, la mangiata, la mangiata, le cari ma anche ad esempio in visica, ad esempio eseguiti una sargenza radiative che decade in modo naturale e mettendo la reazione isotropicamente quindi con la stessa probabilità in tutte le direzioni dello spazio possiamo calcolare la resta distribuzione minimiale qual è la probabilità di avere i miei particeli che ne vengono messe, ad esempio nel sfero in avanti quindi immaginatevi di dividere l'ambuloso olivo in due sferi immaginatevi che ne vengono messe in un certo numero e in un certo tempo lo distrugano a un certo numero di particeli, sopondiamo un milione e allora vi domandate su un milione di particele che ne vengono messe poi nella probabilità di osservare 500 mila particele ne messe in avanti oppure 400 mila o 800 mila, poi si vada, in realtà si avviva l'distribuzione minimiale capite che a legge la cosa di disposizione di un conto è un po' antibrato soprattutto quando si fa con dei grandi uneri per capire per il campo di coefficienti di nociare non venga a gerole perché a legge unente fattoriale deve essere sempre il riconthico un milione di emissioni per se calcolare è un milione fattoriale e capite che possesiva poi è per se non fa in una stessa posizione quindi non sempre è andebolo e poter applicare 500 mila in casi del milanere per le altre proprietà, se vi ricordate era possibile calcolare il valore ognuno delle istituzioni come quello che riesce a percire e la parianza come ende tuttavia per fortuna di lei in alcuni casi è possibile attivare l'distribuzione minimiale con una distribuzione diversa ad esempio quando andate a considerare le elencherà, le elencherianà come la vita è successa molto bassa di P e le zero è un numero che trova molto elevato ma dovrebbe possibile andare a posizionare l'distribuzione minimiale poi la destituzione di questo quindi è un caso particolare e semplicemente di lavorare una forma ricettamente piagghevole di quella di destituzione minimale e a l'intura in alcuni casi quando in uno dirò molto grande questa distribuzione possa essere passimata all'distribuzione diversa che andrà più conveniente per alcuni tipi di conti le istituzioni possono vivere quando è spessa da questa allevazione e dipende da un unico parametro questa è una caratteristia per la destituzione di possone che differisce absolverà la destituzione della sua percente sua dipende da un unico parametri in sicumale del valore medio Qui invece la destituzione di possone dipende unicamente del valore di questo parametro nuo che rappresenta il valore medio della destituzione questa destituzione di probabilità ci formisce autres, la probabilità di osservare i miei elementi quando in media se ne hanno un nuo quindi supponete se anche gli avere con le palette possimamente di vere disposizione di un contador in tair un ricordatore, un ricordatore con le tore che per il suo proposto all'attenzione naturale misura in media trenta di scelre ogni 10 secondi ovviamente questo è un calore medio quindi è il tuo calore a dire miù, miù, mello, mello, mello e allora io vi posso dire quale è la probabilità di misurare 4 o 40 scelre oppure di misurare ne 2 e di misurare ne 0 chiaramente questa probabilità ne è comoda cioè noi in media otteniamo 3 ma a causa delle situazioni statistiche a volte le ottenete di più o a volte di 0 e quindi la probabilità di ottenere un numero diverso da un dolore meglio può essere dal colore al corso, solo questa distribuzione può sonne capite che le probabilità più elevate si hanno in consistenza del dolore medio quindi le probabilità molto comode di imparciare 3 scelre 4 o 2 diventerà più improbabile di misurare ne 20 quando il dolore ne è di 3 non capita che lo vuole non sanno in cera del momento o che faccia quantificare un processo questa formula e noi diremo la responsabilità che dicevo questo colore di carriere eseguire una schere d'attività che interventerà di intervincare spermentamente questa distribuzione ma questa distribuzione non sia anche la solamente al caso a punto d'eternal di tibia quindi la risultata è che la distribuzione è distribuzione ma anche attentissima stile della protezione ad esempio mai le chiamate al cellulare stanno diventando sempre meno frequenti perché abbiamo altri mezzi di comunicazione passati sulle messe gs e stantanea quindi la telefonata ormai diventa sempre più una parata addirittura se pensate la telefonata a casa o se penso che non lo trovate la città del reggione, non so guarda neanche nello casa come lo più mente per dire quante diventano le unità usuale questa quindi, famite che in un caso tipico al cui possiamo abitare la distribuzione di passone non è sempre plantare quando le chiamate ricevono al cellulare il 

metipatore andate a partire con il suo numero, veramente cambiare il giorno e il giorno possiamo vantarne una media e possiamo andare per il tuo mare se il suo numero è chiamato il metipatore si distribuisce secondo una distribuzione di passone camminere non è valido in artistra a spenta in cacantere ma a fare le trate per il nuovo finanzatore di cacantine quindi bisogna chiamare e molto suolte delle tante e se ci invitiamo tutto a venire in tirari è possibile abitare la distribuzione di passone oppure il numero di nasci del giornale di uno spedale camminere comunque si è ovviamente abbastanza rare rispetto a tante nasci che si hanno al verno mondiale in un loro e se ci invitiamo a ciocchiare un spedale veramente sono un periodo di abbastanza rare quindi anche in questo caso potremmo applicare la distribuzione di passone oppure immaginate di andare a raccontare un globo di rossi che si osservano a un microscopio in un docenzano di seguito di un obiettivo e che dovrà farlo a quanta di globo di rossi anche in questo caso di una mantigna dove vengono del misterale con lo rispetto a un numero possibile che potreste meravigliore di invece un attocchiante perché la distribuzione di passone si può applicare anche a un caso di diseggamento di radiativi perché anche se considerate codissime quantità di materiali di attività che li incroppiamo in termini all'or interno sono presenti il numero di un plate enorme dell'orbine del numero di alocando come 10 alla 23 ma un altro colpso di carri di vantimento è sempre molto piccola quindi in mezzo di sempre lo demonstrerò mille goglie su 10 alla 23 vende tanto e quindi questo si può applicare successo per ammetteri e sicuramente si può applicare la distribuzione di passone vi ricordo la forma delle soluzioni di passone che cambiano in base al governo queste sono tre punti ottenuti per tre vantimenti diversi che sono fatte per 10 e vedete effettivamente le asimmetrici e la mano che aumenta in un olmeio è possibile andare a possimare la distribuzione di passone con la distribuzione di passone diventa sempre la distribuzione di passone che è un po' quello che abbiamo fatto quando abbiamo parlato di diseggazione in energia abbiamo detto la risponsione di energia anzi anche l'uricola allora si divende la parte coppia, elettrone, ilionale, sicuramente quello ma questo numero segue la distribuzione di passone e può essere possimato anche con la distribuzione di passone quindi abbiamo fatto tutte le due appassimazioni e dove si tratta la vergrenza medanzitica e binandossiana e scorta la varianza delle distribuzioni di passone per vantare alla distribuzione di energia quindi vedetevi alla base di tanti discorsi che faremo che abbiamo fatto e che faremo e tante aspettiche affrontate di laboratorio e allora conosce la distribuzione di passone standardiale quindi ho scottato questi 10 minuti di cima e ne vevo per riclamare questi concetti dove resta vero che maniera utilizzato una volta in questo televolo e in queste ultime riclamazioni e più allora ragazzi, io ho l'opera un distinto quindi ve lo provo ad unere semplicemente con se andiamo in chidzo e chi certamente non deve ricevere per il problema

\begin{esempio}[Distribuzione dei tempi di arrivo]
   $\mu=\lambda t$
   \begin{equation*}
      P_{\mu}(\nu)
      =e^{-\lambda t} \frac{(\lambda t)^{\nu}}{\nu!}
   \end{equation*}
   \begin{equation*}
      P_1
      =P(t_1 \leq t)
      =1 - P(t_1 > t)
      =1 - P_{\mu}(0)
      =1 - e^{-\lambda t}
   \end{equation*}
   \begin{equation*}
      \dv{P_1}{t}=\lambda e^{-\lambda t}
   \end{equation*}
   \begin{equation*}
      P_2
      =P(t_2 \leq t)
      =1 - P(t_2 > t)
      =1 - \bigl[ P_{\mu}(0) + P_{\mu}(1) \bigr]
      =1 - e^{-\lambda t} - e^{-\lambda t}\lambda t
   \end{equation*}
   \begin{equation*}
      \dv{P_2}{t}=\lambda^2 e^{-\lambda t}
   \end{equation*}
   \begin{equation*}
      \dv{P}{t}=\frac{\lambda^n t^{n - 1}}{(n - 1)!} e^{-\lambda t}
   \end{equation*}
\end{esempio}