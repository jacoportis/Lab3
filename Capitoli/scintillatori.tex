bene, quindi man mano che state inserendo cercate di completarmi prima possibile così iniziamo e vi comunico i turni allora prima di cambiare argomento sui rivelatori a gas abbiamo domande su quello che abbiamo visto oggi anche da foglio ovviamente non mi sembra, ok allora andiamo a guardare adesso un'altra tipologia di rivelatore i consideri scintillatori, ve lo nominate più volte in queste lezioni ora vedremo esattamente che cosa consiste e vi farò vedere anche dal vivo qualche esente di scintillatori ho posto un po' di materiale allora questi materiali sono dei materiali che mettono delle luce infatti si chiama luce di scintillazione quando vengono attraversati da una radiazione quindi in questo caso ciò che produce un segnale è la raccolta di questa luce quindi nel rivelatore a gas abbiamo visto raccolta di carica dovuta la rionizzazione del gas adesso invece abbiamo produzione di luce i fotoni che vengono emessi in questi materiali non sono tantissimi o comunque quelli che riusciamo a raccogliere sono veramente pochi e sono tipicamente dei fotoni che vengono emessi nel ultravioletto grosso modo come reggione dello spettro un V violetto siamo all'unite del visibile la rapporta di questo segnale e luce ci può dare informazioni sulle particelle ad esempio più sull'energia che sarà depositata da una particella ma anche sul tipo di particella molte esistono diversi tipi di scintillatore inoltre sono dei materiali che possono essere lavorati in diverse forme ad esempio in questa foto vedete diverse scintillatori con forme molto differenti dal momento che ci sono diverse tecniche per la produzione di queste scintillatori anche tecniche di estrusione che vuol dire che si creano degli stampi che si rientri di questa sostanza ad alta temperatura in forma liquida poi viene fatta per affreddare e si si quindi lo scintillatore di forma desiderata quali sono i vantaggi degli scintillatori? anzi tutto devo dire che storicamente anche se si conoscerano queste proprietà di questi materiali non furono subito applicati nel tanto della rivelazione come un presero particolare piede venivano sempre utilizzati in rivelazione agas ma per un semplice fatto una semplice questione pratica questa luce di scintillazione che vi dicevo la luce ha abbastanza fiocca che è impercettibile allo più mano veniva inizialmente visualizzata in ambienti scuri attraverso microscopio quindi era una visualizzazione manuale possiamo dire non era per niente comoda quindi era utilizzata più a scopi dimostrativi che a scopi di vere e propria misure immaginate se abbiamo delle misure di particelle che arrivano con una certa frequenza che veramente 

mettersi di a guardarle a occhio lungo non era una cosa pratica quindi inizialmente non presero piede non furono utilizzati molto non appena si svilupparono i primi fotosensori elettronici quindi degli strumenti in grado di andare a misurare della luce anche molto debole e trasformarla in un segnale elettrico allora al proprio punto che scintillatori divennero preso il sopravvento divennero i riviatori più adoperati non so che c'e ne sono di diverse forme ma c'e ne sono di diversi tipi e ogni tipo di scintillatore ha dei vantaggi e dei disvantaggi quindi diciamo che non esiste lo scintillatore ideale dipende dal tipo di applicazione che va vedimento in generale possiamo dire che ci sono diversi vantaggi in anzitutto abbiamo una risposta abbastanza lineare in energia cioè il numero di fotoni che viene prodotto e abbastanza proporzionale all'energia che viene depositata nel materiale e questo è utile perché ci permette di capire quanto energia ha lasciato depositato la particella poi un altro vantaggio e che molti di questi materiali hanno tempi di risposta veloci quindi questa luce viene messa e viene messa anche in tempi abbastanza rapidi quindi se abbiamo poi un'elettronica quindi un fotosensore e un'elettronica che permettono anche di produrre un segnale veloce allora a quel punto capiamo che sono rivelatori che possono essere utilizzati per il timing cosa che invece non è quasi mai vero nel caso dei rivelatori a gas perché i rivelatori a gas tipicamente sono rivelatori abbastanza lenti a causa del fatto che i processi che intervengono sono processi che richiedono una raccolata del segnale che può durare anche centinaia di microsecondi comunque lecide di microsecondi quindi non li possiamo definire rivelatori veloci a meno che non andiamo a partivare rivelatori come l'MRPC che abbiamo visto alla fine della presentazione precedente infine l'ultimo vantaggio è che permettono di effettuare anche una discriminazione nel tipo di particella che è inciso quindi in qualche modo posso distinguere tra l'arrivo di una particella $\gamma$ o l'arrivo di una particella carica e questo si fa andando ad analizzare la forma del segnale che viene prodotto e ora andremo un po' più nel dettaglio alla base di uno scintillatore quindi abbiamo la emissione di questo flash di luce che ha una durata veramente breve, potrebbe durare anche pochi nanosecondi e questo flash di luce deriva da transizioni di elettroni quindi da disecitazione di questo materiale ora secondo il tipo di materiale vedremo che queste transizioni avvengono tra livelli diversi daremo qualche dettaglio più specifico però il principio di base si basso nel fatto che questo materiale si disecita in qualche modo e produce della luce questo è il principio in generale di un rivelatore basato sullo scintillatore quindi ogni rivelatore basato su scintillatore prevede ovviamente una parte che è la parte proprio sensibile di rivelazione così di un blocco di scintillatore poi questo blocco di scintillatore deve essere in qualche modo collegato a un fotosensore qui nello specifico in questa animazione vedete un fotosensore che prene il nome di foto moltiplicatore ma che discuteremo la prossima volta quindi non mi concentrate sulla tipologia di photosensore dovete tutto immaginare che ci deve essere poi un sensor in gravo di misurare questa luce e tradurla in un segno allelectrico che poi può essere inviato di nuovo per uno sistema di acquisizione quindi abbiamo sempre due componenti scintillatore e photosensore oggi parleremo della parte sensibile di rivelazione cioè dei scintillatori qua ci sono le proprietà ideali di uno scintillatore allora innanzitutto vorremmo avere un'altra efficienza di scintillazione cosa vuol dire che se una particella è in deposita energia vorrei che questa energia venisse sfruttata il meglio per produrre fotoni di scintillazione quindi vorrei produrre il maggior numero di fotoni in corrispondenza di una data quantità di energia depositata cioè vorrei una resa in luce elevata quindi questa proprietà dello scintillatore prende il nome di resa e luce rappresenta il numero di fotoni che vedono prodotti però in mezzo di energia depositata e questa resa dipende dal tipo di scintillatore quindi ci sono scintillatori che hanno resa più elevata e scintillatori che hanno resa più bassa però magari hanno altri mantangi era quello che vi dicevo prima magari il scintillatore ottimale non si trova però diciamo che una delle caratteristiche che si va a vedere vedere il caso dell'ascintillatore e innanzitutto la resa in luce inoltre vorrei che la resa in luce fosse proporzionale all'energia depositata perché in questo modo il mio scintillatore può funzionare da rivelatore che venissima l'energia della particella depositata quindi questo è un altro aspetto importante da andare a vedere in uno scintillatore altro aspetto che bisogna andare a considerare riguarda lo spettro della luce a messa questa luce io vi ho detto viene messa normalmente negli UV ma non è sempre così se guardate un po' la figura che vi ho fatto vedere all'inizio è vero che queste materiali sono emettere una luce sul violetto ma questi vede te sono già sul verde o addirittura ce ne sono sono sul rosso quindi in realtà ogni materiale ha un caratteristico spettro di emissione e questo spettro ha un'importanza fondamentale perché questa luce deve essere raccolta quindi immaginate di avere un sensore che non è adatto alla luce che viene emessa dello scintillatore allora diventa un sistema inefficente perché non siamo in grado di andare a mesurare questa luce quindi è fondamentale che lo spettro di emissione sia il più possibile vicino a quella che è la finestra di lunghezza e donna cui il mio cotosensore è sensibile quindi bisogna andare a individuare un accoppiamento ottimale in termini di lunghezza e donna altra caratteristica fondamentale è che lo scintillatore deve essere trasparente alla luce che emette cosa vuol dire? emette della luce questa luce non deve essere fiersorbita ovviamente perché sono la perdo e quindi in qualche modo normalmente questo si realizza attraverso il drogaggio dei materiali scintillanti deve evitare questo effetto di riassorbimento della luce emessa questo equivale a dire se io devo andare a vedere le caratteristiche di lo scintillatore equivale a dire che la lundezza di assorbimento deve essere elevata quindi questa luce che viene messa deve percorrere lunghi tragitti prima di essere assorbita al rispetto importante l'abbiamo detto prima è avere una risposta veloce questo non sempre vero non è vero per i tri scintillatori però c'è alcuni che effettivamente hanno una risposta molto veloce e questo ci aiuta non solo per il timing ma anche per avere delle frequenze di contenti elevate perché più è breve il segnale prima il mio rigalto sarebbe pronto per misurare una nuova la lienzione e quindi posso avere anche delle frequenze di contenti elevate e infine l'ultima caratteristica, ma lo capite l'abbiamo visto quali tipi di materiale stiamo considerando sono materiali di questioni di una molto a plexiglass anche come aspetto deve avere un indice di rifrazione molto simile a quello del vetro e questo è legato al fatto che come vedremo siccome il rivelatore è sempre costituito dallo scintillatore più il fotosensore non sempre le dimensioni di questi due gettico incidono quindi non è detto che voi abbiate un cilindro e che questo sensore sia un cilindro delle stesse dimensioni che possiamo semplicemente apostare magari hanno dimensioni diverse, area diverse quindi sorge l'esigenza anche di guidare la luce verso il fotosensore quindi creare una sorta di guida di luce che viene realizzata normalmente in plexiglass e che quindi deve avere un indice di rifrazione simile a quello del scintillatore vi dicevo che ci sono tantissimi tipi di scintillatore, uno di cui presenta una o più di 

queste caratteristiche però se guardiamo in generale le caratteristiche della luce messa per scintillazione possiamo distinguere diverse componenti allora in generale quello che si nota è che questa luce che viene messa diminuisce nel tempo e diminuisce esponenzialmente quindi abbiamo un flash di luce e questa luce, se vado a vedere la sua evoluzione nel tempo ha un andamento esponenziale le crescente e la pennezza di questo esponenziale dipende dalle transizioni che hanno dato luogo a questa luce in generale si possono evidenziare diverse componenti della luce messa abbiamo una componente che tende il nome di componente prompt veloce che è la luce di fluorescenza questa viene messa con tempi caratteristici molto brevi lo vedete qui in questo grafico è quella che ha la pennezza maggiolare quindi vedete se si esaurisce tiver tiver quantità quantità quantità tempo pochino a secondi quindi questa è la componente più veloce e anche quella vedete più importante quella che ha dall'alicotributo mangiore potremmo avere inoltre anche fenomeni di phosphorescenza che edificamente sono dei fenomeni più lenti anche dell'ordine di milli secondi o secondi e questi sono i materiali che conoscete quelli che andanno a illuminarli con la luce e poi continuano a emettere luce di fosforescenza anche i loro diversi secondi, sicuramente qualcuno di voi ha un'auto estellina attaccata nel soffitto della stanza, ma che è più la ragazza che i ragazzi, però lo conoscenza un materiale fosforescente, quindi vedete questa luce, viene rilasciata anche a montenti molto molto lunghi. Oppure si potrebbe avere anche una fluorescenza ritardata, cioè un'emissione con dei tempi caratteristici che sono ancora maggiori, ma hanno lunghezze simile e lunghezze dono simili a quelle della emissione di fluorescenza. Quindi ricordiamoci che quando parliamo di luce messe per scintillazione, possiamo avere queste componenti, con una componente pronta, oppure una componente più lenta, quindi fluorescenza o fosforescenza. In generale, ci sono tanti materiali che si viscono questi fenomeni che vedepiniamo di luminescenza e la luminescenza si distingue a seconda del tipo di stimolo, perché per emettere della luce il materiale prima deve essere sollicitato, deve ricevere energia e quindi a seconda della fonte di energia andiamo a distinguere diverse modalità di luminescenza. Quindi ad esempio si parla di foto luminescenza quando il materiale viene stimolato dalla luce e poi la riemette sotto forma di luminescenza. Oppure potremmo indurre luminescenza attraverso calore. Allora si parla di termine luminescenza, non so se abbiamo mai sentito le tecniche di datazione passate sulle termine luminescenze, consiste proprio nel fare mette materiali che hanno immagazzinato nel tempo nel corso della storia, hanno hanno elettroni in delle trappole e per queste trappole ne venono liberate attraverso il riscaldamento del materiale. In base a quanto la luce luce messa si sa quanto elettrono, se ho stato intrappolato nel tempo, si sa quanto tempo è trascorso e quindi si risale alle tratti di campione. Quindi tipicamente questo si fa questa tecnica che che una maggi tecnica basata sulla termine luminescenza. Oppure energia trasferita attraverso un onda sonora, un ultrasuono, sono luminescenza. Energia elettrica, elettro luminescenza, anche le deformazioni meccaniche del materiale possono produrre effetti di luminescenza. Allora si parla in questo caso di tribo luminescenza o reazione di chimiche, che è luminescenza. Infine c'è tutta una branca che studia le emissioni di luce da parte di organizzimi viventi, tipicamente quelli che vivano a vedere nel mare, e allora si parla di bioluminescenza. Noi ci concentreremo sulla scintillazione, cioè emissione di fondori a seguito di ecidazione di autobi o molecole che viene indotta da radiazione. Quindi il nostro stimolo in questo caso è una radiazione che può essere o particelle o una radiazione $\gamma$. Quali scintillatori andremo a vedere? Andiamo a distinguere innanzitutto due grandi categorie. Gli scintillatori organici e gli scintillatori inorganici. Allora gli scintillatori organici sono composti ovviamente da molecole fatte di catturino, di di di di di di di catturino, catturino, catturino. E possono essere diverse tipologie, ad esempio potremmo avere dei cristallini organici puri, quindi dei materiali che presentano una struttura cristallina oppure dei materiali amorfi come gli scintillatori plastici. Potremmo anche avere una sottoforma di liquido e che si parla di liquidi organici. Gli scintillatori organici invece possono essere o dei cristalli, o scintillatori a gas o scintillatori a vetro. Ora andremo un po' più nel dettaglio tutti questi scintillatori, perché ognuno di questi ha delle caratteristiche, dei pro e dei contro. Partiamo da gli scintillatori organici, quindi quelli composti appunto da carboni, ossigili e biologici. E cerchiamo di capire quali sono i principi attraverso cui abbiano questa emissione di luce di scintillazione. Allora innanzitutto questa luce di scintillazione è associata a transizioni di elettroni, di valenza liberi. Il senso sono degli elettroni che appartenono alla molecola, ma non sono associati a un specifico atomo della molecola. E se andiamo a vedere i livelli che può occupare un elettrone di questo tipo, ad esempio potremmo trovare degli schemi, come ho mostrato in questa figura, con gli schemi abbastanza complessi, dove vediamo ad esempio i livelli legati a Stati di sinuoletto, quindi basalo spigno sostanzialmente, e dei livelli legati a Stati di tripletto. In questo caso sono stati messi una canta all'altro per evitare di confondervi. Ad Ad se non dovete guardare soltanto la scala verticale, quella quella comanda, quindi in base alla posizione del livello, ovviamente più è alta il livello, più è alta l'energia. Poi il fatto che siano destra e sinistra è una comodità per evitare di soprapporli e rendere un fusionario di grafico. Ad esempio se ci concentriamo su di stati di sinuoletto, ma anche per quali di tripletto, vado, vedete che oltre a questi stati S0 e S1 troviamo altri livelli più fini che sono legati ai modi migrazionali delle molecole. Non so se l'abbiamo fatto per sé la sua istruttura, forse ancora no, comunque lo scopriamo. A breve, quindi oltre a dei livelli che conosciamo su una standard abbiamo molte dei livelli legati al fatto che stiamo lavorando con delle molecole, quindi si possono generare dei modi migrazionali che introducono dei nuovi livelli a disposizione per l'elettrone. Quindi se ad esempio guardiamo gli stati di sinuoletto, vedete che ne ha azio tutto lo stato fondamentale, quello livello di energia più basso e lo stato S0, lo vedete qua. Quindi gli elettroni normalmente si trovano qui. Quando ricevono energia, a questo punto l'energia adatta dalla relazione che è inciso sullo scintillatore, che può essere un gammolo, una particella carica, questa energia può essere utilizzata per compiere una transizione verso un livello elicitato, che sono queste facette che vedete verso l'alto che possono andare verso lo stato S1 o verso uno dei suoi stati citati o verso S2, S3 e così via. Ok? Dipende da quanto energia è stata subita. Questi elettroni che si ritrovano in questi stati citati, a questo punto, ritornano allo stato S1 e ritornano a questo stato attraverso un processo che prende il nome di conversione interna. È un processo che fa sì che questi elettroni ritrovano alla persona questo livello, ma non venga emessa della radiazione. Normalmente quando abbiamo una diseggisazione, questa corrisponde all'emissione di una radiazione, invece il processo di conversione interna, questa discerza dell'elettrone verso il livello S1, avviene senza emissione della radiazione. Quindi ci ritroviamo questi elettroni a livello S1 che possono a questo punto ritornare allo stato S0 fondamentale, con le emissione di luce, di questo caso la luce di flore e scienza, che vi ricordo che la componente è pronta, quella veloce, quella che viene messa in pochi nanosecondi, poche decine di nanosecondi. E notiamo una cosa, che l'elettrone può ritornare proprio allo stato fondamentale o a uno dei livelli eccigati vibrationali, quindi uno di questi livelli tratteggiati. E questo che cosa permette, permette il fatto che questa radiazione che viene messa non ha un'energia sufficiente per essere riassorbita, perché per essere riassorbita questa energia dovrebbe servire a compiere una di queste 

transizioni che abbiamo visto all'inizio, ma come vedete dalla lunghezza della freccia, l'energia di questi fotoni non è sufficiente a far sì che un elettrone salti allo stato S1, e quindi la luce non viene riassorbita, ecco che lo scintilatore è trasparente alla sua luce. Chiaramente se la transizione avviene per sui livelli è proprio proprio in questi casi la radiazione può essere riassorbita, ma solamente in questi casi. Altrimenti quello che può avvenire dallo stato S1 è che l'elettrone va ad aver su un livello di tripletto, uno stato di tripletto T1. Da questo stato di tripletto può risciendere allo stato S0 e mettendo luce di fosforescenza, perché a questo punto il processo processo più lungo e quindi la luce viene messa e viene messa con un certo ritardo. Questa rappresenta la componente più lenta della luce di scintillazione che avevamo visto in quelli esponenziali, qualche se l'hai da fare. All'ilà di tutto il schema di livelli è abbastanza complesso, dove è risultato ricordare il fatto che avvengono delle eccitazioni, quindi da livelli fondamentali, a livelli eccitati, che che l'elettrone nuovamente torna a un stato fondamentale e viene messa con una luce. Grazie a questo schema di livelli la luce messa non può essere riassorbita, comunque sotto una piccola parte viene riassorbita. Infatti se andiamo a guardare lo spettro di emissione e di assorbimento, si trova uno sfasamento, uno shift tra questi due spettri. Ad esempio guardiamo questa figura a sinistra e una figura di tipo quantitativo, vedete che sull'assi o l'incentrale verticale non sono rappresentati dei numeri, ma soltanto delle grandezze, quindi qui abbiamo la lunghezza d'onda e sulla cervicale abbiamo l'assorbimento o l'emissione, dipende quale coppa si sta guardando. Guardate ad esempio in emissione la luce viene messa ad esempio a questo spettro, questo è lo spettro delle lunghezza d'onda e della luce messa. In assorbimenti invece vedete che le lunghezza d'onda che vengono assorbite sono più piccole, cioè energie e mangiori, che corrispondono con quello che avevamo detto prima, lo spettro che è messo, quindi queste transizioni, vengono messa a lunghezza d'onda a mangiore, frequenze minori, infatti vedete l'energia, lo spallo, il salto di energia è più piccolo, quindi vuol dire una frequenza più piccola della reazione che viene messa, lunghezza d'onda a mangiore. Rispetto alla luce assorbida, che vedete corrisponda a salto energetici maggiori, quindi energie maggiori, lunghezza d'onda a mangiori. E come mai il spettro di emissione corrisponde all'unghezza d'onda a mangiore rispetto allo spettro di assorbimento. C'è sempre un overlap, un piccolo overlap, ovviamente vuol dire che la luce è messa, potrebbe essere assorbita, però diciamo si cerca di mantenerlo, il più possibile è piccolo, lo sopè l'upperlap. Andando a guardare magari uno spettro più quantitativo, vediamo qui uno spettro di uno scintillatore plastico. Vedete questa è una sigla, un scintillatore plastico, che si adobrano delle sigle che derivano dalle gasaprojutrici, quindi si identificano questi codici. Quindi questo è uno scintillatore plastico, è abbastanza discuso. Vedete qui le lunghezza d'onda, a a punto possiamo anche cercare di capire a che in che regione stiamo ricalendo, siamo in regione del violetto, del ultravioletto, del rosso modo. Anche ci si sposta anche verso la componente alzzurra, chiaramente abbiamo anche delle code. Vedete come normalmente questi spettri di emissione presentano dei picchi, quindi tante volte alcuni dati che troverete nelle dannelle magari si riferiscono a quel valore che, ad esempio, viene viene l'indice di rifrazione del materiale, viene formito l'indice di rifrazione, in corrispondenza della lunghezza d'onda del picco, perché sapete che l'indice di rifrazione dipende dalla lunghezza d'onda. Quindi bisogna quando si acquista uno scintillatore, andare a vedere ad esempio questa catteristica. Andiamo agli scintillatori adesso in organici. Se prima abbiamo visto che l'emissione per gli scintillatori organici è legato a transizioni tra livelli della molecola, quindi invece abbiamo transizioni tra bande di energia. Quindi abbiamo la classica sul divisione tra banda di valenza e banda di conduzione, con un vietpro di energia prohibita nel mezzo, e quello che può succedere quando arriva una radiazione che c'è dell'energia, è questa energia che può essere utilizzata per promuovere un elettrone dalla banda di valenza alla banda di conduzione. In questo caso, sostanzialmente, stiamo ionizzando il materiale, elettrone di ventone e elettrone libero, e si crea una laguna, quindi abbiamo un elettrone libero e una laguna che sono libere di muoversi nel cristallo. A volte però, oltre a questo processo, potrebbe avvenire la provazione di un ecitone, cioè un elettrone della banda di valenza viene promosso a una banda che si trova leggermente al di sotto della banda di conduzione. Questo che cosa fa sì? Fa sì che l'elettrone in realtà non è proprio libero. Quindi, alla fine, assorbendo energia, si può creare una crocchia elettrone della banda, totalmente libero, oppure creare un ecitone. In entrambi casi, la laguna potrebbe incontrare un atomo di impurezza. Infatti, questi materiali, per per per per per per per quindi, quindi, quindi, quindi, quindi, quindi, ben non odrogati rispetto alla struttura cristallina perfetta che noi utilizziamo, abbiamo a introdurre degli atomi che sono delle impurità, quindi atomi di elementi diversi, quindi che vanno a modificare lo schema dei livelli energetici, sostanzialmente. Quindi quello che può succedere è che una laguna può andare a incontrare uno di questi atomi, uno uno questi impurità, e può ionizzarla. Quindi, uno può sostanzialmente ricominarsi, l'elettrone di questo atomo, di questo impurità, può combinarsi con una laguna. Chiaramente, nella guida rimarrà una laguna. E' appunto nella transizione, con l'emissione di radiazione. È un processo abbastanza complesso, però alla fine capiamo che, proprio grazie a queste impurità, come non introdotta questo drogaggio del materiale, che si inseriscono dei nuovi livelli energetici, che permettono queste transizioni. E E la cosa che ci 

interessa sempre alla base è che la luce è messa, una luce d'onda abbiamo detto maggiore rispetto a quella che viene assorbita. E lo vedete qui. Vedete l'assorbimento avviene tra queste due bande, quindi la distanza, indica l'energia necessaria, e e la frequenza, che di conseguenza la lunghezza d'onda. E se vedete invece, l'emissione avviene tra i bandi che sono più vicini tra di loro in energia, quindi vuol dire una frequenza minore, una lunghezza d'onda maggiore, quindi la stessa cosa che avevamo osservato precedentemente, quindi la luce messa è trasparente e non viene assorbita dal materiale. Spettere di emissione. Anche in questo caso vedete dei tipici spetteri che presentano un massimo minimo. In questo caso abbiamo gli spetteri emessi da diversi scintillatori. Vedete questi sono proprio cristalli. Io duro di sodio quello che c'è trasparente, sia il materiale che vi ho utilizzato per trovare taglio. E questo è è il rivelatore che adoverete l'ambratorio. Io duro di sodio, io io che altro taglio. Io duro di scesio, io duro la tra sodio. Io duro di scesio, io duro la tra taglio. Quindi vedete anche in base al tipo di drogam, che lo cambia l'ospetto di emissione, quindi cambia una caratteristica e dallo scintillatore. Non ho detto quali sono i vantaggi di questi scintillatori e vi aspetto gli scintillatori organici. Allora, gli scintillatori organici, quindi quelli composti da caracoli, ossigine, hidrogeno, hanno il bel tratto di avere una emissione molto veloce. Quindi sono dei rivolatori che venneranno adoverati nel timing. Tuttavia, a causa della densità non troppo elevata, a causa del numero atomico non troppo elevato, non hanno una resa di luce particolarmente grande, perché lo stop in power è più piccolo. Quindi entro una particella, si perde energia, ma non le perde tantissimo. Vi c'è a verse invece, questi scintillatori, gli scintillatori inorganici hanno il vantaggio di avere una resa di luce elevata, perché hanno tipicamente delle densità notevoli, e lo vedrete, per per che hanno portato uno qui, ve lo vorrò vedere quanto è pesante uno scintillatore organico, proprio non è un classico plexi class, si vede che è un materiale molto più denso, ha un elevato numero atomico e quindi un elevato stop in power. Questo è molto utile per la rivalazione del $\gamma$, quindi quando si vuole fare il spettroscopio in $\gamma$, dove è importante che il $\gamma$ interagisca, quindi è importante importante un elevato z o un elevato numero atomico, si utilizzano questi scintillatori. Altra faccia della medaglia, abbiamo una risposta temporale abbastanza scarsa, perché spesso i segnali sono molto lunghi, quindi questa luce viene messa anche in centinaia di nato secondi. Altro spetto negativo è il fatto che molti materiali sono igroscopici, quindi quindi dell'umidità. E questo fa sì che devono essere riusciti a l'interno di un materiale ermetico per proteggerli dall'umidità. Vediamo alcune caratteristiche, questi sono alcuni scintillatori inorganici, quindi quelli che abbiamo visto poco fa, il duro di sovvio, il duro di ceso, il duro di ceso, il duro di scintillatore, non entriamo nella dettaglio di tutti. Vi faccio vedere appunto, sono le caratteristiche che si vanno a vedere quando si deve scegliero uno scintillatore. Allora la densità, vedete, sono densità abbastanza elevate, partiamo dai 367 dell'ultimo di ceso, che quindi è uno dei più leggeri, possiamo dire, arriviamo a 8 per questo ton stanato di pionbo o per questo liso, intorno a 7, ci sono materiali veramente molto densi. Il melting point è più un dato di interesse di chi costruisce politi. Questi scintillatori non interessano l'interesse, per esempio, la lunghezza di radiazione e il raggio di modere. Se vi ricordate ne abbiamo parlato, quando abbiamo descritto di sciami e tromagnetici, d'altra parte, se vogliamo misurare $\gamma$, è possibile che si sviluppe all'interno di sciami e tromagnetici, ma è importante sapere queste caratteristiche del materiale per capire quale sono le energie che posso mesurare che vennero contenuti all'interno del mio rivelatore. La nutelli di interazione, c'è l'intervene di internazio per corso e che è importante, è importante, interno del metron. L'indice di rifrazione, vedete? Sono un valore molto simile al vetro, molto fondamentale. Questa è la carattistica che finisceombi l'impasto. Questi sono egroscopici. Ad esempio, se l'intervene di l'intervene puro di sodio, c'è una storia di un intraverso. perché encapsulata l'interno di un materiale protettivo. Allora, per l'umine scienzo non ci interessa, ci interessa andare a guardare più che altro quest'altra caratteristica. Il tempo di detenimento, con l'unilasloppo, la pennenza, di quell'esponenza le decrescente, che ci dice quanto è stato veloce la risposta l'emissione di questa luce. E vedete, espresse in nanosecondi e abbiamo volte anche due valori che venivano riportati, perché, se mi ricordate, abbiamo detto che ci questo è una componente veloce, una componente lenta. Tante volte è soltanto la componente veloce con la che viene presa in considerazione con quella più importante, quella lenta non è quasi mai presente, comunque trascurabile. Quando viene riportata è perché effettivamente è una componente che non può essere trascurata. Lifehild è l'emissione di luce. Quanti fotomori la resa, quanti fotori venivano prodotti per ogni mevori. In realtà qui è espressa in maniera un po' diversa, non è un numero di fotomori per me, ma si pone a 100 la resa dell'unicello di sodio, che è un materiale di riferimento, e poi rispetto a questa si esprimono tutti e altre. Quindi ad esempio il unicello di cesio emette più luce di un fattore 1,65 rispetto all'unicello di sodio. Spesso si usa anche questo modo di rappresentare questi dati. Comunque sono carnalmente resa e molto levati. In generale la resa nello scintillatore può raggiungere anche valori di 10.000 fotoni per mevori. E capite che questo ha una importanza per la risoluzione dell'energia, perché più fotoni venivano prodotti, più elevati il segnale, migliore sarà la risoluzione. Un po' come quello che abbiamo detto per i rivelatori a gas, andando a vedere il numero di coppie, il numero di cariche. Allo stesso modo di un scintillatore una risoluzione determinata da 20 fotoni venivano effettivamente emessi. Più fotori venivano emessi, migliore sarà la risoluzione dell'energia. Quindi in generale i risoluzioni in organica hanno risoluzioni migliori. Alcuni esempi di scintillatori classici, quindi altri dati riferiti però a scintillatori classici, quindi di tipo orlamico, che cosa cambia rispettato prima. Vediamo i tempi di emissione, che sono molto più rapidi, nanosepondi. E poi anche qui il light ill viene riportato in percentuale rispetto all'light ill di un materiale che non trascende in questo caso specifico. Anche qui abbiamo altri valori di piccolo della lumezza d'onda, lumezza di tenazione e così via. Abbiamo detto che la risposta è luce e in via di principio mi potrebbe fornire un'idea della energia depositata. Questo ovviamente se il numero di fotori prodotto è proporzionale all'energia depositata. E questo in mezzo è vero, ma non sempre così. A volte parte dell'energia che viene depositata non viene utilizzata per produrre fotoni, viene persa attraverso dei processi non radiativi. E questo è così che la 

risposta non è proprio esattamente lineare. Addirittura nel caso di particelle cariche passanti, la linearità effettivamente non è verificata. Lo vedete in questo grafico dove la resa di luce viene mostrata in funzione dell'energia. Allora intanto abbiamo una nipendenza dal tipo di particelle, quindi anche se viene depositata la stessa energia, se questa energia è depositata da un elettrone, viene messa più luce rispetto a quanta meniera è messa se l'energia viene depositata da un prodone. E più vedete anche la linearità non è assicurata per le particelle cariche passanti. Quindi un grosso modo vale, ma a volte bisogna considerare delle correzioni. E proprio per quanto riguarda questa perdita di energia, esiste un modello semi-empirico che è la sassidetta relazione di BX. Allora idealmente abbiamo detto che la resa di luce, quindi quanti fotoni vengono prodotti in base all'energia che viene depositata, dovrebbe essere una relazione lineare che lega le due grandezze, quindi se per un centimetro è stata persa una certa quantità di energia DE, allora mi aspetto proporzionalmente una quantità di luce, quindi numero di fotoni, sempre in DX. Quindi mi aspetto che se non ci sono particolari fenomeni, effetti secondari, esiste una relazione di tipo lineare. E quello che abbiamo detto prima, dove questo fattore di proporzionalità rappresenta un'efficienza massima di sciutinazione. Che sostanzialmente va a stabilire la pendenza di questa curva, possiamo dire. Ma in realtà non è esattamente così che l'abbiamo capito, a volte intervengono degli effetti secondari e non radiativi degli effetti di quenching, quindi in realtà la relazione che dobbiamo prendere in considerazione è una relazione di questo tipo che però vedete tagliata a metà, quindi un attimino lo levo. Quindi rispetto a quello che abbiamo scritto prima, quindi S e DX, c'è un fattore correttivo che vedete riportato qui al denominatore, dove compare questo termine KB che è un parametro che viene ricavato da data sperimentale, come dicevo SMP, abbiamo una relazione per una parte e questa relazione viene ricavata soltanto sperimentalmente dai dati. Quindi ci aspettiamo che a causa di questi effetti non radiativi la relazione effettamente eliminale non vega sempre matta. Cosa ci aspettiamo di vedere in uscita da uno scintillatore? Il segnale che viene prodotto, capite che in realtà noi non possiamo vedere il segnale in luce, ma andiamo a vedere un segnale elettrico che viene prodotto dal fotosensore, quindi quello che si vedrà allo oscilloscopio se noi potessimo visualizzare questo segnale chiaramente è quello che produce un sensore di luce e la sua caratteristica dipendono dall'elettronico e dal tipo di sensore, quindi ad esempio più o meno ci si aspetta comunque sia una forma di questo tipo, si tratta di un segnale analogico perché a questo punto siamo interessati a quanto ha luce stata prodotta per il segnale per avere un'ampiezza proporzionale al numero di fotoni prodotti e normalmente sono segnali che hanno una durata molto breve con dei tempi di discesa di decine fino ai secondi è una durata che in realtà questa dipende un po' più dall'elettronica ma quello che ci interessa è soprattutto il fronte di salita o fronte di discesa se non ha il segnale così dimmi negativo e anche questo lo vedremo in laboratorio, quindi vedremo al oscilloscopio un tipo segnale elettriciatore allora prima di andare avanti vi faccio vedere una cosa di pratico che mi ho portato, più l'altro perché l'ho portato a non vorreste fare troppo tanto però devo ripartare una prossima volta mi stavo domando soltanto come farvele vedere più l'altro farle vedere anche a casa allora cosa vi ho portato alcune cose sono facili da farvele vedere allora questo è un esempio di scintillatore alla fine non è niente disconvolgente perché sembra un pezzo di plexiglas sanzialmente quindi si presenta esattamente come se fosse un plexiglas o un vetro, addirittura questo è uno scintillatore plastico quindi vedrete che la consistenza sembra quella di appunto del plexiglas più che di un vetro, un vetro più pesante di questo quindi densità non particolarmente levate quindi questi sono dei materiali che vengono adoperate molto bene per la rivalazione intanto di particelle cariche già per in $\gamma$ questi non manno proprio benissimo perché abbiamo detto in $\gamma$ devono avere materiali più densi con alto umratomico, ma questo invece non è così questo è stato realizzato con una tecnica di estrusione quindi con uno stampo riempito con lo scintillatore sotto forma liquida che poi viene fatta raffreddare e vedete che la superficie in questo momento è ricoperta con un materiale plastico per proteggerlo, perché comunque si ha alcuni materiali, sono sensibili anche al grasso delle mani quindi si dovranno maneggiare con i guanti la forma che è stata scelta ovviamente è stata scelta da in base allo utilizzo che dovrà fare il problema, quali sarà in questo caso andare a utilizzare un sensore in grado di misurare la luce e messa da questo scintillatore perché come vedremo si propaga in tutte le direzioni e quindi ora il problema sarà cercare di capire come si raccoglie la luce si deve convogliare in qualche modo in un punto della superficie perché è impensabile andare a rivestire tutta la superficie dello scintillatore con dei sensori non è realissimo, normalmente si utilizza uno, due sensori quindi bisogna convogliarle in qualche modo e poi bisogna trovare un opportuno coppimento, un opportuno sensore che divenceri deve avere quindi vi farò vedere, ora con le prossime se la idea è complesso progettare a un ripulatore che è abbastanza semplice ma in realtà bisogna cercare di ottimizzare la raccolta della luce ho lavorato tanto tempo fa con un professore che poi non l'abbiamo conosciuto del professore russo che è il papa di Marco Russo ma poi non l'abbiamo conosciuto una persona in gammo un esperimentale elettronico veramente simpatico anche e avevo lavorato su un progetto dove si utilizzavano scintillatori dove aveva raccogliere questa luce quindi dovevamo cercare di poter ottimizzare la raccolta della luce mi riconno che mi è rimasta impresso questa frase, non può volgare però proprio due due dì, rimanda tra noi lui diciava sempre, i fotoni sono bastardi perché escono è difficilissima anche una leggerissima perdita una leggerissima crepa nella copertura mi comporta delle perdide notevoli in termini di raccolta della luce quindi è un problema veramente difficile a affrontare che si deve studiare anche attraverso delle simulazioni, infatti farò vedere anche le code di simulazione per studiare il trasporto della luce per il materiale ve lo faccio girare, il mostro per farvi lo farvi vedere la seconda cosa che vi volevo far vedere sempre sui scintillatori e cercare di capire dato che sembra un pezzo di plastica come si può vedere la differenza tra un normale pezzo di plexiglass e uno scintillatore e vi racconto questo aneddoto che capitò proprio durante l'attività di produzione di alcuni fotosensori e alcuni rivelatori per l'esperimentalice comunque ci troviamo di fronte a un lavoro che dobbiamo fare qui a Catania dove c'era stato richiesto di realizzare delle piccole guide di luce e di accoppiarle a un sensore, dove avevamo un scintillatore una guida di luce, ora vedremo cos'è comunque lo capito dalla stessa definizione guida la luce verso il fotosensore e quindi ordiniamo queste guide di luce da una fabbrica russa o forse ucraina non mi ricordo e arrivarono queste guide di luce e arriva da certo punto un tecnico che stava lavorando con noi disse queste non sono guide di luce ma sono, quindi non sono semplici pezzi di plexiglass ma sono stati contaminati con dello scintillatore come l'ha fatto, come ha fatto a capirlo semplicemente perché ha preso uno di questi plexiglass e ha messi alla luce infatti se lo vedete l'ho alzato un po' verso la luce e poi la poi la poi un colore violetto un colore violetto, leggermente violetto e ora vi faccio vedere un effetto ovviamente questo è perché sta ricevendo la luce e la riemette sul violetto ma lo faccio vedere ancora meglio stimolando lo scintillatore con una luce che lui riesce a assorbire meglio e quindi vi ha portato qui degli senti di guide di luce che abbiamo proprio adoperato in quel periodo quindi in mezzo a questo gruppetto anzi ora hanno visto qualcuno di voi a venire e provare a fare la distinzione riconoscere così senza ovviamente metterla alla luce il plexiglass contaminato non

è un vero proprio scintillatore è messo contaminato da materiale scintillante quindi se qualcuno vuole venire a dividere è piaciuto questa sfida quindi provo a dividere in i due gruppetti quelle che tu ritieni senza guardarli vedendo risultato quelle che tu ritieni contaminato da scintillatore e quello che ritieni senza poi cosa faremo ho portato anche una lampada ovi una lampada di wood quindi poi accenderemo questa lampada e stimoleremo il materiale scintillante se è presente ok va bene questo non lo so abbiamo fatto la suddivisione ora risvegliamo tutto proviamo ad accendere la lampada di wood non so se riuscite a vederlo due ne ho persi ne ho svegliati vedi della differenza non so se riuscite se riuscite se da fuori mi viene di fare vederlo solo questi due questo è giusto quindi è riuscita abbastanza bella però si è stato imboccata quindi abbiamo un po capito che sembra apparentemente dei vedi ma in realtà sono dei materiali luminescente quindi che mettono grazie a te per la lunga scienza come appare alla fine un scintillatore montato con un fotosensore allora in questo caso vi ho portato questo altro oggetto che vedete qui questo tubo molto lungo dove in realtà il vero è proprio scintillatore e soltanto la parte argentata quindi in questo caso abbiamo un materiale scintillante di forma cilindrica è un aiuduro di sodio sono sbaglio sì in questo caso è un aiuduro di sodio o è quello che andrete ad operare in laboratorio quindi un scintillatore inorganico un cristallo ci aspettiamo se qualcuno vuole vederla prenderlo ne vedrà il peso sto partendo quindi ovviamente c'è anche il peso del sensore però ti rendi conto che è abbastanza pesante è molto pesante se avessi il scintillatore un paragone con il plastico vede se subito che è un materiale molto più pesante e quindi è un materiale ottimo per la rivolazione del $\gamma$ con un altaresa che cosa c'è subito dopo c'è una guida e il sensore vero è proprio che è il foto moltiplicatore che però andremo a discutere la volta prossima non ci interessa nello specifico vi voglio far vedere soltanto che in particolare non possiamo vedere lo scintillatore perché è incapsulato con questa copertura di alluminio perché è uno dei difetti dell'aiuduro di di sodio è il fatto di essere igroscopico e quindi bisogna necessariamente proteggerlo dall'umidità ambientale inoltre il fatto di in realtà nessuno scintillatore lo vedrete mai nudo come lo state vedendo adesso e il motivo lo spiegheremo ora quando viene prodotta questa luce quindi immaginate questo sia ad esempio una quartice ma è interno dello scintillatore e da questo punto vengono a messi i fotoni di scintillazione che abbiamo detto sono fotoni nell'uv comunque di pochi elettronbold e questi fotoni vengono a messi in tutte le direzioni e se vengono a messi in tutte le direzioni è lasciassimo lo scintillatore nudo una ferata è chiaro che nel momento in cui i fotoni raggiungono il bordo dello scintillatore questi fuoriescono su un viscola di frazione e fuoriescono il materiale e le abbiamo persi e quindi una prima operazione che bisogna fare e cercare di evitare questi fotoni fuori escano da rivelatore secondo problema che dobbiamo affrontare l'abbiamo detto prima e cercare di convogliare i fotoni in una piccola regione dove vado a posizionare il mio fotosensore allora quello che si fa tipicamente è rivestire di un materiale possibilmente riflettente dai pareti dello scintillatore eccepto la parete dove si va la parete comunque la zona dove si va a posizionare il fotosensore ecco perché lo vedete anche coperto non soltanto per proteggirlo dall'umidità ma perché in realtà ci sarà uno strato riflettente che appunto permette alla luce di essere riflessa quindi ad esempio questo fotone che vedete qui segue queste riflessioni in questo caso stiamo vedendo una riflessione perfettamente speculare si riflette diverse volte fino ad arrivare al fotosensore capite che con uno schema del genere un fotone potrebbe anche percorrere percorsi molto lunghi prima di arrivare al fotosensore ecco perché andare a vedere la posizietta lunghezza di attenazione è fondamentale perché è vero che magari il luce in chidatore può avere dimensioni della decina di centimetri e quindi voi possiamo dire se le lunghezze di attenazione sono 40 cm sono tranquille, srena in realtà no perché i fotoni possono percorrere tramite queste riflessioni spazi molto lunghi quindi alla fine essere assorbiti non solo possono essere assorbiti dal materiale stesso secondo la legge esponenziale decrescente dove appunto la sloppo di questa esponenziale la pennenza è determinata da questa lunghezza di attenazione ma potremmo avere anche effetti di assorbimento nelle pareti cioè il materiale che vado a introdurre per effettuare questa riflessione potrebbe andare ad assorbire magari non è una perfetta riflessione alcune volte sia una probabilità del 10\% di essere assorbito, il fotone non è assorbito anziché essere riflesso e quindi normalmente quando si introducono questi materiali per la riflessione si parla di coefficiente di riflettività un coefficiente che vada da 0 o 1 e che stabiliscepra un po' la probabilità di avere una riflessione o un assorbimento ad esempio un materiale che noi spesso doperiamo ho messo qua per rivestire le pareti di uno scintillatore può essere un materiale come questo è un astrodessivo un materiale che si chiama un astrodessivo di miler vedete che appunto è una sorta di carta da cuscia tipa alubigno come aspetto e è una superficie abbastanza abilettente quindi questa potrebbe essere una tecnica attiva, vi volevo portare anche le cuti valenti di quella mattonella di cintinatore rivestita però non l'ho trovata, non so dove è finita. Comunque in quella mattonella poi abbiamo avuto un'altra cortezza che vi schiederò a breve per andare a raccogliere la luce quindi capite che più è stesso il rivelatore più la raccolta della luce è difficile perché alla fine dei fotosensori hanno delle superfici abbastanza ridotte, non sono mai degli oggetti troppo grandi quindi più è stesso il rivelatore più abbiamo difficoltà da accoppiare il rivelatore col fotosensore quindi a fare una raccolta di luce efficiente quindi in quel caso si adoprano anche degli strategiani un po' diversi. Fino adesso abbiamo immaginato che la riflessione fosse perfettamente speculare quindi ad esempio in un materiale del genere effettivamente mi aspetto una riflessione speculare a meno che non creo qualche rugetta nello standard il miter ma a volte si preferiscepresse utilizzare dei materiali che provocano diffusione piuttosto che riflessione quindi tificamente dei materiali o con una superficie rugosa e allora vedete la differenza. Qui abbiamo un caso di riflessione perfetta qui abbiamo un caso di diffusione. La diffusione deriva dal fatto che la superficie non è perfettamente piana e quindi quando avviene una riflessione insomma l'angolo di incidenza dipende dalla normale sostanzialmente dipende da un'inclinazione del particolare punto che viene colpito e quindi vedete uno schema molto più disordinato rispetto a una riflessione perfetta. A volte utilizziamo dei materiali diffusivi come ad esempio l'utilizzo di vernici bianche e normalmente si utilizza il biossido di ditaneo, uno di quelle vernici che vanno a rivessire il scintillatore o a volte si utilizza il teflon. Il teflon non solo l'abbiamo mai misso, viene utilizzato ad esempio nell'idraulica e quella sorta di nastro bianco si mette nelle tubature, nei raccordi per evitare le perdite. L'utilizziamo per rivessire gli scintillatori, grazie. Perché anche un materiale bianco produce diffusione e può andare bene effettivamente per il nostro scopo. Quindi dicevamo uno dei problemi che questi fotoni possono anche percorrere spazi molto grandi, lunghezze elevate e questo non solo la conseguenza di poter perdere fotoni, perché magari vengono assorbiti, ma anche conseguenze sulla risoluzione temporale, perché noi fino a adesso ci siamo concentrati sui tempi di emissione di questa luce, che abbiamo detto sono tempi molto brevi e va bene. Ma il problema è anche raccoglierla la luce a questo punto e se la luce deve percorrere spazi lunghi capite che il segnale che viene raccolto al fotosensore chiaramente può anche essere catarizzato da tempi più lunghi di quelli che abbiamo visto fino adesso. Facciamo degli esempi. Immaginate che la luce all'interno di questo scintillatore si muova a una velocità, chiaramente non è proprio la velocità della luce in quel mezzo, cioè la velocità della luce è il vuoto, ma immaginiamo che sia così, in realtà è ancora più massa. Quindi immaginiamo via U al AC e la potreste anche poi far una simulazione, ad esempio estrendo un punto a caso nello scintillatore, possiamo lavorare anche nel piano in prima approssimazione, quindi estrete un punto a caso di questa superficie dello scintillatore, estrete una direzione casuale in tutto il angolo, in tutto il duve greve, quindi 360 gradi e andate a seguire attraverso delle riflessioni il percorso di questa particella finché non arriva alla zona dove è posizionata il fotosensore e possiamo stimare quanto spazio ha percorso il fotone ed è questo sabbiliere quindi la distribuzione dei tempi, quanto tempo è impiegato a percorre quella distanza e capite che quindi idealmente se il fotone si muove se verso il fotosensore avreste tempi abbastanza brevi perché abbiamo 30 centimetri in nanosecondo, quindi se il rivelatore fosse 30 centimetri 

ci metteremo un nanosecondo a raccogliere tutta la luce, ma in realtà non è così, a seguito di queste riflessioni le distanze percorse sono molto diverse, possono essere molto diverse, abbiamo una distribuzione di distanza percorse quindi una distribuzione dei tempi per percorrere queste distanze e quindi questo rappresenta un problema soprattutto quando il rivelatore ha geometrie particolari come ad esempio nel caso di bacchette di scintillatori quindi delle barre molto lunghe con una con sezione piccola, ecco perché vi dicevo in realtà andare a ottimizzare un rivelatore di questo tipo è sempre una cosa banale, i concetti di basso sono semplici, abbiamo la produzione di luce, rifestiamo con materiale riflettente, mettiamo il fotosensore ed è fatta, in realtà purtroppo non lo voglio ridire, ma sappiamo che i fotoni non sono simpatici, bisogna ottimizzare tutto perché si rischia veramente alla fine di avere dei segnali troppo debboli, quindi di raccogliere pochi fotoni da decine di migliaia di fotoni che possono essere prodotti però ogni web alla fine se ne raccolgono qualche decina, cioè immaginate quante perdite si hanno in tutti questi processi e che cosa si può fare? Si può studiare il comportamento del rivelatore e l'arcolte di luce con dei simulatori professionali, ad esempio vi ho già parlato più volte di questo simulatore che si chiama giant di cui inesistono diverse versioni in particolare l'ultima versione giant 4 permette anche di seguire e di trasportare la luce nel materiale, nei materiali, perché la versione precedente considerava soltanto le particelle cariche, quindi i percorsi delle particelle cariche nella materia giant 4 include anche la parte dell'ottica della luce e ad esempio se voleste realizzare una simulazione, quindi seguire il percorso dei diversi fotoni che vengono prodotti per scintillazione in un rivelatore, dovreste andare a definire tantissime informazioni, ad esempio dovreste specificare che tipo di luce viene emesse per scintillazione, quindi lo spettro di emissione dello scintillatore, il tempo di decadimento, quindi quanto tempo ci si impiega a emettere questa luce, oppure si potrebbe andare a specificare l' assorbimento del materiale stesso, quindi attraverso un coefficiente di assorbimento, le proprietà di riflessione del materiale che abbiamo posto sulle pareti, effetti di rifrazione quando si passa da un mezzo a un altro, il tipo di superficie di rivestimento, se una superficie è perfettamente liscia, una superficie scabra, processi di scattering della luce e altre tipi di processi, quindi veramente è un software estremamente complesso. A titolo di esempio vi faccio vedere che si potrebbe andare a introdurre nel codice, una volta che abbiamo scelto un materiale ad esempio scelto questo materiale per andare a rivestire il mio rivelatore, vado a vedere il datasheet che mi fornisce il produttore, il costruttore di questo nastro, e scopro che il suo coefficiente, la sua riflettività cambia in base alla lombazzadonna della luce che incide, quindi non è sempre la stessa, ma cambia e ha quest'andamento, vedete, in base alla lombazzadonna, in particolare se io sono concentrata ad esempio nella riflessione di luce messa da uno scintillatore, quindi tipicamente sui 400 nanometri, vedo che addirittura in questa regione il coefficiente di riflettività varia parecchio,

quindi non è il 97\% come in questa zona ma scende anche all'83-85\% ed è importante perché cambiano tevolmente le prestazioni del rivelatore, e questi sono tutti dati che si possono passare al simulatore. Giusto per farvi vedere alcuni esempi, immaginate queste sono simulazioni che abbiamo fatto, noi a nostro gruppo di avere delle barre di lunghezza 1 metro e di sezione all'incirca 1 centimetro quadro che di voler soddare il percorso della luce all'interno di questa barra con l'idea di raccogliere la luce all'estremità, quindi la luce che viene prodotta deve percorrere anche distanze di decine di centimetri per arrivare all'estremità di vende dove è stata prodotta, di vende dove è passata la particella. E allora guardiamo quanti fotomoni riescono a percorrere una data distanza, in base alle caratteristiche del mio rivelatore, ad esempio ci sono due condizioni diverse di riflettività, 90\% 80\% e poi due condizioni diverse di superficie, se una superficie è levigata o una superficie rulosa. Allora se la superficie è posh levitata vedete che la riflettività ha un suo effetto, considerate che questa è scala logaritmica, quindi da migliaia di fotoni comunque intanto diminuisce esponenzialmente come ci aspettiamo a seguito dell'assorbimento e poi a seconda della riflettività vedete appunto che la curva blue è molto più bassa rispetto a quella rossa, in più se abbiamo una superficie scabra, appunto rivesti il rivelatore, vedete che gli effetti di riflettività ponzono di meno, ma rispetto alla superficie levigata abbiamo una notevoa diminuzione dei fotoni e questo è fondamentale, quindi ci dà delle indicazioni e non è necessario realizzare la prova sperimentale e verificare, conviene simulare prima e poi ovviamente costruire sperimentalmente. Vi dicevo che può tornare utile a volte realizzare delle guide di luce, cioè degli oggetti realizzati in plexiglass che permettano un accoppiamento più agevole tra lo scintillatore e il rivelatore, perché magari hanno proprio geometrie diverse oppure un problema che potrebbe sorgere, deriva dal fatto che magari il mio rivelatore è dentro un campo magnetico e il mio fotosensore per diversi motivi non può lavorare all'interno di un campo magnetico, perché magari sfrutta dei campi elettrici, dei moti di particelle cariche che all'interno di un campo magnetico vi conosco a molte, quindi devo portare il mio sensore fuori dal campo magnetico, come faccio a fare un accoppiamento, si possono realizzare delle guide, quindi prendere questa luce e trasportarla verso il fotosensore, capite che ognuno di questi processi, cioè ogni cosa che aggiungete comporta perdita di fotoni e lo vediamo anche qui, nel caso di una guida di luce non possiamo prendere tutti i fotoni e trasportarli su una superfice più piccola, perché in un certo senso il flusso di fotoni non è comprimibile, non valgono le stesse regole che valgono ad esempio nel trasporto dei fluidi in un condotto dove c'è ovviamente la conservazione della massa, qui al massimo possiamo trasportare una quantità di luce che è proporzionale al rapporto tra le aree, tra area del fotosensore e area dello scintillatore, non di più, quindi non possiamo pensare che tutta la luce che fuoriesce da questa superficie venga convogliata sul fotosensore che ha una superficie più piccola, al massimo possiamo convogliare questa frazione data dal rapporto delle aree, dietro questa semplice osservazione c'è ovviamente molta matematica e le guide di luce si basano sul fatto che insomma si cerca di guidare il più possibile la luce attraverso delle geometrie opportune che facilitano il fenomeno della riflessione totale. E vi dico ci sono geometrie molto varia, molto varie ad esempio questo è un classico esempio di guida di luce che si accoppia al scintillatore come questo ovvio, come questo che abbiamo visto adesso, una mattonella quindi con una sezione con una retta angolare che quindi si accoppia a questa estrenità della guida di luce e dall'altro lato invece vedete una forma cilindrica per permettere l'accoppiamento con un fotosensore che ha una superficie circolare quindi vedete questa è proprio una questione geometria e quello vedete meglio qui ad esempio scintillatore, la guida di luce e il sensore oppure vedete geometrie ancora più elaborate sono veramente delle costruzioni quasi artistiche veramente belle. Ultima cosa che vi volevo dire, mi ricordo se c'è altro, si vabbè ci fermiamo qua facciamo questo e ci fermiamo qua perché devo insegnare anche kit. Allora l'ultima cosa che vi dico è quando abbiamo una barra lunga un metro come vi ho fatto vedere prima e dobbiamo andare a raccogliere la luce dove lo posizioniamo il fotosensore, lo potrei posizionare agli estremi della barra però questo parto ovviamente prevede che la luce debba essere trasportata fino all'estremità e abbiamo visto che c'è una notevole perdita di luce a seguito di diversi effetti. Allora un'alternativa potrebbe essere quella di utilizzare le cosiddette fibre ottiche però sono fibre particolari perché si chiamano wavelength shifter fibers sono delle fibre che mi spostano la lunghezza d'onda della luce quindi assorbono la lunghezza un determinato range di lunghezza e d'onda e la riemettono a una lunghezza d'onda tipica. Ve lo faccio vedere concretamente queste sono appunto fibre di questo tipo. Vedete il colore vuol dire che emettono nel verde perché state vedendo il verde qual è la caratteristica assorbono la luce anche luce che incide lateralmente anzi lo vedete sembra quasi accese ok perché che cosa succede? Allora la normale fibre ottica e non so se abbiamo visto in altri corsi si o no? No non abbiamo mai parlato di fibre ottica? Allora la fibre ottica ha una struttura come quella che vedete qui nella slide è costituita da due cilindri concentrici allora il primo cilindro prende il nome di cor e il cuore della fibra con un determinato indice di rifrazione ed è circondato da un altro strato che prende il nome di cladding che ha un altro indice di rifrazione quindi sono due materiali con indice di rifrazione diversi in particolare l'indice di rifrazione della parte interna è maggiore rispetto a quella della parte esterna. Questo fa sì che la luce come lo vedete qui se entra con un opportuno angolazione può essere guidata attraverso la fibra ottica mediante riflessioni totali. Ricordo che la riflessione totale è permessa 

solamente se si passa da un indice di rifrazione maggiore a uno minore quindi l'idea è riuscire a guidare la luce anche in percorsi non rettiline quindi percorsi curvati attraverso queste riflessioni totali e questo è il principio di funzionamento della normale fibre ottica che viene utilizzata anche vedete lo comunicazione lo conoscete benissimo conoscete più le applicazioni che fosse il principio di funzionamento questa fibra invece la fibra webl n shifter è un po' diversa perché la fibra ottica capite può trasportare la luce solamente se la luce entra dalla sezione viene guidata ed esce dall'altro lato ok ma una luce che incide lateralmente non riesce a entrare nella fibra perché subisce pesos non è fuori ash ok questa fibra invece alla caratteristica di assorbire la luce. Quindi anche la luce che incide lateralmente viene intanto assorbita e poi riemessa alla lunghezza d'onda, in questo caso del verde. La luce che viene riemessa, siccome si trova all'interno della fibra, in automatico viene guidata verso l'esterno. Quindi capite a un funzionamento un po' diverso che a noi torna molto utile perché se io prendo questa fibra e la vado in qualche modo a distendere all'interno dello scintillatore, lei sarà in grado di assorbire la luce e messa dallo scintillatore perché la luce va a colpirare lungo tutta la sua superficie esterna, prende la luce, l'assorbe, la riemette e la guida verso l'esterno e io a quel punto posso collegare il mio fotosensore all'esterno della fibra. Come faccia incapsularla all'interno di una barra a lungo metro? Ci sono diverse tecniche, a volte con la tecnica dell'estruzione che vi ho detto prima si può realizzare una barra che presenta al centro un foro dove fa passare la fibra ottica oppure, come abbiamo fatto noi, abbiamo fatto realizzare sempre per l'estruzione una barra che presenta un solco sulla superficie. Questo solco vado a posizionare la mia fibra ottica e questa è una tecnica molto utile quando si ha a caccare con degli scintillatori di dimensioni molto estese dove diventa difficile avere fotosensori estese, quindi o si mettono tanti fotosensori e si devono far lavorare in coincidenza ma diventa complicato oppure si va a posizionar una fibra ottica e si raccoglie la luce attraverso questa fibra ottica, Wembley shifter e il fotosensore si posiziona all'estremità delle fibra ottica, più anche molto utile perché se ad esempio ho due estremità posso posizionare solamente due rivelatori, li metto in coincidenza, cioè faccio misuro solamente quando entrambi da un segnale perché se viene prodotta la luce e si incanala nella fibra ottica andrà verso il trambe l'estremità, quindi mi aspetto segnale da entrambi l'estremità e questo mi permette anche di ridurre gli eventi di rumore di fondo perché è improbabile che io abbia una coincidenza giusta appunto due segnali insieme insomma è veramente raro. Ad esempio se avessi un rivelatore molto esteso e di forma anche quadrata potrei realizzare un solco e andare a creare una spirale con una fibra ottica questo anche si fa e si va a posizionare poi un sensore da questo punto può avere anche dimensioni molto piccole, lo vedete anche del millimetro quadro e andare a raccogliere la luce con questa tecnica. Spettro di assorbimento e spettro di emissione ovviamente devono essere diversi, spettro di assorbimento si deve adattare allo scintillatore quindi deve essere verso l'UV, spettro di emissione e questo punto si deve adattare al fotosensore perché il fotosensore deve vedere questa luce. Ok ragazzi allora per oggi io ho finito vi chiedo di continuare a convinare lo schema dei gruppi, in maniera tale da poter iniziare a fine messe con il laboratorio e adesso procedo alla consegna del kit multimetro e il prima battuta chiamerò chi già ha avuto il kit arduino ok perché sono sicura che non è impegnato il kit arduino buonasera allora la consegna viene oggi e il ridiro lo dovete fare al giorno ventinue perché appunto il giovedì io non ci sarò solo per questo allora comincio con Francesca lì c'è no ok Soltorena eh sono la casa mia che si fa ok eh soffia per vita c'è ok Qualcuno mi fa da balletto può mamma lo consegnare gli kit ragazzi sono enormi quindi lui lo dovrebbe essere da fuori a vera ragazzi che sono collegati da fuori se vogliono si possono scollegare che stiamo facendo la potenza in un luna l'abbiamo dato la bellvenna luka bonanno ancora ma cosa perché può segnato un po' di che si ok va bene e paprizio buon coraggio non c'è e gabriele primo


\textbf{lez 12}

Buongiorno a tutti ragazzi ho visto che è molto di voi già hanno compilato il foglio per avere i turni ho semplicemente aggiunto un vostro collega che mi ha chiesto di essere aggiunto manualmente e ho spostato un gruppo che ha messo come gruppo provvisorio l'ho modificato le vostre preferenze manca ancora qualsiasi persona io ospitterò un po come poi prenderò la lista degli essenti li andrò a inserire casualmente nei gruppi che sono rimasti parzialmente scoperti. Quindi noi faremo lezione vi dicevo il prossimo lunedì ora metterò un avviso anche su Teams perché giovedì non sono presente e durante la prossima lezione quello che faremo sarà presentare gli esperimenti di laboratorio in maniera tale da poter iniziare poi la settimana successiva ci sono domande sulla parte dei gruppi va bene allora riprendiamo dove avevamo lasciato la volta scorsa se vi ricordate avevamo iniziato a discutere degli scinfillatori perché dopo aver parlato dei rivelatori e gas abbiamo descritto quest'altra tipologia di rivelatori che sono rivelatori per certi versi più facili più maneggevoli rispetto a rivelatori a gas che comunque sia richiedono una alta tensione richiedono dei recipienti a tenuta o addirittura dei recipienti con un flusso continuo di gas gli scintillatori di per sé sono dei rivelatori a passanza basilari perché si basano se vi ricordate sulle missione di luce di scintillazione tipicamente nel UV quindi corrispondente a pochi elettron molto come energia e questa luce quindi viene messa al seguito del deposito di energia da parte di una radiazione che colpisce questo tipo di materiale e abbiamo visto anche diverse tipologie di materiale scintillatore e tipicamente l'abbiamo classificato in due grosse categorie gli scintillatori organici costituiti da composti di carbonio dirogeno sigeno e gli scintillatori inorganici l'abbiamo visto anche alcuni esempi abbiamo detto che ci sono delle proprietà caratteristiche di questi materiali che ogni volta si devono andare ad attenzionare quando si progetta un rivelatore e in particolare si può essere interessati alla resa in luce cioè il numero di fotoni che vengono emessi per ogni MeV di energia depositata perché questa chiaramente può essere un'informazione che ci dà accesso al valore di energia che è stata depositata all'interno dello scintillatore infatti tipicamente la risposta di uno scintillatore è abbastanza lineare quindi il quantitativo di fotoni che vengono emessi dovrebbe essere proporzionale all'energia depositata e chiaramente più fotoni vengono emessi migliore sarà l'efficienza scusate la risoluzione in energia del rivelatore ma anche l'efficienza perché vuol dire che quando incide una radiazione anche se viene depositata poca energia il segnale che si produce comunque un segnale che può essere rivelato e quindi aumenta anche in questo questo l'efficienza del rivelatore oltre a questo aspetto si può andare a vedere anche la risposta temporale dello scintillatore infatti abbiamo detto che in ogni scintillatore possono essere presenti due componenti una componente un po più lenta una componente un po più veloce ciò che si poto non vengono emessi tutti nell'immediato ma vengono emessi con un certo andamento nel tempo che è un andamento esponenziale decrescente che è caratterizzato tipicamente da due componenti una lente e una veloce l'abbondanza dell'une dell'altro dipende dal tipo di scintillatore che si prende in esame quindi più o meno sono queste le caratteristiche che si vanno a studiare in uno scintillatore ci sono scintillatori che magari accellono per la risposta in luce altri per la risposta temporale quindi è difficile trovare sempre lo scintillatore ideale comunque sia eravamo arrivati a discutere il fatto che questa luce una volta che viene prodotta all'interno del materiale deve essere in qualche modo guidata verso uno fotosensore perché a questo punto il problema un po come avveniva nei rivelatori angassa se vi ricordate si producevano delle 

cariche il problema era raccogliere queste cariche da luogo a un segnale elettrico a seguito della formazione di queste cariche qui il problema è è po diverso abbiamo una provizione di fotoni ma alla fine dobbiamo sempre produrre un segnale elettrico che possiamo inviare a un'opportuna elettronica e raccogliendo questi fotoni e quindi la prima cosa che si fa è cercare di convogliare i fotoni verso una porzione della superficie del rivelatore dove si posiziona un fotosensore e vi dico che appunto il fotosensore non aumenta dimensioni più piccole della superficie del rivelatore e quindi è necessario ad esempio guidare attraverso una riflessione i fotoni prodotti all'interno questo si fa se vi ricordate avvolgendo il materiale scintillatore con dei materiali riflettenti ad esempio del dei materiali simili alla carta alluminio alluminizzati oppure anche utilizzando dei materiali che provano la diffusione della luce quindi ad esempio dei nostri bianchi delle vernici bianche vanno bene lo stesso sono molto utilizzati anche in questo questo alla fine è chiaro che ogni fotone seguirà un percorso che dipende dalla punto in cui è stato emesso e dalla direzione iniziale con cui è stato emesso quindi in linea di principio anche se il percorso dal punto di emissione al fotosensore magari di pochi centimetri poche decine di centimetri in realtà se il fotone comincia a assumire delle riflessioni come in questo caso può percorrere anche distanze molto più grande rispetto a quanto previsto e questo comporta ovviamente un aspetto negativo cioè il fatto che questi fotoni possono essere persi senza essere rivelati quindi l'avevamo detto la volta scorsa ne venvano prodotti parecchi di fotoni ma in realtà soltanto una piccolissima percentuale riesce a raggiungere il fotosensore quindi per questo motivo i fotosensori che discuteremo oggi devono essere dei fotosensori intanto che sono sensibili anche a radiazioni poco intensi quindi pochi fotoni e in più devono essere anche in grado di amplificare il segnale quindi non possiamo mettere in qualsiasi fotosensore accoppiato alla scintillatora prima di andare a discutere i fotosensori volevo farvi vedere una galleria di immagini avevamo anche detto appunto che a volte gli scintillatori devono essere scintillatori devono essere accoppiati il fotosensore per questioni geometrica e quindi nasce l'esigenza di costruire delle guide di luce che sfruttano la riflessione totale dei fotoni e che convogliano che guidano la luce verso la superficie del fotosensore e avevamo concluso sono mi sbaglio la volta scorsa parlando di fibre wavelength shifter che sono appunto delle fibre che hanno preso piede nell'arco dell'ultima decina di anni e hanno il vantaggio di assorbire la radiazione che le colpisce e di riametterle e riamettere

questa radiazione verso una in una ezzadonda caratteristica della fibra quello che vi avevo portato in aula se vi ricordate era una fibra che emetteva sul verde quindi riceveva la luce che colpiva anche la superficie laterale della fibra veniva assorbita questa radiazione ad esempio vediamo qui lo spettro di assorbimento che ad esempio copre un determinato intervallo di lunghezzadonda e poi la luce viene riemessa a una lunghezza d'onda diversa shiftata spostata magari spostata verso una lunghezzadonda che va bene si accoppia bene con il sensore che noi andremo a utilizzare il fotosensore ma la cosa interessante è che appunto la luce è riemessa viene guidata all'interno della fibra verso l'estremità della fibra dove possiamo andare a posizionare un fotosensore e avevo detto che queste fibre oltre a essere molto molto cattivanti molto belle come aspetto sono molto utili quando si adoperano dei rivelatori di grandi grandi superfici quindi dei rivelatori in cui la luce viene prodotta su una superficie molto estesa e quindi bisognerà andare a ricoprire in diversi punti la superficie con dei fotosensori per essere sicuri di poter raccogliere tutta la luce prodotta e avere un'efficienza abbastanza uniforme del nostro rivelatore immaginate ad esempio di avere un rivelatore delle superfici di questo tavolo è chiaro che se io nasse ad accoppiare un solo fotosensore magari a un estremità di questo di questo tavolo è chiaro che la luce prodotta all'altra estremità potrebbe essere persa perché i fotoni dovrebbero arrivare attraverso diverse riflessioni verso il mio fotosensore quindi avrei un'efficienza non uniforme magari riesco a vedere bene le particelle che deposita una energia in prossimità del fotosensore ma tutta l'altra zona del mio rivelatore potrebbe presentare efficienza abbastanza bassa quindi ci possono essere diverse soluzioni a questo problema o utilizzare più fotosensori oppure ad esempio ad operare delle fibre web and shifter magari inserite all'interno di solchi scavati nel scintillatore per poter guidare la luce verso l'estremità della fibra dove si vanno a posizionare questi fotosensori rivelatori basati sul scintillatore vengono adoperati in diversi campi chiaramente nel campo della ricerca soprattutto fisica nucleare e astroparticellare ne vedremo alcuni esempi ad esempio per la costruzione di calorimetri vi ricordo che il calorimetro è tipicamente un rivelatore che in grado di misurare l'energia di una particella oppure rivelatori per raggi cosmici ma anche nel campo della fisica applicata vedremo anche qui alcuni esempi o applicazioni mediche qui vediamo non ve l'ho portato ma con quello vedrete poi in laboratorio un tipico esempio di un fotomoltiplicatore appoppiato una scintillatore anzi ne ho portato una volta scorso effettivamente io avevo detto appunto che lo scintillatore era solamente una porzione di quello che vedevate tutti il resto che cos'era era una guida di luce è un fotomoltiplicatore che questo caso era il fotosensore quindi tipicamente voi vedrete sempre uno scintillatore insieme un fotosensore che può essere più o meno meno a seconda del tipo di fotosensore che è stato scelto quindi in questo caso questa parte cilindrica a destra costituisce²isci lo scintillatore accoppiato a un fotomoltiplicatore che vedete sulla sinistra di cui parleremo breve chiaramente si possono anche realizzare delle strutture che comprendono più scintillatori quindi ad esempio ad esempio esempio degli array come quello mostrato in figura dove abbiamo tanti scintillatori che puntano verso una ben precisa porzione di spazio, dove magari avviene una reazione da cui vengono emesse delle particelle, quindi si realizza una corona di rivelatori, in questo caso di scintillatori, che circondano il punto di interazione analogamente. Qui vedete, siccome sono dei materiali, come vi ho detto, che possono essere lavorati facilmente, possono anche assumere delle forme caratteristiche adatte al tipo di misura che si deve effettuare. Poi ogni scintillatore ovviamente ha associato al suo fotomoltiplicatore, quindi quindi sono tantissime geometrie possibili. Oltre al campo della ricerca vi dicevo ci sono anche delle applicazioni, soprattutto in campo medico. Ad esempio gli scintillatori vengono adoperati in esami diagnostici come la PET, la Positron Emission Tomography. Quest'esame diagnostico consiste sostanzialmente nell'andare a misurare dei raggi $\gamma$ che vengono emessi da un determinato punto e vengono emessi in coincidenza, vengono emessi simultaneamente in direzione opposte. Come funziona la PET? Lo scopo, non so se abbiamo avuto un modo di fare una PET conoscete queste esame, consiste proprio nell'andare a individuare l'eventuale presenza di cellule cancerogene, che sono tipicamente delle cellule che hanno un elevato consumo di glucosio. Quindi nel paziente viene iniettato un liquido che contiene il punto del glucosio e un elemento tracciante, un isotopo che tipicamente mette dei positroni e questi positroni una volta messi si ammichilano e danno un luogo a due $\gamma$ a 180 gradi, quindi due gambe identici emessi a 180 gradi. Quindi sostanzialmente questo tracciante si concentra nelle zone all'alto metabolismo e quindi andando a ricostruire questi $\gamma$ che vengono emessi si può andare a individuare delle zone dove c'è una maggiore emissione, quindi dove si è andato a concentrare questo tracciante. E chiaramente per ricostruire due $\gamma$ a 180 gradi bisogna avvolgere in qualche modo con una corona di rivelatori il paziente e quindi in una PET si presentano delle strutture ad arco che sono costituiti da scintillatori in grado di andare a misurare questi $\gamma$. In maniera tale appunto che quando si osserva un segnale in coincidenza quindi simultaneamente un segnale in una coppia di scintillatori opposti si può andare a ricostruire quella che è la direzione di arrivo dei $\gamma$ in qualche modo quindi andando a tracciare tutti questi $\gamma$ si può andare a individuare la sorgente da cui vengono emessi questi $\gamma$. Vedete appunto lo schema della PET con questa tipica corona costituita da fotomoltiplicatori. Ora questi scintillatori fino adesso comunque nelle PET tradizionali di prima generazione erano accoppiati a fotomoltiplicatori, quindi dei fotosensori che ora andremo a vedere nel dettaglio è che avevano però lo svantaggio di non poter lavorare all'interno di campi magnetici. Vi farò vedere la struttura di un fotomoltiplicatore e capirete perché non possono lavorare all'interno di un campo magnetico quindi immaginate di voler invece sviluppare un apparato di agnostico in grado di andare a effettuare una PET all'interno di un campo magnetico perché magari si associa a una una magnetica su ogni uomo e chiaro che questo diventa limitante e per questo motivo si è pensato di andare a sostituire i fotomoltiplicatori con nuovi fotosensori di ultima generazione basati magari su materiali assatospolido quindi ci sono sassate nel corso degli anni delle evoluzioni anche di questa tecnica di agnostica e questo appunto era un tipico esempio di scintillatori di campo medico in questo caso. A questo punto andrei a parlare dei fotosensori perché è l'ultima parte che ci rimane per capire appunto tutta questa catena di riperazione prima di andare avanti ci sono domande sulla parte degli scintillatori anche da fuori eventualmente. Ok, non è è no? E allora andiamo a guardare alcuni esempi di fotosensori partendo da quelli tradizionali, quelli che sono sviluppati solitamente proprio per andare a acquisire la luce e messa da uno scintillatore per passare a rivelatori più moderni che trovano anche diverse applicazioni quindi come vi ho già accennato i fotosensori che devono essere accoppiati a uno scintillatore devono essere dei fotosensori estremamente sensibili quindi in grado di produrre un segnale elettrico anche quando vengono colpiti da pochi fotoni. Addirittura si parla certe volte di rivelatori a singolo fotone quindi anche il singolo fotone può duar luogo a un segnale elettrico rivelabile. Qua li andremo a vedere, vi dicevo i fotomoltiplicatori ma anche rivelatori di più recente produzione quindi in particolare gli avalanci foto di avalanga individuati dalla sigla APD e i silicon photomultipliers che come vedete sono dei fotomoltiplicatori però prodotti realizzate in silice quindi stato solido il materiale solido. Allora cominciamo con il fotomoltiplicatore già questa slide l'avevamo vista comunque questa animazione l'avevamo vista quando avevamo parlato in generale gli scintillatori infatti vi avevo detto che lo scintillatore deve essere sempre accoppiato a un fotosensore e in questa animazione il fotosensore che è mostrato è proprio un 

fotomoltiplicatore anche se l'animazione è molto veloce si può già capire grosso modo non solo lo schema di funzionamento del fotomoltiplicatore ma anche i vari elementi che lo costituiscono quindi la parte in giallo rappresenta il nostro scintillatore vedete arriva una radiazione individuata questa freccia interagisceombi questa radiazione in un punto viene messa della luce in questo caso si segue in percorso di un singolo fotone ma in realtà ce ne possono essere diversi il fotone o attraverso delle riflessioni multiple o perché viene messo in la direzione giusta arriva su questo fotosensore il fotosensore vedete ha una struttura particolare in anzi tutto una finestra di ingresso che prende il nome di fotocatodo il fotocatodo non è altro appunto che una superficie tipicamente vetrosa che viene rivestita di un materiale con un basso potenziale di estrazione e quindi un materiale dove può aprire facilmente effetto fotoelettrico in questo modo quindi se incide un fotone dal luogo effetto fotoelettrico e viene messo un elettrone questo elettrone viene guidato verso questa zona dove sono presenti vedete questi piccoli archi che sono degli elettrodi che prendono il nome di dino di eccellenso un certo numero come vennero guidati chiaramente attraverso un campo elettrico quindi questi dino di vennero posizionati a un potenziale via via crescente quindi ogni dino d'o ha un potenziale maggiore rispetto al dino d'o precedente ok quindi questo che vedete il primo dino d'o il dino d'o successivo ha un potenziale maggiore in maniera tale che gli elettroni vengono guidati verso questa catena di dino di vedete anche uno schema con un partitore di tensione per far capire appunto che l'operatore applica una tensione di alimentazione questa tensione viene suddivisa tra i diversi dino di quindi non so si applicano mille volte ci sono 10 dino di e abbiamo una suddivisione di 100 volte al dino d'o tuttavia l'elettrone che viene generato per effetto fotoelettrico non appena incide sul primo dino d'o vedete permette l'emissione di un certo numero di elettroni quindi viene ceduta dell'energia e questa energia serve per estrarre altri elettroni dal primo dino d'o questi elettroni a questo punto vengono accelerati verso il secondo dino d'o e ognuno di questi elettroni dal luogo lo stesso modo a altri elettroni quindi con un meccanismo di moltiplicazione quindi a ogni dino d'o un elettrono incidente può estrarre un certo numero di elettroni ora vedremo quantificeremo anche quanti elettroni vengono emessi e questo avviene chiaramente lungo tutta la struttura quindi se guardate un po' questa animazione man mano vedete che il numero di elettroni emessi che viaggia attraverso la catena di dino di ovviamente aumenta attraverso una legge di potenza fino a quando arriva all'ultimo dino d'o che non è altro che l'anodo da cui viene prelevato poi il segnale chiaramente segnale in carica è una corrente attraverso l'utilizzo di una resistenza si produce una un segnale intenzione che è quello che poi noi adoperiamo e possiamo visualizzare lo sceloscopio oppure possiamo inviare a un andc per andarne a misurare l'ampiezza quindi è un segnale che l'ina di principio dovrebbe essere proporzionale al numero di elettroni che sono stati generati da questo fotomoltiplicatore e questo numero di elettroni dipende da quanti fotoni hanno inciso sul fotone quindi da quanti fotoni sono stati prodotti durante il processo di 

scintillazione quindi diciamo che è tutta una catena che permette di mantenere una certa proporzionalità e quindi di avere un uscita un segnale che mi dà indicazioni sulla energia depositata all'interno dello scintillatore il fotomoltiplicatore proprio per questo motivo è un oggetto abbastanza esteso non è un oggetto compatto lo abbiamo visto anche prima proprio perché dobbiamo avere uno schema di dinodi che addirittura presenta delle geometrie particolari perché capite questi elettroni devono essere guidati verso i dinodi quindi non posso posizionare i dinodi in qualsiasi punto e devono avere anche delle forme caratteristica ecco perché appunto il nome fotomoltiplicatore perché non solo è un rivelatore di fotoni ma in più è un moltiplicatore quindi da un singolo elettrone che corrisponde alla rivelazione di un fotone si produce una cascata di elettroni verso il dino do con un certo fattore di moltiplicazione allora ricordiamo un po' più nel dettaglio i diversi componenti di questo fotomoltiplicatore partendo innanzitutto dal fotocatodo vi dicevo il fotocatodo presenta sostanzialmente la finestra di ingresso al fotomoltiplicatore quindi è la zona dove devono incidere i fotoni di scintillazione normalmente realizzato appunto su un materiale vetroso che viene ripestito di un materiale caratterizzato appunto da un un basso lavoro di estrazione normalmente vengono utilizzati dei materiali semi conduttori per il semplice fatto che gli elettroni che vengono emessi per effetto fotolettrico riescono più facilmente a lasciare il fotocatodo a foriuscire e chiaramente foriescono con un energia che dipende dalla energia del fotone incidente e dal potenziale di estrazione se vi ricordate il perfetto fotolettrico appunto il fotone che incide cede la sua energia e questa energia serve per estrarre l'elettrone quindi l'elettrone avrà una energia cinetica residua che è data dalla differenza tra l'energia del fotone incidente e il potenziale di estrazione però qua stiamo parlando ovviamente di fotoni di scintillazione che hanno energie di tre elettron molto grosso modo abbiamo detto cadono nell'uv i materiali che si adoperano in questo caso hanno un potenziale di estrazione dell'ordine di un volte e mezzo due volte quindi comunque l'elettrone che viene messo a un'energia abbastanza bassa veramente pochi elettro volt dobbiamo ricordarci inoltre che essendo appunto un materiale con così basso lavoro di estrazione potrebbe verificarsi anche un'emissione spontanea perfetto termico

chiaramente a causa della temperatura ad esempio a temperatura ambiente gli elettroni hanno un'energia media di 0,05 elettron volt e quindi quello che potrebbe avvenire che chiaramente per questioni termiche alcuni questi elettroni potrebbero effettivamente essere messi perfetto termico e diciamo noi non avvertiamo alcuna differenza tra l'elettrone che viene messo perché è arrivato è inciso un fotone o l'elettrone che viene messo per effetto termico quindi per noi sono la stessa cosa sostanzialmente entrammi deranno luogo poi a tutti i processi che abbiamo appena visto lungo tutta la catena di dinodi e capite che questo quindi rappresenta un problema perché rappresentano la fonte di rumore si va a sommare sostanzialmente al reale segnale fisico dovuta alla rivelazione di fotoni di scintillazione quale potrebbe essere la soluzione ma ora lo vedremo comunque più in la onde parleremo del rumore una soluzione banale potrebbe essere quello di abbassare la temperatura e quindi operare a temperature più basse per ridurre la probabilità di emissione di elettroni per effetto termico in questa figura vedete qui in particolare la parte relativa al fotocatodo quindi il fotocatodo dovrebbe accoppiarsi allo scintillatore e realizzato in un materiale vetroso proprio per assicurare un indice di rifrazione simile a quello dello scintillatore o delle eventuali guida di luce che si utilizza per accoppiare lo scintillatore col fotosensore in maniera tale da evitare fenomeni di rifrazione eccessivi vi dicevo appunto che questo effetto termico non è del tutto trascorabile infatti a temperatura ambiente abbiamo questa energia media per gli elettroni e appunto una certa frazione di elettroni può avere un'energia sufficiente quindi superiore a questo valore di energia media per poter scuggire al materiale in generale per dirvi appunto che non è un fattore proprio totalmente trascurabile dovete immaginare che a temperatura ambiente la frequenza di emissione quindi quanti elettroni vengono le messi al secondo a temperatura ambiente nei metalli corrisponde circa 100 elettroni al secondo per ogni metro quadro i materiali semiconductor è anche più elevato 10 alla 6 e 10 all'8 al secondo per metro quadro quindi se andate a prendere la superficie del fotocado possiamo fare un conto di quanti elettroni vi aspettate vengono emessi per ogni secondo per effetto termico quindi non è per niente trascurabile e l'effetto di questa emissione provoca quindi un rumore una corrente possiamo dire di elettroni che prende il nome di dark current perché capite questa è una corrente che è sempre presente anche quando non incide luce sul fotosensore quindi in condizioni di dark di oscurità e purtroppo questo è uno dei parametri che si deve tener conto in un foto moltiplicatore perché rappresenta qualcosa che si insomma al mio segnale un altro aspetto che si deve tener conto quando si sceglie il fotocatodo riguarda l'efficienza di rivelazione perché è chiaro che noi abbiamo schematizzato il tutto dicendo incide la radiazione produce effetto fotoelettrico ma in realtà la probabilità con cui avviene questo effetto fotoelettrico dipende dalla lunghezza d'onda della radiazione incidente e quindi in base al tipo di materiale adoperato possiamo andare a definire quanto vale la quantum efficiency che rappresenta quindi il numero di fotoelettroni emessi su numero di fotoni incidenti quindi se per caso incidessero 100 fotoni di scintillazione e in corrispondenza a ogni fotone di scintillazione ottengo un elettrone o fotoelettrone allora l'efficienza sarebbe al 100\% è chiaro che non abbiamo efficenze così elevate ma abbiamo efficenze più basse infatti e tipicamente ci giriamo intorno al valore di 20-30\% quindi in media 2-3 fotoni su 100 riescono a produrre effetto fotoelettrico e vedete che questa questa quantum efficiency è fortemente dipendente dalla lunghezza d'onda ad esempio in questa figura vedete alcuni esempi di materiali diversi utilizzati per i fotocatodi e la loro risposta in funzione della lunghezza d'onda quindi ci sono fotocatodi che sono 

particolarmente ottimizzati per andare a lavorare nel profondo uv e altri invece che lavorano a lunghezza d'onda più elevate dipende ovviamente dal tipo di scintillatore che state adoperando quindi la scelta del fotocatodo dipende fortemente dallo scintillatore che adoperate e dallo spettro di emissione di questo scintillatore quindi dovete cercare ovviamente di far corrispondere queste due finestre di lavoro l'emissione dello scintillatore e l'assorbimento possiamo dire del fotocatodo a questo punto una volta che sono stati ammessi questi elettroni abbiamo detto gli elettroni hanno energie molto basse e dobbiamo cercare di convobilarli verso il primo dinodo che in questa figura vedete è posizionato in questa zona questa è la finestra del fotocatodo queste sono le diverse diversi punti in cui può avvenire un'emissione perché chiaro che i fotoni ci dono su tutta la superficie del fotocatodo e quindi gli elettroni possono essere messi da diversi punti e pergiunte in diverse direzioni e con energia leggermente diverse che cosa vorremmo vorremmo che la raccolta sia una raccolta efficiente quindi tutti gli elettroni indipendentemente dalla loro energia indipendentemente dal punto in cui sono emessi vorremmo raggiungessero il primo dinodo ma per fare questo li dobbiamo in qualche modo guidare ecco perché questa prima regione presenta può presentare anche degli elettroni di focalizzazione quindi oltre ai dinodi si aggiungono degli elettroni ad esempio vedete qui sono mostrate degli elettroni di piani proprio per accelerare gli elettroni fargli seguire determinate determinati percorsi in base al punto in cui sono emessi e convogliarli verso il primo dinodo infatti allo di là di una questione di efficienza quindi di cercare comunque sia di convogliare tutti gli elettroni c'è un problema anche lo capiamo di timing perché capite che gli elettroni che benvano emessi nelle regioni più periferiche devono percorrere necessariamente un spazio maggiore per arrivare al primo dinodo e quindi rischierei di avere degli elettroni che arrivano dopo e questo comporta una indeterminazione dal punto di vista temporale che ovviamente qualcosa da evitare quando voglio adoperare uno scintillatore un fotomoltiplicatore per misure di timing con questo questo aspetto la profondiremo più là e quindi questa prima zona è una zona molto importante in un fotomoltiplicatore perché ci devono essere dei campi elettrici e addirittura alcune volte utilizzano anche dei campi magnetici proprio per ottimizzare focalizzare gli elettroni verso il primo dinodo a questo punto cerchiamo di capire cosa avviene nei vari dino di l'abbiamo già accennato prima ogni elettrone in grado di emettere ulteriori elettroni quindi elettroni vi diceva che vengono emessi con energia molto basse tipicamente dell'ordine dell'elettron volt e ogni dinodo è posto a un potenziale la differenza di potenziale di circa 100 volt rispetto a precedente in maniera tala al al che gli elettroni vengano guidati verso i dinodi successivi ora quando incide un elettrone lo vedete in questa in questa animazione che però è ve ve oscurata dalla barra sotto forse così lo vedete meglio vedete il tutto parte da un singolo elettrone l'elettrone incide sul dinodo e a quel punto può produrre ovviamente stato accelerato quindi acquisito una certa energia e a questo punto può estrarre dal primo dinodo un certo numero di elettroni e quanti ne estrae? Nel estrae all'incirca una trentina quindi ogni volta le incide un elettrone vemmeno emessi perfetti secondari una trentina di elettroni e questi elettroni devono essere accelerati verso il secondo dinodo ma in realtà non tutti questi elettroni riescono ad arrivare al secondo dinodo perché capite c'è anche qui un'efficienza di raccolta e di picamente solamente una frazione di questi elettroni arriva al secondo dinodo questa frazione mi porta a dire che all'incirca 5 elettroni su 30 riescono effettivamente ad arrivare al secondo dinodo quindi considerato questa efficienza il primo elettrone da era l'uogo in media a 5 elettroni che arrivano al secondo dinodo questo numero che io ho visto dicendo che normalmente intorno a 5 lo chiameremo delta ora vedremo che è importanza in termini di formazione del segnale quindi dovete immaginare che arrivano delta elettrone sul secondo dinodo e ogni uno di questi elettroni darà l'uogo nel terzo dinodo ad altri elettroni e così via allora c'è un ovviamente si evidenza un meccanismo di moltiplicazione lo vedete

anche da queste linee che diventano via via sempre più numerose ma manovesi passa da un dino al successivo quanto vale il fattore di guadagno cioè partendo da un solo elettrone alla fine su l'ultimo dinodo sull'anodo arriva un certo numero di elettroni quindi quanto vale il rapporto tra il numero di elettroni raccolte all'anodo rispetto al numero di elettroni prodotti dal catodo questo rapporto prende il nome di guadagno quindi mi dice sostanzialmente di quanto il fattore moltiplicativo di quanto ho guadagnato in termini di elettroni se parto da uno quanti ne ottengo alla fine e questo guadagno che in questo caso è indicato con G a volte trovate indicato con M dipende dal test consultate se vi fate il conto ogni volta appunto ogni elettrone amplifica di un fattore delta il numero di elettroni quando incide sul dino do questo guadagno sarà dato da questa formula alfa per delta inalzato a n dove n rappresente il numero di dino di questo numero dipende ovviamente da cambio la fotomoltiplicatore a fotomoltiplicatore però tipicamente potrebbe essere dell'ordine della decina che sopra per avere avere alfa invece è un fattore che assume grosso modo il valore uno e allora cerchiamo di capire quanto vale questo guadagno del fotomoltiplicatore se ad esempio abbiamo dieci dino di quindi dieci stadi di moltiplicazione alfa vale uno e delta vale cinque quindi da ogni elettrone incidente sul dino do se ne strabbero il cinque che riescono ad arrivare al secondo dino do allora le guadagno dato da questa formula e quindi vale a cinque inalzato a dieci cioè un valore di dieci alla sette quindi da un singolo elettrone nonostante le abbiamo persi parecchie perché abbiamo detto di 30 ne arrivano cinque quindi nonostante queste perdite riusciamo ad amplificare il numero di elettroni di un fattore dell'ordine di dieci milioni quindi tipicamente i fotomoltiplicatori hanno guadagni dell'ordine di dieci alla sei dieci alla sette sono numeri considerevoli quindi fa effettivamente il suo dovere anche se il segnale è molto debole riesce a produrre comunque sia un uscita un segnale in pensione abbastanza elevato grazie a questo fattore di moltiplicazione il fattore delta vi dicevo ha una sua importanza noi abbiamo fatto l'esempio di delta o la cinque ma in realtà è un valore medio capite che ogni volta di volta in volta questo numero di elettroni che viene messo che riesce ad arrivare al dino do successivo può cambiare su di scelere le fluttuazioni e queste fluttuazioni possiamo immaginare seguano la distribuzione di qua son ok quindi in media magari ne ho cinque però a volte capita che vedono emessi 4 elettroni che arrivano al dino do successivo oppure un numero maggiore sei sette chiaramente con una probabilità che può essere descritta dalla distribuzione di qua son e con una deviazione standard che se vi ricordate la distribuzione di qua son la possiamo immaginare come la radice di delta quindi se io dico che in media ne mettono emessi cinque in realtà sto dicendo cinque più o meno meno radice di cinque come deviazione standard e questo fattore delta ha un'importanza notevole quindi sulle fluttuazioni statistiche quindi il valore di delta è importante per capire anche cosa tendermi nelle fluttuazioni del segnale finale perché alla fine il segnale finale vi dicevo è dato da delta inalzato a n quindi una potenza di delta e quindi possiamo ragionare in termini di fluttuazioni statistiche sia ad esempio delta è uguale a cinque possiamo andare a guardare la distribuzione del numero di elettroni emessi dal primo dinodo allora questa distribuzione vedete cambia a seconda che si va a considerare appunto delta o alla cinque delta o alla quattro delta o alla tre chiaramente più è grande delta più queste fluttuazioni sembrano essere grandi cioè queste 

distribuzioni sono larghe ma in termini relativi effettivamente le fluttuazioni statistiche in termini relativi diminuiscono questo è appunto quello che avevamo detto anche nel caso della distribuzione di possono quindi se vengono emessi in meia cinque elettroni o ne vengono emessi venticinque quello che io mi aspetto è che le fluttuazioni statistiche in termini relativi chiaramente diminuiscono quando vado a considerare una emissione maggiore quindi un delta più grande e questo è una sua sua perché alla fine il segnale che si produce all'anodo tanto più elevato tanto più è probabile che si possa distinguere da un eventuale segnale di rumore perché immaginate ad esempio sul fotocatodo incidono dieci fotoni di questi maveri non li riveliamo tutti ne riveliamo una parte perché abbiamo quella quanto efficienzi di cui abbiamo discorso e quindi in uscita abbiamo un numero di foto elettroni che potrebbe essere due tre che danno luogo a un certo segnale a seguito di questo guadagno di cui abbiamo parlato questo segnale che si produce devo provare a distinguere da eventuali segnali di rumore tolvuto ad esempio e questioni termiche quindi che danno luogo a un segnale che tipicamente è molto frequente ma ha un'ampiezza piccola perché viene emesso un elettrone che da luogo a questa catena la giovitria ed edino di la disposizione è fondamentale perché vi dicevo questi elettroni devono essere guidati da un dino dall'altro e quindi in realtà ci sono tantissime configurazioni che sono state studiate nell'arco della storia in particolare quelle più utilizzate sono di quattro tipi che sono che che che messaggi relativi a zoom sono legati e sono quelli più utilizzati sono questi che vengono mostrati qui vedete a veneziana a box e a a focalizzati di niermento focalizzati circolarmente questi corrispondono a queste geometrie che vedete qui sulla destra non entraiamo nel dettaglio comunque quella che abbiamo visto fino adesso è quella di tipici quindi focalizzati linearmente che ha diversi diversi vantaggi ad esempio un aspetto che si può andare a cercare di capire in base alla configurazione scelta e sono le devienzioni dalla linearità quindi io mi aspetto che questo segnale vi dicevo mantengo una certa proporzionalità rispetto all'energia che viene depositata ma in realtà questo non è sempre vero a volte se il flusso di elettroni è molto elevato si può perdere la linearità quindi finché in questo grafico ad esempio la linea di devienzione della linea di sono pari a zero chiaramente stiamo lavorando in un ottimo regime di lavoro e questo ad esempio sono le curve che si ottengono per queste quattro configurazioni però non non non non non corrispondono questa abici di sono mi sbagliano corrispondono queste quindi ho riportato anche qui sotto la la descrizione del grafico che si legge appena perché l'abbiamo a barra davanti comunque sia quella migliore quella individuata dalla di corrisponde proprio a i i fotonottipicatori focalizzati linearmente infatti vedete che in una ampia regione di lavoro le devienzioni dalla linearità sono lo 0 per 100 ecco perché vi dicevo questa configurazione è una delle più adoverate effettivamente mentre le altre vedete riesco a mantenere le devienzioni della linearità dello 0 per 100 soltanto per basse correnti quindi basso numero di elettroni che viene prodotto e raccolta al lano do come avviene l'alimentazione vi dicevo banalmente si applica una un'alta tensione al tmt quindi voi il laboratorio ad esempio vi ritroverete all'applicare una tensione di 700 volt 700 volt ma in realtà poi questa tensione viene suddivisa tra i diversi dinodi andando a utilizzare dei partitori di tensione in generale vi dicevo c'è una particolare 

attenzione riguardo il primo dino do quindi ciò che viene tra il fotocato del primo dino do proprio perché il punto in cui noi dobbiamo andare a raccogliere in maniera più efficiente gli elettroni che vengono emessi dal fotocato do ci sono due modalità equivalenti di lavoro o si lavora con un fotocato a potenziale negativo e lano do a zero oppure viceversa il fotocato do zero e lano do ha una tensione positiva sono due modi diversi di lavorare ma alla fine l'importante è la differenza di potenziale che si deve passare da un potenziale più basso un potenziale più alta alla fine andiamo a guardare un altro aspetto quindi questa tensione di alimentazione del fotomoltiplicatore che influenza a sul segnale finale questo fattore delta che vi dicevo in realtà io vi ho ho è grosso modo 5 ma può cambiare a seconda della tensione di lavoro più e alta la tensione di lavoro più gli elettroni verranno accelerati tra un dino dell'altro quindi raggiungeranno energie più alte maggiore sarà il numero di elettroni emessi quindi di elettroni che riescono ad aggiungere al dino do successivo quindi questo fattore delta cambia a seconda della tensione di alimentazione allora se immaginiamo di indicare con vi condì la differenza di potenziale tra due dino di consecutivi delta è dato da k per vi condì dove k è una costante di proporzionalità e vi condì questa tensione e allora il guadagno che l'avevamo espresso come alpha per delta inalzato a n si può andare a riscrivere in termini di tensione alpha per k v inalzato a n dove n è il numero di dino di quindi abbiamo una legge di potenza il guadagno dipende dalla tensione non in modo lineare ma attraverso una legge di potenza v inalzato a n Quindi basta anche una leggera variazione nella tensione di lavoro per avere una variazione nel guadagno notevole proprio a causa di questa potenza all'N. Ecco perché gli alimentatori che si adoperano per alimentare i fotomoltiplicatori devono essere degli alimentatori abbastanza stabili, quindi che mantengono il valore di tensione il più stabile possibile. Ma possiamo essere interessati a cercare di capire quanto varia il guadagno quando variamo la tensione di un volt. E allora questa variazione che normalmente si esprima in termini percentuali prende il nome di coefficiente di guadagno ed è qualcosa che andrete a misurare il laboratorio. Infatti misurare il guadagno del fotomoltiplicatore non è un'operazione facile perché per misurare il guadagno voi dovreste lavorare in condizioni di singolo fotoelettrone, quindi mettervi nelle condizioni di far produrre un fotocato di un sole elettrone e vedere cosa si ottiene all'anodo. E questo non è qualcosa di semplice, dovete avere un sistema calibrato, quindi la misura del guadagno non è banale. Però si può studiare l'andamento del guadagno in funzione della tensione e quindi valutare queste variazioni percentuali del

guadagno al vadeare della tensione. Vedete quello che farete il laboratorio, quindi proverete ad acquisire, ad effettuare delle misure con un fotomoltiplicatore, aumentando di volte in volta la tensione, esplorando un certo intervallo di tensioni e andando a cercare di capire quanto cambia il vostro segnale in uscita. Il segnale cambierà ovviamente la sua ampiezza perché aumentando la tensione aumenta il guadagno, quindi aumenta il numero di elettroni che arriva all'anodo e il vostro segnale aumenterà in ampiezza. Quindi voi possiamo studiare di quanto aumenta o diminuisce il vostro segnale quando cambiate la tensione di un certo valore. E quindi possiamo studiare la derivata dell'ampiezza del segnale in funzione della tensione, ritrovare una legge di potenza di questo tipo e valutare il coefficiente di guadagno, che è qualcosa che vincevo, dovete fare in laboratorio e lo farete appunto in questo primo turno di esperienza. Chiaramente il coefficiente di guadagno viene espresso in percentuale su volt. Andiamo a guardare un altro aspetto del fotomultrificatore e cioè la risposta temporale. Allora tipicamente gli elettroni vengono emessi con tempi molto rapidi, un decimo dinano secondo, grossomodo. Tuttavia la cosa che conta di più nella produzione del segnale in termini temporali è il tempo che impiegano questi elettroni per passare dal catodo fino all'anodo. E questo è un tempo che non è trascurabile ed è l'ordine delle decine di l'anno secondi. Quindi per percorrere tutto quel percorso tra i dinodi è necessario un intervallo temporale di circa dieci nanosecondi. Forra se questo tempo fosse fisso, fosse costante, fosse sempre lo stesso ogni volta che viene messo un elettrone dal catodo non avrei alcun problema. Nel senso sarebbe un ritardo noto, non lo fornisce il costruttore quindi so quando effetto una misura di timing che è presente questo ritardo. Quindi da quando incide la radiazione e quando viene rivelata mi aspetto un ritardo fisso. Il problema in realtà è che per diverse ragioni esiste una indeterminazione, una dispersione di questo transit time, di questo tempo di transito che prende il nome di tts transit time spread. Quindi è vero che magari il tempo medio per percorrere il fotomoltiplicatore dell'ordine della decina di nanosecondi però rispetto a questo tempo medio o delle fruttuazioni che possono essere anche dell'ordine di alcuni nanosecondi. E questo chiaramente limita l'utilizzo del fotomoltiplicatore per applicazioni di timing, spinto ovviamente, dove magari voglio risoluzioni temporali particolarmente ottimali dell'ordine e comunque inferiori a nanosecondo. Come si può migliorare questo aspetto? Allora ci sono dei fotomoltiplicatori che magari migliorano questo aspetto con un'opportuna geometria del dinodino di oppure diminuendo il numero di fotoelettroni, quindi lavorando in condizione di illuminazione del fotomoltiplicatore un po' più basse però non sempre è possibile. Un ultimo aspetto che volevo discutere sui fotomoltiplicatori riguarda il fatto che i fotomoltiplicatori devono lavorare lavorando con elettroni di pochi elettron volte l'abbiamo detto e quindi sono elettroni molto poco energetici che possono subire delle deviazioni anche a seguito della presenza del campo magnetico terrestre. Ecco perché i fotomoltiplicatori a seconda di come vengono orientati, in lì e di principio potrebbero funzionare in maniera vengermente diversa perché potrebbero avere un orientamento rispetto al campo magnetico terrestre diverso secondo di come viene posizionato il fotomoltiplicatore. Per evitare questi aspetti, questi problemi, quello che si fa è circondare il fototubo con un materiale che si è in grado di schermare questi campi magnetici non eccessivamente elevati come ad esempio il campo magnetico terrestre e questi

materiali è tipicamente prendere il nome di MuMetal che sono delle levi e metalliche adalta per miabilità magnetica quindi formano un vero e proprio schermo per i campi magnetici, maniera tale che gli elettroni che vengono prodotti all'interno non surviscono effetti di deviazione dovute a questi campi magnetici così poco intensi. Tuttavia capite che certe volte si ha l'esigenza di posizionare i fotomoltiplicatori all'interno dei campi magnetici di valore più elevato perché magari in fisica si vuole provare a misurare l'impulso di una particella deviando il percorso della particella con campi magnetici che possono essere anche molto più intensi rispetto a quelli tipicamente derivanti da fonti naturali. Allora in questo caso i fotomoltiplicatori non possono essere adoperati, ve l'avevo detto anche prima nel caso della PET. Quello che si fa è eventualmente trovare dei degni sostituti del fotomoltiplicatore che ora andremo a vedere e che possono lavorare anche in presenza di campi magnetici. Un aspetto importante riguarda il rumore, il noise già l'avamo visto, la principale fonte di rumore in un fotomoltiplicatore è l'emissione termodionica di elettroni da parte del fotocato, ma anche da parte di altri materiali che costituiscono il fotomoltiplicatore. Tipicamente vi dicevo viene messo un solo foto elettrona alla volta, ecco perché il segnale che si produce un segnale spurio, cioè dovuta rumore, in questo caso un segnale di bassa ampiezza ed è importante quindi avere i segnali fisici invece più elevati in maniera tale da poterli discriminare dal rumore. Può essere diminuito, vi dicevo, abbassando la temperatura, ma questo non lo so, si fa quasi mai. Inoltre un altro aspetto importante sempre in termini di rumore è quello di evitare di esporre il fotomoltiplicatore alla luce anche quando non si sta lavorando col fotomoltiplicatore, quindi immaginate di prendere un fotomoltiplicatore e trasportarlo da una stanza nell'altra dovete avere la cura, la cortezza di evitare l'esposizione alla luce, questo perché i materiali vetrosi che avevano adoperato soprattutto nel cato do, possono emettere luce di fosforescenza anche per tempi molto lunghi, anche nelle ore successive, quindi se per caso il fotomoltiplicatore viene sforza alla luce poi bisogna attendere un tempo che può essere anche dell'ordine delle ore prima di poterlo adoperare, altrimenti potremmo avere effetti indesiderati dovuti proprio questi fenomeni di fosforescenza del vetro. Oltre a questo aspetto poi ci sono altre fonti di rumore in un fotomoltiplicatore che possono essere derivanti ad esempio dalla radiatività del vetro, infatti nel vetro possono essere presenti degli i sottopi radiativi come il potassio 40, al torio che emettono ovviamente radiazione che viene amplificata nel fotomoltiplicatore, radiazione cosmica che incide non solo sul rivelatore ma anche all'interno del tubo oppure ci potrebbero essere correnti di fuga nei supporti degli elettro di 11C, ci vuolano particolare cura nel cercare di isolare gli elettro di presenti in un fotomoltiplicatore. Un altro aspetto che potrebbe verificarsi sono i cosiddetti after pulses, guardate ad esempio questo è un tipico segnale da un fotomoltiplicatore come viene visto in uno oscilloscopio quindi ovviamente quello che si rappresente qui è un asse temporale e l'asse su cui si visualizza il segnale prodotto all'anodo, vedete un segnale molto veloce, la scala non ci aiuta però veramente il tempo di discesse molto rapido dall'ordine del nano secondo, pochi nanosecondi, il segnale risale e dopo che è risalito vedete qui la presenza di piccoli impulsi di bassa ampiezza, questi prendono le nome di after pulses, sono degli impulsi che avvengono dopo il segnale principale e possono avere delle conseguenze importanti soprattutto nelle misure di timing perché le misure di timing ovviamente prevedono un start, uno stop quindi se per caso ci sono degli impulsi spuri lo start e lo stop potrebbero non essere quelli effettivamente desiderati. Le fonti di questi after pulses che sono sempre delle 

ovviamente fonti di rumore per noi possono avere diverse origini o delle reazioni luminose cioè i dino di colpiti dagli elettroni possono emettere dei fotoni che se arrivano al fotocatodo, quindi tornano indietro verso il fotocatodo, possono dare origine ad un effetto fotoelettrico e quindi produrre questi after pulses che sono ovviamente ritardati rispetto all'impusso principale perché dobbiamo considerare il tempo che impiega la luce attorno indietro verso il fotocatodo e i ritardi tipici in questo caso sono dell'ordina di 20, 100 nanosecondi, oppure ionizzazione del gas residui nel fototubo, non ve lo ho detto ma il fototubo, questo tubo che abbiamo visto dove sono inseriti i dino di ovviamente lavora in condizioni di vuoto possiamo dire, però per quanto possa essere spinto il vuoto realizzato all'interno del fototubo sempre sono presenti delle molecole, degli atomi di gas residuo e quello che può succedere è che gli elettroni possono ionizzare questo gas residuo e gli ioni positivi possono migrare verso il fotocatodo perché ovviamente il fotocatodo si trova a un potenziale negativo e liberare elettroni e quindi produrre dei segnali ritardati con ritardi tipici di 100 nanosecondi o anche microsecondi perché dovete immaginare che questo impulso viene prodotto da un ione che ha dovuto viaggiare verso il fotocatodo e quindi ovviamente con una mobilità ridotta oppure potremmo avere effetti di backscattering degli elettroni e quindi durante il processo di amplificazione alcuni di questi elettroni che avevano messi i retino di anziché viaggiare verso il vino del successivo tornano indietro generando un altro segnale ritardato quindi queste sono tutte possibili fonti di after falses che purtroppo sono sempre presenti in un foto moltiplicatore vi dicevo le dimensioni della forma possono essere anche qui molto su variate tipicamente sono oggetti ingombranti in ogni caso hanno normalmente questa forma tubolare a forma di cilindro questo è un tipico fotomoltiplicatore vedete qui la finestra di ingresso il fotocatodo appare proprio come un vetro ma potrebbe avere anche forme un po più particolare ad esempio questo un fotomoltiplicatore di grande area perché sono fotomoltiplicatori che venono ad esempio adoperati nell'esperimento km3 net l'esperimento per la rivelazione dei neutrini che stanno installando il capo passero in quel caso il motivo di utilizzare un fotomoltiplicatore di dimensioni così grandi consiste nel cercare di andare a accattare un quantitativo di luce che sia il più grande possibile perché già di persa la luce prodotta in mare dall'interazione dei neutrini quindi dei muoni è abbastanza bassa quindi bisogna raccogliere il più possibile la luce prodotta vi dicevo ci sono tantissime applicazioni dei fotomoltiplicatori non solo nel campo della della fisica ma anche in medicina e biologia ad esempio una delle applicazioni più note di fotomoltiplicatori nel campo della fisica riguarda l'esperimento supercambio cande non so se ne abbiamo parlato non abbiamo parlato non abbiamo affrontato ancora neutrini niente non abbiamo parlato di neutrini va bene comunque questo è un grosso rivelatore che è stato realizzato all'interno di una miniera abbandonata questa miniera è stata riempita di acqua e le panetti sono state ricoperte vedete da questi fotomoltiplicatori di grande area e il motivo appunto che i neutrini interagendo con l'acqua possono produrre ad esempio muoni che poi emettono ovviamente della radiazione nell'attraversare l'acqua per pareffetto cerenco quindi massiva misurare questa radiazione vedete sono tanti occhi insostanzialmente questo esperimento ha lavorato per diversi anni vedete qui ad esempio una fase di intervento queste sono delle persone su un buon muone che vanno a sostituire comunque a intervenire su alcuni di questi 

fotomoltiplicatori questo esperimento diceva ha preso dati per tantissimi anni diversi anni fa c'è fu un incidente si rupperò praticamente tutti i fotomoltiplicatori attraverso una reazione a cadena cioè si si si è rotto un fotomoltiplicatore e poi sono esposito tutti gli altri comunque ormai mi sembra che non prenda più dati questo estremento oppure ad esempio nel campo del raggi cosmici raggi cosmici nell'attraversare l'attraversfera possono produrre luce di florescenza anche questa è una luce molto debole e utilizzando degli opportuni telescopi di florescenza si può convogliare attraverso un sistema di specchi la luce di florescenza verso un fotosensore che in questo caso è costituito da fotomoltiplicatori ad esempio l'esperimento ger che è il più grande esperimento per la fisica dei raggi cosmici utilizza proprio sistemi di questo tipo oppure al ser non ovviamente ci sono tantissimi rivelatori che utilizzano questo tipo di fotosensori ovviamente e qui il problema è al solito l'utilizzo di campi magnetici e in questa tabella vedete giusto per avermi idea i diversi esperimenti che fanno uso di scintillatori e di fotomoltiplicatori ovviamente sono anche grossi numeri quelli in gioco anche nel campo della ricerca dell'antimateria sulla soluzione spaziale sono installati i rivelatori come ad esempio ams che presenta il termine dei fotomoltiplicatori quindi gli impieghi sono veramente molto ampi anche nel campo della bioluminescenza quindi la rivalazione di luce messi ad organismi viventi anche questo caso siccome la luce messi è molto debole si utilizzano dei fotomoltiplicatori tuttavia i fotomoltiplicatori hanno delle problematiche l'abbiamo detto e anzitutto a volte hanno delle dimensioni troppo grandi rispetto all'area sensibile sono influenzati dai campi magnetici hanno una risposta spettrale che non sempre è adatta alla luce da rivelare una efficienza quantica che abbiamo visto non è città l'ordine di 20 30 per cento la stabilità del guadagno quindi quanto è stabile questo guadagno che abbiamo definito insomma non è ottimale dipende molto dalla tensione inoltre dobbiamo doverare delle tensioni di alimentazione elevate diverse centinaia di volte fino anche al chilo volte quindi ci sono diversi aspetti negativi è proprio per questo motivo nel corso degli anni sono stati sviluppati dei fotossensori più recenti più compatti basati sull'utilizzo di materiale semiconduttore quindi si sono dei fotossensori ha stato solido in particolare oggi vedremo e li avranno scoperto da io si si dicono i propri players i primi prototipi di app di sono stati sviluppati ormai una quarantina d'anni fa inizialmente i primi prototipi avevano dimensioni molto piccole superfici dell'ordine del millimetro quadro inoltre erano particolarmente sensibili all'infrarosso quindi magari non si adattavano quelli scintilatori che mettevano nell'ov avevano un basso guadagno che costa molto davvero sembravano dei rivolatori non particolarmente performanti ma in anni recenti ovviamente sono state migliorate diverse caratteristiche di questi fotossensori quindi adesso abbiamo degli app di che hanno superfici più estese dell'ordine di decine dei millimetri quadri quindi tre per tre millimetri quattro per quattro millimetri insomma sono già dei sensori non superfici più grandi hanno una sensibilità maggiore già nel blu e l'ultravioletto che è quella che ci interessa di più soprattutto per la luce di scintillazione il costo si è abbassato e hanno una guadagno relativamente elevato vi dico relativamente perché comunque sia confrontato con il guadagno del fotomultilitatore che abbiamo detto dell'ordine di 10 alla 6 10 alla 7 qui proprio per il principio di funzionamento dell'app di non si superano guadagni di 100 lo secondo quindi stiamo parlando della decina come il guadagno e hanno l'enorme vantaggio di lavorare con tensioni più massi rispetto a quella dei fotomultilitatori. Venono utilizzati proprio per questo motivo in diverse applicazioni dove sono presenti gli scintillatori. Qual è il principio di funzionamento vedremo un po più nel dettaglio i riferatori sono solido quando parleremo proprio di questa categoria di riferatori per adesso spero che appunto abbiate 

perché conoscenze di base, di fisica, dei semi-conduttori comunque quello che si fa è ad operare dei materiali semi-conduttori però non pure ma drogati, cioè vuol dire che alcuni atomi sono stati introdotti alcuni atomi di un elemento diverso che possono essere elementi donori quindi con un eccesso di elettroni rispetto a un atomo che costituisce voyage cristallo oppure accettori e quindi tipicamente voi troverete materiali drogati di tipo P o di tipo E. Secondo del tipo di elemento che abbiamo utilizzato per drogare i semi-conduttori. Alla no realizzare delle particolari unzioni è possibile creare dei rivelatori con i seguimi conduttori, i rivelatori di particelle non vedremo, ma anche rivelatori di luce. Quindi tipicamente una PD ha una struttura come quella che vedete qui schematizzata, quindi vedete una sequenza di materiali dove abbiamo una parte intrinsega, dove avere intrinsegolo, vuol dire un materiale che non è stato drogato, ma vuol dire un materiale che è stato drogato con la stessa concentrazione di atomi, donori e accentori. Poi abbiamo, vedete, un primostrato drogato di TPP, poi abbiamo un altrostrato drogato di TPP e di TQN. Insomma abbiamo una sequenza di strati dove si viene a realizzare una regione ottimale per la rivelazione, quindi se guardate qui schematicamente la stessa struttura sostanzialmente, quando incide un fotone con energia HN viene prodotta in questa regione, quella regione di rivelazione, una poppia elettrona in lacuna e grazie all'utilizzo di elettro, di vedete qui viene creato una differenza di potenziale tranne o decatodo, questa elettrona emigra, ovviamente viene accelerato e può produrre soprattutto in questa regione P ed N più dei processi di moltiplicazione a valanga, ecco perché avalanci fotondaioz, nuovamente un po' come avveniva per in gas, se vi ricordate, anche qui l'elettrone può arrivare ad avere un'energia sufficiente a estrarre e produrre nuove poppie elettrone lacuna, con ovviamente un guadagno anche in questo caso che vi dicevo non supera il valore di 100, quindi possiamo da un fotone incidente quindi un elettrone prodotto al massimo ottenere un centinaio di elettroni raccolti nel nostro elettro d'ofinale, quindi capito i segnali non menvano amplificati di molto, ma quel che serve per la rivelazione di questi fotoni. Andiamo a guardare alcune caratteristiche di questo APD, allora in azitudine abbiamo un strato anti riflesso in superficie e quindi grazie a questo abbiamo un'elevata efficienza di rivelazione perché la maggior parte dei fotoni coincide riesce effettivamente a essere rivelata. La tensione di alimentazione vi dicevo è più bassa dell'ordine di alcune centinaia di volt, però i guadagni non sono elevatissime, 50-100, grosso modo. Anno inoltre 

come tutti i rivelatori e semiconduttore è purtroppo una dipendenza dalla temperatura, quindi variare la temperatura di lavoro chiaramente comporta delle variazioni nel segnale in uscita, possiamo avere anche qualche per cento di variazione per ogni grado di temperatura, quindi lavorare una temperatura statabile è importante, ma in generale vale per tutti i rivelatori e casse e decomittore. Sono particolarmente adatti per convertire la luce proveniente da Fibre, Wevel e Shifter perché le Fibre, Wevel e Shifter hanno dimensioni piccole, hanno sezioni piccole quindi andare ad accoppiare un rivelatore piccolo è utile in questo caso, o a partire da piccoli scintillatori, chiaro che uno scintillatore di dimensioni grandi non può essere accoppiato come un superficie sufficiente ad assicurare la rivelazione dei fotoni di scintillazione, hanno delle proprietà temporali di timing abbastanza buone e vi dicevo ormai le dimensioni possono arrivare anche a 5 per 5 mm quadri, vedete qui un esempio di APD dove la parte sensibile è questa parte in nero. Il segnale è proporzionale al numero di fotoni incidenti, quindi manteniamo l'informazione sull'energia depositata o comunque sullo numero di fotoni incidenti e vedete qui ad esempio con un APD di questo tipo, giusto per capire le dimensioni, qui abbiamo una monetina non so se un centesimo o docentesimo con una monetina in rame e vedete le dimensioni dei fotodiodi e qui abbiamo un piccolo cristallo come quelli che avevano operato ad esempio per l'aperta che hanno dimensioni veramente piccole, a questo punto potrebbero essere detti tranquillamente con un APD. Una applicazione che abbiamo fatto all'interno dell'esperimento lì dice di cui ci siamo occupati, riguardavo l'utilizzo di fotoni, per andare a leggere da luce raccolta da fibre wavelength shifter, vedete le fibre che vi ho portato una volta scorsa sono esattamente queste e queste fibre corrono all'interno di un modulo che è costituito da strate alternati di piombo e scintillatore che veniva doperato proprio per un calorimetro, in questo caso quindi la luce prodotta nello scintillatore veniva raccolta da queste fibre e le fibre venivano confondinate sulla superficie dell'APD, quindi vedete ci sono diverse applicazioni anche se la superficie non è particolarmente estesa ci possono essere delle soluzioni che permettono di doperare l'APD anche con rivelatori di dimensioni più grandi e qui era fondamentale perché chiaramente si alimenta luce e lavora all'interno di un campo magnetico quindi un fotomultivitore si sarebbe prodotto in caso per l'area. Concludiamo andando a guardare l'ultima categoria di rivelatori un po più recenti di fotosensori in silicon photomultipliers quindi capite il fatto che si adoveri la parola fotomoltiplicatore vi fa capire che evidentemente qui di guadagno è più elevato più simile a un fotomultivitore quindi abbiamo il vantaggio di avere un guadagno elevato ma abbiamo un materiale semiconductor infatti anche in questo caso abbiamo l'utilizzo di un rivelatore che è molto simile all'APD dovete immaginare che un sito è costituito da una matrice di APD di piccolissime dimensioni quindi come se fosse un rivelatore a pixel, un ipixel è una sorta di APD. Ora come funziona? Tutte queste cele che vedete che possono avere diciamo delle dimensioni dalle decine, le centinaia di micron quindi abbiamo densità molto elevate dell'ordine di 1000 pixel per millimetro quadro tutte queste cele lavorano in parallelo quindi se il vione di queste cele viene colpita da un fotone e quindi dal luogo a un segnale io posso andare a sommare tutti questi segnali tra di loro e quindi avere un segnale l'uscita che è proporzionale a quante cele sono state colpite dai fotoni e quindi è proporzionale a quanti fotoni hanno inciso sul rivelatore. Chiaramente tutto questo discorso ha una sua validità finché parte del presupposto che su ogni singola celle incide un solo fotone è chiaro. Se dovessero incidere più fotoni perdo ovviamente questa linearità che è il supposto però normalmente i sito siccome hanno celle di 

dimensioni molto piccole normalmente lavorano in queste condizioni di linearità quindi vedete qui uno schema di questi di questi sito ma dove possiamo vedere le diverse cele ognuna di queste cele ha una sua resistenza di spegnimento di quenching per poter riportare la celle poi con una certa costante caratteristica verso le condizioni di lavoro ottimali e quindi sostanzialmente la vorra come se fosse un rivelatore multiplo quindi tanti rivelatori che lavorano insieme ma alla fine prelevo un solo segnale che è la summa dei segnali di tutte le celle. Con i vantaggi A in anzi tutto lavorare attenzioni molto basse, inferiori a 100 volt, ha un'efficienza quantica che questi sono numeri un po' vecchiotti ma già siamo arrivati anche ai 50\% quindi è abbastanza elevata. Hanno due danie elevati 10 alla 6 proprio perché a questo punto il tutto dipende da quanti pixel abbiamo l'esposizione. La risoluzione temporale è molto buona, sono indipendenti dal campo magnetico, tuttavia ancora si deve lavorare un po' sulle dimensioni e sul dark count rate quindi quel rumore che è sempre presente nei dispositivi, nei fotosensori, anche in essenza di radiazione che in questo caso può essere anche molto elevato, dipende dal produttore ma anche in questo campo ci sono stati dei notevoli sviluppi ormai si arriva anche a un kilo eazzo per milimetro quadro che è qualcosa di molto soddisfacente. Vediamo alcuni esempi, ci sono diverse ditte che producono questi dispositivi, l'esempio la più famosa è una ditta giapponese che prende il nome di Amamatsu e vede appunto questi piccoli dispositivi. Vi dicevo le celle lavorano in parallelo e quindi se andiamo a guardare all'oscilloscopio un tipico segnale prodotto da un silicon photomultiplier questi segnali se li mettete in persistenza, cioè non fate aggiornare i display dell'oscilloscopio ma ad accumulare tutti i segnali che vi arrivano nel tempo, quindi non si cancella, quello che è stato visualizzato prima ma mi rimane nello schermo e non segnale viene sovrapposto a quello che abbiamo visto prima. Quello che osserverete se abbiamo un buon SIPM è una distribuzione dei segnali come quello che vedete qui, che cosa stiamo vedendo? Stiamo vedendo dei segnali che non hanno ampiezza qualunque, hanno ampiezze quantizzate quindi assumono dei valori tipo discreti o ad esempio in corrispondenza di questo livello o in corrispondenza del doppio, del triplo, che cosa stiamo vedendo? Stiamo vedendo la rivelazione dei singoli fotoelettroni, quindi avendo un segnale ad esempio che è questa ampiezza equivale ad aver misurato un solo fotoelettrone, quindi è inciso un solo fotole che ha dato l'uovo a una sola scella colpita. Se ne avevano colpite due abbiamo un segnale che ha una ampiezza doppia, se ne vengono colpite tre vedete il triplo e così via, quindi si vede questa discretizzazione, questa quantizzazione, perché ogni cella lavora come se fosse un guide, quindi un rivelatore on off. Ok, un attimo metto a caricare il computer prima che si scarica, vediamo un po' la situazione della batteria, no, beccia, dovrei fare. Se andate a realizzare uno spettro delle ampiezze di questi segnali, chiaramente troverete una distribuzione a picchi, lo faccio vedere meglio, perché corrisponde al fatto che le ampiezze non sono quelle, ma assumono dei valori ben precisi, quindi ad esempio questo primo picco è quello corrispondente al primo fotoelettrone, il secondo picco è quello corrispondente al due fotoelettrone, così via. Guardiamo l'ultimo aspetto, anche qui abbiamo una efficienza di rivelazione, una photo detection efficient, che però dipende da diversi parametri, abbiamo un aspetto derivante dal film factor per vedere cos'è, l'efficienza che abbiamo visto per le foto cattle di un colore che dipende della condizionata, che è una propapriità di trigger della valore. In particolare il film factor dipende dal fatto che abbiamo una matrice e quindi l'area sensibile ovviamente la parte centrale, ma abbiamo inevitabilmente dei bordi che rappresentano un'area morta, quindi se il fotone dovesse incidere su questi bordi, non riverrebbe perso, quindi questo è un contributo, ovviamente il rapporto tra l'area effettivamente attiva di rivelazione e l'area complessiva, quindi più siamo in grado di realizzare i sottili, maggiore sarà l'efficienza dovuta al film factor. Capite che l'effetto del borbe è tanto più importante quanto più è piccola la dimensione del pixel, quando il pixel è più grande il borbe ha un effetto minore. Comunque normalmente possiamo avere

anche film factor che variano dal 30\% fino all'80\%, dipende chiaramente dallo dispositivo. L'efficienza quantica è più elevata rispetto a quella che abbiamo visto nel caso dei fotomoltiplicatori tradizionali, vedete può assumere valori anche superiori all'80\%, dipende ovviamente dal tipo di sensore e dalla longhezza donna che andiamo a rivelare. Infine abbiamo anche una probabilità di trigger, una volta che ha inciso la radiazione e ha prodotto una poppia riesce a produrre questa poi un segnale sufficientemente elevato, quindi a dare luogo a un trigger, quindi a una rivelazione e questo dipende sostanzialmente dalla tensione. Vedete quei due curve che variano con il campo elettrico applicato, quindi con la tensione e che ciò a vedere appunto che la probabilità di trigger cresce ovviamente con la tensione applicata. Per i SITM ci sono diverse strutture anche qui vediamo diversi strati di materiale del drogato e tipicamente si distinguono in P2N o N su PES, quando di quale la regione è sensibile, se è una regione di TPUN o una regione di coppia, comunque questi sono dettagli più postruttivi. Possono essere adoperati siccome hanno dimensioni molto piccole un po come il caso degli APD per la lettura delle fibre VLS, soprattutto quando abbiamo mandato una sola fibra quindi vi è aspettata poco a luce, il SITM avendo un guadagno notevole è un ottimo candidato per la rivelazione di questa luce. Infatti vedete qui ad esempio degli scintillatori dove sono stati realizzati dei solchi dove è stata messa la fibra VLS ai capi della fibra si va a posizionare un SITM, quindi ci sono tantissime applicazioni di questo tipo. Anche per la PET si sta pensando sostituire il classico cotocontribilitatore con dei silicon photomotivlier, sono degli ottimi candidati quindi riassumendo, confrontando questi fotosensori da un lato abbiamo il PMT che si caratterizza per avere degli ottimi guadagni ma ha una tensione di lavoro molto elevata con un dark count rate abbastanza basso, altro difetto e efficienza quantica bassa ma la selleraria sensibile è abbastanza elevata. Poi abbiamo i materiali semi-conduttori quindi APD e SITM che si caratterizzano soprattutto dalle basse tensioni di lavoro hanno però come svantaggio il fatto di avere delle superfici abbastanza ridotte ma hanno efficienze tipicamente più elevate, con guadagni che possono anche essere confrontabili con quelli del fotomotivlidatore tradizionale. Devo attaccare il carica batteria prima che si scolle da tutto. Anche lo srotchio in realtà lo possiamo immaginare come un fotosensore, non so se conoscete la struttura interna dell'occhio, sapete che appunto la luce alla fine dopo aver attraversato diversi strati di materiale diverso. Arriva a incidere sulla retina, la retina dovete immaginare come se fosse un vero e proprio rivelatore anche se costituita da una sorta di pixel che sono i nostri fotorecettori. Infatti la retina è costituita da coni e basso in celli, secondo se siete interessati alla visione di urna, dei colori o alla versione notturna crepuscolare. Intervengono dei recettori, fotorecettori diversi, in particolare i coni sono destinati alla visione di urna, quindi la percezione dei colori sono un numero più ridotto rispetto ai basso in celli, tipicamente si concentrano vicino nella parte centrale della retina, vicino all'asse ottico. I basso in celli invece che sono utilizzati per la visione notturna sono molti di più e sono distribuiti un po' in tutta la regione della retina. Comunque perché faccio questo discorso? Perché potremmo cercare di capire che prestazioni ha il nostro occhio in termini di auto rivelazione. Alla fine noi abbiamo parlato di strumenti che vanno a percepire della luce che il nostro occhio è lì in principio e non è in grado di percepire, o comunque non è facile percepirli. Allora in totale posso immaginare di avere un occhio standard all'incirca 100 milioni di fotorecettori, quindi equivale al dire 100 megapixel se dovessimo adoperare un termine simile a quello che doveriamo nel campo della rivelazione e in una singola immagine vengono interessati simultaneamente all'incirca 7 megapixel. Tuttavia l'occhio vede continuamente in modo ciclico anche 100 volte al secondo e questo ne aumenta la risoluzione. Quindi alla fine potremmo dire che complessivamente il nostro occhio può essere approssimo all'autofotosensore con 570 megapixel, quindi immaginate ovviamente delle prestazioni notevoli. Chiaramente la sua risposta dipenderà dalla lunghezza d'onda della luce che incide, in particolare di nostro occhio ottimizzato per vedere le lunghezze d'onda intorno al verde e giallo che appunto corrispondono anche col massimo di emissione della nostra stella nel sole, quindi non è un caso ovviamente che noi abbiamo una maggiore risposta proprio in corrispondenza di queste lunghezze d'onda. Concludo facendovi vedere questo articolo che uscito qualche anno fa a Sonnecer, è stato appunto una notizia particolare, io insegnavo a Topp, fin l'anno scorso e quindi mi interessavo anche di questi aspetti, sostanzialmente è stato condotto uno studio per cercare di capire se l'occhio umano fosse in grado di rivelare di riperrare i singoli fotoni, cosa che noi normalmente facciamo appunto con della strumentazione anche molto costoso e complessa, ma in realtà è stato dimostrato sperimentalmente che l'occhio umano è in grado di misurare i singoli fotoni, è stato fatto appunto uno studio sperimentale sotto determinate condizioni e si è arrivato questo risultato che ha meritato la pubblicazione su Nexar, non so se conoscete vostre riviste, penso di sì, quindi sapete che non è rivista di assoluto prestigio e questo è contestato il riassunto di questo risultato sperimentale. Allora ragazzi io finirei qui oggi, avevo un'altra cosa da mostrarvi molto velocemente ma possiamo riprendere la prossima volta dove quando parleremo delle esperienze, quindi chiedo a chi doveva restituire il kit di portarmelo qui e noi ci vediamo l'uno è di prossimo, mi scollego anche dall'esterno, vi vediamo ragazzi, adioderci, vi vediamo.

\textbf{lez 13}

Allora mi sentite da fuori? Sì, però se adesso la sentiamo. Scusate, è ritardo e ci sono state una serie di domande quindi abbiamo attivato il collegamento con un po' di ritardo. Allora iniziamo con David al monitor. Allora ragazzi possiamo iniziare. Come vi dicevo oggi descriverò le esperienze che vi ritroveremo da fare a partire dalla prossima settimana. Quindi Joe Vidi vi dico che è scalta la lezione perché nuovamente sono fuori per lavoro però inizieremo con i turni di laboratorio. Ok a breve vi farò avere questo schema dei turni. Ho completato i gruppi perché molti di voi già avevate inserito il nome e altri li ho inseriti io dato che non erano presenti e quindi li ho inseriti io casualmente nei gruppi che erano rimasti scoperti. Quindi se hai andato a consultare quel Fone XL che aveva condiviso trovate i gruppi come sono stati definiti se ci sono dei problemi che sono mai fatte nello presente. Siccome già è stato ammirato se per caso abbiamo difficoltà per alcune date di laboratorio perché vi ha detto non tutti con molti in tutti i giorni previste per il laboratorio eventualmente se riuscite a trovare una soluzione tra di voi cioè far uno scambio con un collegamento e magari magari altro giorno deve fare la stessa esperienza possiamo farlo basta di te ne lo comunicate quando venite il laboratorio altrimenti ne lo fate presente proviamo a trovare un'altra soluzione. Comunque si aveva previsto anche dei turni di recupero eventualmente per chi proprio perde troppe lezioni eventualmente alla fine del mesi di gennaio ho previsto qualche turno di recupero. Vi ricordo nuovamente il discorso del certificato di sicurezza insomma il corso di sicurezza che abbiamo eseguito io ho ricevuto dalla secretaria una lista di studenti che aveva sostenuto questo corso infatti mi sono appuntata a alcuni nominativi a quelli di voi mi hanno inviato via email o un screenshot della scirmata insomma del cosso che abbiamo sostenuto gli altri mi raccomando cercate di averlo pronto prima di entrare il laboratorio e me lo consegnate. Chi l'ha fatto il primo anno dovrebbe valere perché se non mi sbagli non aveva di di di di di 5 anni io vi posso far vedere che lista mi è arrivata perché a me è arrivata una lista però non era per niente completa cioè nel senso ne mancavano parecchi. Eccolo chi ha l'X era in questa lista quindi non riuscite a leggerla? Vendranno il discontomo. Quindi verificate questa è la lista che mi è stata data circa una settimana fa e in ordine alfabetico. Vedete? Vedete? Molti mancano ora non so se perché non l'abbiamo fatto pura perché... Ok vabbè ma comunque risolviamo se poi semplicemente mi fate uno screenshot e mi mandate la scirmata. Non c'è bisogno di portarmi nulla il laboratorio, io so che voi abbiamo fatto il corso quindi chi ha l'X certamente non mi deve dare nulla. Chi non ha l'X magari se si fa uno screenshot semplicemente che compare scritto ora qualcuno di voi me l'ha mandato il corso e il punteggio non vi ricordo come arriva. È un problema, ci sono difficoltà su questo... come non so come funziona, non non qualcuno sociato così no? Ecco magari se mi scambiate tradito di questa informazione perché io non lo so sinceramente. Oppure se mi manda gli elenchi passati io non lo mando, io mi mangio di sì infatti è sempre lo stesso quindi basta di mi trasmette un elenco delle persone. Comunque i primi apporzi il problema saranno quel è il gruppo 1 perché entreranno in laboratorio il giorno 29, prossima settimana. Allora ragazzi vi dicevo in questi turni di laboratorio faremo parte delle esperienze di la fine della tesina. Per questioni logissiche ho selezionato 3 di queste esperienze rispetto a quelle che sono elencate nel programma più vi farò fare anche un'esercitazione che però non è un oggetto di tesina quindi un'esercitazione che farete con l'attività laboratoriale solo utilizzando i contatori guide per la distribuzione di quassonna quindi per lo studio della distribuzione di quassonna. Quindi contemporaneamente ci saranno 4 gruppi che lavoreranno in laboratorio, suddivisi in due ambienti non so se abbiamo avuto modo di vedere dove sono gli ambienti del laboratorio 3 sono 2 e uno è all'inizio di questo corridoio e l'altro praticamente alla fine del corridoio che porta verso i laboratori 1 e 2. Quindi vado ad arrivare poi vi dirò in quale ambiente lavorerete perché sono due esperienze in un locale e due esperienze in un altro locale. Quindi una esperienza che farete riguarderà certamente la misura del coefficiente di assorbimento degli elettroni in diversi tipi di materiale utilizzando come rivelatore il contatore guide. Quindi a questo punto date le lezioni che abbiamo fatta abbiamo tutte le conoscenze necessarie per poter svolgere questa esperienza. Cosa state facendo? State semplicemente andando a studiare come gli elettroni vengono assorbiti da spessori di materiale di natura diversa. Quindi la perdida di energia degli elettroni in un dato materiale che è una cosa che abbiamo visto, abbiamo detto avviene attraverso diversi meccanismi può essere anche parametrizata con un esempio formato di vede blocco soprattutto se stiamo lavorando a base energia come sarà ovviamente il laboratorio e si tratta di verificare sperimentalmente una legge semimilpirica che abbiamo descritto a lezione. I rivelatore che doverate è un contatore guide. L'abbiamo visto anche a lezione la volta scorsa, l'abbiamo studiato dal punto di vista del principio di funzionamento. In realtà il rivelatore guide che doverete è un po' diverso, non è proprio quello che vi ho dato l'altra volta lezione ma è sempre un contatore guide. Quindi la solita geometria cilindrica che conoscete e in particolare verrà gestito con una apparatosperimenta, un dispositivo che permette sia di applicare la differenza di potenziale tra gli elettrodi sia di conteggiare le particelle che arrivano quindi a un display che vi permette anche di sapere quante particelle sono passate in un determinato intervallo di tempo. Quindi l'apparatosperimentale si presenta grosso modo così schematizzato. abbiamo sostanzialmente una sorgente beta in particolare una sorgente di stronzi 90 e 390 che sono due isotopi che decadono beta. Ora vedremo anche lo schema di decadimento per capire l'endpoint e il tempo di dimezamento. Dopodiché abbiamo una serie di assorbitori di diverso tipo con vari spessori. In particolare c'è un supporto dove possiamo andare a posizionare diversi spessori, direttura possiamo andare a sommare insieme quindi non so ad esempio abbiamo diversi spessori di alluminio, uno da mezzo millimetro, uno da un millimetro, possiamo metterli insieme per fare uno spessore da un millimetro e mezzo. Ok quindi possiamo combinarli in maniera tale da sabbiliere diverse valori di spessore. In particolare in base al tipo di materiale che decidete di utilizzare ha senso arrivare fino a uno spessore massimo perché quello che avviene che quando superate uno spessore pari a range chiaramente andate ad assorbire tutti gli elettroni e messi dell'assorgente quindi anche se aggiungete altri spessori chiaramente non cambierà nulla nella vostra misura. Aveva disposizione in particolare alluminio, ottone che è una leda e quindi l'assorbimento dipende anche un po' dalla percentuale di questi elementi presenti nella leda e plexiglass. La rivalazione vi diceva che avviene attraverso un contatore Geiger che ha in particolare una finestra di ingresso che in questo momento non la vedete comunque dovete immaginare l'altra base del cilindro con la esposta verso l'assorgente quindi una finestra sottile proprio per ridurre al minimo la perdita di energia degli elettroni che entrano nel contatore Geiger. Sono altra visuale diciamo dello stesso apparato sperimentale, grosso modo e in particolare il contatore Geiger attraverso un carbo e collegato vi dicevo per chiatura elettronica. Questa vi permetterà di stabilire la differenza di potenziale tra gli elettroni del contatore Geiger perché se vi ricordate di principi di funzionamento del Geiger è basato su una camera cilindrica dove l'involgore esterno rappresenta il cattodo e un filo che percorre l'asse del cilindro 

rappresenta l'anodo. La differenza di potenziale applicata tra cattodo e anodo permette la raccolta del segnale quindi quando una particella carita attra come l'elettrone attraversa il contatore Geiger produce coppie elettrone e ione in particolare poi attraverso questo campo elettrico queste coppie migliano ovviamente gli elettroni verso l'anodo e gli ioni positivi verso il cattodo e gli elettroni riescono a produrre delle ionizzazioni successive quindi con un meccanismo diciamo avvalanga. Questo porta formazione di un segnale e questo segnale sempre attraverso questo cavo viaggia verso questo dispositivo e viene contato quindi ogni qualvolta si presenta un segnale e nel display il numero che è presente nel display aumenta la immunità. Quindi diciamo dal punto di vista operativo non c'è nulla da stabilire ad esempio la differenza di potenziale si può essere variata però lavorerete con la differenza di potenziale fissa perché se vi ricordate per lavorare con il contatore il Geiger bisogna lavorare nel piano rotto lo Geiger ha mostrato una curva e quindi bisogna scegliere un'opportuna attenzione di lavoro ma voi la troverete impostata quindi non dovete fare nessuna operazione si tratterà soltanto di far partire il contatore e a zerarlo. Chiaramente dovete effettuare tante misure con diversi spessori per poter costruire una curva di assorbimento. Il timer non è disponibile perché possiamo utilizzare benissimo quello del cellulare quindi vi basate su questo tanto non è necessaria una precisione, appartato del timer del cellulare una buona precisione ma comunque non è certamente l'errore sul tempo a comandare in questo tipo di misura. Ripassiamo un po cosa vi dovete aspettare. La legge di assorbimento degli elettroni grosso nodo segue un andamento di tipo esponenziale di crescente questa è una una semi-empirica se vi ricordate in particolare quindi vi aspettate che l'intensità si riduca attraverso una funzione esponenziale dove la pennenza dell'esponenziale è determinato da questo parametro mu che è il coefficiente di assorbimento proprio ciò che volete determinare da questa misura. Il coefficiente mu solle sprimete in unità lineari e espresso in centimetri alla meno uno o metri alla meno uno insomma una lunghezza alla meno uno. Tuttavia se andate a dividerlo per la densità ottenete il coefficiente di assorbimento massivo che diventa praticamente quasi indipendente dal materiale cioè otterrete un valore che è molto simile anche per materiali condensità diversa quindi ad esempio troverete che per l'alluminio il mu lineare è molto diverso da l'uniu lineare dell'ottone ma se andate a considerare il coefficiente di attenuazione massivo in quel caso i due mu di due coefficienti di assorbimento per ottone e dalluminio saranno molto simile tra di loro. In questa legge che vedete qui chiaramente voi non andate a misurare un'intensità ma andate a misurare un rate di conteggio una frequenza di conteggio quindi quante particelle arrivano sul conto Turing diver al secondo. Per questo motivo è importante che durante tutte le misure non venga modificata la distanza trasorgente in rivelatore perché altrimenti chiaramente non sono più misure confrontabili quindi questa i sarà espressa in Hertz. I con zero è una misura che si effettua senza alcuno spessore di materiale quindi tutto ciò che proviene della sorgente che intercetta il rivelatore quindi rispetto a questo I con zero che il valore massimo che possiamo misurare è man mano che andate a interporre degli spessori vi aspettate che questa I diminuisca. Quello che si otterrà quindi dovrebbe essere un andamento di questo tipo vedete questa è una misura realizzata con diversi spessori di alluminio ogni punto rappresentano una misura realizzata per un determinato spessore quindi vedete abbiamo lo spessore nullo che vuol dire non interporre nulla e poi abbiamo via via spessori che variano di mezzo millimetro quindi zero 5 1 5 2 e così via la scala che trovate riportato è i millimetri sull'assorizzontale perché lo spessore è sulla sé verticale trovate proprio il rate quindi quante particelle si misura al secondo in scala logaritmica in questo modo relazione precedente capite si trasforma viene linearizzata quindi si si trasforma sostanzialmente in una relazione lineare quindi in questo modo è più facile verificare che effettivamente i punti seguano un andamento esponenziale decrescente vedete i valori in grosso modo che sono più o meno quelli che otterrete il laboratorio è chiaro che la cosa migliore da fare e cercare di posizionare il rivelatore quanto più vicino alla sorgente però lasciando uno spazio utile per l'inserimento di tutti gli spessori che volete inserire perché è importante fare questa questa cortezza perché in questo questo diminuite la distanza e capite aumentate l'angolo solido sul teso del rivelatore quella che noi avevamo definito come accettanza geometrica questo comporta il fatto di permettere una un rate di opposizione più elevato quindi da un punto di vista statistico riuscire a raccogliere più dati in meno tempo sostanzialmente quindi cercate di mantenere non solo la distanza fissa ma quella più piccola possibile tra sorgente e rivelatore e quindi con questa cortezza grosso modo quello che vi aspettate sono valori di questo tipo quindi vedete quando non interponete nessuno spessore siamo a valori intorno a 20 hertz quindi comunque sono valori elevati capite che già misure di pochi minuti portano all'accumulo di centinaio di conteggi e lo dovete pensare da un punto di vista statistico perché ogni volta che poi effettuato una misura dovete pensare all'incertezza che abbiamo su quella misura quindi se io arriva ad acquisire 100 conteggi l'incertezza sarà la randesce di 100 quindi 10 e 10 su 100 quindi avere una incertezza di 10 su 100 misurate vuol dire un'incertezza del 10 per cento che comunque è grande si può migliorare certamente quindi più statistiche accumulate migliori sarà la vostra incertezza relativa e chiaramente il problema non si pone quando non abbiamo alcun alcun spessore o spessori molto sottili perché vedete qui le frequenze sono molto elevate il problema si pone quando gli spessori diventano notevoli tanto da praticamente arrestare tutte le particelle che propengono dalla sorgente e infatti qua desentù possiamo avere anche valori di 0,1 quindi abbastanza bassi e capite che se volete avere la stessa incertezza statistica su tutti i punti questo vorrà dire che le prime misure dureranno di meno le misure con spessori più elevati dovranno durare di più quindi quello che mi viene chiesto il laboratorio innanzitutto è anche fare una stima dei tempi delle diverse misure e organizzarvi il lavoro abbiamo a disposizione due ore abbiamo a disposizione diversi materiali questo non vuol dire che dovete fare tutti i materiali magari ne fate almeno due per confrontarlo ovviamente e ma per ogni materiale dovete esplorare diversi spessori quindi per vi dire vi richiediamo una maturità maggiore rispetto ai laboratori precedenti perché in questo caso vi ritrovate proprio a progettare una misura a stabilire voi come suddividere il tempo a disposizione ok quindi questo cosa vuol dire vuol dire che magari andate a valutare quanto è la frequenza nelle condizioni migliori quanto è la frequenza di conteggio nelle condizioni peggiori e a quel punto stabilite quanti spessori analizzare quanto tempo dedicarci a spuna misura in maniera da suddividere bene il tempo ok infazio successiva di analisi se poi non verrà estratta questa esperienza chiaramente il tutto consisterà poi nel estrarve da questi dati quindi dalla pennenza di questa curva il coefficiente di assorbimento e confrontarlo con una valore tabulata in letteratura guardiamo un po più nel dettaglio la sorgente la sorgente che è stata doverando io ho detto una sorgente mista non è misto scusate una sorgente di stronzi o 90 e 3 o 90 sapete che

le sorgenti beta non sono monocromatiche proprio perché si tratta di un decadimento a tre corbi quindi abbiamo nello stato finale il nupio residuo più l'elettrone l'anti neutrino e questo passì che l'elettrone può assumere valori di energia diverse abbiamo un doppio decadimento lo stronzi o 90 decade in e poi il 90 decada a sua volta non so se lo sclama di determinato no comunque nella nella scheda dell dell esperienza lo troverete ma la cosa che ci interessa quindi è un decadimento possiamo dire a cadena da lo stronzi 90 abbiamo come è prodotto l'itrio 90 poi solta decade nuovamente in beta sono neque di imbrio secolare perché appunto i tempi di dimazzamento sono molto diversi tra di loro e quindi comanda sostanzialmente quello che ha il tempo di dimazzamento più grande quindi vuol dire che ogni qual volta c'è un decadimento dello stronzi 90 subito dopo c'è il decadimento del littrio 90 e come spettri chiaramente lo spettro o conflessivo sarà la sovrapposizione dei due spettri quindi quello singolarmente dello stronzi 90 quello singolarmente del littrio 90 quindi quando andate a effettuare una misura dello spettro in energia o lo andate a simulare con la teoria di fermi e la forma dello spettro ha questo andamento che in realtà deriva dalla sovrapposizione di due spettri diversi uno relativo al decadimento dello stronzi o è uno relativo al decadimento dell'itrio e quindi vedete una prima parte a bassa energia un primo spettro che ha come un po in grosso modo e mezzo mezzo e poi un secondo spettro che si estende energia e pellevate dell'ordine di 2.2 mezzo come un po in ovviamente non possiamo distinguere quali sono gli elettroni provenienti del litrio quali provenienti dello stronzi non ci interessano eppure ci interessa sapere soltanto che è molto probabile che l'elettrone che state misurando ha un energia bassa perché vedete questa distribuzione questo spettro presenta dei valori molto elevati proprio basse energie diventa più improvabile avere elettroni più energetici cosa comporta questo comporta il fatto che man mano che inserite degli sfessori è chiaro che i primi elettroni essere sorbiti sono quelli relativi alla parte di bassa energia soltanto quelli più energetici riescono ad attraversare ovviamente i spessori maggiori questo è giusto per capire appunto cosa ci si aspetta attraverso delle simulazioni e riguardo appunto questo assorbimento degli elettroni ad esempio sapessimo elettroni monocromatici l'andamento sarebbe completamente diverso se vi ricordate ci aspettiamo l'andamento quasi al gradino però un gradino molto smussato non vi fa di niente in in da questo grafico semplicemente qui abbiamo una scala logarittimica verticale se rappresentassimo una scala lineare avremo un gradino molto smussato quando invece vai a considerare degli spetti di continui come quello dello stronzio novanta dell'itria novanta l'andamento è appunto un andamento in un modo disponenziale decrescente questa è la componente dello stronzio novanta che è quella che comanda bassa energia quindi questo primo spettro e questo dell'itria novanta come si vedono le scala orizzontali ma si si può vedere che vediamo uscendo scosate dalla presentazione se si veda le scale sono molto diverse vedete l'astronzione novanta che ha con end point o in questo questo mezzo MeV fa sì che gli elettroni arrivino al massima per correre 0,25 mm quindi un quarto di millimetro e lo stronzio e l'itria novanta invece che mette elettroni con energia più elevata anche di qualche MeV in questo caso vediamo che gli elettroni possono percorrere anche alcuni millimetri di questo materiale mi sbaglio all'uminio. Ovviamente noi non andiamo a distinguere le mie componenti vi ho detto vedremo il risultato complessivo però giusto per aver l'idea di come incidono i due isotopi. Se si vanno a confrontare i diversi materiali cosa ci si aspetta? Chiaramente l'annamento di queste curve è differente quindi il valore di miù mi aspetto che sia più elevato nel caso di materiali più densi come ad esempio l'ottone e quindi vedete abbiamo una discesa molto veloce addirittura con un millimetro di ottone già fate fuori tutti gli elettroni sostanzialmente mentre l'alluminio e la situazione intermedia ad esempio rispetto a utilizzare del cartone che è stato riportato un altro un altro materiale però non abbiamo a disposizione comunque il plexi ovviamente si comporta in maniera più simile al cartone piuttosto che all'alluminio con un coefficiente più basso. Vi dicevo quella forma di queste curve si può approssimare un'espolenziale detrescente quindi vedrete ovviamente delle leggeri deviazioni rispetto a quanto ho premisto da una pure espolenziale detrescente. Se si vuole essere approfondire questi aspetti di perdere energia bisognerà ricorrere assimolazioni molto dettagliate con dei software professionali come il software giant che vi ho ci dato più volte. Giant non è complesso da adoperare ci sono diverse versioni il giant giant e il giant giant vi dico che qualche vostro collega ad esempio in salle di esame presentando la tessina sia anche cimentato a fare dei conti cini con questi software un po più evoluti proprio per capire cosa mi aspetto in determinate condizioni si può simulare tutto con giant a far fare dalla sorgente, il materiale che abbiamo interposto, il rivelatore si può veramente simulare tutto e ad esempio questo è un un risultato che si ottiene con giant a 3 dove è stato riprodotto l'apparato sperimentale e sono stati riportati i dati quindi che ci si attende con questo apparato sperimentale poi si può fare un confronto con i dati sperimentali. Chiaramente non è richiesto non è un obbligatore però io ve ne parlo perché appunto sono anche espunti eventualmente per la tessina insieme al materiale che vi fornirò per questa esperienza vi fornisco anche un articolo che abbiamo scritto il professor Riggi riguatto proprio a questa questa quindi questo è l'ultimo che è un'autilizzare per lo sviluppo della tessina. Cosa andrete a fare quindi materialmente abbiamo scritto l'apparato sperimentale andrete innanzitutto studiare il fondo perché è la prima cosa di cui vi accorgerete che anche in assenza disorgente il conto della idea è di armisura qualcosa e questo è legato alla radiazione naturale quindi chiaramente tutto ciò che viene messo dai muri la radiazione cosonica interviene quindi fa parte costituzione una sorta di fondo che è presente sempre nelle vostre misure quindi anche quando sarà presente la sorgente parte delle particelle che abbiamo misurato in realtà non è un problema della sorgente ma sono legate al fondo di radiatività naturale quindi bisogna stimarlo per poterlo andare a sottrarre dalle misure. A quel punto si vanno a effettuare le misure per 

valutare il coefficiente di assorbimento con un dato materiale ad esempio possiamo partire dall'alluminio che quello più semplice anche perché ha spessori diciamo macroscopici anche più facile da misurare rispetto all'ottone e poi ripetere tutto per almeno un altro materiale quindi ho l'ottone o il plexiglass. I aspetti su cui fare attenzione ve lo detto riguardano i tempi certamente di misura perché questo stabilisceombra l'errore statistico da attribuire a ciascun punto sperimentale e l'aspetto della sottrazione del fondo e che è importante soprattutto per le misure con spesso rippie elevati perché quando andate a interporre 3 millimetri di alluminio che fermano praticamente tutte le particelle messa dalla sorgente è chiaro che quello che è messurato è praticamente al fondo quindi dovete stimarlo bene il fondo per poterlo sottrarre e quindi quindi quindi quindi quindi quindi quindi la la su questa esperienza quindi io vi fornirò. Oltre alla presentazione di oggi ovviamente una scheda ve la faccio vedere che descrive l'esperienza quindi in maniera più discorsiva rispetto alla presentazione di viene descritta l'esperienza quindi richiamando la parte teorica ciò che vi aspettate l'apparato sperimentale e poi le diverse parti dell'esperienza. Vi fornirò una scheda di attività che è qualcosa che dovete man mano fare durante l'esperienza anche per rendermi conto che tutto sta andando come effettivamente dovrebbe andare e quindi ad esempio quando andate a fare la misura del fondo vi chiedo a valutare il tasso di conteggi del fondo e l'incertezza statistica corrispondente quindi non so abbiamo fatto una misura l'unica abbiamo acquisito mille conteggi, andate a valutare anche l'incertezza statistica qui ad esempio vi viene fornita una tabella da riempire in base alla misura che abbiamo settualato la possibilità di andare a già costruire un primo grafico ma non lo mente perché ogni volta che si fa una misura è utile comunque non solo guardare il dato che abbiamo ottenuto ma anche provare a rappresentarlo banalmente in grafico, giusto per capire se le cose torno non pure no, sapete quanta volta mi arrivano tessine dove c'è l'andamento di tutti i punti che segue un certo trend e poi c'è un punto sparato da un'altra parte ma se le sono accorti dopo a posteriori la misura andava bene se invece si fa un controllo puntuale insomma bisogna saper valutare quello che si sta misurando quindi questa scheda di attività vi può aiutare anche in questo ovviamente il laboratorio sono presente io, il tecnico di laboratorio saranno presenti anche il ragazzo che farà il tutorial, ci siamo qui si abbiamo il dubbio ma manovitiamo durante le due ore di laboratorio e l'articolo che vi avevo detto prima quindi questo sarà il materiale accorrendo di questa esperienza. Ci sono domande su questa esperienza? Sì. C'è l'ostronzo dei caldi con elettroni un poche energetici, tanto che ci sia un enno spessore di aria tra la sagente e il rivelatore, quanto può influire come errore? Alla rada. Che Che si sono particelle altre, quindi non si cargano subito.

Il collega ha chiesto per chi ha casa ha chiesto nel soprattutto per quanto riguarda gli elettroni meno energetici, se ci sono dei problemi per il fatto che questi elettroni dovranno attraversare una piccola porzione di aria prima di essere livellati in realtà hanno praticamente la perdita di energia in aria di elettroni veramente trascurabile rispetto a quella che avviene dei materiali solidi quindi si può riteneri praticamente trascurabile. Altra domanda? Ok, mi pari di no. Allora andiamo avanti con le altre esperienze. Altra esperienza che vi ritroverete a fare è la spettrometria $\gamma$. Quindi in questo caso la sorgente riadoperata è una sorgente $\gamma$ e per misurare in $\gamma$ con una buona efficienza se vi ricordate abbiamo detto è utile ad operare scintillatori. In particolare quello che adopererete in laboratorio è uno uno aiuduro di sodio drogato al tallio che è uno scintillatore in organico, un cristallo. Uno di quei cristalli che vi diceva la difficoltà di essere igroscomico e infatti non lo vedrete nel suo stato naturale, nel senso lo vedrete encapsulato dentro un rivestimento di alluminio che non solo permette la riflessione della luce, la raccolta della luce e lo scintillatore ma in più lo protegge dall'umidità. Quindi avrete a disposizione diverse sorgenti $\gamma$, in particolare certamente il Cessio 137 e il Cobalto 60 che sono delle sorgenti di cui vi ho fatto vedere gli spettri più volte. Perché vi cito questi? Perché innanzitutto sono quelli che hanno un'attività maggiore nel nostro caso e quindi permettono delle misure anche con buona statistica in tempi brevi ma in più utilizzando solamente queste due sorgenti possiamo già effettuare una calibrazione nell'energia. Cioè abbiamo a disposizione già tre valori energetici perché il Cessio 137 è decada in $\gamma$ e mette $\gamma$ a 662 keV. Il Cobalto 60 addirittura ha due dettaglimenti $\gamma$ e quindi abbiamo altri due valori di energia e questo vedremo e ci ci utile per effettuare una calibrazione in energia. Quindi basterebbero soltanto queste tre sorgenti ma in realtà ne avrete a disposizione anche altri. Poi abbiamo un rivelatore scintillatore che abbiamo discusto la volta scorsa, vi ho detto di un duro di sodio drogato al taglio e ovviamente un foto sensore, in questo caso è un foto multiplicatore. In questo caso troverete un minimo di elettronica. Ancora l'elettronica non l'abbiamo affrontata come argomento quindi non abbiamo parlato di preamplificatore però diciamo una catena elettronica molto semplice, quello che faremo è vedere innanzitutto il segnale lo sceloscopio di questo foto multiplicatore così cominciate a approcciarvi per la prima volta con i segnali elettrici quindi anche l'elettronica che c'è di conseguenza. Però non vi preoccupate per questo discorso dell'elettronica perché questo è stesso appareto sperimentale e poi verrà utilizzato per la seconda esperienza il secondo turno di laboratorio. Quindi in quel caso poi avremo già presentato in diversi modi elettroniche quindi potremmo guardarli con altri occhi. Questo è lo schema dei livelli delle due sorventi di cui vi parlavo prima, quindi il CSI 137 vedete che il decadimento $\gamma$ avrebbe sempre assaigliato di un altro decadimento e quindi in particolare qui vedete è messo un $\gamma$ 662 keV. Mentre per quanto riguarda il cobalto vengono emessi due $\gamma$ 1 a 1 e 17 keV e l'altro a 1 e 33. Capite ad esempio che nel caso del successio questo $\gamma$ che arriva sullo scintillatore interagisce solamente o per effetto fotoelettrico o per effetto quantum. Ad esempio la produzione di coppie qui non può avvenire perché non siamo all'obbi di sopra della famosa energia di soglia per cui può avvenire produzione di coppie. Cosa ci aspettiamo di osservare? Questo è un aspetto che non abbiamo discosso. Noi abbiamo parlato da un lato dei $\gamma$. Abbiamo detto sono monocromatici e da un lato abbiamo parlato di scintillatore. Abbiamo detto in via di principio con uno scintillatore siamo in grado di misurare l'energia. Quindi cosa mi aspetterei da uno spettro $\gamma$? Mi aspetterei uno spettro mono energetico. Quindi sempre lo stesso valore di energia all'ina di principio, dato che in $\gamma$ sono mon energetici. Pensiamo ad esempio in $\gamma$ emessi dalla sorgente di Cesio. Sono tutti emessi con energia di 662 keV. Quindi se io avessi un rivelatore come lo scintillatore in grado di misurare l'energia dovrei dire che l'energia di misura è sempre 662 keV. Non dovrei ottenere altri valori. In realtà la situazione è più complessa. Se ad esempio il rivelatore avesse dimensioni infinite. In quel caso sono sicure che effettivamente tutta l'energia del $\gamma$ viene misurata. Perché cosa può fare il $\gamma$ quando incide sullo scintillatore? Può dare l'uovo a quei tre effetti che abbiamo detto prima. Effetto fotoelettrico, effetto quantum, produzione di coppie. Se si produce effetto fotoelettrico siamo fortunati perché come prodotto dell'effetto fotoelettrico abbiamo l'emissione di un elettrone. Questo elettrone si prende praticamente tutta l'energia del fotone incidente. Perché ad esempio se il fotone ha 662 keV, questi 662 keV servono in parte per liberare l'elettrone. Quindi una piccolissima parte viene utilizzata per strapare l'elettrone all'atomo. Tutta la restante parte se la prende l'elettrone come energia cinetica. Ora quanto vale quanto è l'energia necessaria per strapare un elettrone lo abbiamo detto pochi elettronvolta, una decina di elettronvolta, venti elettronvolta non di più. Quindi vuol dire che su 662 keV venti elettronvolta venono spensi per strapare l'elettrone. E capite che è una quantità ridicola in risoria e possiamo arrivare ad affermare che l'elettrone viene messo con l'energia per far far quella quella diamo incidente. Ok, questo elettrone nel materiale e e ovviamente interagisce laundry laundry i meccanismi che abbiamo visto quando abbiamo parlato della perdida di energia degli elettroni percorre qualche millimetro non l'abbiamo visto in caso dell'alluminio però ho detto detto fa percorre tra millimetri in grosso modo nel caso del Plexiglas quindi nel caso di unuscitillatore percorrere a un po di più 5-6 mm ma in pochissimo spazio perde tutta la sua energia e questa energia viene convertita in luce e quindi chiaramente a maggior ragione se il rivelatore è in dimensioni infinite sono sicure che tutta l'energia viene dissipata all'interno dello scintillatore e quindi in qualche modo il segnale più dotto sarà una misura della energia del $\gamma$ incidente e questo vi dicevo è il caso più semplice quando avviene effetto fotolettrico se avviene effetto Comton oltre a un elettrone diffuso abbiamo il $\gamma$ diffuso come prodotti dell'effetto Comton ora l'elettrone segue la stessa cosa lo stesso percorso che abbiamo detto prima quindi perde la sua energia e viene fermato in pochissimo spazio l'altro fotofotone diffuso può interagire con lo scintillatore attraverso sempre quei meccanismi che abbiamo detto effetto fotolettrico effetto Comton in produzione di coppia dipende ovviamente dalla sua energia e della probabilità di ciascun processo tuttavia se le dimensioni del rivelatore sono infinite anche se il $\gamma$ non interagisce Cut, Cut, prima o poi dobra interagire e quindi attraverso questi meccanismi prima o poi tutta l'energia verrà nuovamente depositata nello scintillatore e quindi in questo caso anche se avvenuto effetto Comton tutta l'energia viene ripostruita per il semplice fatto che ho un 

rivelatore di dimensioni infinite pensate invece di avere un rivelatore dimensioni finite come le le d'avere di laboratorio che ne so 5 cm per 5 cm per 5 cm per 5 cm cm cm il $\gamma$ prodotto dall effetto Comton potrebbe anche non interagire potrebbe fuori uscire dal materiale abbiamo detto il $\gamma$ ha una probabilità di interazione molto più basse rispetto a quella delle particelle cariche quindi potrebbe fuori uscire e noi lo perdiamo quindi quello che noi andiamo a misurare lo scintillatore soltanto l'energia che si è trasportato l'elettrone una parte dell'energia del $\gamma$ incidente quindi viene un po' a mancare quello che avevamo detto c'è che con un scintillatore io sono in grado di misurare l'energia della particelle incidente non è detto dipende cosa viene se avviere effetto Comton c'è una forte probabilità che parte dell'energia mena persa perché il fotone diffuso fuoriesce senza interagire dopo a mente per la produzione di coppia invece possiamo dire anche qui dato che si produce una coppia elettrone positrone che anche in questo caso si ricostuisce l'energia complessiva del $\gamma$ incidente quindi il grosso problema sta nell'effetto Comton se avessimo un rivelatore di dimensioni infinite non sarebbe un problema prima o poi raccogliamo tutta l'energia ma se il rivelatore ha dimensioni finita come è ovviamente nella realtà potrebbe accadere che parte dei prodotti delle interazioni come ad esempio i fotoni 

diffusi perfetto Comton fuoriescano dal rivelatore e non vendono misurati quindi l'energia vero riuscirà solamente una parte dell'energia iniziale questo porta ha uno spettro abbastanza complesso quindi quello che misurerete il laboratorio non sarà una riga in corrispondenza di un determinato valore di energia ma uno spettro che alla fine uno spettro continuo dove si intravedono determinate strutture quindi in particolare avremo un picco che è il il picco fotoelettrico questi eventi che poi misurate a questo valore di energia corrispondono agli eventi in cui si raccoglie tutta l'energia del $\gamma$ incidente e si chiama picco fuor elettrico perché questo si verifica soprattutto quando c'è effetto fotoelettrico abbiamo detto se avviene effetto fotoelettrico l'elettrone emesso viene sostanzialmente assorbito dal materiale cioè viene targa e tutta la sua energia materiale quindi ricostruiamo tutta l'energia e quindi è ragionevole che questo picco si trovi mi dica che l'energia è pari a 662 KG nel caso del senso dell'assorbiente di 637 quindi l'energia del $\gamma$ incidente ma oltre a questi casi possono capitare una serie di eventi in cui l'energia ricostruite solo una parte e quindi i valori di energia vedete più bassi e qui è tutto un continuo dove eventualmente si distinguono solamente due picchi picchi, cioè le strutture che prendere nel nome di picco di backscattering e di spalla cotton si possono valutare quindi il valore di energia a cui attendersi questi due picchetti si vogliamo chiamare picchi si può calcolare ricorrendo alla formula che conoscete benissimo valida per l'effetto cotton dove è possibile calcolare l'energia del fotone diffuso conoscendo l'energia del fotone incidente e l'angolo di scattering questa è una forma che conoscete e in particolare ad esempio per il 637 dove l'energia HN del fotone incidente 662 keV si può trovare che il picco di backscattering corrisponde a l'energia che ha il fotone quando l'angolo di scattering è di 180 gradi quindi quando teate 180 gradi si può fare il conto calcolare questo HN primo attraverso questa formula e si trova che il picco di backscattering è atteso a 184 keV e infatti se torniamo indietro ci ritroviamo qui questo per il 637 ovviamente per un altro risotto può cambiare la posizione di questo picco mentre la spalla cotton si valuta a considerare l'energia che ha l'elettrone in corrispondenza di questa condizione quando teate 180 in fatto banalmente si calcola come differenza tra l'energia del fotone iniziale e l'energia del fotone scatterato e quindi in questo caso 478 keV ed è proprio che abbiamo trovato qui quindi interessante quando andate a fare una misura verificare che il vostro spettro si presentino effettivamente questi picchi il picco foteletrio è quello più importante ovviamente è che ha una forma vedete grosso modo blausiano comunque non è strettissimo ma questo deriva dalla risoluzione in energia del vostro appartamento sperimentale e poi abbiamo questa struttura continua dove possiamo mettere in evidenza queste queste ulteriori picchetti Cosa dovrete fare il laboratorio? All'alena, anzi, tutta, studierete un po' il fotosensore. In questo caso, ti ho detto, un foto moltiplicatore. Il foto moltiplicatore lavora con un'altra tensione, in particolare quello che adopererete voi sui 600 keV. E quindi possiamo cercare di capire come va l'area, l'ampiezza del segnale prodotto quando cambiate la tensione, quindi apparità di sorgente. Stati inviando sempre gli 60 $\gamma$, ottenete ad esempio un picco fotoelettrico in corrispondenza di un determinato canale dell'ATC. Se cambiate la tensione, sostanzialmente cambiate il fattore di guadagno del foto moltiplicatore, quindi l'amplificazione cambia ed è il segnale elettrico. Quindi possiamo avere un segnale elettrico con ampiezza maggiore se aumentate la tensione e ampiezza minore se la diminuita. E quindi l'effetto sarà sostanzialmente che questo picco si sposta a destra o a sinistra, al secondo del valore di tensione che state considerando. Justo per capirci, vedete ad esempio qui abbiamo uno spettro espresso ovviamente in canali di ADC e questi sono dei picchi che sono stati ottenuti con diversi valori di tensione partendo da 550 volte fino ad arrivare al 600 volte. Vedete che man mano il picco si sposta verso destra come ci si aspetta perché il segnale all'aumentare della tensione del foto moltiplicatore aumenta la sua ampiezza e quindi l'ADC che è quel modulo che serve per andare a misurare l'ampiezza del segnale fornirà un valore in uscita che sarà via via più grande all'aumentare della tensione. Questi sono proprio i picchi fode elettrici come visto prima, non stiamo guardando la spalla conto, tutta la parte conto è tutta a sinistra e tagliate il grafico. Quindi la prima parte dell'esperienza è consistere a modificare la tensione e andare a studiare di quanto si sposta questo canale, quindi la posizione del picco ad esempio il passare da 550 550 e 5 valuto di quanti canali distano i due picchi, quindi questo delta C lo divido per la differenza di tensione, quindi quel caso 5 volt, lo divido per il valore di canale meglio tra i due e per i due picchi, ottengo quindi la variazione percentuale del canale per un volt di tensione, il coefficiente di guadagno, quindi di quanto varie il percentuale il mio segnale quando varie la tensione di un volt. Capite che se l'appalato è molto sensibile anche piccole variazioni di tensione comportano un grosso spostamento del picco, quindi un coefficiente di guadagno elevato, invece se lo strumento è meno sensibile anche se la tensione scilla fluttua di poco non mi accorvo il quasi nulla, il picco rimane sempre nella sua posizione. La seconda parte dell'esperienza consiste nel sabilire una calibrazione in energia, perché vi ho detto alla fine, voi abbiamo a disposizione una DC che vi dice l'ampiezza del segnale espressa in un numero che va da 0 a in questo caso 2048 perché abbiamo una DC a 11 bit. Come facciamo a capire ogni canale a che valore di energia corrisponde? Semplicemente utilizzando delle energie note e vedendo il picco fotoelettrico dove cade e quindi basta ad operare delle sorgenti per andare a realizzare una retta di calibrazione dove da un lato riportate le energie note quindi ad esempio abbiamo misurato lo spettro del cesio che emette a 662 keb, quindi siamo qui questo punto, abbiamo misurato un picco fotoelettrico che cade al canale 510, ecco questo rappresenta un punto di questo grafico. Chiaramente se abbiamo a disposizione diverse sorgenti, quindi con diverse energie, possiamo costruirvi questo grafico con diversi punti e valutare la retta di calibrazione, la relazione dovrebbe essere lineare è quello che ci si appende. L'altra parte dell'esperienza consiste a poi nell'analizzare lo spettro, quindi c'era dove si trova il picco fotoelettrico, quanto è il lavbo, perché questo mi da un'indicazione sulla risoluzione in energia, dove si trova la spalla conto, il picco in backscattering. In tutto ciò bisogna anche realizzare una misura del fondo, perché abbiamo sempre a che fare con un rivelatore, quindi questo rivelatore come nel caso il contatore Giver misurerà anche una radiazione di fondo e quindi dobbiamo fare una misura senza sorgente, che poi andrà sottratta dagli spettri misurati con le sorgenti. Considerate che nel caso del fotomultiplicatore, oltre alla radiazione di fondo, in realtà ci sarà una serie di misure, di eventi che vogliamo misurare di piccola ampiezza e questi sono 

il rivelca segnali dovuti al rumore del fotomultiplicatore. Vi ricordo che il fotomultiplicatore è uno strumento molto sensibile, ma anche molto rumoroso, tanto che, diciamo, bisogna avere la cortezza di non esporlar la luce, di non aumentare troppo la temperatura, perché ad esempio per effetto di emissione termica potrebbe essere prodotto un segnale spurso, un segnale che non deriva dalla misura di luce, ma semplicemente a causa di questa emissione di elettroni per effetto termico. Quindi in realtà ci saranno sempre una serie di segnali e di passi a un piazzale rappresenta un rumore elettronico, che può però cuocere sottratto realizzando una misura, in assenza di sorgente, che poi va sottratta allo spettro misurato. Vedete qui un esempio di spettro dove vedete il picco foteletrico che è stato fittato con una gaussiana e gli altri due bicchetti di backscattering e di contour, e si può andare a vedere appunto la corrispondenza con l'energia a cui ci si aspetta, si presentino queste strutture. In realtà troverete anche un bicchetto dovuto all'emissione di X, questo deriva dal decadimento sempre della sorgente, quindi eventualmente sapete una buona statistica, si può anche intravedere questo piccolo piccetto all'inizio. Si possima la l'attività della sorgente, poi abbiamo a disposizione una sorgente, diverse sorgenti, che hanno la loro attività. Chiaramente l'attività è quella che era stata fornita dal produttore tempo addietro, adesso bisogna valutare in base al tempo di dimensione a quanto si è ridotta l'attività, quindi voi se fate una misura la fate dell'attività odierna, quindi eventualmente dovreste provare a capire cosa bisogna attendersi a seguito del fatto che sono trascorsi anni da quando la sorgente è stata acquistata. Comunque detto questo l'attività è chiaramente non consiste semplicemente nel valutare quanti eventi andate a misurare al secondo, perché se vi ricordate c'è il problema che il vostro rivalatore non va a misurare tutto l'angolo solido, quindi tutte le particelle messe alla sorgente, ne misura solo una certa porzione legata alla sua accettanza geometrica e quindi con questo apparato si può provare a valutare l'attività conoscendo ovviamente l'efficienza geometrica. L'efficienza geometrica la possiamo valutare con delle formula un po' approssimate in base alla geometria che abbiamo all'isposizione oppure con delle simulazioni montetarlo che poi faremo materialmente durante le esercitazioni di rulto. Quindi avrete gli strumenti per poter valutare l'efficienza geometrica. Capite che conoscendo l'efficienza geometrica, l'efficienza intrinsica la possiamo considerare uno. Conoscete il rate di misura, la frequenza di conteggio, il branching resso e il noto in base al tipo di sorgente che conoscete possiamo valutare l'attività e confontarla quanto effettivamente vi aspettate in base agli altri resso e il risorgente. Ok, con questa esperienza oltre ovviamente la presentazione mi fornirò una nota descriviva dell'esperienza più una scheda di attività. Andiamo a vedere la terza esperienza di esame che in questo caso è lo spettrofotometro. Lo spettrofotometro è uno strumento per andarmi a suonare spettri luminosi, quindi spettri disorgente luminose. Avrete a disposizione di diversi tipi di disorgente che dovrete effettuare una rispura. In particolare vi farò vedere i vantaggi di operare un spettrofotometro rispetto alla strumentazione che abbiamo adoperato fino ad essere elaborato in un web specifico. Che cosa avrete a disposizione? Lo strumento per misurare lo spettro, quindi lo spettrofotometro digitale, diverse sorgenti luminose sia di spettri continui che di spettri all'iglia. Ovviamente ci limitiamo alla zona visibile, quindi ciò che avvede l'occhio umano anche se in realtà lo spettrofotometro permette di misurare anche un po' dell'UV e un po' dell'infrarosso. Esempi di spettri che andranno a misurare. In particolare come spettri continui possiamo adoperare la lampada neon, il ciasso fitto, una lampada incandescenza, faretti all'olgeni, quindi abbiamo a disposizione diversi sorgenti. Lo spettro che misurate vi permette di stabilire le caratteristiche di quella sorgente, quindi tipicamente ad esempio una sorgente calda, non so se vedete mai questa lampadina, penso che si è raccapitata, tante volte le lampadine si caratterizzano dalla temperatura, del valore di temperatura, perché in base alla temperatura il colore della luce è più caldo o più freddo. Quindi vi renderete conto anche analizzando lo spettro di una sorgente che viene definita calda, quindi con una componente molto forte del rosso e del giallo rispetto a una sorgente di luce fredda, come può essere ad esempio un LED bianco del cellulare, la luce del cellulare con una sorgente di luce molto fredda. Ma abbiamo anche a disposizione delle sorgenti che mettono spettri a righe, in particolare ovviamente una sorgente migliore in questo caso sono le lampade spettrali, alcuni di queste probabilmente le abbiamo adoperate, quindi le gassarare fatti in cui viene prodotto una scarica e quindi a seguito della disecitazione del litato mi viene messa della luce nemo cromatica, caratteristica di quella sostanza. Anche la luce solare, in principio un esempio di spettro che potreste andare a misurare, il problema è che non entra luce il laboratorio, già ho visto di cosa sarebbe bella perché si vanno a vedere dei rivi di fra un'over però una

misura è purtroppo il laboratorio non si può fare. Questo ad esempio è un tipico spettro di una lampada in condescenza, cosa stiamo vedendo vedete la lunghezza donna è l'intensità, in questo caso viene data un'intensità relativa rispetto a un valore massimo che corrisponde al 100\%, quindi non è espresse l'unità assolute però è importante per capire le diverse componenti cromatiche come giocano all'interno dello spettro. Vedete qui una fascia nera da un certo valore di lunghezza donna in poi, corrisponde alla parte dell'infrarosso quindi più legata al calore però il sensore che viene adoperato permette anche di misurare questa porzione di spettro. Chiaramente dal punto di vista vissivo ci interessa più la parte dei colori e quindi ad esempio questo è un tipico esempio di luce calda perché vedete la componente fredda è parecchio attenuata rispetto alla parte calda. Chiaramente un spettro continuo non essendo presenti ai veri propri picchi si possono fare delle analisi del tipo su di vedere lo spettro in due regioni definirla magari come regione fredda, regione calda, cercare di capire la percentuale di spettro in una data regione per classificare il tipo insurgente. Più interessante mi va a vedere spettri a riga quindi spettri che esprimono le caratteristiche della sorgente quindi questo ad esempio una lampa da popoli di mercurio dove si trovano delle rivi caratteristiche si possono andare a confrontare anche con ciò che riportate il letteratura. Oltre a ciò che si osserva in emissione, quindi andate a posizionare il vostro spettro fotometro e andate a ricevere della luce emessa da una sorgente, si può andare anche a affrontare un'altra problematica cioè l'assorbimento quindi che cosa viene a mancare quando io interpongo qualcosa nel mezzo tra la mia fibra ottica e l'assorgente che un po' quello che si studia ad esempio per l'astronomia per andare a capire la composizione delle stelle. E tutt'è reddè qualche misura in assorbimento, un tempo ne facevamo anche diverse con delle soluzioni di sostanze dimiche poi non capito che era una cosa un po' complicata da gestire il laboratorio quindi ci limiteremo l'utilizzo di filtri colorati quindi l'effetto del filtro è ovviamente quello di attenuare una determinata componente cromatica e farne passare dell'altra, quindi ad esempio un filtro rosso fa passare la luce rossa e dovrebbe attenuare tutte le altre componenti cromatiche. Cosa abbiamo adoperato poi finora il laboratorio per misurare degli spettri? Tipicamente degli spettroscopi o al reticolo di diffrazione o a prisma, penso più a reticolo, a tutte e due le abbiamo adoperati, ok perfetto. In quel caso capite l'apparato sperimentale è puramente meccanico, non c'è nulla di elettronico, la precisione di questa tecnica sta avvenito sostanzialmente dal goniometro, quindi alla precisione con cui andrà da misurare l'angolo, il corso che è il quale andrà da misurare una ben precisa riga e mi dico che questa strumentazione ovviamente è molto precisa, cioè mi permette di stabilire dall'ungezza donna di una riga con una ottima precisione. Qual è lo subantaggio? Che chiaramente è una misura manuale, quindi voi dovete percorrere diversi gradi fino a quando non incontrate la riga e appuntate manualmente il valore e inoltre non abbiamo alcuna informazione sull'intensità diminosa di quella riga. Pentre lo spettro fotometro vi permette con una sola misura di andare a misurare tutto lo spettro della l'ungezza donna, anche la dove non è presente lucidia di principio, e darvi un'idea delle intensità relative dello spettro in corrispondenza delle diverse lumezza seconda, quindi un'informazione più completa a scapito però della precisione, quindi ad esempio nelle righe spettramoli. Abbiamo una risoluzione che chiaramente non può competere con quella di uno spettroscopio al reticolo o a prisma, quindi lo spettro fotometro permette di misurare l'intensità a diverse lumezza donna, ma ha una risoluzione più scarsa rispetto a quello dello spettroscopio. Come fatto all'interno lo spettro fotometro, che vedrete con una scatoletta chiusa. Innanzitutto abbiamo una fibra ottica, che già vi dico da adesso, maneggiate con cura perché la fibra ottica si può curvare in nata per questa, ma c'è anche il produttore, un angolo massimo di curvatura, oltre a cui si rompe, quindi evitate curvature eccessive. Questa fibra ottica vi servirà proprio per guidare la luce messa dalla sola luce all'interno del dispositivo, quindi sostanzialmente l'estremità libera verrà puntata verso la corrente luminosa. La luce potrà entrare all'interno dello spettro fotometro e c'è una serie di specchi collimatori, vedete qui la luce entra in tutte le sue componenti cromatiche, viene riflessa da questo specchio che ha la funzione di collimare la luce verso questo reticolo di diffrazione, quindi alla fine il principio è 

sempre lo stesso, c'è sempre un reticolo all'interno. La luce a questo punto viene riflessa e diffratta in tutte le sue componenti cromatiche verso uno specchio anche qui focalizzatore e uno specchio fa incidere la radiazione sul sensore. Il sensore che c'è all'interno è un CCD, un charge de caplet de l'ice che è lo stesso sensore che vi è ritrovato sostanzialmente nelle fotocamera e dei cellulari, un sensore di luce. Ne parleremo un po' quando parleremo del rivelatore e semiconduttore. Comunque dovete immaginare che questo rivelatore è un rivelatore a pixel, quindi ha tanti quadratini, ovvio uno verrà investito da una radiazione di colore diverso, perché proprio geometricamente vedete la luce rosse incide in questa estremità del sensore, la luce gru in corrispondenza dell'altra estremità. Quindi ogni pixel sarà sensibile in base alla sua posizione e alla sua larghezza, a una porzione di lunghezza e donna. Capite che la risoluzione a questo punto del sensore, tutto l'apparato, la risoluzione interna di lunghezza e donna è stabilita da quanti pixel abbiamo e delle loro dimensioni. Quindi più piccoli sono i pixel, più pixel abbiamo, ma già la risoluzione del vostro apparato è sperimentale. Nello specifico quello che vi aperete in laboratorio ha una risoluzione di circa mezzonanometro di che vuol dire che insomma questo mezzonanometro sarà stabilito dai pixel di questo sensore, il suo CCD. Se avessimo avuto un sensore più performante con una maglia di pixel, di dimensioni più piccole, questa risoluzione potrebbe essere anche più piccola. Abbiamo questo utilizziamo questo. Dal CCD ovviamente esce fuori esce un segnale, questo segnale viene elaborato anche in questo caso da un ADC, ovviamente voi non vedete nulla e tutto dentro la scatoletta, e alla fine abbiamo un software di acquisizione che vi permette proprio di produrre degli spettri di questo tipo. Quindi vi viene già la volata informazione, vi viene formita la lunghezza d'onda e l'intensità. Guardiamo un po' molto velocemente le sorgenti che abbiamo a disposizione. Per spettere continui vi dicevo l'ampada in caddescienza dei LED, anche se i LED sembra monocromatici in realtà sono delle sorgenti di luce, possiamo dire continua, ma c'è comunque interessa in una porzione di lunghezza d'onda non trascurabile. Faretti all'oggi l'ampa dei flore scenti che sono quelle del soffitto. Anche le fibre Web and Shifter che abbiamo visto l'altra volta allezione, le possiamo illuminare con una luce qualsiasi e la luce che viene emessa dalla fibra può essere misurata. Per andare a vedere se l'ospettro che è andato a misurare normalmente sul verde corrisponde con lo spettro che viene fornito dal costruttore. Spettri di emissione righe in particolari lampade spettrali, lampade miscelate al vapore di mercurio, un piccolo montatore laser, quindi vedrete un po' delle sorgenti monocromatiche. Questo spettro di assorbimenti di soluzione non lo facciamo, facciamo soltanto, vi dicevo, i filtri. Qui vediamo una carrella di immagini, questo ad esempio una lampada miscelata al vapore di mercurio. Quindi vedete è una sorta di spettro intermedio di spettro arricche e spettro continuo perché abbiamo una base continuo dove si vedono dei picchi caratteristici degli elementi presenti in questa lampada. La lampada interdessenza l'abbiamo vista anche prima, che è un classico esempio di spettro continuo. Un'infaretta l'oggiere è un altro spettro continuo che si può analizzare anche diverse tensioni di lavoro. Chiaramente la parte più interessante vi dicevo riguarda le lampade spettrali, ne abbiamo a disposizione diverse e quindi possiamo ad esempio fare una misura dello spettro di una lampade idrogeno e andare a confrontare quello che ottenete sperimentalmente con la serie di Balmer che abbiamo studiato, in struttura sì, sì, ok, comunque quando farete la tesina certamente l'avrete studiata. Chiaramente lampade più complessi, gli elementi più complessi sono un po' più difficili da verificare con dei conti o con delle formule, non esistono delle formule, ma trovate sicuramente dei valori tabulati e letteratura. Questo ad esempio è una, questo è una lampada spianéon, lampade spettrali non ne ho qui, comunque ci sono lampade spettrali che hanno spettri anche molto complessi con diverse righe. Vedete gli spettri LED, non sono per niente monocromatici, guardate la riviezza ad esempio di un LED rosso e di un LED giallo, quindi quello che ha, che sembra un colore definito in realtà non è per nulla, monocromatico. Questo è il tipico spettro di emissione della fibra Webless shifter, quella che abbia fatto vedere la volta scorsa che mette tipicamente nel verde, verde e blu, quindi questo è quello che si ottiene. Questa parte vi diceva che non si fa più l'assorbimento attraverso delle soluzioni chimiche, ma farete certamente la parte di assorbimento attraverso un filtro, quindi prendete una sorgente di riferimento che può essere ad esempio il faretta l'ogge, non fate una misura di questa luce e poi fate una misura interponendo il filtro. Quello che si osserva è ad esempio che una parte della componente cromatica viene totalmente soppressa come questo caso e rimane la componente del relativo al colore del filtro, quindi in questo caso è un filtro rosso e quindi quello che viene fatto passare soprattutto alla lunghezza d'ondo del rosso. Chiaramente il filtro è tanto più professionale quanto più filtra. Questi ovviamente sono le filtri didattici, dei filtri professionali, hanno delle bande passanti, lo stiamo a dire molto più selezionate. Guardate ad esempio questo il filtro verde, ma il riferimento è un po' di verde 

questa, ma oltre al verde passa anche un faretto rosso e un faretto rosso, quindi certamente non è un filtro professionale questo. L'ultima esperienza non ha una presentazione e quella diciamo dei Geiger che non è esperienza ad esame, avrete a disposizione questa scheda di attività, la scorriamo molto velocemente. La prima parte è una parte del tutto generica introduttiva sul riferatore Geiger che già abbiamo discusso insieme, comunque la possiamo leggere per ricordarvi alcune cose. A questo punto viene presentato il contatore come l'abbiamo visto la volta scorsa, ne avrete a disposizione per ogni gruppo tre contatori, quindi ovviamente siete in quattro per lo più, ma magari c'è una persona che prende appunti, prende le misure degli altri che misurano. Vedete che questa scheda di attività vi presenta anche la distribuzione di quassone e vi chiede di fare delle attività intermedie, ad esempio sua distribuzione di quassone vi chiede di valutare il valore predetto della distribuzione per un valore o meno di due e cinque, oppure di riportarlo graficamente, oppure realizzare delle misure brevi con il contatore Geiger, riportarle qui, valutare in certezza soluta, in certezza relativa, poi effettuare delle misure più lunghe e ogni qualmolta appunto vi viene chiesto anche di riflettere sui risultati che abbiamo ottenuto. Comunque al momento non si prende a discutere durante il turno. Quindi una scheda che prevede diverse attività e che alla fine ha come scopo quella di verificare che le misure che abbiamo effettuato il laboratorio seguono effettivamente la distribuzione di quassone. Quindi da un lato dovete calcolare il ciò che è predetto dalla distribuzione di quassone, dall'altro dovete fare delle misure e quindi avrete a disposizione questa scheda. L'ultima cosa che vi volevo mostrare oggi si ricollega agli spettri $\gamma$. Vi ho detto che lo spettro $\gamma$ ha una forma abbastanza caratteristica. Io voglio far vedere come lo spettro $\gamma$ che è misurato il laboratorio è caratteristico del vostro apparato sperimentale. Addirittura abbiamo detto che se avessimo rivelatore di dimensioni infinite lo spettro $\gamma$ sarebbe semplicissimo, avremmo solamente il picco fotoelettrico. Quindi capite che la geometria del rivelatore incide tantissimo su la percentuale di eventi fotoelettrici e di eventi con ton. Vediamo in che modo. Chiaramente questo aspetto si può studiare sperimentalmente proprio utilizzando rivelatore di forme diverse oppure si può simulare. In particolare questi risultati che vi faccio vedere sono un affutto di una simulazione, ma una simulazione sempre proprio per questo. Dal momento che non ho rivelatore di qualsiasi dimensione posso provare a simulare e vedere l'effetto, cosa attendermi. Immaginate di simulare quindi un rivelatore come quello che ha operato il laboratorio, io dico di sodio d'orga taltaglio, due pollici per due pollici, un cilindro di due pollici per due pollici, questo è la sorgente di cesio 137 e inizialmente adopero questa configurazione per cui la distanza tra la sorgente di cesio e il centro del mio rivelatore è di 3 cm. Immagino di avere una certa risoluzione, di ripostruire l'energia, questo è un altro dato che bisogna passare agente, in qualche modo sporcare il risultato per tener conto del reale funzionamento di un rivelatore e vedete ad esempio gli effetti delle dimensioni del rivelatore, questo è quello che ci si aspetta quando il rivelatore ha una lunghezza di 2,54 cm. Vedete questa è la parte relativa al Comton, al continuo Comton questo è l'effetto fotoelettrico e vedete possiamo dire l'effetto fotoelettrico insomma si verifica la maggior parte dei casi quindi nella maggior parte dei casi riusciamo a ripostruire quasi per intero l'energia del fotone incidente, cosa cambia se modifico la lunghezza del rivelatore ad esempio la dimesso, anziché 2,54 la faccio di 1,27 che torna indietro per vedere l'effetto e chiaramente succede quello che ci aspettiamo ma un rivelatore diventa sempre più piccolo e è probabile che parte dell'energia men la persa perché magari qualche prodotto secondario di fotone diffuso può fuori uscire e infatti la percentuale di eventi Comton incomincia a crescere rispetto agli eventi del fotoelettrico questo si accendo ancora di più se si riduce nuovamente la lunghezza del rivelatore di quel cilindro e vedete come ad esempio per 0,254 quindi un decimo della dimensione di partenza a questo punto interviene una forte componente conto non rispetto al fotoelettrico veniamo qui a confronto appunto i tre casi questo era quello di iniziare 2 pollici per 2 pollici e man mano un dimezamento della lunghezza del rivelatore è un passore 10 la risoluzione immaginiamo l'un per un certo di risoluzione vedete come viene ricostruito molto bene il picco fotoelettrico molto stretto si ottiene sempre lo stesso valoro quasi se peggioriamo la risoluzione effetto chiaramente quello di osservare un picco via via più largo 10 per 100 20 per 100 e capito che questo diventa un problema quando ad esempio come nel caso della sorgente del combattuto dove vengono emessi due $\gamma$ con valori di energia abbastanza vicini tra di loro allora in quel caso i due picchi possono arrivare anche a confondersi si sovrappongono certamente di troverete 

sovrapposti perché noi abbiamo un 10\% di risoluzione in grosso modo ma ancora di più se la risoluzione fosse peggiora potrebbe ad avvertura non essere più distinti l'uno dall'altro la direzione dei $\gamma$ i $\gamma$ vengono emessi in tutte le direzioni della sorgente cosa comporta qualsiasi interme di rivelazione a un discorso che i $\gamma$ incidono ortogonalmente la superficie del rivelatore un altro discorso che magari incidono in maniera inclinata quindi promieni da direzioni inclinate quello che succede che se lo spessore di rivelatore è attraversato è piccolo cosa che avviene quando la direzione è molto inclinata magari toccata un angolo del rivelatore non vi detto che il fotone incida interagisca ok perché il fotone ha una certa probabilità di interazione chiaramente più è lungo il percorso che effettuano interno del rivelatore maggiore sarà la probabilità che questo interagisca se invece rivelatore viene toccato solamente nel bordo è chiaro che potrebbe capitare che il fotone non riesce a interangire il $\gamma$ interagisca quindi non venga misurato che cosa cambia questo tambien termini efficienza l'efficienza se vi ricordate rappresenta il numero di particelle rivelate su un numero di particelle incidente dipende cosa si va a vedere se efficienza è in trinsega o efficienza complessiva comunque l'efficienza in trinsega in particolare rappresenta il numero di particelle rivelate su quelle che incidono ok quindi la particella è inciso la toccata il rivelatore ma qual è la probabilità che produga un segnale e quindi media una rivelazione completa allora dipende molto anche dalla direzione dei $\gamma$ se per assurdo vessimo $\gamma$ provenienti tutti quanti della stessa direzione in gilenti perpendicularmente rivelatore efficienza complessiva sarebbe del 75\% invece con una distribuzione di solo troppa quindi tutte le direzioni dello spazio vedete come l'efficienza si ammassa naturalmente e questo ovviamente uno svantaggio per l'esperimentatore perché per accumulare statistica sono richiesse dei tempi ovviamente più lunghi bene questo era un corollario possiamo dire la parte degli spettri che avevamo discorso in precedenza quindi le tre esperienze che vi ho detto sono esperienze di same spettrometria $\gamma$ assorbimento dei beta e spettrofotometra digitale e poi c'è l'esercitazione sui geiger allora per regolarvi comunque ora ve lo mando questo specchietto e lunedì inizierà il gruppo 1 gruppo 1 gruppo 2 gruppo 3 gruppo 4 esatto esatto delle 11 quindi i primi 4 gruppi sono questi forse qualcuno già l'ha fatto presente che ha la difficoltà per lunedì prossimo del caso appunto vi ho detto provate a trovare qualche scambio in grandisco allora adesso io procederò con la consegna dei kit perché a me rimandare in circa la ventina di persona che deveva dire a kit invece che aveva il kit multimetro può portarlo qui e lasciarlo qui ovviamente se ci sono dei problemi me lo segnalate la prossima consegna aspetto ve lo dico quando verrà allora per i kit ulti vedrò la prossima consegna la facciamo giorno 29 quindi la prossima settimana però il laboratorio ok quindi dovete venire alle 11, le lezione prima, le lezione 11 e la prossima con arduino chi deve andare può andare ovviamente si si allora c'è stata una domanda scusate ragazzi nel caso dello spettrofotometro in oltre all'immagine che viene per me c'è un modo per stare arredato e si vi farò vedere che lo spettrofotometro produce tre file due sono delle immagini una come quella che vi possiamo vedere è un'altra è un'immagine proprio del sessibili dominato e il terzo file è un file testo che vi riporta proprio i dati quindi per ogni lunghezza donna si intensita la mesurata quindi poi apposteriò ovviamente insieme ad analisi si può lavorare su questi dati dovete lavorare su questi dati allora quindi mi dicevo un edifico di prossimo consegno i prossimi kit ultimetro volete già sapere chi deve venire forse meglio perché sono venite tutti poi non lo vado tutti ovviamente allora quindi cominciò il nodi alfabetico francesca lì può venire non c'è arena santo si può venire lo ha disegno ok alisia attuglio non ti sento scusa aiarduino ok va bene allora rimandiamo la rimandiamo la passiamo l'auto e barbita già l'ho avuto luca buon anno non c'è fabbrizio buon coraglio non c'è gabriele bruno ti posso segnare poi luca garbona posso segnarti ok cerruto simona aiarduino ok andiamo avanti compagnini domenico andorno di bella posso segnare ok toscati piaci non vedo giordana di fede andrea di fini andrea d'unzo ok ti posso segnare per la prossima volta per il 29 sei tu si ok francesco fiorino si professoressa si dimmi si posso segnarti ok va bene salvatore garofi ok devi prender tu dito di letto e gincoli arduino emmanuele della ferla posso segnarti martina rossa ok lucia lattuga posso segnarti giù e alaudani di mandatuino alessandro li pani li pani scusate non mi ricordo mai e la cazzo guarda la faccia la no non mi ha mi ok niente e alfio lo castro ademri il cardino secondo me non ti consulta in 30 secondi non potrei ed e guarda lo dici francesca lo che però ha preso ha preso il ruino si giusto e antonio macchione ah no l'ha avuto già e marco micci cani si professoresso posso segnarti ok giorgio migliora deve si prender al ruino e andrea mia bella grazie grazie a non moravico posso segnarti ok alessia mosumeci aggacelai scusa era gian e giallucci giallucci bacino posso segnare mattheo marisi e ricapisci vello gioelle proietto gian arugusa che fossera samisante io si posso venire a giorno 29 e donata ragnolo deve prender duino carmen al spagnola non c'è simone rocca ti posso segnare ok salvo sant'angelo posso venire ok simona scalora disegno e francesco scarcevino ok saras crudo per estressa ancora me manca arduino ok si ok però quindi devi venire a ritirare anche arduino si quindi a me quando puoi poter rarduino e considera che ora questo questo profumo terraria arduino per almeno una decina di giorni in questo modo quindi verrà riconsegnato non questo giovetti ma l'altro e quindi la riconsegna viene tra due settimane ok ok giorgio sessa prenderà a due i no giuglia spina ha c'è la risposta la ragione e dario ballone prenderà a due i no so ce la chiara zisa ok va bene il massimo abbiamo intorno che sono risciotto però la chi è stato segnato che è la prossima lunettiglia non è necessario che prenda la corsa ah ok si forse la provo segnato è una mia collega che doveva portarle oggi testa ma non si segnate bene si deve consegnare chi deve consegnare anche chi arduino quindi chi i testa negli auguri ne devo consegnerli oggi e chi ce l'ho si chi non ce l'ha me lo consegna la che si segnata la corsa e il giorno domani per il binario metteteli tutti qua per il diodo si si ma lo sai che si prega per la corsa lo posso mancare controllo di ccia e di chi ha risciletto si, la la si grazie mille, qui te lasciate qui allora si 40 otto le listane si, non penso di aver veduto ah va bene si sta rantendo si preoccupa, 40-50 la non fa di ferma si, riguarda più si, possiamo lasciarla qua si, non si non Alicia a curio si ciao si, non si non si fatto un'imprensione con lei però non lo so si, non so che mi sia capito non so, non so va bene, allora, lo fai va bene, ok se lo trovo, posso venire a dare la mano si, si, si, si, si, si, un'ora, è un'ora e per un'ora, è un'ora è un'ora è va bene, va bene, va bene, ah, va bene ok ha dato benvenno si, si, si ah, va bene ok, Simone, è ciò di mudo? si, si va bene chi ne ha avratto sale? facciamo conto con tre, ragazzi chi ha bisogno? io, Danna Danna, mettiamo quattro ah, ma non c'è benvenuto ha dato saltato va bene, va bene va bene, va bene per il cinque, come è scusato? Garuffo ah, garuffo garuffo per il cinque, poi gente che ha avuto un cinque di diretto ok la vi consegno, se abbiamo tutte quelli ragazzi, che non avrò andato

\textbf{credo sia lez 14 poi vedo dove metterla}

Buongiorno. Allora ragazzi iniziamo, come state? Tutto posso? Sono Sono bene, insami? Allora, come avevo già anticipato alla fine del primo semestre, noi riprendiamo con delle lezioni in aula che non dureranno molto, ormai ci mancano un po' chi argomenti, oggi completeremo la parte dei rivelatori, spero di arrivarci, poi dovremmo fare qualcosina di elettronica, presentare le esperienze del secondo semestre e poi rimane una parte, che in realtà è una parte pratica, che riguarda l'utilizzo del software root e quindi saranno delle esercitazioni che faremo in aula sostanzialmente, quindi vi chiederò poi di portare il computer, di metterci qui a lavorare insieme, quindi più o meno meno programma è completo, io immagino che nell'arco di tre settimane, massimo un mese, noi noi tutto il programma e addirittura pensavo anche di procedere un po' in parallelo, quindi dopo qualche settimana di lezioni in aula, magari cominciare già il laboratorio, quindi una volta che abbiamo tutte le nozioni per poter iniziare le esperienze, io inizierei subito. Vi do un'altra comunicazione che riguarda il fatto che al secondo semestre sarà presente un altro docente, non so se lo sapete, è un nuovo ricercatore che è entrato da poco il dipartimento, in se non si frequenta il dipartimento da tempo perché fa il dottorato e assegna di ricerca qui e adesso da ricercatore gli hanno assegnato per quest'anno accadenico un certo numero di ore per il corso di laboratorio tre, si chiama professore Leonardo, non so se lo conoscete, comunque poi avrò modo di presentarvelo, quindi avremo l'assistenza anche di questa di quest'interiore persona. Allora, se non ci sono questioni, qualcuno mi ha già affermato per dirmi che c'è qualche sovrapposizione di orario per il giorno di giovedì, quindi eventualmente cercate di valutare quante sono le persone che hanno queste problematiche e di farmelo presente perché possiamo anche valutare la possibilità di utilizzare un altro pomeriggianzi che quello del giovedì, se per caso questo problema riguarda più persone, quindi eventualmente fatemelo sapere. Finché facciamo lezione, ovviamente io posso registrare, quindi chi si perde la lezione può riascoltarla, il discorso che noi appunto le lezioni teoriche le finiremo a breve, quindi comunque il problema si porrà poi per la presenza di laboratorio. Allora, vi dicevo oggi tratteremo l'ultimo argomento sui rivelatori, abbiamo abbiamo a trattato i rivelatori che abbiamo utilizzato il laboratorio per le prime esperienze, quindi quindi rivelatori a gas, i rivelatori a scintillazione. Adesso l'ultima tipologia di rivelatore che rimane da ad introdure sono i rivelatori a semiconduttore, è detto anche rivelatore stato solido. Quindi per introdure questi rivelatori dobbiamo richiamare qualche concetto di base della fisica dei semiconduttori che io immagino abbiate affrontato già in altre materie, probabilmente in strutture della materia o in altri corsi, datevi un feedback su questo. Ma che laboratorio giude? Ha il laboratorio 2 quindi abbiamo fatto qualche simile su semiconduttori, semiconduttori drogati, immagino in vio di semiconduttori, quindi qualche concetto lo abbiamo. Noi andremo invece a analizzare i semiconduttori per il loro utilizzo come rivelatori. Allora, questi rivelatori e semiconduttori come diceva la stessa parola si basano sull'utilizzo di materiali semiconduttori che sapete sono dei materiali cristallini come ad esempio quello che vedete qui in questa figura, quindi una struttura ordinata di atomi in un reticolo. Generalmente i rivelatori e semiconduttori sono rivelatori tipicamente o al silicio o al germanio, in in adesso esistono anche altre tipologie di rivelatori però quelle più comunemente utilizzate, comunque i trindi che sono stati inventati si basano sull'utilizzo di silicio e germanio, ovviamente non insieme, silicio o germanio. Sono anche detti rivelatori a stato solido perché hanno una densità che è circa mille volte maggiore rispetto a quella dei gas, quindi questa è una notevole differenza rispetto a quello che abbiamo studiato al primo semestre, rivelatori a gas si basano sull'utilizzo di un gas che sapete a densità molto basse, qui abbiamo invece un materiale solido e questo comporta delle differenze delle caratteristiche che andremo ad approfondire. Sono rivelatori che ormai hanno una loro storia perché i primi prototipi sono stati sviluppati negli anni 60, da allora chiaramente ci sono stati enormi sviluppi e oggi abbiamo rivelatori al silicio particolarmente performanti pensati per diversi tipi di applicazione, quindi quello che noi faremo in questa presentazione è introdurre il concetto di base di rivelazione basata sul semiconduttore, quindi quindi sostanzialmente è stato inventato e rivelato dal semiconduttore, però poi alla fine di questa presentazione vi farò vedere alcune applicazioni, alcuni esempi di come i rivelatori al semiconduttore sono stati sviluppati e migliorati nel corso degli anni. Quelli sono i principali vantaggi di questi rivelatori. Allora in anzitutto hanno un'ottima risoluzione energetica. Se vi ricordate durante il primo ciclo di esperienze avevate la possibilità di misurare la risoluzione in energia con un rivelatore a scintillazione. Se vi ricordate venivano fuori quegli spettri acquisiti utilizzando l'esorgentiga, ma è andando a guardare la larghezza del picco cocaelettrico possiamo valutare la risoluzione in energia. A qualcuno l'ha fatto lì sul momento perché magari ha avuto il tempo e gli strumenti per poterlo fare, altri lo rifaranno quando magari dovranno analizzare i dati in sede di stessura della tessina. Comunque se qualcuno l'ha fatto si è risolvonto che un sistema basato sull'utilizzo di uno scintillatore e di un fotomoltiplicatore porta ad avere risoluzioni in energia che non sono particolarmente spinte, stiamo parlando di risoluzioni dell'ordine del 5-10\%. Vi ricordo che la risoluzione in energia esprime la capacità di uno rivelatore di distinguere due valori in energia molto vicini tra di loro. Quindi se la risoluzione non è buona quello che succede è che si rischia di non riuscire a distinguere due valori in energia molto vicini tra di loro. I rivelatori semi-conduttore invece presentano delle buone risoluzioni energetiche.

Altra caratteristica che noi ritengiamo un vantaggio è l'elevato stopping power, che infatti il fatto di avere a disposizione un materiale solido fa sì che la radiazione che penetra all'interno del rivelatore venga più facilmente arrestata, quindi perda più facilmente energia. Pensiamo ad esempio a un elettrone che deve entrare in un rivelatore a gas per delle energie oppure entrare in un rivelatore a silicio per delle energie. Nel rivelatore a silicio percorre veramente pochi millimetri, un rivelatore a gas potrebbe percorre diversi centimetri. Quindi questo fa sì che affinché ad esempio il rivelatore venga utilizzato per misurare tutta l'energia di una particella sono sufficienti spessori e dimensioni compatte. Ad esempio lo vedrete in laboratorio, questi rivelatori noi utilizzeremo per misurare particelle alfa da 5 mezzo, grosso modo, e le particelle alfa dei 5 mezzo vengono arrestati in poche decine, di micron, di silicio. Quindi è sufficiente il rivelatore ad esempio spesso 50 micron come quello che utilizzate il laboratorio per essere sicuri che le particelle alfa vengono arrestate all'interno del rivelatore, quindi andiamo a misurare tutta l'energia della particella alfa. Altro vantaggio è che in generale questi rivelatori che vedremo sono sostanzialmente delle raggiunzioni PN polarizzate inversamente richiedono delle basse tensioni di alimentazione. E questo è un vantaggio da un punto di vista pratico. Se Se confrontiamo ad esempio con i fotomoltiplicatori che richiedono centinaia di volt, se vi ricordate abbiamo modificato la tensione di lavoro del fotomoltiplicatore usato per i $\gamma$, e li arrivate anche 600 volt, ma ci sono fotomoltiplicatori che richiedono tensioni anche più elevate. Oppure il contatore Geiger magari non vi siatere si conto dell'attenzione di alimentazione perché non l'abbiamo modificata, però anche lì siamo dell'ordine di 300-400 volt. Invece per il rivelatore al silicio lo vedrete sono sufficienti decine di volt per la polarizzazione. Un altro vantaggio riguarda la risposta del rivelatore con una risposta abbastanza veloce. Questo è utile soprattutto per le misure di timing. Cosa che invece non avevamo ad esempio per i rivelatori a gas. Se vi ricordate vi ho sempre detto sono dei rivelatori piuttosto solenti proprio per i meccanismi e i fenomeni che avvengono all'interno del rivelatore a gas. Chiaramente ci sono anche degli svantaggi. Il Il svantaggio è il fatto che questi rivelatori sono parecchio sensibili alla temperatura, quindi le condizioni di lavoro possono cambiare a seconda della temperatura, ambiente della temperatura in cui si sta operando. A volte alcuni di questi richiedono addirittura proprio un sistema di raffreddamento perché altrimenti il rumore sarebbe eccessivamente elevato e questo è il caso del germanio. In generale i rivelatori al germanio necessitano di un raffreddamento. Altro svantaggio che però a noi non interessa per le cose che facciamo in laboratorio riguarda il danneggiamento della radiazione. Infatti essendo il semiconduttore un reticolo, una struttura ordinata in cristallo quando viene sottoposto a un'elevata dose di radiazione questa dose può produrre dei danni al reticolo quindi modificare in qualche modo la struttura ordinata del reticolo e causare delle conseguenze sul funzionamento del rivelatore. Ora vi dicevo questo non è un problema per le esperienze che svolgiamo in laboratorio, dal momento che noi utilizziamo una sorgente alpha che è comunque un'attività abbastanza bassa. Questi problemi si presentano laddove questi rivelatori devono essere adoperati in presenza di alte radiazioni, frussi di dose alta radiazione come può essere ad esempio il caso di esperimenti sotto fascio, quindi quindi un acceleratore oppure esperimenti nello spazio dove la radiazione cosmica diventa importante perché non si è più schermati dal filtro dovuto all'atmosfera terrestre. Infine un ultimo svantaggio che qui non è riportato, in realtà era stato riportato come vantaggio cioè le dimensioni compatte perché da un lato è bella avere il rivelatore piccolino facile da trasportare. Dall'altro immaginate però di dover rivestire una superficie molte stesse con un rivelatore semiconduttore. Questo diventa estremamente costoso e anche impegnativo dal punto di vista, lo vedremo di elettronica, di consumo, quindi in generale quando si deve realizzare un rivelatore al silicio o un rivelatore con un cassone conduttore di dimensioni stesse, nascono altre problematiche e può essere non facile affrontarle. Quindi richiamiamo molto velocemente le proprietà dei semiconduttori. Voi sapete che i materiali solidi si distinguono in tre categorie diverse a seconda dello struttura a ban de livelli di energia. Quindi in generale noi sappiamo che se queste ban di valenze di conduzione sono separati da un gap abbastanza esteso abbiamo un materiale isolante, se invece il gap è inesistente il materiale è un materiale conduttore, una situazione intermedia invece l'abbiamo per i materiali solidi conduttori dove questa gap è presente, è una gap dove non possono essere presenti dei livelli energetici, una gap di energia proibita, però non ha una dimensione molto grande.

Parliamo di alcuni electron volt quindi esiste ma è abbastanza piccola. Questo cosa comporta? Comporta il fatto che se noi proviamo ad applicare un campo elettrico a questi tre eticologie di materiale, chiaramente un campo elettrico applicato in isolante non comporta il passaggio di corrente perché tutti gli elettroni si trovano nella banda di valenza e quindi non abbiano conduzione. In un conduttore invece si dovrebbe osservare il passaggio di una certa corrente e il semiconduttore si osserva una piccolissima corrente legata al fatto che alcuni elettroni che noi definiamo elettroni termici che hanno acquisito un'energia proprio perfetti termici, sufficiente a superare il getto energetico, questi elettroni possono condurre, però chiaramente capite che è un effetto proprio curamente statistico dovuta alla agitazione termica e quindi si genera una corrente e una corrente abbastanza debole. Ne semiconduttori abbiamo un'altra caratteristica fondamentale, cioè il fatto che i portatori di carica, cioè le cariche che generano una corrente possono essere di due tipologie, elettroni e lacune. Voi sapete che le lacune sapete che sono delle assenze, delle vacanze di elettroni e anche se concorrono alla valutazione della corrente in un semiconduttore, quindi ogni volta che noi andremo a parlare di questi effetti di corrente, faremo riferimento sia elettroni che lacune. In un semiconduttore, chiaramente poi ci siamo fermando ancora i semiconduttori puri, ok? In un semiconduttore normalmente questi sono due fenomeni che sono in competizione l'uno con l'altro. Il primo fenomeno è quello della creazione di coppie elettrone lacuna che abbiamo già accennato, cioè per effetti termici effettivamente un elettrone potrebbe passare dalla banda di valenza alla banda di conduzione e quindi si viene a creare una coppia elettrone lacuna. Viceversa potremmo avere un fenomeno posto, cioè quello della ricombinazione, cioè un elettrone potrebbe ricombinarsi con una lacuna, perché ovviamente valori di energia e di impulso lo consentono. Capite che in generale in un semiconduttore puro, ho detto anche in trinceco, il numero di elettroni che si trova nella banda di conduzione è esattamente uguale al numero delle lacune in banda di valenza. Questo è abbastanza chiaro perché ogni volta che un elettrone viene promosso, salta dalla banda di valenza, la banda di conduzione, crea la sua lacuna corrispondente. E vi dicevo che sia gli elettroni che le lacune contribuiscono alla conducibiliate elettrica del semiconduttore. E allora, quanto vale questa concentrazione di elettroni e di lacune? La indichiamo genericamente con N con I, dal momento che in questa fase c'è un semiconduttore auguro il numero di elettroni corrisponde al numero di lacuna. Quindi la chiamiamo N con I, che rappresenta la concentrazione di elettroni e di di E chiaramente questo N con I dipende innanzitutto dal numero di possibili stati che abbiamo nella banda di conduzione e nella banda di valenza. Quindi ritroviamo qui questa radice di N con C per N con I. E poi abbiamo questo fattore esponenziale, un po' la Botsman, dove compare la temperatura e compare E con G, cioè la gap energetica. L'energia della gap proibita, quanto è ampia. Ed è chiaro che queste dipendenze sono abbastanza intuitive perché se la gap è piccola sarà più facile per un elettrone lasciare la banda di valenza e andare verso la banda di conduzione. Ed effettivamente avere qui un esponenziale di meno e con G esprime proprio questo aspetto.

Viceversa la temperatura più elevata, la temperatura più sarà probabile che un elettrone perfetto termico possa nuovamente passare alla banda di conduzione. Ed ecco perché ci ritroviamo qui al denominatore in questo esponente. Ora, utilizzando la statistica di Fermidira, è possibile andare a valutare questo numero di stati nella banda di conduzione e dalla banda di valenza e si trova che corrisponde a un fattore A, una costante per T inalzata a tre mezzi. Quindi abbiamo semplicemente espresso il fatto che la concentrazione di elettroni di lacune che mi ritrovo libere per effetto termico semplicemente dipende dalle caratteristiche del semiconduttore, quindi in particolare dai con G e dalla temperatura. Ora quanti sono questi elettrone e queste lacune che ci ritroviamo magari a temperatura ambiente? Supponiamo di avere una temperatura di 300 gradi Kelvin e andiamo a sostituire questo valore qui all'interno di questa formula considerando due semiconduttori di tipo diverso il germanio e il silicio che si differenziano per l'energia e con G, la dimensione del laghetto. E allora se fate questo conto troverete che nel caso del germanio questa N con I corrisponde all'incirca 2,5 per 10 alla 13 su centimetro cubo che sembra un numero enorme così presa a sé. Ma in realtà se considerate che in media abbiamo 10 alla 22 atomi ogni centimetro cubo questo numero che abbiamo estratto prima corrisponde in realtà a dire che soltanto un elettrone su un miliardo di atomi si trova nella banda di conduzione quindi effettivamente non sono tantissimi e questo corrisponde a quello che avevamo detto all'inizio se vi ricordate se applichiamo un campo elettrico a un materiale sempliconduttore si genera una corrente ma una corrente molto piccola ed effettivamente così ti roviamo poche coppie elettrone, le lacune lacune ed effetti termici. Il silicio che è una gap leggermente più ampia ha in media 1,5 per 10 alla 10 e le coppie elettrone lacune per centimetro cubo che corrisponde in questo caso a uno su 10 alla 12 atomi quindi anche qui è un numero abbastanza basso. Se noi appliciamo un campo elettrico sotto razione di un campo elettrico e elettrone lacune cominciano a muoversi, cominciano a derivare e possiamo valutare la velocità con cui abbia questa deriva. La velocità di elettroni e la velocità delle lacune sono proporzionali al valore del campo elettrico come giusto che sia e alla mobilità che sono diverse per elettroni e lacune. Queste mobilità dipendono non sono costanti, dipendono in realtà dal valore del campo elettrico e della temperatura e quindi in generale ad esempio per il silicio a temperature normali troviamo che le mobilità risultano essere costanti per valori di campo elettrico inferiori a mille volte per centimetro quindi finché vi mantengo con campi elettrici abbastanza piccoli miucone e miuconacca le posso ritenere costanti. Poi incominciano ad avere delle dipendenze dal valore del campo elettrico quindi in particolare una dipendenza come uno sul radice di E nella zona intermedia tra 10 alla 3 e 10 alla 4 volte su centimetro è una dipendenza come uno su E per valori di campo elettrico maggiori di 10 alla 4 volte su centimetro. Questo comporta il fatto sostanzialmente che è arrivato a un certo punto si giunge a una saturazione della velocità a un valore massimo di 10 alla 7 centimetri al secondo legato al fatto che l'energia che viene acquisita da questi elettroni e queste lacune viene poi persa per gli urti con il reticolo quindi non si va oltre comunque una certa velocità di saturazione. La conuttività quindi si può esprimere andando a considerare entrambi i contributi. Quella dovuta gli elettroni e quella dovuta le lacune quindi è semplicemente dato dal prodotto della carita per la concentrazione di elettroni o la punna per la mobilità. Vedete appunto è la somma di entrambi i contributi. E questa parte che abbiamo fatto fino adesso riguarda la produzione di coppie che era il primo dei fenomeni che avviene in un semi-conductor in trincepo quindi creazione di coppie elettrone e lacuna per effetti termici. Ora andiamo a vedere l'altro processo, quello posso di riconvinazione. Può venire una riconvinazione spontanea cioè legato al fatto che elettrone e lacune che hanno opportuni valori di energia ed impulso possono riconvinarsi e da luogo all'emissione di un fotone. Ora questo che sembra un meccanismo molto facile in realtà è un meccanismo raro perché capite che non è sufficiente che elettroni e lacune si trovino vicine per ricominarsi ma davvero aveva anche dei valori opportuni di energia e impulso. Ed è un processo raro tanto che la vita media di una coppia che è stata creata per effetti termici è dell'ordine di un secondo quindi se si viene a creare una coppia a causa di effetti termici questa coppia il media sopravvive un secondo e poi per qualche motivo c'è una ricombinazione e quindi questa coppia cessa di esistere. Ora un secondo sembra un tempo piccolo ma in realtà è confrontato con i tempi della rivelazione, tempi che che tuttavia quello che si osserva sperimentalmente è che la vita media di una coppia è molto più bassa di un secondo quindi evidentemente ci sono altri meccanismi oltre alla ricombinazione spontanea che portano alla scomparsa della coppia. Questi meccanismi sono legati alla presenza dei cosiddetti centri di ricombinazione. Infatti a causa di difetti della struttura cristallina si possono presentare dei livelli nella zona proibita, sono dei livelli non definiamo profondi nel senso che sono abbastanza distanziati sia dalla banda di conduzione che dalla banda di valenza, li vedete ad esempio qui tratteggiati.

In questi livelli si può verificare una ricombinazione perché sono livelli che possono attrarre elettroni dalla banda di conduzione e la pune della banda di valenza permettendo una ricombinazione, ecco perché vengono definiti come centri di ricombinazione, quindi oltre alla ricombinazione spontanea possono essere presenti dei centri di ricombinazione che fanno sì che alla fine la vita media di una coppia sia effettivamente più bassa. Da un punto di vista della rivelazione cosa ci interessa? Ci interessa che se il rivelatore semi-conduttore come effetto del passaggio di una particella produce coppie elettrone e lacuna, un po' in analogia a quella quella abbiamo visto nelle rivelatori a gas. Se vi ricordate il passaggio della particella produceva coppie elettrone che noi poi andavamo a raccogliere agli elettrodi. Qui vedremo che in un rivelatore semi-conduttore il segnale sarà basato sulla creazione di coppie elettrone lacuna, quindi io sono interessata a raccogliere queste cariche e non voglio che si ricombinino prima che io riesco a raccoglierle quindi non voglio che la vita media sia effettivamente eccessivamente bassa perché se queste coppie si ricombinano troppo rapidamente io il segnale lo perdo. Quindi questo che cosa mi porta a dire? Mi porta a dire che i semi-conduttori che devo utilizzare nei rivelatori devono essere dei semi-conduttori esseramente puri, devono presentare pochi difetti perché più difetti ci sono più aumentano i centri di ricombinazione. Alcuni centri di ricombinazione sono anche definiti dei centri trappola, sono in realtà dei livelli dove magari viene catturato soltanto una tipologia di carica quindi ad esempio viene catturato un elettrone dalla banda di conduzione e l'elettrone rimane di per parecchio tempo, ecco perché venivano definiti dei livelli trappola. I difetti ci possono essere in un reticolo cristallino come quello di un semi-conduttore. Vedete in questa figura sono rappresentate le principali difetti di un reticolo allora potremmo avere ad esempio una vacanza come quella che vedete qui cioè l'assenza di un atomo nel reticolo qui era previsto un atomo non è presente, questa è una vacanza. Ogni difetto del reticolo comporta una modifica dello schermo dei livelli energetici a questa conseguenza poi potremmo avere un altro caso come quello che vedete qui il numero 2 dove il difetto è autointerstiziale cioè è proprio l'opposto della vacanza vedete un atomo dello stesso tipo ad esempio se questo è silicio un atomo di silicio si trova in più rispetto a quanto previsto dello schermo del reticolo. Poi potremmo avere un altro difetto come quello che vedete qui in alto interstiziale che si differenze dal precedente per il fatto che abbiamo sempre un atomo in più rispetto a quanto previsto però è un atomo di natura diversa. E poi gli ultimi due il 4 e il 5 vedete sono dei difetti sostituzionali cioè un atomo di silicio ad esempio viene sostituito con un atomo di tipologia diversa però affinché questa venga le dimensioni devono essere abbastanza simili, i raggi dei due automi devono essere similari. Scegliendo in maniera opportuna degli elementi da aggiungere al reticolo cristallino possiamo andare a generare degli ulteriori livelli energetici che questa volta sono superficiali quindi da un lato abbiamo visto che i difetti comportano la formazione di centri di riconvinazione o centritratola che sono uno svantaggio per noi quindi vorremmo effettivamente dei cristalli puri. Dall'altro però ci si resi conto che con opportuni drogaggi del semiconduttore si è in grado di modificare la struttura dei livelli energetici al nostro favore in particolare allo scopo di generare dei livelli che sono vicini o alla banda di conduzione o alla banda di valenza quindi sono dei livelli superficiali. E quindi in realtà si utilizzano semiconduttori drogati. Che cosa cambia un semiconduttore drogato? Sostanzialmente si vanno a introdurre, a sostituire gli atomi del semiconduttore quindi può essere siddicio germano con atomi di elementi diversi ad esempio normalmente nel caso del silicio si dice il germano sono entrambi i tetravalenti quindi hanno quattro elettroni di valenza e è necessario introdurre per drogare o atomi trivalenti o atomi pentavalenti. I semiconduttori drogati sono anche detti semiconduttori estrinseci. Capite che quello che succede sarà a cambiare la proporzione tra elettroni e lacune mentre in un semiconduttore puro il numero di elettroni coincide con il numero delle lacune generati per effetti termici qui drogando il materiale chiaramente andiamo a creare unis ecolibrio. E allora ad esempio un semiconduttore drogato di tipo N si va a sostituire un atomo del reticolo con un atomo pentavalente come può essere l'ascenico, il fosforo, l'antimonio quindi vedete che questo atomo chiaramente avrà 5 elettroni di cui quattro vanno a legarsi con i quattro atomi di sitiscio che si tromano attorno, il quinto elettrone rimane come un elettrone in eccesso che rappresenta un portatore di carica, può essere appunto un elettrone di conduzione. Questo equivale sostanzialmente a generare nella zona dell'elettro prohibito, nell'energia di cap proibita, è un livello che è un livello superficiale, un livello molto vicino alla banda di conduzione. Quindi questo elettrone si troverà in questo livello discreto, molto vicino alla banda di conduzione quando dicono vicino intendiamo ad esempio a una distanza di 0,01 elettronvolte nel caso del Germania o 0,05 nel caso del Siricio e quindi un elettronico che con una piccolissima energia è in grado di passare alla banda di conduzione, quindi un elettrone quasi libero sostanzialmente. Con valori di drogaggio tipici si possono raggiungere ad esempio un numero di elettroni di conduzione di 10 alla 17 su 100 metro cubo, mentre il numero di lacune si riduce a 10 alla 3 su 100 metro cubo. Ecco perché i simiconduttori drogati di tipo N, i portatori di carica maggioritari sono gli elettroni, 

mentre le lacune rappresentano i portatori di carica minoritari. Situazioni opposte in simiconduttore drogato di tipo P, in questo caso si va a sostituire un atomo del reticolo con un atomo trivalente come il gallio, il borro, l'indio e in questo caso quello che rimane in eccesso è una lacuna. Questa lacuna si troverà in un livello discreto, vedete qui, che si trova nella banda di energia probita, a una distanza molto piccola dalla banda di valenza. In questo caso le lacune rappresentano i portatori di carica maggioritari e gli elettroni portatori di carica minoritari, quindi abbiamo la situazione esattamente speculare. Si può dimostrare che, indipendentemente del tipo di drogaggio, il prodotto della concentrazione di elettroni per la concentrazione di lacune è sempre pari a N con I al quadrato dove N con I è la concentrazione del simiconduttore intrinceco, quella che avevamo visto precedentemente. Quindi che cosa succede? Che se aumentano gli elettroni per effetto di un drogaggio, la quantità delle lacune che si formano nel simiconduttore non rimane costante ma diminuisce e viceversa nel caso opposto. Questa legge prende il nome di legge dell'azione di massa. Allora, partendo dalla legge precedente, ad esempio, un insiamiconduttore di tipo N, mi aspetto che il numero di atomi accettori sia zero, gli atomi accettori sono come gli atomi trivalenti, mentre il numero di atomi donori N con I comporta la formazione di un certo numero N di elettroni. Allora, andando a sostituire alla legge di prima apposso di N e N con I, troviamo questa relazione e possiamo andare a esprimere la conducibilità del materiale in questo modo. Alla logamente, per un simiconduttore drogato di tipo P. Quindi vedete la differenza di prima dove la conducibilità veniva espressa come sommatoria di due termini, uno dovuta agli elettroni o uno dovuta alla elettrocone. Qui alla fine la conduzione elettrica è affidata ai portatori maggiori di carica, quindi gli elettroni nel caso dei simiconduttori drogati di tipo N e le lacune nei casi semiconduttori drogati di tipo P. Quindi se volete avere una corrente più elevata, quello che dovete fare è drogare maggiormente il materiale. Quindi se abbiamo un drogaggio di tipo N dovete aumentare la concentrazione degli atomi donori. Mi c'è versato il simiconduttore di tipo P, dovete aumentare la concentrazione di atomi acettori. Troverete a volte delle sigle un po diverse, P più N e più EI, che cosa stanno a indicare queste sigle? Può essere utile andare a drogare i simiconduttori con elevate concentrazioni. Si parla anche di 10 a 20 atomi su centimetro cubo, anziché i classici 10 a 13 a 1 su centimetro cubo. Questi materiali risultano essere altamente conduttivi per quello che abbiamo detto prima. Questi materiali sono indicati proprio per indicare che la concentrazione è elevata e vengono tipicamente adoperati per l'enerizzazione dei contatti elettrici. Voi sapete che in qualsiasi rivelatore per poter estrarre un segnale elettrico abbiamo bisogno di contatti elettrici. Quindi nel caso, ad esempio, del del a gas, se vi ricordate, avevate a Anodo e Cato Do, da cui potevate prelevare un segnale. Qui il materiale semiconduttore non ha dei contatti, bisogna crearli e non è facile crearli, lo vedremo un po più là, perché nel momento in cui io provo a utilizzare un materiale metallico a contatto con un semiconduttore purtroppo introduco degli effetti collaterali. Allora, per evitare questo, quello che si fa è andare a realizzare delle zone ad alto drogaggio proprio là dove voglio andare a creare il contatto omico con un metallo. Comunque questo lo approfondiremo dopo. Per adesso mi interessa semplicemente darvi la definizione di P più N più. Addirittura troverete anche N più più P più più per dire che sono concentrazione ancora più elevate. Invece la I, perché cosa sta? Sembrerebbe stare per intrinceco, ma in realtà non è esattamente così. Nel senso un materiale di tipo I si comporta come un materiale intrinceco, un materiale puro, ma in realtà prende il nome di materiale compensato, perché comunque sia un semiconduttore che è stato drogato, sia di tipo N che di tipo P con la stessa concentrazione. Quindi l'effetto risultante è semplicemente riportare il semiconduttore alla condizione di intrinceco, di puro, ma dovete sempre ricordarvi che è il testato drogato. Chiaro? E sono materiali che hanno una resistività elevata, ovviamente. Allora facciamo un passo avanti e parliamo delle giunzioni, perché questo è il principio di base di un rivelatore, quindi una giunzione PN o Np. Se andate a prendere due materiali, uno drogato di tipo N e uno drogato di tipo P e li mettete accostati uno vicino all'altro, in realtà non è così semplice come lo sto dicendo, ma ma che sia così, ma vengono dei fenomeni. State a metterne a contatto dei materiali che hanno concentrazione di carite diverse l'una dall'altro, quindi quindi mettendo vicino un materiale di tipo N dove c'è un'elevata concentrazione di elettroni liberi. È un materiale di tipo P dove abbiamo un'elevata concentrazione di la puna. Cosa succede? Chiaramente comincia a avere una diffusione di elettroni verso il materiale di tipo P e di la puna verso il materiale di tipo N, a causa proprio di questa differente concentrazione. Ora gli elettroni che si diffondono nella zona di tipo P incontrono le lacune e si ricombinano con le lacune. Viceversa lo stessa viene nel materiale di tipo N, con le lacune che si sono diffuse nel materiale di tipo N. Ora siccome le regioni inizialmente erano neutre, ecco mi sono dimenticata di specificare una cosa, torno un attimo indietro, qui anche se noi andiamo a introdurre un'impurezza, un attimo di tipo diverso, il materiale rimane pur sempre neutro, questo deve essere chiaro perché ad esempio qui dove è introdotto un'impurità di tipo N, un donore, vedete c'è l'elettrono in eccesso, ma non vuol dire che il 

materiale carico negativamente perché abbiamo la carica del nucleo che compensa ovviamente l'elettrone libero. Quindi complessivamente abbiamo un materiale neutro, stessa cosa per il materiale di tipo P. Quindi detto questo inizia di un'impurità un'impurità tipo tipo tipo che si viene invece a creare una carica nella regione dell'aggiunzione, ora lo vediamo come un disegno, ecco. Dovete immaginare che questo è il il N, questo è il il P, quindi inizialmente in N erano presenti gli elettroni liberi portato di maggiorità di carica, qui qui avevamo le lacune, gli elettroni sono passati da N a P e si sono ricombinati con le lacune. Cosa è successo? Che qui nel reticolo sono rimasti degli ioni negativi, fissi nel reticolo. Biceversa nella zona di tipo N, le lacune si sono ricombinati con gli elettroni e quindi sono rimasti fissi nel reticolo gli ioni positivi. Cosa comporta questo? Comporta la formazione di un potenziale, si crea un campo elettrico, un potenziale di contatto. Vedete qui come si sono deformate i livelli energetici. In questo disegno vedete sull'asse verticale sempre i livelli in energia. L'asse orizzontale lo dovete immaginare come un'asse spaziale che corrisponde alla figura che vedete in alto. Qui ci ritroiamo nella zona dell'aggiunzione, qui nella zona B e qui nella zona N. Se creato una differenza di potenziale, dovuta la presenza di queste cariche fisse, nel neritico dei materiali di tipo P di tipo N, una differenza di potenziale che infidisce ulteriore passaggio e diffusione di cariche libera. Se creato una regione che dal punto di vista della rivelazione è ottimale, perché è una regione in cui non sta circolando carica libera. E se per caso dovesse passare una particella che comporta la formazione di nuove cariche, queste cariche possono essere spazzate via da questo potenziale e raccolte. Sulla destra vedete grafici analoghi sempre relativi a questa situazione della aggiunzione, dove vedete la densità di carica e il relativo campo elettrico. Questo potenziale di contatto è abbastanza piccolino e circa un volt. La zona che si è venuta a creare prende il nome di zona di suotamento, perché è una zona dove ho eliminato tutte le cariche libere. Sono presenti delle cariche fisse, che determinano un campo elettrico, ma non sono presenti i cariche liberi, che potrebbero rappresentare l'umore per me, perché sono cariche che non sono dettate dalla passaggio di una radiazione, ma sono cariche che erano presenti nel materiale semiconductor. Invece non ci sono in questa regione, quindi una regione utile per la rivelazione. Ma quanto si estende, quanto è grande questa regione? Lo possiamo valutare andando a guardare la concentrazione degli atomi acettori e degli atomi donori. Se ad esempio la concentrazione di carica, la densità di carica, ha un andamento che vedete qui nella figura in alta a destra, quindi è un andamento sostanzialmente costante fino a una certa profondità. X con m, in questo caso, è x con p nel caso del materiale di tipo p. Questa densità di carica la posso esprimere, ad esempio, con questa espressione, con questa funzione, per entrambe le zone, zona di tipo n, quelle in alto, e zona di tipo p, quelle in basso. Utilizzando le equazioni di Poisson, posso andare a determinare la forma del potenziale e la profondità, x con n e x con p, in cui si va a estendere la regione di svuotamento. Vedete che x con n e x con p dipendono dal potenziale di con zero, il potenziale di contatto e dalle concentrazioni. All'inizio della formula, che può essere complicata, ci interessa vedere una cosa che, se, torniamo un attimo indietro, ho un materiale n estremamente drogato con un alto drogaggio, allora la giunzione si estenderà maggiormente nella zona p. O viceversa, se il materiale p è più drogato rispetto a quello n, la giunzione si estenderà di più nella regione n. E tutto dipende, sostanzialmente, da come è stato drogato il materiale. Tuttavia, in ogni caso, sono regioni di svuotamento molto piccole, parliamo tipicamente di dimensioni dell'organicento micron. In queste condizioni, un rivelatore basato su una giunzione pn avrà prestazioni abbastanza limitate in termini di rumore, risoluzione e stop-in-power. Quindi, come possiamo migliorarlo? Semplicemente ampliando questa regione di svuotamento e questo si fa polarizzando la giunzione, cioè applicando una differenza di potenziale esterna per aumentare la regione in cui si verifica lo svuotamento. Quindi è una polarizzazione inversa, cioè vuol dire che, vedete, il potenziale positivo viene applicato al materiale di tipo n, quello negativo al materiale di tipoppi. In questo modo si allarga la zona di svuotamento, che rappresenterà proprio il volume sensibile per la rivelazione delle particelle. Cosa succede se la giunzione viene polarizzata direttamente? L'abbiamo studiato questo? Si contrae e si conduce. Sì, sostanzialmente si produce luce. Per esempio, un LED si basa su questo principio. Qui abbiamo esattamente l'opposto. Noi siamo interessati ad aumentare la regione di svuotamento e a utilizzare questa zona per la rivelazione. In questo modo possiamo aumentare la regione di svuotamento anche a valori dell'ordine del millimetro. Non si può andare oltre un certo valore perché poi abbiamo limite dettato dalla resistività del materiale. Quindi, vi assumendo, rivelato alla semiconductor come funziona, è un'aggiunzione PN polarizzata inversamente. La regione di svuotamento è la regione attiva, quella che noi utilizziamo sostanzialmente per la rivelazione. Vedete ad esempio qui schematizzato un rivelatore a semiconductor, dove abbiamo la zona N, che vedete qui, che rappresenta la parte principale di questa aggiunzione. Poi abbiamo dei contatti, qui realizzati con un P più, e con un N più dall'altro lato. Quando passa una particella all'interno della regione di svuotamento, la particella deposita energia. Queste energie viano utilizzate per produrre coppie, 

elettrone e lacuna, e queste coppie cominciano a migliorare verso gli elettrodi per poi raccolte, per innurre il segnale che poi noi andiamo a misurare. Quindi vedete, è qualcosa di molto simile a quello che abbiamo visto in un rivelatore a gas, dove quello che cambia è il mezzo in cui avviene il processo, e anche il tipo di processo cui produciamo coppie elettrone e lacuna, nel caso di un rivelatore a gironizzazione si produce i gironizzazioni. Questo è un po' uno schema riassuntivo di quello che abbiamo visto finora nel campo della rivelazione. Vedete che infatti i rivelatori a gas, i rivelatori a semiconductor, presentano sostanzialmente uno schema molto similare. I rivelatori a scintillazione ovviamente hanno un comportamento diverso, si basano su principi fisici abbastanza diversi. Ma come si realizzano un rivelatore a semiconductor? Quindi come si realizzano queste junzioni? Ci sono diverse tecniche, diciamo diversi processi, non utilizzati per creare la barriera. E noi andremo a vedere molto velocemente, molto rapidamente, alcuni di questi, quelli più utilizzati. Ad esempio, i rivelatori a diffusione, questi vengono realizzati facendo diffondere delle impurità di tipo N, come ad esempio il fosforo, in un'estremità di un semiconduttore di tipo P. E per fare questo sono necessarie delle elevate temperature, anche mille gradi. Si gioca un po' sui tempi di diffusione, sulle concentrazioni, in maniera tale da avere una giunzione adeguata. Tuttavia, il principale problema di questo tipo di rivelatori è che la giunzione non si forma in superficie, veloce si forma, vedete qui in questa figura, a una profondità di alcune decine di micron. Questo vuol dire che ad esempio una particella che incide sul rivelatore dovrà attraversare prima questa zona, che è una zona per noi morta, perché non è una zona utile per la rivelazione, e poi entrare nella regione di suotamento. Quindi perdiamo comunque sia parte dell'informazione trasportata dalla particella. Quindi il principale svantaggio è questo, che limita certamente le misure di energia. E un altro svantaggio sono le alte temperature che si adobrano, perché aumentano il rumore e tendono a diminuire la vita media dei portatori di carica, che abbiamo detto non è una cosa che vanno a trovare un svantaggio, perché noi vorremmo andare a misurare queste cariche prodotte. I vantaggi sono certamente la robustezza e le basse contaminazioni superficiali. Superano questo problema della barriera, a una certa profondità quindi della presenza di una zona morta, questi rivelatori, i rivelatori è barriera superficiale. Questi sono basati su dei dio di shopki, cioè dei dio di che si formano non con due semiconduttore, pensi con un semiconduttore e un metallo. Infatti quando andate ad accostare, vedete, un metallo con un semiconduttore, quello che succede è la formazione anche in questo caso di un aggiunzione. Quindi si può ad esempio adoperare dell'oro su un materiale di Tqn o dell'alluminio su un materiale di Tqnp. La produzione di questi rivelatori avviene innanzitutto trattando la superficie chimicamente, ossidandola e poi depositando lo strato metallico per evaporazione. Il tutto infine viene montato su un anello isolante con delle superficie metallizzate per assicurare il contatto 

elettrico. Questo ad esempio è il rivelatore che noi adobberemo per l'esperienza della misura della radiazione alfa e si presenta così con questo aspetto. Quindi vedete qui l'esterno e l'anello su cui viene montato il tutto. All'interno è posizionato il rivelatore, che è questo. Il connettore ad habitare permette appunto il passaggio del segnale. I vantaggi è che in questo caso abbiamo dei rivelatori totalmente svuotati, non abbiamo nessuna zona morta come avveniva invece nel caso precedente. Possono essere profondi anche diversi millimetri. Nel nostro caso non ci interessa, però potrebbe essere utile per la rivelazione di altre particelle. Il processo di lavorazione avviene a temperatura ambiente, al solito a differenza di quanto abbiamo visto per il rivelatore di diffusione. Lo spontaggio quale è che lo strato depositato è talmente sottile, un strato metallico depositato così sottile che purtroppo non isola dalla luce e quindi sono rivelatori che possono essere sensibili alla luce. Infatti la radiazione luminosa, i fotoni del visibile hanno un'energia sufficiente a poter creare delle coppie elettorane la puna. Inoltre sono anche sensibili a possibili contaminazioni superficiali. Altre tipologie di rivelatori si basano invece sull'utilizzo dell'impiantazione ionica, quindi quello che avviene, il drogaggio, avviene attraverso l'utilizzo di acceleratori che accelerano dei fasci di ioni che sono le nostre impurità per impiantarle all'interno di un materiale semiconduttore. Capite di un processo proprio violento, un bombardamento sostanzialmente del materiale semiconduttore e questo fa sì che alla fine sia necessario un annealing a 500 gradi per poter ripristinarli da anni eventualmente causati da questo processo di impiantazione. I vantaggi che sono dei rivelatori molto starabili con finestre di ingresso molto sottili, quindi la zona morte eventualmente di poche decine di nanometri, ma l'osvantaggio che sono parecchio costosse, soprattutto per i processi di produzione che richiedono l'utilizzo di un acceleratore. L'ultimo rivelatore di cui volevo parlare è il rivelatore che prende il nome di Silly, rivelatore a deriva di Lithio, Silly, appunto che viene denominato Silly. Questi rivelatori cercano di risolvere il problema delle piccole dimensioni della regione di suotamento. Vi ho detto che comunque c'è sempre un limite a questa regione di suotamento, perché poi il diolo ha rottura. Allora per risolvere questo problema si utilizza una cosiddetta aggiunzione PIN, dove al materiale di tipo P e al materiale di tipo N si frappone un materiale di tipo compensato, il tipo di cui avevo parlato prima. Allora in questo modo è possibile sostanzialmente ampliare la zona in cui può avvenire la rivelazione. Si possono arrivare anche a spessori di 10-15 mm, che è molto utile laddove, abbiamo bisogno di range elevati per poter fermare una parte cella e misurare nell'energia. Questo è il processo con cui avviene la realizzazione di questi rivelatori, quindi nello specifico. Qui abbiamo del materiale dopato di tipo P, il litio viene posizionato in superficie, viene fatto diffondere attraverso l'applicazione di un campo elettrico, e alla fine quello che si ottiene è un'aggiunzione di tipo PIN, perché qui rimane la zona P, perché il litio non è arrivato fino a questa estremità. La regione centrale diventa di tipo compensato, perché è la drogata di tipo P e gli aggiungiamo il litio di tipo N, e poi invece rimane la zona di tipo N dovuta essenzialmente a un'alta concentrazione di tipo P. I vantaggi sono ovviamente questi spessori elevati, quindi vennero utilizzati per la spettroscopia beta o per i raggi X, bassa energia. I svantaggi è il fatto di si debba doverare a temperature basse, a causa di un rumore termico, e inoltre anche la conservazione dovrebbe avvenire a temperatura basse per mantenere inalterata la zona intrinzeca. L'ultimo aspetto che volevo discutere sul rivolto da Siriccio riguarda proprio la realizzazione dei contatti. Ve l'avevo accennato poco fa. Non è possibile banalmente creare un contatto omico utilizzando un materiale metallico sul semiconduttore, perché l'abbiamo visto, quello che si viene a creare è sostanzialmente un dio dosciocchi, quindi un ulteriore aggiunzione, un ulteriore regione di svuotamento. Questo non è desiderato. Allora quello che si fa, sostanzialmente prima di applicare un contatto metallico è realizzare una regione altamente dopata. Quindi, ad esempio, qui vediamo una regione di tipo N più prima del contatto metallico. Adesso, una volta concretata questa generalità sul funzionamento del rivolto da Siriccio, vi chiede innanzitutto se ci sono domande o qui o anche da fuori. Sì, dimmi. Madonna, penso sia un stupide, ma come mai il rivelatore si dice se ci serve per qualcun motivo di rivestire una superficie ampia non diciamo fare banalmente più grandi? In essenzione, Nici. Allora, si possono fare più grandi. Considera, ti puoi fare un'idea anche dai processi di produzione. Un po' hai visto i meccanismi con cui vengono prodotti. Fa sì che realizzare tutto in una superficie piccola d'esempio, pensa l'impiantazione, l'ionica.

Tu hai un acceleratore, un discorso è realizzare un rivelatore di superficie piccolina, un altro di superficie stesa. Ma il vero motivo è legato al fatto che questi rivolatori di Siriccio normalmente venono utilizzati come rivolatori segmentati per dare una informazione aggiuntiva alla posizione quindi dove ha inciso la radiazione. E quindi, ora lo vedremo nel corso di questa presentazione, si utilizzano delle tecnologie diverse di rivolatori al Siriccio che possono essere strip, drift o pixel. Ad esempio, pensa i pixel. I pixel possono avere anche dimensioni molto piccole e rivestire superfici dell'ordine del metroquadro con pixel di dimensioni di 20 micron per 20 micron. Non capisci che praticamente diventa una matrice fittissima. E hai a che fare con una equivalente elettronica, perché poi ogni canale ogni pixel ha la sua elettronica associata, che consuma anche energia e potenze e quindi può portare anche a un riscaldamento. In più considera la complicazione di andare a leggere un numero di canali enormi dell'ordine di miliardi di canali, miliardi di segnali. Quindi la difficoltà, oltre a una difficoltà realizzativa che può essere legata al processo di produzione, è anche poi una difficoltà pratica, perché spesso questi rivelateli si disciò, vengono adoperati più che altro come rivelatori di posizione per andare a rivestire delle superficie stesa. Quindi questa è un po' la problematica. Certo, anche questo si può creare una zona morta tra un pixel e il successivo, quindi esiste un free factor anche qui che normalmente viene mantenuto il più piccolo possibile. Ci sono altre domande? Ok, mi sembra di no. Allora andiamo a guardare un po' le caratteristiche invece di questi rivelatori. Abbiamo accennato all'inizio che sono dei rivelatori che da un, per diversi aspetti, ci aiutano, hanno delle prestazioni migliori rispetto ad altri rivelatori che abbiamo studiato. Allora, partiamo da una cosa fondamentale, la linearità. In ogni rivelatore che misura l'energia, ci aspettiamo un ottimo grado di linearità. Quindi se arriva una particella con un'energia E, mi aspetto che il segnale prodotto sia proporzionale a E, in maniera tale da avere una risposta lineare. In quel caso di un rivelatore semiconductor, chiaramente il punto di partenza è che la particena, innanzitutto, si debba fermare all'interno del rivelatore, quindi abbiano spessore sufficiente ad arrestare la particella e a misurare tutta l'energia. Allora, a quel punto mi aspetto che il segnale intenzione che viene indotto sia proporzionale alla carica prodotta, quindi alle coppie elettrone e lacuna che sono state create a seguito della perdida di energia della particella all'interno del rivelatore, diviso la capacità. Ovviamente, perché alla fine il dio non è altro che una capacità sostanzialmente. Questo è uguale a cosa posso indarli esprimere in un seguito modo, q, che è il numero di coppie elettrone e lacuna, lo posso indarli esprimere come l'energia E, che è stata persa all'interno del rivelatore, diviso q2, che invece è l'energia media per creare una coppia. In realtà qui viene introdotto un ulteriore fattore Epsilon, perché è vero che magari si produce un certo numero di coppie. La generale è l'energia depositata, ma magari non tutte le coppie vengono effettivamente raccolte, quindi abbiamo un certo grado di efficienza che è espressa da questo fattore Epsilon, potrebbe essere ad esempio il 90\%, quindi 90 su 100 coppie vengono raccolte, le altre purtroppo le perdiamo, però l'importante è conoscere questo fattore e l'importante è che si mantenga comunque sia la linearità tra il segnale prodotto e l'energia depositata nel rivelatore. Ora, la risposta dei semi-conduttori in generale è abbastanza indipendente dal tipo di particella, quindi se entro nell'elettrone o se entro nella particella alfa, normalmente se si deposita la stessa energia viene prodotto lo stesso segnale. Tutto questo è vero, almeno che non si va a considerare l'arrivo di Ioni, quindi nel caso di Ioni purtroppo non si generano degli effetti di plasma, quindi delle nuvole di elettroni particolarmente dense che vanno a modificare e distorcere il campo elettrico all'interno del rivelatore, quindi la linearità non è del tutto assicurata. Capite che se il rivelatore non è sufficientemente spesso, viene invece misurata semplicemente una perdida di energia, quindi a volte i rivelatori a semi-conduttore venivano utilizzati come rivelatori in trasparenza, nel senso con lo scopo di essere attraversati per perdere alle particelle una quantità di energia molto piccola. In questo caso capite che la risposta non è lineare con l'energia della particella. La risoluzione in energia, l'abbiamo detto all'inizio, la risoluzione di questi rivelatori è tipicamente abbastanza buona. Infatti, se vi ricordate, servono pochi elettron volt per generare una coppia elettrone lacuna, che dobbiamo confrontare con l'energia media per creare ad esempio una coppia ionelettrone in un rivelatore a gas. Se vi ricordate, in quel caso, erano necessarie energie dell'ordine dei venti, 40 elettron volt per creare una coppia, mentre qui siamo un fattore 10 più basso, quindi abbiamo bisogno di meno energia per creare una coppia. Qual è la conseguenza? Che a pariutà di energia depositata, nel caso di un rivelatore al silicio, si produce un numero di coppia che è 10 volte superiore rispetto a quanto si produrrebbe in un rivelatore a gas. E il fatto di avere un numero di coppia elevato fa sì che la risoluzione sia migliore. Ora faremo dei passaggi matematici molto semplici proprio per vedere questo effetto. Comunque alla base c'è proprio il fatto che si generano un numero di coppia più elevato rispetto ad altri rivelatori.

Ora, quanta energia è necessaria? Vi ho detto qualche elettro molto, ovviamente dipende dalla gap di energia prohibita. Vedete qui, sono riportate quelle per il silicio e per il germaneo a due temperature diverse. Vedete che per il silicio dell'ordine di 3-4 letto, molto per il germaneo dell'ordine di 3 letto molto più basso. Queste sono in realtà le energie per produrre una coppia. Le energie del gap sono in realtà più piccole. Se vogliamo vedere il letteratura quanto vale il gel del silice del germaneo, sono un gel molto leggermente più piccoli. Vi ritrovate invece un'energia per creare una coppia un po' più grande, legata a fatto che una parte dell'energia viene spesa, viene persa per eccitare il reticolo per creare delle vibrazioni reticolari. Quindi ci ritroviamo ad avere bisogno di almeno 3-4 letto molto per creare una coppia. Allora possiamo fare qualche passaggio, sono di funzione al tablet. La miglior condivisione. Allora quando vogliamo misurare l'energia, in realtà l'energia che noi misuriamo la coppia è stata, è viene depositata nel rivelatore diviso W, energia media per creare una coppia. Abbiamo detto questo vale sia per i rivelatori a gas che per i rivelatori al silice, con l'unica differenza che nei rivelatori a gas, W è dell'ordine all'incirca di 30 eV, mentre nel rivelatore al silice o comunque al semiconduttore W è dell'ordine di circa 3 eV, quindi 1.10. Chiaramente quando io dico che il numero di coppia è N, ovviamente parlo di un valore medio, perché sì è vero che si producono queste coppia ma poi le coppie che danno l'uomo al segnale sono una parte perché appunto ci sono dentro tanti fenomeni di fluttuazioni statistiche e quindi il numero N medio che io misuro è che indice della mia energia, il numero N che io misuro in realtà può variare da evento a evento, quindi ad esempio se io invio il rivelatore al silice delle particelle alfa come quello che vi troverete a fare a breve elaboratorio di 5 MeV, questi 5 MeV vengono persi nel rivelatore e deranno luogo a un certo numero di coppie che produrranno il segnale, ma questo numero di coppie fluttua ovviamente a seguito di fenomeni in natura sadistica e quindi quando io dico che sono andando a misurare lo spettro in energia e le particelle alfa e trovo un picco molto stretto come quello in figura, questo rappresenta il numero di particelle alfa che hanno una certa energia. Ora quello che io riporto sull'asse orizzontale, l'energia, in realtà equivale a N, equivale a dir a quante cariche, quante coppie ho generato e quindi il fatto di non avere una delta di dirac come io mi dovrei aspettare perché sto mandando 5 MeV, mi dovrei, dovrei trovare sempre 5 MeV, in realtà trovo a volte dei valori un po' più grandi a volte un po' più piccoli per quale motivo perché ci sono delle flutuzioni dovute proprio al numero di coppie che creano il segnale, questo numero fluttua, coccilla leggermente quindi è come dire che quello che gli osservo solo spettro in energia è una conseguenza del numero di coppie N che ogni volta si producono e che vengono raccolte che non sono sempre le stesse. Quindi posso valutar un N medio, quindi dalla formula che abbiamo visto prima ma poi rispetto a questo N meglio ogni volta ho delle fluttuazioni. La risoluzione di energia come viene valutata, se vado a ricordare, andando a guardare il picco e studiandone la larghezza, in particolare mi concentro sulla larghezza a metà altezza, quindi se questo ad esempio è il massimo del picco vado a considerare la metà altezza, quindi il massimo diviso 2 e vado a vedere la larghezza del picco a metà altezza, quella che è indicata normalmente con la sigla FWHM Full Width At Alph Maximum. La risoluzione in energia la possiamo indicare con R e data dalla larghezza a metà altezza diviso l'energia, possiamo moltibricarla per 100, per averla in percentuale però questa è la formula. Se il picco ha una forma gaussiana si può dimostrare che questa larghezza a metà altezza è all'incirca 2,35 per la deviazione standard della gaussiana diviso l'energia. Quindi se abbiamo una gaussiana siete agevolati perché ad esempio possiamo realizzare un best fit della vostra distribuzione con una funzione gaussiana e strare il parametro sigma che rappresenta la deviazione standard e utilizzarlo per valutarla la larghezza a metà altezza semplicemente moltirricando per 2,35. Altrimenti l'alternative è guardare la distribuzione dei valori che abbiamo misurato, guardare la larghezza a metà altezza materialmente come abbiamo fatto qui adesso. Allora come mettiamo in relazione la risoluzione al numero di coppia? Abbiamo detto abbiamo questa relazione che abbiamo scritto qui in alto n medio uguale ae su w quindi quando vado a riscrivere la risoluzione al denominatore posso andare a sostituire ae il numero n per w. Al numeratore mi riscrivo 2,35 per come sigma? Sigma è la deviazione standard di questa distribuzione che posso immaginare appunto essere anch'essa legata alla sigma della distribuzione del numero di coppia quindi posso andare a scrivere che sigma con n non è altro che sigma con e diviso w. Andando quindi a sostituire questo diventa w per sigma con n, w e w posso semplificare e quindi la mia risoluzione banalmente la posso andare a scrivere la scrivo qua sotto r come 2,35 sigma con n diviso n. Quindi ho riscritto banalmente la risoluzione in termine di n numero di coppia quindi la risoluzione sarà dato dalla larghezza di questa distribuzione del numero di coppia su numero medio di coppia generato. Se la distribuzione, dato che stiamo parlando di numero di coppia, la distribuzione che regola la produzione di queste coppia, la distribuzione di tipo possaniana, posso approssimare sigma alla radice di n e quindi capite che questo diventa 2,35 radice di n su n. Quindi se canciglie un attimo, andiamo a fare il calcolo per i due casi. Quindi r l'abbiamo scritto come 2,35, 1 su radice di n. E allora ad esempio se nel selicio si producono, abbiamo detto in media, 10 volte in più di coppia, capite che la risoluzione non rivoluta alla selicio sarà dato da 2,35 diviso 1 sulla radice di 10n. E quindi sostanzialmente vedete c'è un fattore all'incirca un terzo, quindi la risoluzione di un rivelatore alla selicio è all'incirca un terzo rispetto alla risoluzione di un rivelatore a grassa. Suffizialmente per il fatto che si producono un fattore di 10 in più di coppia. Quindi dire che è un terzo vuol dire che è più piccolo e quindi vuol dire che ovviamente è una risoluzione migliore. Nei rivolatori al selicio si possono raggiungere anche le risoluzioni dell'1\% e 

anche più bassa volta, dipende dal modo di operare. Ci sono domande su questi passaggi. Sensibilità ed efficienza in trince. In trince che ha di un rivelatore a semiconduttore è dell'ordina del 100\%. Vi ricordate cos'è l'efficienza in trince che rappresenta il rapporto tra il numero di particelle che hanno dato l'uogo a un segnale diviso il numero di particelle incidenti. Quindi arriva una particella, incide sul rivelatore, perde energia. Questo da luogo a un segnale rivolabile non è detto, in questo caso sì, praticamente prossima al 100\%. In altri rivolatori l'efficienza in trince dovrebbe essere più bassa anche a secondo del tipo di particella. Quali sono i limiti? I limiti potrebbero essere la presenza di una soglia minima necessaria sul segnale da rivelare. Legato al fatto che comunque sì e ci sono delle correnti di dispersione e l'elettronica che introducono un minimo di rumore. Quindi possono creare dei segnali che in realtà sono segnali di rumore. Sono segnali ovviamente di bassa ampiezza. Questo fa sì che se non vogliono essere confusi con segnali fisici devono essere in qualche modo discriminati, cioè eliminati dell'acquisizione. Questo può essere fatto imponendo una soglia minima al di sotto della quale non rivelare nulla, non acquisire nulla. Quindi questo però da un lato ovviamente vi aiuta a eliminare il rumore. Dall'altro però vi potrebbe far perdere qualche segnale fisico di interesse, quindi abbassare leggermente l'efficienza in trincega. La presenza di una zona morta potrebbe anche qui imporre una certa perdita nell'efficienza perché magari i particelli di bassa energia riescono ad attraversare la zona morte, quindi non vengono misurate. Quindi hanno inciso sul rivelatore ma non sono state misurate. La densità, vi dicevo questo è un vantaggio perché permette di avere un elevato stopping power. Ad esempio guardate qui in questo grafico, come non è facilmente leggibile a causa del della qualità. Viene riportata l'energia della particella da 1 a 100 membo e sulla sé verticale viene riportato il range in micron e i grafici che vedete qui le linee corrispondono a diverse tipologie di particelle, protone ed autoni, trizzi, ediotree e alfa. Concentriamoci ad esempio sulla situazione di terci e di troveremmo il laboratorio, quindi i particelle e l'alfa da 5 Mav che incidono sul sidicio. Questo grafico ci permette di capire che spessore deve avere il rivelatore per assicurarci che tutta l'energia venga persa all'interno del rivelatore. Ad esempio a alfa da 5 Mav ci ritroviamo qui grosso modo, vedete che siamo a range dell'ordine della ventina di micron e quindi è sufficiente un rivelatore di questo spessore per poter far sì che le particelle alfa perdano tutta la loro energia. Questo vi dicevo è un vantaggio perché sono necessari spessori veramente ridotti, quindi rivelatori compatti per poter andare a misurare energia di particelle anche molto energetiche. Infine, un'ultima qualità di questi rivelatori che abbiamo una diffusione estremamente più bassa, più piccola rispetto a quelle rivelatori a gas, si fa sì che questi rivelatori possono essere utilizzati più facilmente come rivelatori di posizione, quindi per andare a misurare la posizione della particelle incidente.

Il tempo di risposta, vi ho detto, sono rivelatori molto veloci in generale possono essere utilizzati come rivelatori di timing certamente molto più performanti rispetto ai classici rivelatori a gas. Tipicamente i tempi di salidità di un segnale sono dell'ordine dei nano secondi quindi estremamente veloci. Altre rivelatori veloci che avevamo visto erano ad esempio gli scintillatori, se vi ricordate abbiamo fatto qualche misura lo sceloscopio. Un problema però di questi rivelatori è il cosiddetto danneggiamento a rada raviazione, cioè cosa succede se questi rivelatori vengono esposti a notevoli dosi di raviazione. Purtroppo questo avviene soprattutto in alcuni ambienti quando possono essere appunto gli esperimenti sotto fascio, sotto accederatore o esperimenti nello spazio e questo modifica le qualità e le proprietà del rivelatore. In generale i danni che si possono creare sono di due diversi tipi, possono essere o danni del bike, cioè proprio del corpo, della massa che costituisce leaponduttore oppure dei danni sulla superficie, quindi sull'ossido che normalmente rivesse questi rivelatori. Gli effetti sul bike appunto sono effetti che si manifestano come danneggiamento al arreticolo e in generale sono dovuti alle cosiddette radiazioni nile cioè radiazioni non ionizzanti. Cosa comporta poi ne satti concretamente che magari può cambiare l'attenzione di svuotamento. Quindi se ad esempio voi stavate lavorando con il vostro rivelatorio a 10 volt e attenevate determinate prestazioni, vi rendete conto che a seguito di questo deneggamento non sono più sufficienti 10 volt, ma dovete doverare 12 volt, 15 volt. Quindi cambia la vostra attenzione di lavoro a cui ottenevate un buon svuotamento della regione. Potrebbe aumentare la corrente. Quindi una corrente che comporta un rumore e quindi questo potrebbe richiedermi l'utilizzo di un raffreddamento, oppure diminuire l'efficienza nella raccolta della carica. Qui ad esempio possiamo vedere come cambia l'attenzione di lavoro man mano che aumenta la dose. Questi sono dei dati sperimentali realizzati su un determinato tipo di rivelatore. Vedete come l'attenzione di lavoro man mano cambia. L'importante è ovviamente studiare per bene i possibili deneggiamenti di radiazione. Questo è una parte importante di quello che costituisce hierarchica la progettazione di un esperimento. Ad esempio tutti gli esperimenti acceleratori come può essere l'HCO o altri acceleratori richiedono anni e anni di progettazione anche per studiare questi effetti. Perché sono arrivelatori che ovviamente richiedono lo sforzo di anni di lavoro e devono operare per almeno 5, 10 anni. quindi bisogna assicurarsi che il rivalatore non subnisca un deterioramento eccessivo e che quindi possa lavorare per l'intervallo di tempo in cui dovrebbe lavorare. Quindi spesso la tecnologia non esiste nel momento in cui si immagina di dover costruire il rivalatore, quindi si lavora per costruire e arrivare a questi obiettivi. Gli effetti sulla superficie invece sono tipicamente dovuti a radiazione ionizzante e questo comporta sostanzialmente la presenza di cariche positive che si accumulano sugli ossidi e poi può anche influire sull'umore e sulla tensione di rottura. Gli effetti che vengono prodotti dal deneggiamento e da radiazione possono essere o effetti cumulativi che si vanno sommando nel tempo e sono dei danni che non si possono riparare, sono dute proprio all'esposizione prolungata alla radiazione che è un po' quello che vi ho descritto prima, ha immaginato il rivalatore deve essere adoperato in un esperimento sotto fascio, chiaramente è un accumulo costante nel tempo che comporta ovviamente dei danni irreparabili oppure possono verificarsi e questi si studiano pure degli eventi singoli che possono essere proprio o eventi transitori, immaginate ad esempio banalmente il cambio di un bit da 0 a 1 questo potrebbe essere un effetto transitorio o eventi catastrofici permanenti ma questi sono singoli eventi quindi che potrebbero accadere oggi come tra 10 anni quindi non sono legati proprio a un acuno come abbiamo visto nel caso precedente ragazzi mi fermerei qui oggi e riprendiamo questa parte la prossima volta quindi sono ormai soltanto applicazioni del rivalatore a ciricio e parleremo poi del introdurro una delle esperienze che farete quella appunto un rivalatore a ciricio e particelle alfa e parleremo delle tecniche di vuoto perché appunto quando adopererete questo rivelatore dovrete lavorare con una cameretta da vuoto quindi introdurremo qualche concetto di questo tipo allora noi ci vediamo giovedì pomeriggio mi sembra in aula e mi sbaglio ci hanno spostati allora ragazzi qua interrompo la registrazione e ci vediamo giovedì

\textbf{lez 15}

Allora ragazzi, riprendiamo quello che avevamo iniziato la volta scorsa. Se vi ricordate avevamo in prodotto i rivelatori a semiconduttore, era l'ultima tipologia di rivelatori che affrontiamo in questo corso. Vi ricordo molto molto velocemente la struttura di base di un rivelatore a semiconduttore e vi ricordo che si basa su un'aggiunzione PN, quindi quindi realizzata con un materiale semiconduttore drogato di tipo N e accanto un materiale drogato di tipo P. In questo questo è possibile realizzare nella zona dell'aggiunzione una regione di svuotamento che è una regione priva di cariche libere. Invece sono presenti delle cariche fisse che sono i milioni presenti nel nerreticolo che generano una potenziale, detto potenziale di contatto. Quindi vedete qui ad esempio la forma del campo elettrico proprio in corrispondenza della giunzione e qui come vengono deformate le bande energetiche dei materiali. Questa è una regione che si è adatta bene allo scopo della rivelazione perché nel momento in cui dovesse passare una particella e depositare dell'energia questa energia verrà utilizzata per produrre nuove coppie elettrone e lacuna, chiaramente il numero di coppie proporzionale all'energia che viene depositata nel materiale e queste coppie possono essere raccolte proprio grazie a questa differenza di potenziale. Chiaramente poi è necessario andare a mettere degli elettrodi per poter raccogliere la tarica, tuttavia essendo un potenziale molto piccolo la regione di svuotamento comunque molto ridotta e quindi per aumentarla quello che si fa è polarizzare inversamente l'aggiunzione e quindi andare ad applicare un potenziale positivo dalla parte n è un potenziale negativo del lado p. In questo modo la regione di svuotamento aumenta e diventa ovviamente una regione ancora più adatta alla rivelazione. Tante volte questi rivelatori semi-conduttore vengono utilizzati per misurare radiazione alfa che viene arrestata in poche decine di micron e quindi in realtà queste giunzioni non devono avere delle regioni di svuotamento particolarmente stesse, però per la rivelazione di altre radiazioni come ad esempio gli x, le regioni anche dell'ordine del millimetro e quindi abbiamo visto anche delle tecniche costruttive di realizzazione di questi rivelatori per arrivare a estensioni dell'ordine dei millimetri. Avevamo concluso un po' tutta questa parte, eravamo arrivati anche a discutere le caratteristiche di questi rivelatori al silicio e il danneggiamento della radiazione e dovevamo concludere questo argomento parlando delle applicazioni nella spettroscopia di particelle cariche. Questi rivelatori hanno delle proprietà molto positive, abbiamo detto hanno una risposta temporale abbastanza pronta, abbastanza veloce quindi sono adatti ad esempio per applicazioni di timing, dove volete andare a misurare delle differenze di tempo tra segnali elettrici, quindi forniti ovviamente da rivelatori. La risoluzione è ottimale, stiamo parlando di risoluzioni dell'ordine dell'1\%, quindi molto più bassa rispetto a quelle che abbiamo visto con altre tipologie di rivelatori e quindi proprio per queste proprietà questi rivelatori venivano utilizzati nel campo della spettroscopia di particelle cariche. Inoltre sono disponibili come abbiamo detto in un'ampia varietà sia di spessori che anche di area sensibile, hanno un'efficienza prossima al 100\% nel caso di protoni, particelle alfa ma anche ioni pesanti e di ricevo che lo spessore da doperare chiaramente si deve valutare in base all'applicazione, quindi in base al tipo di particella che si vuole andare a parlare in base anche alla sua energia. E proprio per questo motivo nel campo della fisica moderna, più che fisica moderna della fisica attuale, soprattutto fisica delle alte energie, questi rivelatori vengono utilizzati ampiamente, hanno appunto le prestazioni proprio adatte per andare ad affrontare anche degli ambienti spavorevoli, come ambienti in cui si ha una elevata dose di radiazione, una elevata densità di particelle e quindi negli ultimi decenni ci sono stati enormi sviluppi soprattutto in questo campo con ovviamente ritadute poi in altre tipologie di applicazioni come ad esempio il campo della medicina, quindi tante volte alcune tecnologie si sviluppano nel campo della ricerca e poi vengono trasferite in automatico nel campo, in altri tipi di campo, campo arespaziale, campo della medicina. Un altro vantaggio di questi rivelatori è che spesso è utile avere dei rivelatori molto subtili, vi ho detto che questi rivelatori possono essere adoperati anche come rivelatori in trasparenza, vengono attraversati della particella, viene effettivamente depositata dell'energia, ma questa energia è molto piccola, quindi è come se fosse una piccola perturbazione effettivamente del percorso e della particella nelle sue proprietà. E questo è un vantaggio perché da un lato abbiamo comunque sia un segnale che mi indica il fatto che è passato una particella e quindi eventualmente per problematiche di tracciamento di tracking, quindi sapere attraverso quali punti la particella è passata è utile avere questi rivelatori perché sono in grado proprio di misurare la posizione della particella senza perturbarne eccessivamente le caratteristiche. Quindi ormai tutti, quasi tutti gli esperimenti di alta energia e nello spazio utilizzano ampiamente questi rivelatori al silicio. Questo è un grafico che ormai è un po' datato perché insomma risale ormai quasi a vent'anni fa, però era giusto per darvi un'idea vent'anni fa di quali erano gli esperimenti che tipicamente utilizzavano rivelatori in questo caso al silicio di tecnologia strip, ora vi dirò un po' qualcosa qualcosa questa tecnologia. Comunque al di là di questo dettaglio vedete colorati le diverse applicazioni, questi sono acceleratori, quindi quindi verde e rosso si riferiscono a esperimenti prestoacceleratori e l'HC è l'acceleratore al momento più potente che riesce ad accelerare alle massime energie fino a la realizzata mentre in blu vengono riportati gli esperimenti nello spazio. Quindi vedete che appunto... ma che è questo rumore ragazzi? Sto riuscendo a favori. Ansì rispetto a qualche giorno fa, a qualche qualche di giorni fa. A inizio febbraio abbiamo fatto un esame del dottorato, un esame fino a il dottorato qui in quest'aula, è stato un inferno veramente, non si capiva niente. Comunque, vi dicevo, vedete appunto la varietà di applicazioni qui sotto sono riportati proprio nel dettaglio le sigle degli esperimenti che fanno uso di questa tecnologia al siri. Cioè, ad esempio, quello che mostra la maggiora utilizza, ad esempio l'esperimento CMS, non so se l'abbiamo mai seguito nominare, un esperimento presso l'acceleratore e l'HC. Diciamo uno dei esperimenti che ha contribuita alla scoperta del bosone X. E vedete qui il grafico riporta l'area in metri quadri. Quindi Quindi questo caso, vedete, CMS utilizza 214 metri quadri di rivelatori, in questo caso al siriscio in strippo. Ma altri rivelatori non sono da meno, comunque una tecnologia particolarmente utilizzata. Guardiamo ora un po' più più dettaglio alcune forme, alcuni rivelatori più moderni che utilizzano sempre la tecnologia basata sul siriscio. In generale, si possono avere due possibili scelte quando si vuole utilizzare il siriscio per ricostruire la posizione della particella. Quindi la particella incide sul rivelatore, voglio sapere esattamente in che punto è passata la particella. Allora, in questo caso si possono adoperare due scelte distinte. O si utilizza un readout continuo o un readout discreto. Capiremo ora la differenza. E in particolare faremo riferimento a queste tipologie di rivelatori al siriscio. I rivelatori a strip, che sono quelli di cui vi ho parlato nella slide precedente, quelli a pixel o pad e quelli a drift. Quindi vedete già un'immediata differenza, soltanto nella rappresentazione. Strip sta a indicare in inglese una striscia. E infatti vedete che qui questo rivelatore, ora andremo a vedere un po' più nel dettaglio, presenta queste strisce parallele, l'un all'altro, che prendono nel nome appunto di strip. Pixel o pad, ovviamente conoscete i pixel perché siete abituati ovviamente con le fotocamere, ecco qualcosa di similare anche in questo caso. Vedete una matrice di elementi che prendono nel nome appunto di pixel. Infine quelli drift vedete qui nuovamente una striscia a strip, però evidentemente c'è qualche differenza a rispetto al caso delle strip. E a seconda del tipo di rivelatore avremo un readout continuo o un readout discreto. 

Partiamo ad esempio a rivelatori a strip o micro strip quando queste strip hanno dimensioni di pochi micron. Allora vediamo un po' la struttura di questo rivelatore. Si basa tutto sempre su giunzioni piene quindi non è qualcosa di totalmente diverso da quello che abbiamo fatto fino ad esso. Però in più si sfrutta il fatto di poter realizzare gli elettrodi con determinate geometrie. Quindi qui la caratteristica è che gli elettrodi di lettura sono costituiti da queste strip che vengono posizionati parallelamente, l'un all'altro ha una distanza opportuna tipicamente della decina di micron. Il segnale che viene prodotto all'interno del rivelatore induce il segnale sull'elettrodi lettura e chiaramente andrà in dur, il segnale sull'elettrodi più vicino. Quindi in base a dove passa la particella, ad esempio guardate questa linea che rappresenta una particella che è attraversata il rivelatore, qui questa particella avrà prodotto un certo numero di coppie elettrone e lacuna che migliano verso gli elettrodi inducendo un segnale ed è chiaro che la strip che viene interessata ci fornirà la coordinata spaziale, cioè il punto attraverso cui è passata la particella. Chiaramente è un'informazione monodimensionale perché l'unica informazione che abbiamo è lungo questa direzione. Quindi se viene colpita ad esempio il rivelatore al centro immagino che viene interessata la strip centrale se invece la particella colpisce il rivelatore più a sinistra sarà la strip di sinistra dal segnale e così via. Chiaramente qui vedete soltanto una porzione del rivelatore ma normalmente sono presenti centiraia di canali in base all'essenzione del rivelatore. Quindi questo rivelatore è diciamo di posizione in questo caso monodimensionale perché ci dà l'innicazione su una sola coordinata spaziale sul piano ovviamente. Poi il meccanismo, il principio di funzionamento è esattamente sempre lo stesso quello che abbiamo visto prima quindi sempre giunzioni pn polizzate. Se volessimo avere un'informazione bidimensionale sul piano cosa potremmo fare? Una tecnica che viene aggoperata è semplicemente andare a sovrapporre due strati di rivelatore a strip come vedete qui con una certa inclinazione ovviamente. Questo vi fornirà l'informazione bidimensionale grazie al fatto di conoscere le strip interessate su ciascun piano però può portare eventualmente a dei casi di ambiguità perché se passa una sola particella attraverso entrambi gli strati allora non c'è alcuna ambiguità perché verrà interessata una strip per piano ma se per caso simultaneamente come può avvenire ad esempio in un esperimento sotto fascio sono più particelle ad attraversare i due strati allora inevitabilmente si crea un'ambiguità. Guardate qui l'esempio riportato in basso a destra. I punti verdi sono i punti reali attraverso cui è passata la particella, sono passate due particelle simultaneamente allora in basso a questa posizione ora magari il grafico è piccolino però possiamo immaginare qual è la strip di ciascun strato interessato anzi ciascun strato avrà due strip interessate una per particella quindi nel momento in cui voglio risalire alla posizione devo andare a considerare tutte le possibili combinazioni di queste coppie di strip e quindi cosa succederà che ne ricostruirò due correttamente ma ce le saranno altre due che sono quelle segnate in rosso che sono diciamo legittime nel senso derivano proprio dalla combinazione delle strip colpite ma in realtà nascono insomma sono in realtà spuri non sono collegata al reale passaggio di particelle quindi è vero che è utile sono a porre questi strati di strip però bisogna stare attenti a questi casi di ambiguità quanto vale la risoluzione chiaramente la risoluzione spaziale quindi la capacità di ricostruire la precisione con cui si ricostruisceotomia la posizione dipende dalla distanza tra le strip più queste sono vicino tra di loro migliore sarà la risoluzione spaziale in particolare se vi ricordate già questo discorso l'avevamo fatto nel caso delle camera fili, qui è un discorso analogo, data la distanza tra due elettrodi, la risoluzione data da questa distanza diviso la radice di 12. Quindi se ad esempio la distanza è 20 micron, la risoluzione reale sarà 20 diviso la radice di 12. Questa è una conseguenza del teorema del limite centrale, dava un discorso durante il primo semestre. Quindi vedete una risoluzione molto spinta, si si arrivare anche a 5 micron, tempi di raccolta delle cariche molto veloci, anche decine di nanosecondi. Un unico difetto è questa eventuale ammiguità nella ricostruzione della posizione. E questo è un esempio di rivelatori al redout discreto, perché qui la posizione viene data grazie alla discretizzazione dell'elettrodo, cioè il fatto di avere tanti elettrodi separati. In un rivelatore ad drift invece abbiamo un redout continuo. Che cosa cambia rispetto a prima? In anzitutto il rivelatore ad drift fornisce entrambe le coordinate su un piano, quindi un'informazione bidimensionale. Come fa? La struttura sembra molto simile a quella strip. Vedete qui abbiamo sempre degli elettrodi che sono realizzati come strip, paralleli l'un all'altro. Ma queste in realtà hanno una funzione diversa rispetto a quella che abbiamo visto prima. Ciò che raccoglie il segnale sono in realtà questi anodi che vedete qui sotto forma di pad, cioè di piazzola. Andiamo a vedere cosa succede quando passa una particella in grado di produrre coppie elettroni di lacuna. Queste sono gli elettroni e le lacune che vengono prodotte, gli elettroni incominciano a migrare verso gli anodi grazie a una differenza di potenziale che viene realizzata attraverso queste strip. Quindi questo strip se vedete qui c'è un partitore di tensione. Questi strip vengono messe a potenziale evidentemente via via crescente in maniera tale che gli elettroni che si producono a seguito del rilascio di energia nel rivelatore, questi elettroni vengono guidati verso l'anodo. Quindi come facciamo ad avere la coordinata bidimensionale? Allora una coordinata che è quella lungo questo ass, penso che segue la manina del cursore, viene fornita da quale anodo è stato interessato, quindi quale anodo ha raccolto la carica. Evidentemente l'anodo più vicino raccoglierà la carica. L'altra coordinata questa, quella verso cui, lungo cui si muovono gli elettroni, viene fornita dal tempo di deriva, cioè da quanto tempo impiedono gli elettroni ad arrivare all'anodo. Capite che se la produzione di questi elettroni avviene qui a questa estremità, impiegaranno un certo tempo, se venisse invece in questa estremità gli elettroni arriverebbero subito, quindi il tempo di deriva sarebbe più piccolo. Quindi il tempo con cui questi elettroni derivano verso l'anodo ci fornisce la coordinata lungo questa direzione. Ecco che quindi abbiamo una informazione di natura bidimensionale. Sto vedendo che la risoluzione è completamente diversa. Io sono molto più allungata, voi vede tutto più compatto. Tuttavia qual è l'osvantaggio? Questi sono rivelatori più lenti perché bisogna aspettare una deriva che può essere anche abbastanza lunga degli elettroni e inoltre necessitano di temperature stabili. Questo è un esempio di rivelatore al redauto continuo perché a questo punto è vero che abbiamo una discretizzazione lungo questa direzione dovuta agli anodi, ma lungo è l'altra direzione. La lettura è continua, cioè questo tempo di deriva assume un valore che non è discreto, ma è continuo. È chiaro? Se ci sono domande interrompete. Vede qualcosa che è molto simile a un dispositivo che voi adoperate spesso. Il CCD, il Charged Couple Device, voi lo utilizzate spesso perché è l'elemento di base delle fotocamere. Anche questi CCD sono rivelatori al redauto continuo, si dicono anche rivelatori a memoria perché gli elettroni che vengono prodotti a seguito della ionizzazione non sono rimossi immediatamente, quindi non vengono raccolti immediatamente ad un elettrodo, ma vengono fatti passare, innanzitutto vengono confinati all'interno di un pixel e poi a poco a poco vengono fatti derivare fino alla raccolta finale, un po' come avviene qui, che gli elettroni vengono prodotti e poi man mano vengono fatti derivare. Ora vedremo che qualcosa di simile vale anche per i CCD. Vediamo proprio un'animazione che fa sia la cosa più semplice se fa per messerlo a questa. Niente, è peccato perché la vengono se clicco. Ah, perché funziona. Allora, vedete il CCD è costituito da la matrice di pixel, che sono questi quadratini che vedete. Questi pixel costituiscono righe e colonne. Quando viene attraversato, questo sensore viene attraversato in questo caso da luce, viene generata della carica e questa carica non viene raccolta immediatamente da un elettrodo, ma in realtà a seconda del punto in cui si è creata viene guidato lungo la colonna corrispondente. Quindi, su per esempio di trovarvi qui, la carica è stata prodotta qui grazie a un campo elettrico opportuno, quindi adere differenze di potenziale, è possibile guidare la carica prodotta da pixel a pixel fino ad arrivare a quest'ultima riga dove avviene la lettura. Quindi proprio grazie a questo sistema è possibile andare a risalire a quale pixel è interessato. Qui sotto vedete nel dettaglio proprio la fase di deriva. Questi sono i pixel di una colonna, vedete che inizialmente la carica è confinata qui dove si trova un potenziale positivo, dopo di che il potenziale positivo viene creato nel secondo pixel e azzerato nel primo, quindi in questo modo riuscite a far derivare le cariche fino all'ultima riga. I cct sono quindi rivelatori bidimensionale, vi viene fornita effettivamente la posizione in cui è venuta la rivelazione, hanno acuratezze spaziali anche molto elevate dell'ordine di 10 nigron, però hanno come difetto principale in un tempo di lettura molto lungo, quindi almeno hanno dell'ordine dei 10 mili secondi. E questo fa sì che questi rivelatori normalmente non possono essere adoperato nel campo della fisica delle alte energie perché sono troppo lenti. Succederebbe che ad esempio se pensiamo in un esperimento in un acceleratore ovviamente avvengono un certo numero di collisioni al secondo, se il rivelatore ancora è impegnato a misurare ciò che è avvenuto nell'evento precedente, succede il cosiddetto pile up, una sovrapposizione di eventi che chiaramente deve evitare. Quindi Quindi rivelatori non vengono adoperati in questo campo, ma vengono adoperati parecchio nel campo dell'astronomia. Hanno una bassa risoluzione energetica e spaziale, quindi vengono utilizzati nel caso dell'astronomia per spettroscopie di ragilis. Questo è un altro esempio giusto per farvi capire come vengono identificati i pixel interessati. Vedete, questa è una matrice piccolina di 4 x 3 pixel, anzi 3 x 3 per questo poi è la riga di rettura. Vedete incidono quattro fotoni che sono sostanzialmente quelli che colpiscono i pixel sottostanti, se vedete qui vedete i pixel interessati. Si produce in ciascuna di queste regioni la carica e questa carica ovviamente lungo ogni colonna viene fatta migrare verso la riga di rettura e alla fine quello che succede è che si ricostruiscono i pixel che sono stati effettivamente interessati nella radiazione per andare a ricostruire l'immagine. Chiaramente questo è un esempio esempio di 3 x 3 ma dovete immaginare matrici con milioni di pixel. Funziano in maniera analoga i cosiddetti rivelatori a pixel, quindi quelli sono i ccd. Ora vediamo invece nel caso della fisica nucleare cosa possiamo usare di analogo che sia però molto più veloce rispetto a un ccd. In questo caso noi adoperiamo il rivelatorio a pixel anche si costituiti da matrici di pixel che possono essere anche dell'ordine delle centinaia di milioni compressivamente a seconda della superficie da rivestire e quindi sono ovviamente dei canali indipendenti. Qui ritorniamo a un caso di redout discreto quindi ognuno è come se fosse un rivelatore indipendente dall'altro. Le dimensioni di queste pixel ovviamente capite che più sono piccole migliore sarà la risoluzione spaziale del mio rivelatore. Ormai si è arrivato anche a pixel dell'ordine di 15-20 micron di lato, quindi veramente piccoli. Sono rivelatori bidimensionali perché viene fornita sia la coordinata x che la coordinata y. Vi dicevo ormai abbiamo dimensioni particolarmente spinte dell'ordine delle poche decine di micron. Sono rivelatori che hanno un livello di rumore molto basso, il il dipende dalla capacità di questi rivelatori e quindi un buon rapporto segnale sul rumore. Cioè vuol dire che quando vengono attraversati da una radiazione si produce una carica quindi un segnale che si sopraeleva rispetto al fondo, quindi il rapporto segnale sul rumore è ottimale. Sono rivelatori veloci e sono particolarmente adatti nel caso di alte densità di particelle. Quindi immaginate di avere una situazione in cui abbiamo tante particelle l'una vicino all'altra, ma le volete distinguere e rivelare indipendentemente. Quindi la probabilità che due particelle cadano in un 

pixel di dimensioni così piccole è veramente remota, veramente bassa. Ecco perché questi rivelatori normalmente vengono posizionati nella regione a più alta densità. Quindi se immaginate un esperimento, un collider dove avviene un'interazione tra fasci, una contro l'altra, capite che è la zona più vicina alla collisione, al punto di collisione sarà quella con la più più densità di particelle. Più vi allontanate, più avere rivelatori così discreti non è necessario. Quindi tipicamente rivelatori al silizio non sono posizionati proprio all'inizio in prossimità della collisione. quali sono gli svantaggi ovviamente. Capite che abbiamo un numero di canali enormi da leggere? Ora vedremo un esempio concreto, giusto per rendersci conto di che numero li stiamo parlando. In lì è di principio ogni rivelatore ha bisogno di un connettore e di un cavo. Se parliamo di un rivelatore macroscopico, immaginate ad esempio il rivelatore che abbiamo utilizzato, lo scintillatore, col fotomultiplicatore. Dal fotomultiplicatore usciva un cavetto che trasportava il segnale. Ora immaginate di doverlo fare per miliardi di pixel di queste dimensioni. L'ansi è tutto già realizzare un filo di quelle dimensioni capite che diventano un problema, ma ora lo discuteremo, chiaramente una sfida tecnologica. Qui se vedete un ingrandimento di quello che è un rivelatore a pixel, questo è una matrice e qua sotto non so se riuscire a riferli. Sono dei piccoli fili microscopici che sono dei bonding, c'è proprio delle connessioni che vengono realizzate con delle opportune macchine, perché non si possono realizzare a mano, ma si realizzano con delle macchine, si chiamano bondatrici e vanno a realizzare come se fosse un filo però di natura microscopica, quindi dal rivelatore verso l'elettronica. Sono ovviamente molto fragili e richiedono tecnologia all'avanguardia. Perché? Perché innanzitutto il primo problema è vi dicevo quello di trasportare questo segnale fuori da ciascun pixel e allora come si fa? Capite ogni pixel è un rivelatore a 60, quindi ad esempio se il segnale che fu riesce da questo rivelatore ha bisogno di un'opportuna catena elettronica, questa catena elettronica dovrebbe essere replicata per tutti i pixel. E quello che si fa normalmente, che si faceva tempo addietro, ora questa è una tecnologia che si sta superando, era quella di realizzare l'elettronica, ovviamente sempre su strade di silicio, miniaturizzata e delle stesse dimensioni dei pixel. Vete quindi si aveva sostanzialmente una matrice di pixel che erano i rivelatori e sotto una matrice con la corrispondente elettronica. Come si fa la connessione si realizzava attraverso la tecnica che prende il nome di bump bonding, cioè un bonding non realizzato con un filo, ben sì con una pallina, come la vedete qui. Qui sotto abbiamo il sensore, qui sopra abbiamo l'elettronica, questa pallina di materiale conduttore costituisceurasse il contatto. Chiaramente la corrispondenza deve essere esatta di sensore e di pixel di elettronica. Il tutto viene a temperatura controllata perché comunque questa è una sorta di non dico stagno ma comunque un materiale che si deve sciogliere deve realizzare un contatto stabile e tuttavia questa tecnica aveva grossi difetti. Qui vedete degli ingrandimenti con il microscopio, per farvi rendere al conto della forma di questi contatti. Tuttavia aveva dei difetti enormi, un'efficienza estremamente bassa quindi spesse volentieri si buttavano tanti buffer di silicio perché il contatto, il bonding non era venuto bene. Inoltre era un processo molto costoso e capite che il limitato era la dimensione del pixel quindi anche se la tecnologia ci permetteva di ridurre le dimensioni del pixel, l'elettronica non si riuscì a compatterli più di tanto e anche il bonding non poteva essere effettuato su pixel di dimensioni troppo piccole. Inoltre sempre nell'ottica di rendere il materiale il rivelatore il più possibile è trasparente inserire una pallina di materiale, di un qualsiasi materiale, introduce ovviamente una perdida di energia se la particella deve attraversare anche questa piccola pallina è una perdida di energia. Sembra una cosa ridicola perché hanno dimensioni veramente di decine, venti micron, tuttavia per la rivolazione delle particelle può rappresentare un problema anche di multiposcheting. Come siamo andati oltre? Adesso la tecnologia ci permette di realizzare i così detti maps, cioè rivelatori monolitici. Cosa vuol dire che non lo stesso substrato di silicio si realizza sia il sensore che l'elettronica. Capite che non abbiamo più bisogno del bonding quindi guadagniamo in termini di efficienza e diventato ormai un processo abbastanza commerciale. La carica, vedete questo è il sensore e qui abbiamo tutta la parte di elettronica sostanzialmente. La carica che si produce viene fatta migrare verso l'elettrodo e poi c'è la relativa elettronica. Non sono necessarie connessioni elettriche ed è possibile diminuire le dimensioni del pixel. Tutta via qual è stato il problema iniziale di questa tecnologia era il fatto che questa tecnologia era poco resistente alla radiazione quindi perdeva in prestazioni man mano che si accumulava a dose. Ormai questi limiti sono stati superati quindi questa tecnologia viene utilizzata in tantissimi esperimenti. Come esempio vi riporta appunto il nuovo tracciatore dell'esperimento Alicia che è stato installato proprio lo scorso anno. Settestrati interamente di pixel, una superficie di 10 metri quadri per un totale di oltre 10 miliardi di pixel. Quindi come se restavò a disposizione una macchina fotografica di un smartphone un miliardi di pixel. La vostra macchina fotografica è un esempio del cellulare che il numero di pixel ha in grosso modo. 12 mega pixel ormai forse anche di più. Giusto per avere un ordine di grandezza. Ok, ci sono domande? Allora, visto che abbiamo parlato dei rivolatori al silicio, presentiamo una delle esperienze che dovete fare al secondo semestre. Per spiegare queste esperienze ora non parleremo delle esperienze, del rivelatore, delle modalità con cui viene seguita. Alla fine di questa presentazione introdurremo le tecniche da vuoto perché questa è un'esperienza che deve essere svolta utilizzando una cameretta da vuoto. Quindi magari alcuni concetti sul vuoto prendete li così come sono. Ora nella presentazione successiva ne parleremo. Allora, in che cosa consiste questa misura? La misura è la rivelazione di particelle alfa, in particolare spettrometria. Per spettrometria si intende l'analisi, lo studio di uno spettro in energia. In questo caso di particelle alfa, il studio della perdita di energia di queste particelle in opportuni materiali. Il rivelatore che doveremo è un rivelatore al silicio. Qual è lo scopo della misura? Allora, innanzitutto utilizzare un rivelatore al silicio che ancora non abbiamo adoperato per la misura di particelle cariche. Valutare la risoluzione del rivelatore per vedere se effettivamente la risoluzione in energia è molto più piccola rispetto a quella che abbiamo misurato nel caso dello scintillatore più fotomultiplicatore. Utilizzare un apparato per la produzione e la misura del vuoto, questa è un'altra cosa che per vuoto risulterà nuova. Misurare la perdita di energia di particelle cariche in un materiale per poi confrontarlo con quanto atteso dalla relazione di Bethe Block. Questa era una delle esercitazioni che vi avevano 

suggerito nel corso dell'anno proprio anche a questo scopo, perché attraverso la forma di Bethe Block se siete in grado di integrare questa relazione possiamo benissimo confrontare quello che misurate sperimentalmente con quanto previsto della teoria. Infine, non solo osservare ma anche quantificare gli effetti di straggling nella perdita di energia. Quando introdurre un materiale tra l'assorgente e il rivelatore, questo materiale, oltre a far perdere parte dell'energia alla particella, produrrà un effetto di straggling energetico, quindi con un relativo allargamento del picco misurato. Ora vedremo alcuni esempi. In cosa consiste l'appato sperimentale? Tutto molto contatto, perché in realtà è tutto contenuto all'interno di questo modulo. Questo è un modulo che non vedete la figura completa e inserito, vedete su queste guide, all'interno di quello che noi chiamiamo crate, che poi vedete... ah, l'abbiamo visto in crate laboratorio? Ad esempio, quando abbiamo fatto la misura con lo scintillatore, c'era il modulo, quello per la tensione di alimentazione che era posizionato proprio su quella struttura. In realtà quello è semplicemente un contenitore che permette di inserire questi moduli e prendere dal retro delle alimentazioni che servono per far funzionare il modulo. Quindi magari il crate è collegato di per sé alla 220, insomma, la tella corrente che noi utilizziamo. E da questa avevo utilizzando dei trasformatori, vi produce sul retro, su quello che vedete qui, su questo connettore, delle tensioni standard che sono una più 5 volt, una 12 volt, continue, che sono quelle che vengono adoperate tipicamente da questi moduli di elettronica. Quindi una volta che voi inserite il modulo nel crate, magari questo poi quando torneremo il laboratorio lo guarderemo dal vivo così me ne rendete conto più facilmente, quindi quando si inserisce ine sceglie questo modulo avviene il contatto sul retro. Il modulo può prendere le tensioni che li servono. Quindi questo è tutto un modulo di questa azienda Ortec, pensato per studiare le alfa particelle attraverso il rivelato dal sidiccio. Questo sportellino si può aprire e all'interno è presente la cameretta da vuoto. In questa cameretta si può montare un rivelatore sul soffitto della cameretta, che è il nostro rivelatore al silicio, e posizionare sotto una sorgente alfa. Quindi l'apparato sperimentale prevede diverse sorgenti alfa, poi una serie di assorbitori che noi possiamo andare a interporre tra la sorgente e il rivelatore per vedere quanta energia viene persa in questi spessori. Capite che sono spessori piccolissimi, perché se le alfa si fermano in venti micron di silicio, io non posso mettere uno spessore di alluminio di un millimetro, perché è chiaro che non passano. Quindi sono spessori di decine di micron al massimo. Poi abbiamo un rivelatore al silicio che, come vi dicevo, viene posizionato sul soffitto della cameretta. Abbiamo la cameretta da vuoto per realizzare un vuoto che non è un vuoto spinto. Ora ne parleremo di questo, per capire se il vuoto che produciamo è sufficiente. Per produrre il vuoto è necessario un sistema di pompaggio, quindi l'utilizzo di una pompa per strarre l'aria a residua nella cameretta. E poi, sempre contenuta all'interno del modulo, abbiamo un minimo di elettronica per la gestione del segnale prodotto del rivelatore al silicio e un sistema di acquisizione datica, è esattamente identico a quello che abbiamo adoperato, ad esempio, nel caso degli spettri in $\gamma$. Quindi vi ritroverete esattamente lo stesso software. Qui, appunto, oltre alla fotografia, abbiamo riportato anche un elenco delle caratteristiche di questo modulo. In questo modulo vedete, c'è anche questa valvola che può essere posizionata in diverse modalità. Serve sostanzialmente per aprire o chiudere la connessione con il sistema di aspirazione. Quindi immaginate di avere una pompa che attraverso un condotto aspirala l'aria, tra il condotto e l'ingresso della cameretta c'è questa valvola. Quindi se voi la chiudete e come se stessi solando la camera rispetto al sistema da vuoto, altrimenti se l'aprite permette alla pompa di aspirare l'aria della cameretta. Poi l'ultima posizione che è possibile questo vento serve per far rientrare l'aria dentro la cameretta, perché altrimenti capite che si si può aprire con un po' di forza, però la differenza di pressione non aiuta sinceramente in questa operazione. E allora cosa vogliamo misurare? Abbiamo detto particelle alfa, quindi ritiamiamo alcune caratteristiche delle particelle alfa per capire cosa ci aspettiamo di osservare. Il decadimento alfa prevede l'emissione di particelle che sono dei nuclei di helio con un'energia specifica proprio perché è un decadimento che avviene semplicemente mettendo questa particella, quindi nello stato finale abbiamo il nucleo residuo più la particelle alfa. La particella alfa si prende tutta l'energia a disposizione mentre il nucleo residuo chiaramente rimane fermo. Questa energia quanto vale? Tipicamente abbiamo energia dell'ordine di 5 MeV, 4 o 5 MeV per i principali i soto, Pi, Alfa, in natura e posso in realtà avvenire anche diversi decadimenti verso i livelli eccitati dell'isotopofilio. 

Ad esempio qui vedete lo schema di decadimento dell'americio 241 che è uno di questi soto, Pi, utilizzeremo il laboratorio. Nella maggior parte dei casi vedete nel 86\% dei casi questo decadimento avviene verso il primo livello eccitato del nettugno 237. Cosa vorrà dire? Che questo nettugno poi decadrà e mettendo $\gamma$ in questo caso, però noi i $\gamma$ non li misuriamo quindi vedremo sostanzialmente delle alfa derivanti da questo decadimento con un'opportuna energia. Tuttavia questo non è l'unica modalità di decadimento, vedete ad esempio nel 12,5\% l'americio decade verso il secondo livello energetico del nucleo figlio. Questo vuol dire che le alfa emesse avranno un'energia un po' più bassa. Ok? Poi c'è un'altra probabilità di decadere a questo livello eccitato e una probabilità anche questa piccola dello 0,3\% di decadere allo stato fondamentale. Insomma da questo schema di decadimento cosa traiamo? Che informazioni traiamo? Ci aspettiamo che l'americio presenti 4 picchi sostanzialmente. Uno ovviamente che sarà più popolato rispetto agli altri, che corrisponde a questo decadimento verso il primo livello e gli altri in proporzione, ovviamente al branching resso, avranno una popolazione differente. Ora riusciamo effettivamente a distinguere questi picchi l'uno dall'altro. Se vediamo l'energia in gioco sono energie molto vicine all'uno all'altro che differiscono magari di poche decine di keV. Come faccio a capire se riesco a vederli oppure no, devo conoscere la risoluzione in energia. Quindi immaginiamo ad esempio che questo rivelatore si dice abbia una risoluzione in energia dell'1\%, questo mi dice che io posso distinguere energie che differiscono dell'1\% di 5 Mb. Ok? Quindi questo già mi dice se io posso andare a distinguere questi picchi o se sono sovrapposti l'uno all'altro. Quello che noi osserviamo sperimentalmente sono quindi dei picchi che sono abbastanza stretti perché la risoluzione buona ma chiaramente non sono piccolissimi e questo deriva dalla risoluzione del nostro rivelatore. Giusto per farvi capire appunto che il rivelatore che stiamo adoperando è sufficiente fermare le particelle in questo grafico vedete sull'asse verticale l'energia della particella che noi stiamo mandando verso il materiale e questo caso verso il silicio e sull'asse orizzontale il range quindi quanto spazio viene percorso da quella particella fino ad arrestarsi. E allora se io vado a vedere in questa scala dove si trova l'energia corrispondente grosso modo delle particelle alfa siamo su questa linea rossa, sono circa 5 Mb. A me interessa andare a vedere le alfa che sostanzialmente corrispondono a questa linea continua e quindi è la linea più in alto questo mi dice vedete che il range delle alfa in silicio, alfa da 5 Mb in silicio grosso modo se io scendo sotto, corrisponde a 10-20 Mb. Quindi mettere un rivelatore di 50 Mb certamente mi assicura di fermare tutte le particelle alfa che vengono emesse da questi sottopi. Come mai dobbiamo realizzare questa misura con un sistema da vuoto? Andiamo a vedere quanta energia perdono le alfa in aria. Questo è un grafico simile a quello di prima però gli assi sono scambiati quindi sull'asse orizzontale ci ritroviamo l'energia sull'asse verticale il range questa volta espressi in centimetri. Le curve si riferiscono probabilmente a due stime diverse, a due conti diversi però sono molto simili un all'altro e sono riferite a particelle alfa quindi se io vado a concentrarmi nella legione dei 5 Mb e vado a vedere quanti centimetri si percorrono prima di 

arrestare le particelle vedete che mi ritrovo intorno a 3,5 cm quindi la cameretta non è immensa, non è grande, è piccolina, però la distanza tra l'assorgente rivelatore e di alcuni centimetri certamente questo fa sì che magari le particelle alfa riescano ad arrivare al rivelatore ma nel frattempo in quei 2 cm di aria hanno perso praticamente quasi tutta la loro energia quindi la misura che vogliamo andate ad effettuare effettivamente la misura non realistica non è andata a misurare la reale energia delle particelle alfa quindi è necessario praticare un cerchio a livello di uoto. Addirittura gli elettroni hanno un range ancora più elevato se guardate appunto 3 Mb di elettroni vedete in aria che distanze percorrono anche 1200 cm quindi ad esempio nel caso di di elettroni chiaramente la perdita di energia in questo caso è molto più piccola in aria e non si vorrebbe il problema del stesso problema delle alfa. Allora la domanda è devo praticare il vuoto che vuoto devo creare? Un vuoto spinto ora vedremo esistono diversi gradi di vuoto a seconda della pressione che si raggiunge. Praticare il vuoto vuol dire sostanzialmente diminuire la pressione rispetto al valore della pressione atmosferica. Come faccio a stimare che pressione devo raggiungere anche perché vedremo che praticare il vuoto non è un'operazione facile e richiede delle opportune apparecchiature che cambiano a seconda del livello di vuoto che voglio raggiungere. E allora banalmente si può fare questa considerazione abbiamo detto che in generale pressionato atmosferica le particelle alfa percorrono pochi centimetri in aria. Se io diminuissi la pressione di un fattore mille cosa vuol dire? Che il range delle particelle alfa aumenterebbe di un fattore mille quindi se prima percorrevano due mille e due centimetri di minuello di un fattore mille la pressione arriverebbero a percorrere due mille centimetri. Quindi capite che la partia di energia diventa questo punto del tutto trascura. Quindi già soltanto diminuire di un fattore mille mi va più che bene. Chiaramente capite l'effetto perché avviene questa proporzionalità inversa tra pressione e range. Semplicemente perché variare la pressione e qui va a cambiare la densità e quindi il numero di urti che avvengono all'interno del gas. Quindi certamente diminuendo la pressione da mille mille bar che la pressionata atmosferica sta andando grossomodo a un mille bar mi permette di assicurarmi che queste alfa percorrerebbero anche decine di metri prima di fermarsi. Quindi in un paio di centimetri la perdita di energia ed è tutto irrisoria. Noi raggiungeremo pressioni dell'ordine di 10 alla men 1, 10 alla men 2 mille bar. Quindi va più che bene. Quindi un fattore 10 alla 4, 10 alla 5 è più basso rispetto alla pressionata atmosferica. E allora quali sono le misure che andremo a fare? In anzitutto andremo a misurare lo spettro in energia per realizzare anche una calibrazione che è un po' la stessa operazione che abbiamo fatto nel caso dei $\gamma$, anche lì mi suravate dei $\gamma$ di energia nota, però dovete andare a vedere in termini di canali, quindi la scala orizzontale che vi compariva sul grafico, ogni canale a che energia corrispondereva. Allo stesso modo lo dobbiamo fare per le alfa e si partirà con l'utilizzo di una sorgente mixed, cioè una sorgente dove sono presenti più i sottopinoti, in particolare gli sottopi che sono presenti in questa sorgente sono questi elentati qui, vedete, Nettunio, Americe e Cuyo. La Americe e Nettunio hanno anche, e anche Cuyo, hanno anche dei picchi satellite, cioè a causa dello schema di detadimento che può avvenire verso i livelli eccitati del nucleo o figlio, abbiamo l'emissione di alfa di diverse energia, chiaramente con intensità diverse, quindi è molto più probabile andare a misurare, ad esempio nel caso dell'Americe, questo decalimento piuttosto che quest'altro, però potrebbero essere presenti picchi satellite che possono essere anche identificate se andate a effettuare una misura ad alta statistica. Ora quando andrete a fare questa misura, cosa dovete considerare? Innanzitutto il tempo che abbiamo a disposizione, questa è una delle misure che dovete fare, quindi dovete decidere quanto tempo dedicare a questa prima misura, che è una misura comunque sia importante, perché da questa deve venire fuori la calibrazione e in più possiamo studiare la risoluzione del vostro sistema, del vostro e giusto come esempio vi riporto qui un tipico aspetto che viene fuori, addirittura qua già è stato realizzato un best fit con una funzione gaussiana, una per picco, ecetto che nel caso del netto unico che vedete qui dove fortunatamente la statistica ci permette di mettere in evidenza la presenza di un picco satellite, quindi capite che di tutto questo schema noi riusciamo a vedere solamente uno di questi picchi satellite, probabilmente le date al fatto, ma questo dovete valutare voi durante l'esperienza, due tal fatto sia che alcuni picchi hanno intensità molto basse, sia che alcuni picchi sono troppo vicini di uno all'altro e venono sostanzialmente confusi col picco principale. È chiaro che aumentare la statistica migliorerebbe certamente questo aspetto però abbiamo a disposizione un tempo limitato e quindi dovete decidere dove fermare la misura. Dalla posizione di questi picchi che vedete qui riportate in canali possiamo andare a fare una corrispondenza, canale, energia e chiaro che i picchi che vedete qui sono i picchi principali quindi ad esempio nel caso del netto unico e questo picco qui a 4.788 MeV, nel caso dell'americio e quello a 5.486 e infine nel caso del curio è 5.805. Tenendo questi valori che sono riportati sulla severità e facendoli corrispondere al centroide di questi picchi è possibile realizzare la calibrazione in energia, quindi una retta che vi vedete 

\textbf{parte 2}