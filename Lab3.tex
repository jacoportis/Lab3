\documentclass[openany,12pt]{book}
\usepackage[utf8]{inputenc}
\usepackage[letterpaper,top=2cm,bottom=2cm,left=3cm,right=3cm,marginparwidth=1.75cm]{geometry}
\usepackage{wrapfig}
\usepackage[psdextra, colorlinks=true, allcolors=black]{hyperref}
\usepackage[italian]{babel}
\usepackage{afterpage}
\newcommand\blankpage{%
    \null
    \thispagestyle{empty}%
    \newpage}
\usepackage{import}
\usepackage{physics}
\usepackage{mhchem}
\usepackage{amsfonts}
\usepackage{graphicx}
\usepackage{amssymb}
\usepackage{amsmath}
\usepackage{physics}
\newcommand{\notimplies}{%
  \mathrel{{\ooalign{\hidewidth$\not\phantom{=}$\hidewidth\cr$\implies$}}}}
  \newcommand{\notimpliedby}{%
  \mathrel{{\ooalign{\hidewidth$\not\phantom{=}$\hidewidth\cr$\impliedby$}}}}
\usepackage{enumitem}
\usepackage{array}
\usepackage{tikz}
\usepackage{ambienti}
\usepackage[open]{bookmark}
\ProvidesFile{puenc-greek.def}
\usepackage{textgreek}
\usepackage{float}
\newcommand*\circled[1]{\tikz[baseline=(char.base)]{
            \node[shape=circle,draw,inner sep=2pt] (char) {#1};}}
\setlength\parindent{0pt}%e si gode, toglie lo spostmento a destra di una nuova riga
\usepackage{caption}
\usepackage{subcaption}
\newcommand{\comment}[1]{}

%simboli
\usepackage{halloweenmath}
\usepackage{pifont}
\newcommand{\E}{È 
\hspace{0.1mm}}

\newcommand{\A}{Å 
\hspace{0.1mm}}

\begin{document}

\thispagestyle{empty}
\begin{center}

\begin{minipage}[c]{0.45\textwidth}
\begin{flushleft}
\includegraphics[width=0.8\textwidth]{logo-unict-orizzontale-grigio.png}
\end{flushleft}
\end{minipage}
\hfill
\begin{minipage}[c]{0.45\textwidth}
\begin{flushright}
\includegraphics[width=\textwidth]{logo_dfa_orizzontale}
\end{flushright}
\end{minipage}\\
\medskip
\hbox to \textwidth{\hrulefill}

\vfill
\vfill


\uppercase{\sc{ \Large{\textbf{Laboratorio di fisica III}}}}\\

\vfill
\large{A cura di Peppino Salumieri}

\vfill
\vfill
\hbox to \textwidth{\hrulefill}
{\sc anno accademico 2021-2022}
\end{center}

\afterpage{\blankpage}
\newpage

\tableofcontents

\newpage

\chapter{Arduino}
%\import{./Chapter/1-Sections/}{1-primo}

\textit{Carusi non ho tempo poi la faccio sta parte.}

\chapter{Sorgenti di radiazione}
\import{./Capitoli}{sorgenti_radiazione}
%\import{./Chapter/2-Sections/}{1-primo}

\chapter{Perdita di energia-particelle cariche}
\import{./Capitoli}{perdita_energia_particelle_cariche}
%\import{./Chapter/3-Sections/}{1-primo}

\end{document}
