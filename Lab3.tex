\documentclass[12pt,openany]{book}
\usepackage[utf8]{inputenc}
\usepackage[letterpaper,top=2cm,bottom=2cm,left=3cm,right=3cm,marginparwidth=1.75cm]{geometry}
\usepackage{wrapfig}
\usepackage[psdextra, colorlinks=true, allcolors=black]{hyperref}
\usepackage[italian]{babel}
\usepackage{epigraph}
\usepackage{afterpage}
\newcommand\blankpage{%
    \null
    \thispagestyle{empty}%
    \newpage} %serve a lasciare una pagina vuota
\usepackage{import}
\usepackage{physics}
\usepackage[version=4]{mhchem}
\usepackage{amsfonts}
\usepackage{mathtools}
\usepackage{graphicx}
\usepackage{amssymb}
\usepackage{amsmath}
\usepackage{physics}
\usepackage{enumitem}
\usepackage{array}
\usepackage{tikz,pgf}
\usetikzlibrary{snakes}
\usetikzlibrary{shapes}
\usepackage{lipsum}

%servono a mettere i simboli greci nei titoli
\usepackage[open]{bookmark}
\ProvidesFile{puenc-greek.def}
\usepackage{textgreek}

\usepackage{float}
\setlength\parindent{0pt}%e si gode, toglie lo spostmento a destra di una nuova riga
\setlength{\epigraphwidth}{0.5\textwidth}
\usepackage{caption}
\usepackage{subcaption}
\usepackage{fancyhdr}

\newcommand{\comment}[1]{}

\usepackage{ambienti} %pacchetto dove sono definiti i vari ambienti esercizio, approfondimento ecc

%simboli
\newcommand{\E}{È 
\hspace{0.1mm}}

\newcommand{\A}{Å 
\hspace{0.1mm}}

\begin{document}

\thispagestyle{empty}
\begin{center}

\begin{minipage}[c]{0.45\textwidth}
\begin{flushleft}
\includegraphics[width=0.8\textwidth]{logo-unict-orizzontale-grigio.png}
\end{flushleft}
\end{minipage}
\hfill
\begin{minipage}[c]{0.45\textwidth}
\begin{flushright}
\includegraphics[width=\textwidth]{logo_dfa_orizzontale}
\end{flushright}
\end{minipage}\\
\medskip
\hbox to \textwidth{\hrulefill}

\vfill
\vfill

\uppercase{\sc{ \Large{\textbf{Laboratorio di fisica III}}}}\\

\vfill
\large{A cura di Joey Butchers}

\vfill
\vfill
\hbox to \textwidth{\hrulefill}
{\sc Anno 2024}
\end{center}

\afterpage{\blankpage}
\newpage

\clearpage                       % Otherwise \pagestyle affects the previous page.
{                                % Enclosed in braces so that re-definition is temporary.
  \pagestyle{empty}              % Removes numbers from middle pages.
  \fancypagestyle{plain}         % Re-definition removes numbers from first page.
  {
    \fancyhf{}%                       % Clear all header and footer fields.
    \renewcommand{\headrulewidth}{0pt}% Clear rules (remove these two lines if not desired).
    \renewcommand{\footrulewidth}{0pt}%
  }
  \tableofcontents
  \thispagestyle{empty}          % Removes numbers from last page.
} %roba per mettere l'indice senza numero di pagina ne marks

\newpage

\pagestyle{fancy}
\fancyhf{}
\fancyhead[LE]{\nouppercase{\textbf{\thepage}\hfill\leftmark}}
\fancyhead[RO]{\nouppercase{\rightmark\hfill \textbf{\thepage}}}
\fancypagestyle{plain}{%
\fancyhf{} % cancella tutti i campi di intestazione e pi\‘e di pagina
%\fancyfoot[C]{\bfseries \thepage} % tranne il centro
\renewcommand{\headrulewidth}{0pt}
}

\chapter*{Guida al lettore}

\epigraph{There's a thin line between being a hero and being a memory.}{\textit{Optimus Prime}}

Questi appunti costituiscono una raccolta non esaustiva delle lezioni svolte dalla professoressa Paola La Rocca per il corso di Laboratorio di Fisica III del c.d.l. triennale in Fisica dell'università di Catania. Non costituiscono tuttavia un'alternativa alle lezioni e non sono pertanto sufficienti a prepararsi per sostenere l'esame. All'interno di tale testo appariranno spesso esempi e approfondimenti. Gli esempi, che appaiono così

\begin{esempio}[L'esperimento ALICE]
  Lorem ipsum dolor sit amet, consectetur adipiscing elit, sed do eiusmod tempor incididunt ut labore et dolore magna aliqua. Ut enim ad minim veniam, quis nostrud exercitation ullamco laboris nisi ut aliquip ex ea commodo consequat. 
\end{esempio}

riportano argomenti trattati a lezione e messi in evidenza solo per questione di ordine. Viceversa, gli approfondimenti appaiono così

\begin{approfondimento}[L'esperimento PLR]
  \footnotesize
  Duis aute irure dolor in reprehenderit in voluptate velit esse cillum dolore eu fugiat nulla pariatur. Excepteur sint occaecat cupidatat non proident, sunt in culpa qui officia deserunt mollit anim id est laborum.
\end{approfondimento}

e sono stati aggiunti in maniera autonoma dall'autore, pertanto non costituiscono parte degli argomenti trattati e il lettore può tranquillamente decidere di trascurarli senza per questo incappare in una comprensione incompleta.

\thispagestyle{empty}

\comment{

\part{Tecniche e strumenti di laboratorio}

\chapter{Arduino}
\import{./Capitoli}{1.arduino}

\chapter{La tecnologia del vuoto}
\import{./Capitoli}{11.tecnologie_vuoto}
\textit{Ho necessità di skippare questo capitolo al momento. Scusate per il disagio.}

}

\thispagestyle{empty}

\part{Interazione radiazione-materia}

\chapter{Sorgenti di radiazione}
\import{./Capitoli}{2.sorgenti_radiazione}

\chapter{Perdita di energia per particelle cariche pesanti}
\import{./Capitoli}{3.perdita_energia_particelle_cariche}

\afterpage{\blankpage}
\newpage

\chapter{Perdita di energia per elettroni}
\import{./Capitoli}{4.perdita_energia_elettroni}

\afterpage{\blankpage}
\newpage

\chapter[Interazione dei \texorpdfstring{$\gamma$}{\textgamma} con la materia]
{Interazione dei $\boldsymbol{\gamma}$ con la materia}
\import{./Capitoli}{5.interazione_gamma_materia}

\afterpage{\blankpage}
\newpage

\part{Rivelatori di particelle}

\chapter{Proprietà generali dei rivelatori}\label{chap:caratteristiche_rivelatori}
\import{./Capitoli}{6.caratteristiche_rivelatori}

\chapter{Rivelatori a gas}
\import{./Capitoli}{7.rivelatori_a_gas}

\afterpage{\blankpage}
\newpage

\chapter{Scintillatori}
\import{./Capitoli}{8.scintillatori}

\afterpage{\blankpage}
\newpage

\chapter{Fotosensori}
\import{./Capitoli}{9.fotosensori}

\afterpage{\blankpage}
\newpage

\chapter{Rivelatori a semiconduttore}
\import{./Capitoli}{10.rivelatori_a_semiconduttori}

%}

\part{Elementi di elettronica}

\chapter{Segnali dai rivelatori ed elettronica associata}
\import{./Capitoli}{12.elettronica}

%\chapter{Modulistica elettronica}\label{chap:modulistica_elettronica}
%\import{./Capitoli}{13.modulistica_elettronica}

\comment{%inizio commento

\chapter{Metodo Monte Carlo}
\import{./Capitoli}{15.montecarlo}

\chapter{Esperienze di laboratorio}
\import{./Capitoli}{16.esperienze}

\appendix

\chapter{Distribuzione di Poisson}
\import{./Capitoli}{A.distribuzione_Poisson}

\chapter{Multimetro digitale}
\import{./Capitoli}{B.multimetro}

}%fine commento

\end{document}
